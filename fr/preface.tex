\chapter*{Préface}
\addcontentsline{toc}{chapter}{Preface}
\label{chap:preface}

La gestion de source distribué est encore un territoire peu exploré
et par conséquent a grandis très rapidement grâce à la volonté de
ses explorateurs.

Je rédige un libre sur ce sujet car je crois que c'est un sujet 
important qui mérite bien un guide du ``terrain''. J'ai choisi d'écrire
un livre sur Mercurial car c'est l'outil le plus simple pour découvrir
ce nouveau monde et que, en outre, il répond très bien au besoin de
réels environements, là où d'autres outils de gestion de source n'y
parviennent pas.

\section{Cet ouvrage est un travail en cours}

Je publie ce livre tout en continuant à l'écrire, dans l'espoir qu'il
vous sera utile. J'espère aussi que les lecteurs pourront ainsi contribuer
si ils souhaitent.

\section{A propros des exemples de ce livre}

Ce livre a une approche particulière des extrait de code. Ceci sont 
toujours ``dynamique''---chacun est le résultat d'un script shell qui
exécute les commandes mercurial que vous voyez. Chaque fois qu'une 
image du livre est construite tout les scripts d'exemple sont exécutés
automatiquement, et les résultats comparés à ceux attendus.

Cette approche a l'avantage de garantir que les exemples sont toujours
juste; ils montrent \emph{exactement} le comportement de la version de
Mercurial spécifié dans la couverture de ce livre. Si je met à jour cette
version, et que les commandes changent, la génération du livre échouera.

Il y a un petit désavantage à cette approche, qui que les dates et les
temps onl

There is a small disadvantage to this approach, which is that the
dates and times you'll see in examples tend to be ``squashed''
together in a way that they wouldn't be if the same commands were
being typed by a human.  Where a human can issue no more than one
command every few seconds, with any resulting timestamps
correspondingly spread out, my automated example scripts run many
commands in one second.

As an instance of this, several consecutive commits in an example can
show up as having occurred during the same second.  You can see this
occur in the \hgext{bisect} example in section~\ref{sec:undo:bisect},
for instance.

So when you're reading examples, don't place too much weight on the
dates or times you see in the output of commands.  But \emph{do} be
confident that the behaviour you're seeing is consistent and
reproducible.

\section{Colophon---this book is Free}

This book is licensed under the Open Publication License, and is
produced entirely using Free Software tools.  It is typeset with
\LaTeX{}; illustrations are drawn and rendered with
\href{http://www.inkscape.org/}{Inkscape}.

The complete source code for this book is published as a Mercurial
repository, at \url{http://hg.serpentine.com/mercurial/book}.

%%% Local Variables: 
%%% mode: latex
%%% TeX-master: "00book"
%%% End: 
