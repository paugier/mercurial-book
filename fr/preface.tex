\chapter*{Préface}
\addcontentsline{toc}{chapter}{Préface}
\label{chap:preface}

La gestion de source distribué est encore un territoire peu exploré
et qui, par conséquent, a grandi très rapidement grâce à la seule 
volonté de ses explorateurs.

Je rédige un livre sur ce sujet car je crois que c'est un sujet 
important qui mérite bien un guide du ``terrain''. J'ai choisi d'écrire
ce livre sur Mercurial car c'est l'outil le plus simple pour découvrir
ce nouveau monde et que, en outre, il répond très bien au besoin de
réels environnements, là où d'autres outils de gestion de source n'y
parviennent pas.

\section{Cet ouvrage est un travail en cours}

Je publie ce livre tout en continuant à l'écrire, dans l'espoir qu'il
vous sera utile. J'espère aussi que les lecteurs pourront ainsi contribuer
si ils souhaitent.

\section{À propros des exemples de ce livre}

Ce livre a une approche particulière des exemples d'exécution. Ils sont 
toujours ``dynamiques''---chacun est le résultat d'un script shell qui
exécute les commandes Mercurial que vous voyez. Chaque fois qu'une 
image du livre est construite à partir des sources, tous les scripts d'exemple
sont exécutés automatiquement, et les résultats comparés à ceux attendus.

Cette approche a l'avantage de garantir que les exemples sont toujours
justes ; ils montrent \emph{exactement} le comportement de la version de
Mercurial spécifiée dans la couverture de ce livre. Si je met à jour cette
version, et que les commandes changent, la génération du livre échouera.

Il y a un petit désavantage à cette approche, les dates et les
durées que vous verrez dans ces exemples ont tendances à être
``réduits'' de manière très différente d'une exécution manuelle. Un être humain
ne peut exécuter qu'une commande toutes les secondes, alors que mes scripts
automatisés en exécutent plusieurs en une seule seconde.

Ainsi, en une seule seconde, plusieurs ``commits'' peuvent avoir lieu
au sein d'un exemple. Vous le constatez, entre autres, dans les 
exemples sur \hgext{bisect}, dans la section~\ref{sec:undo:bisect}.

En conséquence, quand vous lisez les exemples, n'accordez pas trop
d'importance aux dates et aux durées d'exécution des commandes. Mais
\emph{soyez sûr} que le comportement que vous voyez est cohérent et
reproductible.

\section{Colophon---Cet ouvrage est libre}

Ce livre est publié sous la licence ``Open Publication License''
%\footnote{Pour plus de renseignements : 
%\url{http://opencontent.org/openpub/}{Open Publication License} }
, et est construit uniquement à l'aide de logiciels libres. Il est mis
en forme avec \LaTeX{} ; et les illustrations sont réalisées avec 
\url{http://www.inkscape.org/}{Inkscape}.

L'ensemble des fichiers sources de cet ouvrage sont publiés dans un
dépot mercurial  \url{http://hg.serpentine.com/mercurial/book}.

%%% Local Variables: 
%%% mode: latex
%%% TeX-master: "00book"
%%% End: 
