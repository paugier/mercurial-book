\chapter{Introduction}
\label{chap:intro}

\section{À propos de la gestion source}

La gestion de sources est un processus permettant de gérer différentes
versions de la même information. Dans sa forme la plus simple, c'est
ce que tout le monde fait manuellement : quand vous modifiez
un fichier, vous le sauvegarder sous un nouveau nom contenant un numéro,
à chaque fois plus grand que celui de la version précédente.

Ce genre de gestion de version manuelle est cependant facilement sujette 
à des erreurs, ainsi, depuis longtemps, des logiciels existent pour
résoudre cette problématique. Les premiers outils de gestion de sources
étaient destinés à aider un seul utilisateur, à automatiser la gestion
des versions d'un seul fichier. Dans les dernières décades, cette cible 
s'est largement agrandie, ils gèrent désormais de multiples fichiers, et
aident un grand nombre de personnes à travailler ensemble. Les outils les
plus modernes n'ont aucune difficultés à gérer plusieurs milliers de 
personnes travaillant ensemble sur des projets regroupant plusieurs 
centaines de milliers de fichiers.

\subsection{Pourquoi utiliser un gestionnaire de source ?}

Il y a de nombreuses raisons pour que vous ou votre équipe souhaitiez
utiliser un outil automatisant la gestion de version pour votre projet.
\begin{itemize}
\item L'outil se chargera de suivre l'évolution de votre projet, sans
que vous n'ayez à le faire. Pour chaque modification, vous aurez à votre
disposition un journal indiquant \emph{qui} a fait quoi, \emph{pourquoi}
ils l'ont fait, \emph{quand} ils l'ont fait, et \emph{ce} qu'ils ont
modifiés.
\item Quand vous travaillez avec d'autres personnes, les logiciels de 
gestion de source facilitent le travail collaboratif. Par exemple, quand
plusieurs personnes font, plus ou moins simultanément, des modifications
incompatibles, le logiciel vous aidera à identifier et résoudre les conflits.
\item L'outil vous aidera à réparer vos erreurs. Si vous effectuez un changement
qui se révèlera être une erreur, vous pourrez revenir à une version
antérieure d'un fichier ou même d'un ensemble de fichiers. En fait, un outil de
gestion de source \emph{vraiment} efficace vous permettra d'identifier à quel
moment le problème est apparu (voir la section~\ref{sec:undo:bisect} pour plus
de détails).
\item L'outil vous permettra aussi de travailler sur plusieurs versions différentes
de votre projet et à gérer l'écart entre chacune.
\end{itemize}
La plupart de ces raisons ont autant d'importances ---du moins en théorie--- que
vous travailliez sur un projet pour vous, ou avec une centaine d'autres
personnes.

Une question fondamentale à propos des outils de gestion de source, qu'il s'agisse
du projet d'une personne ou d'une grande équipe, est quelles sont ses  
\emph{avantages} par rapport à ses \emph{coût}. Un outil qui est difficile à 
utiliser ou à comprendre exigera un gros effort d'adaptation.

Un projet de cinq milles personnes s'effondrera très certainement de lui même
sans aucun processus et outil de gestion de source. Dans ce cas, le coût 
d'utilisation d'un logiciel de gestion de source est dérisoire, puisque 
\emph{sans}, l'échec est presque garanti.

D'un autre coté, un ``rapide hack'' d'une personne peut sembler un contexte
bien pauvre pour utiliser un outil de gestion de source, car, bien évidement
le coût d'utilisation dépasse le coût total du projet. N'est ce pas ?

Mercurial supporte ces \emph{deux} échelles de travail. Vous pouvez apprendre
les bases en quelques minutes seulement, et, grâce à sa performance, vous pouvez
l'utiliser avec facilité sur le plus petit des projets. Cette simplicité 
signifie que vous n'avez pas de concepts obscurs ou de séquence de commandes
défiant l'imagination, sans aucune corrélation avec \emph{ce que vous êtes 
vraiment entrain de faire}. En même temps, ces mêmes performances et sa 
nature ``peer-to-peer'' vous permet d'augmenter, sans difficulté, son 
utilisation à de très grand projet.

Aucun outil de gestion de source ne peut sauver un projet mal mené, mais un
bon outil peut faire une grande différence dans la fluidité avec laquelle
vous pourrez travailler avec.

\subsection{Les multiples noms de la gestion de source}

La gestion de source\footnote{NdT: J'ai utilisé systématiquement le terme
``gestion de source'' à travers tout l'ouvrage. Ce n'est pas forcement la
meilleur traduction, et ceci peut rendre la lecture un peu lourde, mais je
pense que le document y gagne en clarté et en précision.} est un domaine
divers, tellement qu'il n'existe pas une seul nom ou acronyme pour le désigner.
Voilà quelqu'uns des noms ou 
acronymes que vous rencontrerez le plus souvent\footnote{NdT: J'ai conservé la
liste des noms en anglais pour des raisons de commodité (ils sont plus
``googelable''). En outre, j'ai opté  conserver l'ensemble des opérations de
Mercurial (\textit{commit},\textit{push}, \textit{pull},...) en anglais, là
aussi pour faciliter la lecture d'autres documents en anglais, et aussi
l'utilisation de Mercurial}.

:
\begin{itemize}
\item \textit{Revision control (RCS)} ;
\item Software configuration management (SCM), ou \textit{configuration management} ;
\item \textit{Source code management} ;
\item \textit{Source code control}, ou \textit{source control} ;
\item \textit{Version control (VCS)}.
\end{itemize}

Certains personnes prétendent que ces termes ont en fait des sens
différents mais en pratique ils se recouvrent tellement qu'il n'y a pas
réellement de manière pertinente de les distinguer.

\section{Une courte histoire de la gestion de source}

Le plus célèbre des anciens outils de gestion de source est \textit{SCCS (Source
Code Control System)}, que Marc Rochkind conçu dans les laboratoire de recherche de Bell 
(\textit{Bell Labs}), dans le début des années 70. \textit{SCCS} ne fonctionnait que sur des fichiers individuels, et obligeait à chaque personne travaillant sur le projet d'avoir un accès à un répertoire de travail commun, sur le même système.
Seulement une seule personne pouvait modifier un fichier au même moment, ce fonctionnement était assuré par l'utilisation de verrou (``lock''). Il était courant que des personnes verrouillent
des fichiers, et plus tard, oublient de le déverrouiller; empêchant n'importe qui d'autre de 
travailler sur ces fichiers sans l'aide de l'administrateur...

Walter Tichy a développé une alternative libre à \textit{SCCS} au début des années 80, qu'il
nomma \textit{RSC (Revison Control System)}.  Comme \textit{SCCS}, \textit{RCS}
demander aux développeurs de travailler sur le même répertoire partagé, et de verrouiller les
fichiers pour se prémunir de tout conflit issue de modifications concurrentes.

Un peu plus tard dans les années 1980, Dick Grune utilisa \textit{RCS} comme une brique de base pour un ensemble de scripts \textit{shell} qu'il intitula cmt, avant de la renommer en \textit{CVS (Concurrent Versions System)}.  La grande innovation de CVS était que les développeurs pouvaient travailler simultanément et indépendamment dans leur propre espace de travail. Ces espaces de travail privés assuraient que les développeurs ne se marchent pas mutuellement sur les pieds, comme c'était souvent le cas avec RCS et SCCS. Chaque développeur disposait donc de sa copie de tous les fichiers du projet, et ils pouvaient donc librement les modifier. Ils devaient néanmoins effectuer la ``fusion'' (\textit{``merge''}) de leurs fichiers, avant d'effectuer le ``commit'' de leur modifications sur le dépôt central.

Brian Berliner repris les scripts de Grune's et les réécris en~C, qu'il publia en 1989. Depuis, ce code a été modifié jusqu'à devenir la version moderne de CVS. CVS a acquis ainsi la capacité de fonctionner en réseau, transformant son architecture en client/serveur. L'architecture de CVS est centralisée, seul le serveur a une copie de l'historique du projet. L'espace de travail client ne contient qu'une copie de la dernière version du projet, et quelques métadonnées pour indiquer où le serveur se trouve. CVS a été un grand succès, aujourd'hui c'est probablement l'outil de gestion de contrôle le plus utilisé au monde. 

Au début des années 1990, Sun Microsystmes développa un premier outil de gestion de source distribué, nommé TeamWare. Un espace de travail TeamWare contient une copie complète de l'historique du projet. TeamWare n'a pas de notion de dépôt central. (CVS utilisait RCS pour le stockage de l'historique, TeamWare utilisait SCCS).

Alors que les années 1990 avançaient, les utilisateurs ont pris conscience d'un certain nombre de problèmes avec CVS. Il enregistrait simultanément des modifications sur différents fichiers individuellement, au lieu de les regrouper dans une seule opération cohérente et atomique. Il ne gère pas bien sa hiérarchie de fichier, il est donc assez aisé de créer le chaos en renommant les fichiers et les répertoires. Pire encore, son code source est difficile à lire et à maintenir, ce qui agrandit largement le ``niveau de souffrance'' associé à la réparation de ces problèmes d'architecture de manière prohibitive. 

En 2001, Jim Blandy et Karl Fogel, deux développeurs qui avaient travaillés sur CVS, initièrent un projet pour le remplacer par un outil qui aurait une meilleur architecture et un code plus propre. Le résultat, Subversion, ne quitte pas le modèle centralisé et client/server de CVS, mais ajoute les opérations de ``commit'' atomique sur de multiples fichiers, une meilleure gestion des espaces de noms, et d'autres fonctionnalités qui en font un meilleur outil que CVS. Depuis sa première publication, il est rapidement devenu très populaire.

Plus ou moins de simultanément, Graydon Hoare a commencé sur l'ambitieux système de gestion distribué Monotone. Bien que Monotone corrige plusieurs défaut de CVS's tout en offrant une architecture ``peer-to-peer'', il va aussi plus loin que la plupart des outils de révision de manière assez innovante. Il utilise des ``hash'' cryptographique comme identifiant, et il a une notion complète de ``confiance'' du code issue de différentes sources.

Mercurial est né en 2005. Bien que très influencé par Monotone, Mercurial se concentre sur la facilité d'utilisation, les performances et la capacité à monter en charge pour de très gros projets.

\section{Tendances de la gestion de source}

Il y a eu une tendance évidente dans le développement et l'utilisation d'outil de gestion de source depuis les quatre dernières décades, au fur et à mesure que les utilisateurs se sont habitués à leur outils et se sont sentis contraint par leur limitations.

La première génération commença simplement par gérer un fichier unique sur un ordinateur individuel. Cependant, même si ces outils présentaient une grande avancée par rapport à la gestion manuelle des versions, leur modèle de verrouillage et leur utilisation limitée à un seul ordinateur rendait leur utilisation possible uniquement dans une très petite équipe. 

La seconde génération a assoupli ces contraintes en adoptant une architecture réseau et centralisé, permettant de gérer plusieurs projets entiers en même temps. Alors que les projets grandirent en taille, ils rencontrèrent de nouveau problèmes. Avec les clients discutant régulièrement avec le serveurs, la montée en charge devint un réel problème sur les gros projets. Une connexion réseau peu fiable pouvant complètement empêcher les utilisateurs distant de dialoguer avec le serveur. Alors que les projets \textit{Open Source} commencèrent à mettre en place des accès en lecture seule disponible anonymement, les utilisateurs sans les privilèges de ``commit'' réalisèrent qu'ils ne pouvaient pas utiliser les outils pour collaborer naturellement avec le projet, comme ils ne pouvaient pas non plus enregistrer leurs modifications.

La génération actuelle des outils de gestion de source est ``peer-to-peer'' par nature. Tout ces systèmes ont abandonné la dépendance à un serveur central, et ont permis à leur utilisateur de distribuer les données de leur gestion de source à qui en a besoin. La collaboration à travers Internet a transformée la contrainte technologique à une simple question de choix et de consencus. Les outils moderne peuvent maintenant fonctionner en mode déconnecté sans limite et de manière autonome, la connexion au réseau n'étant nécessaire que pour synchroniser les modifications avec les autres dépôts.

\section{Quelques avantages des gestionnaire de source distribué}

Même si les gestionnaire de source distribués sont depuis plusieurs années
assez robustes et aussi utilisables que leur prédécesseurs, les utilisateurs
d'autres outils n'y ont pas encore été sensibilisés. Les gestionnaires
de sources distribués se distinguent particulièrement de leurs équivalents
centralisés de nombreuses manières.

Pour un développeur individuel, ils restent beaucoup plus rapides que les
outils centralisés. Cela pour une raison simple : un outil centralisé doit
toujours dialoguer à travers le réseau pour la plupart des opérations, car
presque toutes les métadonnées sont stockées sur la seule copie du serveur
central. Un outil distribué stocke toute ses métadonnées localement. À tâche
égale, effectuer un échange avec le réseau ajoute un délai aux outils 
centralisés. Ne sous-estimez pas la valeur d'un outil rapide : vous allez
passer beaucoup de temps à interagir avec un logiciel de gestion de sources.

Les outils distribués sont complètement indépendants des aléas de votre serveur,
d'autant plus qu'ils répliquent les métadonnées à beaucoup d'endroits. Si
votre serveur central prend feu, vous avez intérêt à ce que les médias de 
sauvegardes soient fiables, et que votre dernier ``backup'' soit récent et
fonctionne sans problème. Avec un outil distribué, vous avez autant de 
``backup'' que de contributeurs.

En outre, la fiabilité de votre réseau affectera beaucoup moins les
outils distribués. Vous ne pouvez même pas utiliser un outil centralisé
sans connexion réseau, à l'exception de quelques commandes, très limitées. 
Avec un outil distribué, si votre connexion réseau tombe pendant que vous
travaillez, vous pouvez ne même pas vous en rendre compte. La seule chose
que vous ne serez pas capable de faire sera de communiquer avec des dépôts
distants, opération somme toute assez rare par comparaison aux opérations
locales. Si vous avez une équipe de collaborateurs très dispersés ceci peut
être significatif.

\subsection{Avantages pour les projets \textit{Open Source}}

Si vous prenez goût à un projet \textit{Open Source} et que vous
décidez de commencer à toucher à son code, et que le projet utilise
un gestionnaire de source distribué, vous êtes immédiatement un "pair"
avec les personnes formant le ``cœur'' du projet. Si ils publient
leurs dépôts, vous pouvez immédiatement copier leurs historiques de
projet, faire des modifications, enregistrer votre travail en utilisant
les même outils qu'eux. Par comparaison, avec un outil centralisé, vous
devez utiliser un logiciel en mode ``lecture seule'' à moins que 
quelqu'un ne vous donne les privilèges de ``commit'' sur le serveur
central. Avant ça, vous ne serez pas capable d'enregistrer vos 
modifications, et vos propres modifications risqueront de se 
corrompre chaque fois que vous essayerez de mettre à jour à votre
espace de travail avec le serveur central.

\subsubsection{Le non-problème du \textit{fork}}

Il a été souvent suggéré que les gestionnaires de source distribués
posent un risque pour les projets \textit{Open Source} car ils 
facilitent grandement la création de ``fork''\footnote{NdT:Création 
d'une version alternative du logiciel}. %%% TODO: Link to Wikipedia
Un ``fork'' apparait quand il y des divergences d'opinion ou d'attitude
au sein d'un groupe de développeurs qui aboutit à la décision de ne 
plus travailler ensemble. Chacun parti s'empare d'une copie plus ou moins
complète du code source du projet et continue dans sa propre direction.

Parfois ces différents partis décident de se réconcilier. Avec un 
serveur central, l'aspect \emph{technique} de cette réconciliation
est un processus douloureux, et essentiellement manuel. Vous devez
décider quelle modification est ``la gagnante'', et replacer, par un
moyen ou un autre, les modifications de l'autre équipe dans l'arborescence
du projet. Ceci implique généralement la perte d'une partie de l'historique 
d'un des partie, ou même des deux.

Ce que les outils distribués permettent à ce sujet est probablement
la \emph{meilleur} façon de développer un projet. Chaque modification
que vous effectuez est potentiellement un ``fork''. La grande force de 
cette approche est que les gestionnaires de source distribués doivent être
vraiment très efficasses pour \emph{fusionner}\footnote{NdT:j'ai choisi de
traduire ici \textit{merging} par ``fusionner'' pour des raisons de clarté}
des ``forks'', car les ``forks'', dans ce contexte, arrivent tout le
temps.

Si chaque altération que n'importe qui effectue, à tout moment, est vue
comme un ``fork'' à fusionner, alors ce que le monde de l'\textit{Open 
Source} voit comme un ``fork'' devient \emph{uniquement} une problématique 
sociale. En fait, les outils de gestions de source distribués \emph{réduisent} 
les chances de ``fork'':
\begin{itemize}
\item Ils éliminent la distinction sociale qu'imposent les outils centralisés
	entre les membres du projets (ceux qui ont accès au ``commit'') et ceux de l'
	extérieur (ce qui ne l'ont pas).
\item Ils rendent plus facile la réconciliation après un ``fork'' social, car
	tout ce qu'elle implique est juste une simple fusion.
\end{itemize}

Certaines personnes font de la résistance envers les gestionnaires de source
distribués parce qu'ils veulent garder un contrôle ferme de leur projet, et
ils pensent que les outils centralisés leur fournissent ce contrôle. Néanmoins,
si c'est votre cas, sachez que si vous publier votre dépôt CVS ou Subversion
de manière publique, il existe une quantité d'outils disponibles pour récupérer
entièrement votre projet et son historique (quoique lentement) et le récréer 
ailleurs, sans votre contrôle. En fait, votre contrôle sur votre projet est 
illusoire, vous ne faites qu'interdire à vos collaborateurs de travailler
de manière fluide, en disposant d'un miroir ou d'un ``fork'' de votre
historique.
%%%TODO: Fussy, those last sentences are not really well translated:
%%%no problem for me (wilk)
%However, if you're of this belief, and you publish your CVS or Subversion 
%repositories publically, there are plenty of tools available that can pull 
%out your entire project's history (albeit slowly) and recreate it somewhere 
%that you don't control.  So while your control in this case is illusory, you are
%forgoing the ability to fluidly collaborate with whatever people feel
%compelled to mirror and fork your history.

\subsection{Avantages pour les projets commerciaux}

Beaucoup de projets commerciaux sont réalisés par des équipes éparpillées
à travers le globe. Les contributeurs qui sont loin du serveur central
devront subir des commandes lentes et même parfois peu fiables. Les 
solutions propriétaires de gestion de source, tentent de palier ce problème 
avec des réplications de site distant qui sont à la fois coûteuses à mettre
en place et lourdes à administrer. Un système distribué ne souffre pas
de ce genre de problèmes. En outre, il est très aisé de mettre en place
plusieurs serveurs de références, disons un par site, de manière à ce qu'il
n'y est pas de communication redondante entre les dépôts, sur une connexion
longue distance souvent onéreuse.

Les systèmes de gestion de source supportent généralement assez mal la 
montée en charge. Ce n'est pas rare pour un gestionnaire de source centralisé 
pourtant onéreux de s'effondrer sous la charge combinée d'une douzaine 
d'utilisateurs concurrents seulement. Une fois encore, la réponse à cette problématique 
est généralement encore la mise en place d'un ensemble complexe de serveurs
synchronisés par un mécanisme de réplication. Dans le cas d'un gestionnaire
de source distribué, la charge du serveur central --- si vous avez un--- est
plusieurs fois inférieure (car toutes les données sont déjà répliquées ailleurs),
un simple serveur, pas très cher, peut gérer les besoins d'une plus grande
équipe, et la réplication pour balancer la charge devient le
travail d'un simple script.

Si vous avez des employés sur le terrain, entrain de chercher à résoudre un soucis sur
le site d'un client, ils bénéficieront aussi d'un gestionnaire de source
distribués. Cet outil leur permettra de générer des versions personnalisées,
d'essayer différentes solutions, en les isolant aisément les une des autres,
et de rechercher efficacement à travers l'historique des sources, la cause
des bugs ou des régressions, tout ceci sans avoir besoin de la moindre 
connexion au réseau de votre compagnie.

\section{Pourquoi choisir Mercurial?}

Mercurial a plusieurs caractéristiques qui en font un choix particulièrement
pertinent pour la gestion de source:
\begin{itemize}
	\item Il est facile à apprendre et à utiliser ;
	\item il est léger et performant ;
	\item il monte facilement en charge ; 
	\item il est facile à personnaliser ;
\end{itemize}

Si vous êtes déjà familier d'un outil de gestion de source, vous serez
capable de l'utiliser en moins de 5 minutes. Sinon, ça ne sera pas beaucoup
plus long\footnote{NdT: Pour appuyer le propos de l'auteur, je signale que 
j'utilise Mercurial comme outil d'initiation à la gestion de contrôle dans
des travaux pratique à l'ESME Sudria (\url{http://www.esme.fr}) et que les
élèves le prennent en main sans difficulté majeur malgré l'approche distribuée.}. 
Les commandes utilisées par Mercurial, comme ses fonctionnalités, sont 
généralement uniformes et cohérentes, et vous pouvez donc ainsi garder en tête 
simplement quelques règles générales, plutôt qu'un lot complexe d'exceptions.

Sur un petit projet, vous pouvez commencer à travailler avec Mercurial en
quelques instants. Ajouter des modifications ou des branches, transférer 
ces modifications (localement ou via le réseau), et les opérations 
d'historique ou de statut sont aussi très rapide. Mercurial reste hors de 
votre chemin grâce à sa simplicité d'utilisation et sa rapidité d'exécution.

L'utilité de Mercurial ne se limite pas à des petits projets: il est 
aussi utilisé par des projets ayant des centaines ou même des milliers
de contributeurs, avec plusieurs dizaines de milliers de fichiers, et des
centaines de méga de code source.

Voici une liste non exhaustive des projets complexes ou critiques utilisant 
Mercurial :
%TODO
% For both spanish and english version, add the following examples:
\begin{itemize}
	\item Firefox ;
	\item OpenSolaris ;
	\item OpenJDK (utilisant en outre l'extension ``forest'' pour gérer
	ses sous modules);
\end{itemize}
% TODO: Also add appropriate link.

Si les fonctionnalités cœur de Mercurial ne sont pas suffisantes pour vous, 
il est très aisé d'en construire dessus. Mercurial est adapté à l'utilisation
de scripts, et son implémentation interne en Python, propre et claire,
rend encore plus facile l'ajout de fonctionnalité sous forme d'extensions. Il
en existe déjà un certain nombre de très populaires et très utiles, 
dont le périmètre va de la recherche de bugs à l'amélioration des performances.

\section{Mercurial comparé aux autres outils}

Avant que vous n'alliez plus loin, comprenez bien que cette section
reflète mes propres expériences, et elle est donc (j'ose le dire)
peu objective. Néanmoins, j'ai utilisé les outils de gestion de source
listé ci dessous, dans la plupart des cas, pendant plusieurs années.
%% TODO: Fussy translation.

\subsection{Subversion}

Subversion est un outil de gestion de source les plus populaire, il fût 
développé pour remplacer CVS. Il a une architecture client/server centralisée.

Subversion et Mercurial ont des noms de commandes très similaires pour 
les mêmes opérations, ainsi si vous êtes familier avec l'un, c'est facile
d'apprendre l'autre. Ces deux outils sont portable sur les systèmes 
d'exploitation les plus populaires\footnote{NdT:Mercurial fonctionne sans problèmes
sur OpenVMS à l'ESME Sudria \url{http://www.esme.fr}, compte tenu que Subversion a été 
développé en C, je ne suis pas sûr que son portage aurait été aussi aisé.}.
%TODO: Backport this statement in english and spanish 

Avant la version 1.5, Subversion n'offrait aucune forme de support pour les fusions. Lors 
de l'écriture de ce livre, ces capacités de fusion étaient nouvelles, et réputés pour être
\href{http://svnbook.red-bean.com/nightly/en/svn.branchmerge.advanced.html#svn.branchmerge.advanced.finalword}{complexes
et buggués}.

Mercurial dispose d'un avantage substantiel en terme de performance par rapport à 
Subversion sur la plupart des opérations que j'ai pu tester. J'ai mesuré
une différence de performance allant de deux à six fois plus rapide avec
le système de stockage de fichier local de Subversion~1.4.3 
(\emph{ra\_local}), qui est la méthode d'accès la plus rapide disponible. Dans
un déploiement plus réaliste, impliquant un stockage réseau, Subversion 
serait encore plus désavantagé. Parce que la plupart des commandes Subversion
doivent communiquer avec le serveur et que Subversion n'a pas de mécanisme
de réplication, la capacité du serveur et la bande passante sont devenu des
goulots d'étranglement pour les projets de tailles moyennes ou grandes.

En outre, Subversion implique une surcharge substantielle dans le stockage local
de certaines données, pour éviter des transactions avec le serveur, pour 
certaines opérations communes, tel que la recherche des fichier modifiées
(\texttt{status}) et l'affichage des modifications par rapport la révision 
courante (\texttt{diff}). En conséquence, un répertoire de travail Subversion
a souvent la même taille, ou est plus grand, qu'un dépôt Mercurial et son
espace de travail, bien que le dépôt Mercurial contienne l'intégralité de
l'historique.

Subversion est largement supporté par les outils tierces. Mercurial est
actuellement encore en retrait de ce point de vue. L'écart se réduit, néanmoins,
et en effet certains des outils graphiques sont maintenant supérieurs à leurs
équivalents Subversion. Comme Mercurial, Subversion dispose d'un excellent
manuel utilisateur.

Parce que Subversion ne stocke pas l'historique chez ses clients, il est 
parfaitement adapté à la gestion de projets qui doivent suivre un ensemble
de larges fichiers binaires et opaques. Si vous suivez une cinquantaine de
versions d'un fichier incompressible de 10MB, l'occupation disque coté client
d'un projet sous Subversion restera à peu près constante. A l'inverse, 
l'occupation disque du même projet sous n'importe lequel des gestionnaire
de source distribués grandira rapidement, proportionnellement aux nombres
de versions, car les différences entre chaque révisions sera très grande.

En outre, c'est souvent difficile ou, généralement, impossible de fusionner
des différences dans un fichier binaire. La capacité de Subversion de 
verrouiller des fichiers, pour permettre à l'utilisateur d'être le seul
à le mettre à jour (``commit'') temporairement, est un avantage significatif
dans un projet doté de beaucoup de fichiers binaires.

Mercurial peut importer l'historique depuis un dépôt Subversion. Il peut
aussi exporter l'ensemble des révisions d'un projet vers un dépôt Subversion.
Ceci rend très facile de ``prendre la température'' et d'utiliser Mercurial et Subversion
en parallèle, avant de décider de migrer vers Mercurial. La conversion de 
l'historique est incrémentale, donc vous pouvez effectuer une conversion 
initiale, puis de petites additions par la suite pour ajouter les nouvelles
modifications.

\subsection{Git}

Git est un outil de gestion de source distribué qui fût développé pour gérer
le code source de noyau de Linux. Comme Mercurial, sa conception initiale à 
était inspirée par Monotone.

Git dispose d'un ensemble conséquent de commandes, avec plus de~139 commandes
individuelles pour la version~1.5.0. Il a aussi la réputation d'être difficile
à apprendre. Comparé à Git, le point fort de Mercurial est clairement sa 
simplicité.

En terme de performance, Git est extrêmement rapide. Dans la plupart des
cas, il est plus rapide que Mercurial, tout du moins sur Linux, alors que 
Mercurial peut être plus performant sur d'autres opérations. Néanmoins, sur
Windows, les performances et le niveau de support général fourni par Git, 
au moment de l'écriture de cet ouvrage, est bien derrière celui de Mercurial.

Alors que le dépôt Mercurial ne demande aucune maintenance, un dépôt Git
exige d'exécuter manuellement et régulièrement la commande ``repacks'' sur
ces métadonnées. Sans ceci, les performances de git se dégrade, et la 
consommation de l'espace disque augmente rapidement. Un serveur qui contient
plusieurs dépôts Git qui ne sont pas régulièrement et fréquemment ``repacked''
deviendra un vrai problème lors des ``backups'' du disque, et il y eu des
cas, où un ``backup'' journalier pouvait durer plus de~24 heures. Un dépôt
fraichement ``repacked'' sera légèrement plus petit qu'un dépôt Mercurial,
mais un dépôt non ``repacked'' est beaucoup plus grand.

Le cœur de Git est écrit en C. La plupart des commandes Git sont implémentées
sous forme de scripts Shell ou Perl, et la qualité de ces scripts varie
grandement. J'ai plusieurs fois constaté que certains de ces scripts étaient
chargés en mémoire aveuglément et que la présence d'erreurs pouvait s'avérer
fatale.

Mercurial peut importer l'historique d'un dépôt Git.

\subsection{CVS}

CVS est probablement l'outil de gestion de source le plus utilisé aujourd'hui
dans le monde. À cause de son manque de clarté interne, il n'est plus 
maintenu depuis plusieurs années.

Il a une architecture client/serveur centralisée. Il ne regroupe pas les
modifications de fichiers dans une opération de ``commit'' atomique, ce
qui permet à ses utilisateurs de ``casser le \textit{build}'' assez
facilement : une personne peut effectuer une opération de ``commit'' 
sans problème puis être bloqué par besoin de fusion, avec comme conséquence
néfaste, que les autres utilisateurs ne récupèreront qu'une partie de ses
modifications. Ce problème affecte aussi la manière de travailler avec 
l'historique du projet. Si vous voulez voir toutes les modifications d'une
personne du projet, vous devrez injecter manuellement les descriptions et les
\textit{timestamps} des modifications de chacun des fichiers impliqués (si
vous savez au moins quels sont ces fichiers).

CVS a une notion étrange des \textit{tags} et des branches que je n'essayerais
même pas de décrire ici. Il ne supporte pas bien les opérations de renommage d'un 
fichier ou d'un répertoire, ce qui facilite la corruption de son dépôt. Il n'a
presque pas pour ainsi dire de contrôle de cohérence interne, il est donc 
pratiquement impossible de dire si un dépôt est corrompu ni à quel point. Je
ne recommanderais pas CVS pour un projet existant ou nouveau.

Mercurial peut importer l'historique d'un projet CVS. Néanmoins, il y a 
quelques principes à respecter; ce qui est vrai aussi pour les autres
outils d'import de projet CVS. À cause de l'absence de ``commit'' atomique
et gestion de version de l'arborescence, il n'est pas possible de reconstruire
de manière précise l'ensemble de l'historique. Un travail de ``devinette''
est donc nécessaire, et les fichiers renommées ne sont pas détectés. Parce 
qu'une bonne part de l'administration d'un dépôt CVS est effectuée manuellement, 
et est donc, sujette à erreur, il est courant que les imports CVS rencontrent 
de nombreux problèmes avec les dépôt corrompues (des \textit{timestamps} 
de révision complètement buggé et des fichiers verrouillés depuis des années 
sont deux des problèmes les moins intéressants dont je me souvienne).

Mercurial peut importer l'historique depuis un dépôt CVS.

\subsection{Commercial tools}

Perforce a une architecture client/serveur centralisée, sans aucun
mécanisme de mise en cache de données coté client. Contrairement à la plupart
des outils modernes de gestion de source, Perforce exige de ses 
utilisateurs d'exécuter une commande pour informer le serveur
central de tout fichier qu'il souhaite modifier.

Les performances de Perforce sont plutôt bonnes pour des petites
équipes, mais elles s'effondrent rapidement lorsque le nombre 
d'utilisateurs augmente au delà de la douzaine. Des installations 
de Perforce assez larges nécessitent le déploiement de proxies pour 
supporter la montée en charge associée.

\subsection{Choisir un outil de gestion de source}

A l'exception de CVS, tous les outils listés ci-dessus ont des 
forces qui leur sont propres et qui correspondent à certaines
formes de projet. Il n'y a pas un seul meilleur outil de gestion
de source qui correspondent le mieux à toutes les situations.

Par exemple, Subversion est un très bon choix lorsqu'on travaille
avec beaucoup de fichiers binaires, qui évoluent régulièrement, grâce
à sa nature centralisée et sa capacité à verrouiller des fichiers.

Personnellement, je préfère Mercurial pour sa simplicité, ses 
performances et sa bonne capacité de fusion, et il m'a très bien rendu service
de plusieurs années maintenant.

\section{Migrer depuis un outil à Mercurial}

Mercurial est livré avec une extension nommée \hgext{convert}, qui
peut de manière incrémentale importer des révisions depuis différents
autres outils de gestion de source. Par ``incrémental'', j'entends que
vous pouvez convertir l'historique entier du projet en une seule fois,
puis relancer l'outil d'import plus tard pour obtenir les modifications
effectués depuis votre import initial.

Les outils de gestion de source supportés par \hgext{convert} sont :
\begin{itemize}
	\item Subversion
	\item CVS
	\item Git
	\item Darcs
\end{itemize}

En outre, \hgext{convert} peut exporter les modifications depuis Mercurial
vers Subversion. Ceci rend possible d'essayer Subversion en parallèle 
avant de choisir une solution définitive, sans aucun risque de perte de
données.

La commande \hgxcmd{conver}{convert} est très simple à utiliser. Simplement,
indiquez le chemin ou l'URL du dépôt de source, en lui indiquant éventuellement
le nom du chemin de destination, et la conversion se met en route. Après cet
import initial, il suffit de relancer la commande encore une fois pour 
importer les modifications effectuées depuis.

%%% Local Variables: 
%%% mode: latex
%%% TeX-master: "00book"
%%% End: 
