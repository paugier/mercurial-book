\chapter{Colaborar con otros}
\label{cha:collab}

Debido a su naturaleza descentralizada, Mercurial no impone política
alguna de cómo deben trabajar los grupos de personas. Sin embargo, si
usted es nuevo al control distribuido de versiones, es bueno tener
herramientas y ejemplos a la mano al pensar en posibles modelos de
flujo de trabajo.

\section{La interfaz web de Mercurial}

Mercurial tiene una poderosa interfaz web que provee bastantes
capacidades útiles.

Para uso interactivo, la interfaz le permite visualizar uno o varios
repositorios. Puede ver el historial de un repositorio, examinar cada
cambio(comentarios y diferencias), y ver los contenidos de cada
directorio y fichero.

Adicionalmente la interfaz provee feeds de RSS de los cambios de los
repositorios. Que le permite ``subscribirse''a un repositorio usando
su herramienta de lectura de feeds favorita, y ser notificado
automáticamente de la actividad en el repositorio tan pronto como
sucede. Me gusta mucho más este modelo que el estar suscrito a una
lista de correo a la cual se envían las notificaciones, dado que no
requiere configuración adicional de parte de quien sea que está
administrando el repositorio.

La interfaz web también permite clonar repositorios a los usuarios
remotos, jalar cambios, y (cuando el servidor está configurado para
permitirlo) publicar cambios en el mismo.  El protocolo de tunneling
de Mercurial comprime datos agresivamente, de forma que trabaja
eficientemente incluso con conexiones de red con poco ancho de banda.

La forma más sencilla de iniciarse con la interfaz web es usar su
navegador para visitar un repositorio existente, como por ejemplo el
repositorio principal de Mercurial \url{http://www.selenic.com/repo/hg?style=gitweb}.

Si está interesado en proveer una interfaz web a sus propios
repositorios, Mercurial provee dos formas de hacerlo.  La primera es
usando la orden \hgcmd{serve}, que está enfocada a servir ``de forma
liviana'' y por intervalos cortos.  Para más detalles de cómo usar
esta orden vea la sección~\ref{sec:collab:serve} más adelante. Si
tiene un repositorio que desea hacer permanente, Mercurial tiene
soporte embebido del \command{ssh} para publicar cambios con seguridad
al repositorio central, como se documenta en la
sección~\ref{sec:collab:ssh}.  Es muy usual que se publique una copia
de sólo lectura en el repositorio que está corriendo sobre HTTP usando
CGI, como en la sección~\ref{sec:collab:cgi}.  Publicar sobre HTTP
satisface las necesidades de la gente que no tiene permisos de
publicación y de aquellos que quieren usar navegadores web para
visualizar el historial del repositorio.

\subsection{Trabajo con muchas ramas}

Los proyectos de cierta talla tienden naturlamente a progresar de
forma simultánea en varios frentes. En el caso del software, es común
que un proyecto tenga versiones periódicas oficiales. Una versión
puede entrar a ``modo mantenimiento'' por un tiempo después de su
primera publicación; las versiones de mantenimiento tienden a contener
solamente arreglos de fallos, pero no nuevas características. En
paralelo con las versiones de mantenimiento puede haber una o muchas
versiones futuras pueden estar en desarrollo. La gente usa normalmente
la palabra ``rama'' para referirse a una de las direcciones
ligeramente distintas en las cuales procede el desarrollo.

Mercurial está especialmente preparado para administrar un buen número
de ramas simultáneas pero no idénticas. Cada ``dirección de
desarrollo'' puede vivir en su propio repositorio central, y puede
mezclar los cambios de una a otra de acuerdo con las necesidades. Dado
que los repositorios son independientes, uno del otro, los cambios
inestables de una rama de desarrollo nunca afectarán una rama estable
a menos que alguien explícitamente mezcle los cambios.

A continuación un ejemplo de cómo podría hacerse esto en la
práctica. Digamos que tiene una ``rama principal'' en un servidor
central.
\interaction{branching.init}
Alguien lo clona, hace cambios locales, los prueba, y los publica allí
mismo.

Una vez que la rama principal alcanza una estado de versión se puede
usar la orden \hgcmd{tag} para dar un nombre permanente a la revisión.
\interaction{branching.tag}
Digamos que en la rama principal ocurre más desarrollo.
\interaction{branching.main}
Cuando se usa la etiqueta con que se identificó la versión, la gente
puede clonar el repositorio en cualquier momento en el futuro
empleando \hgcmd{update} para obtener una copia del directorio de
trabajo exacta como cuando se creó la etiqueta de la revisión que se
consignó.
\interaction{branching.update}

Adicionalmente, justo después de que la rama principal se etiquete,
alguien puede clonarla en el servidor a una nueva rama ``estable'',
también en el servidor.
\interaction{branching.clone}

Alguien que requiera hacer un cambio en la rama estable puede clonar
\emph{ese} repositorio, hacer sus cambios, consignar y publicarlos
posteriormente al inicial.
\interaction{branching.stable}
Puesto que los repositorios de Mercurial son independientes, y que
Mercurial no mueve los cambios de un lado a otro automáticamente, las
ramas estable y principal están \emph{aisladas} la una de la otra.
Los cambios que haga en la rama principal no ``se filtran'' a la rama
estable o vice versa.

Es usual que los arreglos de fallos de la rama estable deban hacerse
aparecer en la rama principal también.  En lugar de reescribir el
arreglo del fallo en la rama principal, puede jalar y mezclar los
cambios de la rama estable a la principal, Mercurial traerá tales
arreglos por usted.
\interaction{branching.merge}
La rama principal contendtrá aún los cambios que no están en la
estable y contendrá además todos los arreglos de fallos de la rama
estable.  La rama estable permanece incólume a tales cambios.

\subsection{Ramas de Características}

En proyectos grandes, una forma efectiva de administrar los cambios es
dividir el equipo en grupos más pequeños. Cada grupo tiene una rama
compartida, clonada de una rama ``principal'' que conforma el proyecto
completo.   Aquellos que trabajan en ramas individuales típicamente
están aislados de los desarrollos de otras ramas.

\begin{figure}[ht]
  \centering
  \grafix{feature-branches}
  \caption{Ramas de Características}
  \label{fig:collab:feature-branches}
\end{figure}

Cuando una rama particular alcanza un estado deseado, alguien del
equipo de características jala y fusiona de la rama principal hacia
la rama de características y publica posteriormente a la rama principal.

\subsection{El tren de publicación}

Algunos proyectos se organizan al estilo``tren'': Una versión se
planifica para ser liberada cada cierto tiempo, y las características
que estén listas cuando ha llegado el momento ``tren'', se incorporan.

Este modelo tiene cierta similitud a las ramas de características. La
diferencia es que cuando una característica pierde el tren, alguien en
el equipo de características jala y fusiona los cambios que se fueron
en la versión liberada hacia la rama de característica, y el trabajo
continúa sobre lo fusionado para que la característica logre estar en
la próxima versión.

\subsection{El modelo del kernel linux}

El desarrollo del Kernel Linux tiene una estructura jerárquica
bastante horizontal, rodeada de una nube de caos aparente. Dado que la
mayoría de desarrolladores usan \command{git}, una herramienta distribuida
de control de versiones con capacidades similares a Mercurial, resulta
de utilidad describir la forma en que el trabajo fluye en tal
ambiente; si le gustan las ideas, la aproximación se traduce bien
entre Git y Mercurial.

En el centro de la comunidad está Linus Torvalds, el creador de Linux.
Él publica un único repositorio que es considerado el árbol
``oficial'' actual por la comunidad completa de
desarrolladores. Cualquiera puede clonar el árbol de Linus, pero él es
muy selectivo acerca de los árboles de los cuales jala.

Linus tiene varios ``lugartenientes confiables''.  Como regla, él jala
todos los cambios que ellos publican, en la mayoría de los casos sin
siquiera revisarlos.  Algunos de sus lugartenientes generalmente
aceptan ser los ``mantenedores'', responsables de subsistemas
específicos dentro del kernel.  Si un hacker cualquiera desea hacer un
cambio a un subsistema y busca que termine en el árbol de Linus, debe
encontrar quién es el mantenedor del subsistema y solicitarle que
tenga en cuenta su cambio.  Si el mantenedor revisa los cambios y está
de acuerdo en tomarlos, estos pasarán al árbol de Linus de acuerdo a
lo expuesto.

Cada lugarteniente tiene su forma particular de revisar, aceptar y
publicar los cambios; y para decidir cuando hacerlos presentes a
Linus.  Adicionalmente existen varias ramas conocidas que mucha gente
usa para propósitos distintos. Por ejemplo, pocas personas mantienen
repositorios ``estables'' de versiones anteriores del kernel, a los
cuales aplican arreglos de fallos críticos necesarios. Algunos
mantenedores publican varios árboles: uno para cambios
experimentales; uno para cambios que van a ofrecer al mantenedor
principal; y así sucesivamente. Otros publican un solo árbol.

Este modelo tiene dos características notables. La primera es que son
de ``jalar exclusivamente''.  Usted debe solicitar, convencer o
incluso rogar a otro desarrollador para que tome sus cabmios, porque
casi no hay árboles en los cuales más de una persona pueda publicar, y
no hay forma de publicar cambios en un árbol que otra persona controla.

El segundo está basado en reputación y meritocracia.  Si usted es un
desconocido, Linus probablemente ignorará sus cambios, sin siquiera
responderle.  Pero un mantenedor de un subsistema probablemente los
revisara, y los acogerá en caso de que aprueben su criterio de
aplicabilidad.  A medida que usted ofrezca ``mejores'' cambios a un
mantenedor, habrá más posibilidad de que se confíe en su juicio y se
acepten los cambios.   Si usted es reconocido y matiene una rama
durante bastante tiempo para algo que Linus no ha aceptado, personas
con intereses similares pueden jalar sus cambios regularmente para
estar al día con su trabajo.

La reputación y meritocracia no necesariamente es transversal entre
``personas'' de diferentes subsistemas.  Si usted es respetado pero es
un hacker en almacenamiento y trata de arreglar un fallo de redes,
tal cambio puede recibir un nivel de escrutinio de un mantenedor de
redes comparable con el que se le haría a un completo extraño.

Personas que vienen de proyectos con un ordenamiento distinto, sienten
que el proceso comparativamente caótico del Kernel Linux es
completamente lunático.  Es objeto de los caprichos individuales; la
gente desecha cambios cuando lo desean; y la fase de desarrollo es
alucinante. A pesar de eso Linux es una pieza de software exitosa y
bien reconocida.

\subsection{Solamente jalar frente a colaboración pública}

Una fuente perpetua de discusiones en la comunidad de código abierto
yace en el modelo de desarrollo en el cual la gente solamente jala
cambios de otros ``es mejor que'' uno  en el cual muchas personas
pueden publicar cambios a un repositorio compartido.

Tícamente los partidarios del modelo de publicar usan las herramientas
que se apegan a este modelo.  Si usted usa una herramienta
centralizada de control de versiones como Subversion, no hay forma de
elegir qué modelo va a usar: La herramienta le ofrece publicación
compartida, y si desea hacer cualquier otra cosa, va a tener que
aplicar una aproximación artificial (tal como aplicar parches a mano).

Una buena herramienta distribuida de control de versiones, tal como
Mercurial soportará los dos modelos.   Usted y sus colaboradores
pueden estructurar cómo trabajarán juntos basados en sus propias
necesidades y preferencias,  sin depender de las peripecias que la
herramienta les obligue a hacer.

\subsection{Cuando la colaboración encuentra la administración ramificada}

Una vez que usted y su equipo configurar algunos repositorios
compartidos y comienzan a propagar cambios entre sus repositorios
locales y compartidos, comenzará a encarar un reto relacionado, pero
un poco distinto:  Administrar las direcciones en las cuales su equipo
puede moverse.   A pesar de que está intimamente ligado acerca de cómo
interactúa su equipo, es lo suficientemente denso para ameritar un
tratamiento en el capítulo~\ref{chap:branch}.

\section{Aspectos técnicos de la colaboración}

Lo que resta del capítulo lo dedicamos a las cuestiones de servir
datos a sus colaboradores.

\section{Compartir informalmente con \hgcmd{serve}}
\label{sec:collab:serve}

La orden \hgcmd{serve} de Mercurial satisface de forma espectacular
las necesidades de un grupo pequeño, acoplado y de corto
tiempo.  Se constituye en una demostración de cómo se siente usar los
comandos usando la red.

Ejecute \hgcmd{serve} dentro de un repositorio, y en pocos segundos
iniciará un servidor HTTP especializado; aceptará conexiones desde
cualquier cliente y servirá datos de este repositorio mientrs lo
mantenga funcionando. Todo el que sepa el URL del servidor que ha
iniciado, y que puede comunicarse con su computador por la red, puede
usar un navegador web o Mercurial para leer datos del repositorio. Un
URL para una instancia de \hgcmd{serve} ejecutándose en un portátil
debería lucir algo \Verb|http://my-laptop.local:8000/|.

La orden \hgcmd{serve} \emph{no} es un servidor web de propósito
general. Solamente puede hacer dos cosas:
\begin{itemize}
\item Permitir que se pueda visualizar el historial del repositorio que
  está sirviendo desde navegadores web.
\item Hablar el protocolo de conexión de Mercurial para que puedan hacer
  \hgcmd{clone} o \hgcmd{pull} (jalar) cambios de tal repositorio.
\end{itemize}
En particular, \hgcmd{serve} no permitirá que los usuarios remotos
puedan \emph{modificar} su repositorio.  Es de tipo solo lectura.

Si está comenzando con Mercurial, no hay nada que le impida usar
\hgcmd{serve} para servir un repositorio en su propio computador, y
usar posteriormente órdenes como \hgcmd{clone}, \hgcmd{incoming}, para
comunicarse con el servidor como si el repositorio estuviera alojado
remotamente. Lo que además puede ayudarle a adecuarse rápidamente para
usar comandos en repositorios alojados en la red.

\subsection{Cuestiones adicionales para tener en cuenta}

Dado que permite lectura sin autenticación a todos sus clientes,
debería usar \hgcmd{serve} exclusivamente en ambientes en los cuáles
no tenga problema en que otros vean, o en los cuales tenga control
completo acerca de quien puede acceder a su red y jalar cambios de su
repositorio.

La orden \hgcmd{serve} no tiene conocimiento acerca de programas
cortafuegos que puedan estar instalados en su sistema o en su red. No
puede detectar o controlar sus cortafuegos.  Si otras personas no
pueden acceder a su instancia \hgcmd{serve}, lo siguiente que debería hacer
(\emph{después} de asegurarse que tienen el URL correcto) es verificar
su configuración de cortafuegos.

De forma predeterminada, \hgcmd{serve} escucha conexiones entrantes en
el puerto~8000.  Si otro proceso está escuchando en tal puerto, usted
podrá especificar un puerto distinto para escuchar con la opción
\hgopt{serve}{-p} .

Normalmente, cuando se inicia \hgcmd{serve}, no imprime nada, lo cual
puede ser desconcertante.  Si desea confirmar que en efecto está
ejecutándose correctamente, y darse cuenta qué URL debería enviar a
sus colaboradores, inícielo con la opción \hggopt{-v}.

\section{Uso del protocolo Secure Shell (ssh)}
\label{sec:collab:ssh}

Usted puede publicar y jalar cambios en la red de forma segura usando
el protocolo Secure Shell (\texttt{ssh}).  Para usarlo satisfactoriamente,
tendrá que hacer algo de configuración a nivel de cliente o el
servidor.

Si no está familizarizado con ssh, es un protocolo de red que le permite
comunicarse con seguridad con otro computador.  Para usarlo con
Mercurial, estará estableciendo una o más cuentas de usuario en un
servidor de forma tal que los usuarios remotos puedan entrar y
ejecutar órdenes.

(Si ssh le \emph{es} familiar, encontrará probablemente elemental una
porción del material a continuación.)

\subsection{Cómo leer y escribir URLs de ssh}

Los URLs de ssh tienden a lucir de la siguiente forma:
\begin{codesample2}
  ssh://bos@hg.serpentine.com:22/hg/hgbook
\end{codesample2}
\begin{enumerate}
\item La parte ``\texttt{ssh://}'' indica a Mercurial que use el
  protocolo ssh.
\item El componente ``\texttt{bos@}'' indica el nombre del usuario que
  está entrando al servidor.  Puede omitirl si el usuario remoto
  coincide con el usuario local.
\item ``\texttt{hg.serpentine.com}'' es el nombre del servidor al cual
  se desea entrar.
\item El ``:22'' identifica el número del puerto en el servidor al cual
  se conectará.  El predeterminado es el~22, así que solamente
  necesitará especificar esa porción si \emph{no} está usando el
  puerto~22.
\item La última porción del URL es la ruta local al repositorio en el
  servidor.
\end{enumerate}

El componente de la ruta del URL para ssh es una fuente de confusión,
puesto que no hay una forma estándar para que las herramientas puedan
interpretarlo.  Algunos programas se comportan de manera distinta a
otros cuando manipulan estas rutas.  No es la situación ideal, pero
es muy poco probable que vaya a cambiar.  Por favor lea los párrafos
siguientes cuidadosamente.

Mercurial trata la ruta al repositorio en el servidor como relativo al
directorio chasa del usuario remoto.  Por ejemplo, si el usuario
\texttt{foo} en el servidor tiene el directorio casa
\dirname{/home/foo},
entonces un URL ssh que contenga en su ruta a \dirname{bar}
\emph{realmente} se refiere al directorio \dirname{/home/foo/bar}.

Si desea especificar una ruta relativa a otro directorio de usuario,
puede usar una ruta que comience con un caracter tildado, seguido del
nombre del usuario(llamémosle \texttt{otrousuario}, así
\begin{codesample2}
  ssh://server/~otrousuario/hg/repo
\end{codesample2}

Y si realmente desea especifica una ruta \emph{absoluta} en el
servidor, comience con el componente de la ruta con dos barras como
en el siguiente ejemplo:
\begin{codesample2}
  ssh://server//absolute/path
\end{codesample2}

\subsection{Encontrar un cliente ssh para su sistema}

Casi todos los sistemas tipo Unix vienen con OpenSSH preinstalado.  Si
usted está usando un sistema de estos, ejecute \Verb|which ssh| para
identificar dónde está instalada la orden \command{ssh} (usualmente
estará en \dirname{/usr/bin}).  Si por casualidad no está presente,
vea la documentación de sus sistema para lograr instalarlo.

En Windows, tendrá que escoger primero un cliente adecuado para
descargarlo.  Hay dos alternativas:
\begin{itemize}
\item El excelente paquete PuTTY~\cite{web:putty} de Simon Tatham, que
  ofrece un suite completo de órdenes de cliente ssh.
\item Si tiene alta tolerancia al dolor, puede usar el porte de Cygwin
  para OpenSSH.
\end{itemize}
En cualquier caso, tendrá que editar su fichero \hgini\ para indicarle
a Mercurial dónde encontrar la orden real del cliente.  Por ejemplo, si
está usando PuTTY, tendrá que usar la orden \command{plink} como un
cliente de línea de órdenes.
\begin{codesample2}
  [ui]
  ssh = C:/ruta/a/plink.exe -ssh -i "C:/ruta/a/mi/llave/privada"
\end{codesample2}

\begin{note}
  La ruta a \command{plink} no debería contener espacios o caracteres
  en blanco, o Mercurial no podrá encontrarlo correctamente (por lo
  tanto, probablemente no sería buena idea colocarlo en 
  \dirname{C:\\Program Files}
\end{note}

\subsection{Generar un par de llaves}

Para evitar la necesidad de teclera una clave de forma repetitiva cada
vez que necesita usar el cliente, recomiendo generar un par de llaves.
En un sistema tipo Unix, la orden \command{ssh-keygen} también se
comportará bien. En Windows, si está usando PuTTY, la orden
\command{puttygen} es la que necesitará.

Cuando genera un par de llaves, se aconseja \emph{comedidamente} 
protegerlas con una frase de clave.  (La única oportunidad en la cual
usted querría identificarse una única vez, es cuando está usando
el protocolo ssh para tareas automatizadas en una red segura.)

No basta con generar un par de llaves.  Se requiere adicionar una llave
pública al conjunto de llaves autorizadas para todos los usuarios
remotos que se vayan a autenticar.  Para aquellos servidores que usen
OpenSSH(la gran mayoría), significará añadir la llave pública a la
lista en el fichero llamado \sfilename{authorized\_keys} en su
directorio \sdirname{.ssh}.

En sistemas tipo Unix, su llave pública tendrá la extensión
\filename{.pub}.  Si usa \command{puttygen} en Windows, puede
guardar la llave pública en un fichero de su elección, o pegarla desde
la ventana en la cual se despliega directamente en el fichero
\sfilename{authorized\_keys}.

\subsection{Uso de un agente de autenticación}

Un agente de autentitcación es un daemonio que almacena frases clave en
memoria(olvidará las frases clave si sale y vuelve a entrar).  Un cliente
ssh notará si está corriendo, y solicitará una frase clave.  Si no hay
un agente de autenticación corriendo, o el agente no almacena la frase
clave necesaria, tendrá que teclear su frase clave cada vez que 
Mercurial intente comunicarse con un servidor para usted(p.e.~cada vez
que jale o publique cambios).

El problema de almacenar frases claves en un agente es que es posible
para un atacante bien preparado recuperar el texto plano de su frase
clave, en alguntos casos incluso si su sistema sea muy alternante.
Es su decisión si es un riesgo aceptable.  Lo que si es seguro es que
evita reteclear.

En sistemas tipo Unix, el agente se llama \command{ssh-agent}, y
usualmente se ejecuta automáticamente cuando usted entra.  Tendrá que
usar la orden \command{ssh-add} para añadir frases claves al agente.  En
Windows, si está usando PuTTY, la orden \command{pageant} actúa como
el agente.  Añade un ícono a su barra del sistema que le permitirá
almacenar frases clave.

\subsection{Configurar el lado del servidor apropiadamente}

Dado que puede ser dispendioso configurar ssh si usted es nuevo, hay 
una variedad de cosas que podrían ir mal.  Añada piense primero en
Mercurial y hay mucho más en qué pensar.  La mayor parte de estos
problemas potenciales occuren en el lado del servidor, no en el cliente.
Las buenas noticias es que una vez tiene una configuración funcional,
usualmente continuará trabajando indefinidamente.

Antes de intentar que Mercurial hable con un servidor ssh, es mejor
asegurarse que puede usar la orden normal \command{ssh} o \command{putty}
para comunicarse con el servidor primero.  Si tiene problemas usando
estas órdenes directamente, de seguro Mercurial no funcionará.  Pero aún,
esconderá el problema subyacente.  Cuando desee revisar un problema
relacionado con ssh y Mercurial, debería asegurarse primero que las
órdenes de ssh en el lado del cliente funcionan primero, \emph{antes}
de preocuparse por si existe un problema con Mercurial.

Lo primero para asegurar en el lado del servidor es que puede entrar
desde otra máquina.  Si no puede entrar con \command{ssh} o 
\command{putty}, el mensaje de error que obtenga le puede dar pistas
de qué ha ido mal.  Los problemas más comunes son los siguientes:
\begin{itemize}
\item Si obitene un error de ``conexión rehusada'', es posible que no 
  haya un daemonio SSH corriendo en el servidor o que no pueda accederse
  a él por configuraciones de cortafuegos.
\item Si obtiene un error de ``no hay ruta hasta el servidor'', puede
  tener la dirección del servidor incorrecta o un cortafuegos con
  bloqueo agresivo que no permitirá su existencia.
\item Si obtiene un mensaje de ``permiso denegado'', puede que haya
  tecleado mal el usuario en el servidor, o que haya tecleado
  incorrectamente la frase clave o la clave del usuario remoto.
\end{itemize}
En resumen, si tiene problemas al comunicarse con el daemonio ssh del
servidor, primero asegúrese de que está corriendo.  En muchos sistemas
estará instalado, pero deshabilitado de forma predeterminada.  Una vez
que haya hecho este paso tendrá que revisar si el cortafuegos del
servidor está configurado para recibir conexiones entrantes en el
puerto en el cual el daemonio de ssh está escuchando(usualmente el~22).
No trate de buscar otras posibilidades exóticas o configuraciones
erradas haste que haya revisado primero estas dos.

Si está usando un agente de autenticación en el lado del cliente para
almacenar las frase claves de sus contraseñas, debería poder entrar al
servidor sin necesidad de que se le solicite frases claves o
contraseñas.  Si se le pregunta alguna, a continuación algunas
posibilidades:
\begin{itemize}
\item Puede haber olvidado usar \command{ssh-add} o
  \command{pageant} para guardar la frase clave.
\item Puede haber almacenado una frase clave errónea para la llave.
\end{itemize}
Si se le solicita la clave del usuario remoto, hay otras posibilidades
que deben revisarse:
\begin{itemize}
\item O bien el directorio del usuario o su directorio \sdirname{.ssh}
  tiene permisos excesivamente abiertos.  Como resultado el daemonio
  ssh no creerá o leerá su fichero \sfilename{authorized\_keys}.  
  Por ejemplo, un directorio casa o \sdirname{.ssh} causará aveces
  este síntoma.
\item El fichero de usuario \sfilename{authorized\_keys} puede tener
  un problema.  Si alguien distinto al usuario es dueño o puede
  escribir el fichero, el daemonio ssh no confiará o lo leerá.
\end{itemize}

En un mundo ideal, debería poder ejecutar la siguiente orden
exitósamente, y debería imprimir exactamente una línea de salida,
la fecha y hora actual.
\begin{codesample2}
  ssh miservidor fecha
\end{codesample2}

Si en su servidor tiene guión que se ejecuta a la entrada e imprimie
letreros o cualquier otra cosa, incluso cuando se ejecutan órdenes no
interactivas como esta, debería arreglarlo antes de continuar, de
forma que solamente imprima algo si se ejecuta interactivamente.  De
otra forma estos letreros al menos llenarán la salida de Mercurial.
incluso podrían causar problemas potenciales cuando se ejecuten
órdenes de forma remota.  Mercurial intenta detectar e ignorar los
letreros en sesiones no interactivas de \command{ssh}, pero no es
a prueba de tontos.  (Si edita sus guiones de entrada en el servidor,
la forma usual de ver si un guión de script se ejecuta en un shell
interactivo, es verificar el código de retorno de la orden
\Verb|tty -s|.)

Cuando verifique que el venerado ssh funciona en su servidor, el
paso siguiente es asegurar que Mercurial corre en el servidor.  La
orden siguiente debería ejecutarse satisfactoriamente:
\begin{codesample2}
  ssh miservidor hg version
\end{codesample2}
Si ve un mensaje de error en lugar de la salida usual de 
\hgcmd{version}, será porque no ha instalado Mercurial en
\dirname{/usr/bin}.  No se preocupe si este es el caso; no necesita
hacerlo.  Pero debería revisar los posibles problemas presentados a
continuación:
\begin{itemize}
\item Está instalado Mercurial en el servidor?  Se que suena trivial
  pero es mejor revisar!
\item Tal vez la ruta de búsqueda de la interfaz de órdenes
  (normalmente vía la variable de ambiente \envar{PATH}) simplemente
  está mal configurada.
\item Puede ser que su variable de ambiente \envar{PATH} soalamente
  apunte al lugar en el cual está el ejecutable \command{hg} si la
  sesión de entrada es interactiva.  Puede suceder si establece la
  ruta en el guión de shell de entrada incorrecto.  Consulte la
  documentación de su línea de órdenes.
\item La variable de ambiente \envar{PYTHONPATH} puede requerir la
  ruta a los módulos de Mercurial en Python.  Puede que nisiquiera
  está establecida; podría estar incorrecta; o puede ser que se
  establezca únicamente cuando hay entradas interactivas.
\end{itemize}

Si puede ejecutar \hgcmd{version} sobre una conexión ssh,
felicitaciones!  Ha logrado la interacción entre el cliente y el 
servidor.  Ahora debería poder acceder a los repositorios de
Mercurial que tiene el usuario en el servidor.  Si tiene problemas
con Mercurial y ssh en este punto, intente usar la opción
\hggopt{--debug} para tener información más clara de lo que está
sucediendo.

\subsection{Compresión con ssh}

Mercurial no comprime datos cuando usa el protocolo ssh, dado que
el protocolo puede comprimir datos transparentemente.  Pero el
comportamiento predeterminado del cliente ssh es \emph{no}
solicitar compresión.

Sobre cualquier red distinta a una LAN rápida(incluso con una red
inalámbrica), hacer uso de compresión puede mejorar el rendimiento
de las operaciones de Mercurial que involucren la red.  Por ejemplo,
sobre WAN, alguien ha medido la compresión reduciendo la cantidad
de tiempo requerido para clonar un repositorio particularmente
grande de~51 minutos a~17 minutos.

Tanto \command{ssh} como \command{plink} aceptan la opción
\cmdopt{ssh}{-C} que activa la compresión.  Puede editar fácilmente
su \hgrc\ para habilitar la compresión para todos los usos de
Mercurial sobre el protocolo ssh.
\begin{codesample2}
  [ui]
  ssh = ssh -C
\end{codesample2}

Si usa \command{ssh}, puede reconfigurarlo para que siempre use
compresión cuando se comunique con su servidor.  Para hacerlo,
edite su fichero \sfilename{.ssh/config}(que puede no existir
aún), de la siguiente forma:
\begin{codesample2}
  Host hg
    Compression yes
    HostName hg.ejemplo.com
\end{codesample2}
Que define un alias, \texttt{hg}.  Cuando lo usa con la orden
\command{ssh} o con una URL de Mercurial con protocolo\texttt{ssh},
logrará que \command{ssh} se conecte a \texttt{hg.ejemplo.com}
con compresión.  Que le dará un nombre más corto para teclear y
compresión, los cuales por derecho propio son buenos.

\section{Uso de CGI a través de HTTP}
\label{sec:collab:cgi}

Dependiendo de qué tan ambicioso sea, configurar la interfaz CGI
de Mercurial puede tomar desde unos minutos hasta varias horas.

Comenzaremos con el ejemplo más sencillo, y nos dirigiremos hacia
configuraciones más complejas.  Incluso para el caso más básico
necesitará leer y modificar su configuración del servidor web.

\begin{note}
  Configurar un servidor web es una actividad compleja, engorrosa y
  altamente dependiente del sistema.  De ninguna manera podremos
  cubrir todos los casos posibles con los cuales pueda encontrarse.
  Use su discresión y juicio respecto a las secciones siguientes.
  Esté preparado para cometer muchas equivocaciones, y emplear
  bastante tiempo leyendo sus bitácoras de error del servidor.
\end{note}

\subsection{Lista de chequeo de la configuración del servidor web}

Antes de continuar, tómese un tiempo para revisar ciertos aspectos de
la configuración de su sistema:

\begin{enumerate}
\item ¿Tiene un servidor web?  Mac OS X viene con Apache, pero otros
  sistemas pueden no tener un servidor web instalado.
\item Si tiene un servidor web instalado, ¿Está ejecutándose?  En la
  mayoría de sistemas, aunque esté presente, puede no estar habilitado
  de forma predeterminada.
\item ¿u servidor está configurado para permitir ejecutar programas
  CGI en el directorio donde planea hacerlo?  Casi todos los
  servidores de forma predeterminada explícitamente inhiben la
  habilidad de ejecutar programas CGI.
\end{enumerate}

Si no tiene un servidor web instalado, y no tiene cierta experiencia
configurando  Apache, debería considerar usar el servidor web
\texttt{lighttpd} en lugar de Apache.  Apache tiene una reputación
bien ganada por su configuración barroca y confusa.
A pesar de que \texttt{lighttpd} tiene menos características que
Apache en ciertas áreas, las mismas no son relevantes para servir
repositorios de Mercurial.  Definitivamente es mucho más sencillo
comenzar con \texttt{lighttpd} que con Apache.

\subsection{Configuración básica de CGI}

En sistemas tipo Unix es común que los usuarios tengan un subdirectorio
con un nombre como \dirname{public\_html} en su directorio personal,
desde el cual pueden servir páginas web.  Un fichero llamado \filename{foo}
en este directorio será visible en una URL de la forma
\texttt{http://www.example.com/\~username/foo}.

Para comenzar, encuentre el guión \sfilename{hgweb.cgi} que debería
estar presente en su instalación de Mercurial.  Si no puede
encontrarlo rápidamente una copia local en su sistema, puede
descargarlo del repositorio principal de Mercurial en
\url{http://www.selenic.com/repo/hg/raw-file/tip/hgweb.cgi}.

Tendrá que copiar este guión en su directorio \dirname{public\_html},
y asegurarse que sea ejecutable.
\begin{codesample2}
  cp .../hgweb.cgi ~/public_html
  chmod 755 ~/public_html/hgweb.cgi
\end{codesample2}
El argumento \texttt{755} de la orden \command{chmod} es un poco más
general que hacerlo ejecutable: Asegura que el guión sea ejecutable
por cualquiera, y que el ``grupo'' y los ``otros'' \emph{not}  tengan
permiso de escritura.  Si dejara los permisos de escritura abiertos,
, el subsistema \texttt{suexec} de  Apache probablemente se negaría
a ejecutar el guión.  De hecho, \texttt{suexec} también insiste en que
el \emph{directorio} en el cual reside el guión no tenga permiso de
escritura para otros.
\begin{codesample2}
  chmod 755 ~/public_html
\end{codesample2}

\subsubsection{¿Qué \emph{podría} resultar mal?}
\label{sec:collab:wtf}

Cuando haya ubicado el CGI en el sitio correspondiente con un navegador
intente visitar el URL \url{http://myhostname/~myuser/hgweb.cgi},
\emph{sin} dejarse abatir por un error.  Hay una alta probabilidad de
que esta primera visita al URL sea fallida, y hay muchas razones posibles
para este comportamiento.  De hecho, podría toparse con cada uno de los
errores que describimos a continuación, así que no deje de leerlos
cuidadosamente.   A continuación presento los problemas que yo tuve en
un sistema con Fedora~7, con una instalación nueva de Apache, y una
cuenta de usuario que creé específicamente para desarrollar este
ejercicio.

Su servidor web puede tener directorios por usuario deshabilitados. Si
usa Apache, busque el fichero de configuración que contenga la
directiva \texttt{UserDir}.  Si no está presente en sitio alguno, los
directorios por usuario están deshabilitados.  Si la hay, pero su
valor es \texttt{disabled}, los directorios por usuario estarán
deshabilitados. La directiva \texttt{UserDir} en caso contrario tendrá
el nombre del subdirectorio bajo el cual Apache mirará en el
directorio de cada usuario, por ejemplo \dirname{public\_html}.

Los permisos de sus ficheros pueden ser demasiado restrictivos.  El
servidor web debe poder recorrer su directorio personal y los
directorios que estén bajo \dirname{public\_html}, además de tener
permiso para leer aquellos que estén adentro.  A continúación una
receta rápida para hacer que sus permisos estén acordes con las
necesidades básicas.
\begin{codesample2}
  chmod 755 ~
  find ~/public_html -type d -print0 | xargs -0r chmod 755
  find ~/public_html -type f -print0 | xargs -0r chmod 644
\end{codesample2}

Otra posibilidad con los permisos es que obtenga una ventana
completamente en blanco cuando trata de cargar el script. En cuyo
caso, es posible que los permisos que tiene son \emph{demasiado
 permisivos}.  El subsistema \texttt{suexec} de Apache no ejecutará un
script que tenga permisos de escritura para el group o el planeta, por
ejemplo.

Su servidor web puede estar configurado para evitar la ejecución de
programas CGI en los directorios de usuario.  A continuación presento
una configuración predeterminada por usuario en mi sistema Fedora.

\begin{codesample2}
  <Directory /home/*/public_html>
      AllowOverride FileInfo AuthConfig Limit
      Options MultiViews Indexes SymLinksIfOwnerMatch IncludesNoExec
      <Limit GET POST OPTIONS>
          Order allow,deny
          Allow from all
      </Limit>
      <LimitExcept GET POST OPTIONS>
          Order deny,allow
          Deny from all
      </LimitExcept>
  </Directory>
\end{codesample2}
Si encuentra un grupo de instrucciones de \texttt{Directory} similares
en su configuración de Apache,  la directiva a revisar es \texttt{Options}.
Adicione \texttt{ExecCGI} al final de esta lista en caso de que haga
falta y reinicie su servidor web.

Si resulta que Apache le muestra el texto del script CGI en lugar de
ejecutarlo, necesitará o bien descomentar(si se encuentra presente) o
adicionar una directiva como la siguiente:
\begin{codesample2}
  AddHandler cgi-script .cgi
\end{codesample2}

Otra posibilidad es que observe una traza de Python en colores
informando que no puede importar un módulo relacionado con
\texttt{mercurial}.  Esto es un gran progreso!  El servidor es capaz
de ejecutar su script CGI.  Este error solamente ocurrirá si está
ejecutando una instalación privada de Mercurial en lugar de una
instalación para todo el sistema.  Recuerde que el servidor que
ejecuta el programa CGI no cuenta con variables de ambiente de las
cuales usted si dispone en una sesión interactiva.  Si este error le
ocurre, edite su copia de \sfilename{hgweb.cgi} y siga las indicaciones
dentro del mismo para establecer de forma adecuada su variable de
ambiente \envar{PYTHONPATH}.

Finalmente, si encuentra \emph{otra} traza a todo color de Python al visitar
el URL: Esta seguramente se referirá a que no puede encontrar
\dirname{/path/to/repository}.  Edite su script \sfilename{hgweb.cgi}
y reemplaze la cadena \dirname{/path/to/repository} con la ruta
completa al repositorio que desea servir.

En este punto, cuando trate de recargar la página, deberá visualizar
una linda vista HTML de el historial de su repositorio. Uff!

\subsubsection{Configuración de lighttpd}

En mi intención de ser exhaustivo, intenté configurar
\texttt{lighttpd}, un servidor web con creciente aceptación, para
servir los repositorios de la misma forma como lo describí
anteriormente con Apache. Después de superar los problemas que mostré
con Apache, muchos de los cuáles no son específicos del servidor.  Por
lo tanto estaba seguro de que mis permisos para directorios y ficheros
eran correctos y que mi guión \sfilename{hgweb.cgi} también lo era.

Dado que ya Apache estaba en ejecución correctamente, lograr que
\texttt{lighttpd} sirviera mi repositorio fue rápido(en otras
palabras, si está tratando de usar \texttt{lighttpd}, debe leer la
sección de Apache).  Primero tuve que editar la sección
\texttt{mod\_access} para habilitar \texttt{mod\_cgi} y
\texttt{mod\_userdir}, los cuales estaban inhabilitados en mi
instalación predeterminada.  Añadí posteriormente unas líneas al final
del fichero de configuración, para hacer lo propio con los módulos.
\begin{codesample2}
  userdir.path = "public_html"
  cgi.assign = ( ".cgi" => "" )
\end{codesample2}
Hecho esto, \texttt{lighttpd} funcionó inmediatamente para
mí. Configuré \texttt{lighttpd} antes que Apache, tuve casi los mismos
reparos a nivel de configuración del sistema que con Apache.  De todas
maneras, considero que \texttt{lighttpd} es bastante más sencillo de
configurar que Apache, a pesar de haber usado Apache por lo menos por
una década, y esta fue mi primera experiencia con \texttt{lighttpd}.

\subsection{Compartir varios repositorios con un script CGI}

El guión \sfilename{hgweb.cgi} permite publicar únicamente un
repositorio, una restricción frustrante.  Si desea publicar más de uno
sin complicarse con varias copias del mismo guión, cada una con un
nombre distinto, resulta mucho mejor usar el guión
\sfilename{hgwebdir.cgi}.

El procedimiento para configurar \sfilename{hgwebdir.cgi} tiene una
porción adicional frente al trabajo requerido con
\sfilename{hgweb.cgi}.  Primero se debe obtener una copia del
guión. Si no tiene una a mano, puede descargar una copia del ftp
principal del repositorio de Mercurial en
\url{http://www.selenic.com/repo/hg/raw-file/tip/hgwebdir.cgi}.

Necesitará una copia del guión en su directorio \dirname{public\_html},
y asegurarse de que sea ejecutable.
\begin{codesample2}
  cp .../hgwebdir.cgi ~/public_html
  chmod 755 ~/public_html ~/public_html/hgwebdir.cgi
\end{codesample2}
Con la configuración básica, intente visitar en su navegador
\url{http://myhostname/~myuser/hgwebdir.cgi}.  Debería mostar una
lista vacía de repositorios.  Si obtiene una ventana en blanco o un
mensaje de error, verifique la lista de problemas potenciales en la 
sección~\ref{sec:collab:wtf}.

El guión \sfilename{hgwebdir.cgi} se apoya en un fichero externo de
configuración.  En principio, busca un fichero llamado
\sfilename{hgweb.config} en el mismo directorio.  Tendrá que crear el
fichero, y permitir lectura de todo el mundo.  El formato del fichero
es similar a un fichero ``ini'' de Windows, que puede interpretar el módulo
\texttt{ConfigParser}~\cite{web:configparser} de Python.

La forma más sencilla de configurar \sfilename{hgwebdir.cgi} es con
una sección llamada \texttt{collections}.  Esta publicará automáticamente
\emph{todos} los repositorios en los directorios que usted
especifique.  La sección debería lucir así:
\begin{codesample2}
  [collections]
  /mi/ruta = /mi/ruta
\end{codesample2}
Mercurial lo interpreta buscando el nombre del directorio que esté a la
\emph{derecha} del símbolo ``\texttt{=}''; encontrando repositorios en
la jerarquía de directorios; y usando el texto a la \emph{izquierda}
para eliminar el texto de los nombres que mostrará en la interfaz
web.  El componente restante de la ruta después de esta eliminación
usualmente se llama ``ruta virtual''.

Dado el ejemplo de arriba, si tenemos un repositorio cuya ruta local es
\dirname{/mi/ruta/este/repo}, el guión CGI eliminará la porción inicial
\dirname{/mi/ruta} del nombre y publicará el repositorio con una ruta
virtual \dirname{este/repo}.  Si el URL base de nuestro guión CGI es
\url{http://myhostname/~myuser/hgwebdir.cgi}, el URL completo al
repositorio será
\url{http://myhostname/~myuser/hgwebdir.cgi/this/repo}.

Si reemplazamos \dirname{/mi/ruta} en el lado izquierdo de este
ejemplo con \dirname{/mi}, \sfilename{hgwebdir.cgi} eliminará solamente
\dirname{/mi} del nombre del repositorio, y nos ofrecerá la ruta
virtual \dirname{ruta/este/repo} en lugar de \dirname{este/repo}.

El guión \sfilename{hgwebdir.cgi} buscará recursivamente en cada
directorio listado en la sección \texttt{collections} de su fichero de
configuración, pero \texttt{no} hará el recorrido recursivo dentro de
los repositorios que encuentre.

El mecanismo de \texttt{collections} permite publicar fácilmente
repositorios de una forma ``hacer y olvidar''.  Solamente requiere
configurar el guión CGI y el fichero de configuración una vez.
Después de eso puede publicar y sacar de publicación un repositorio en
cualquier momento incluyéndolo o excluyéndolo de la jerarquía de
directorios en la cual le haya indicado a \sfilename{hgwebdir.cgi} que
mirase.

\subsubsection{Especificación explícita de los repositorios a publicar}

Además del mecanismo \texttt{collections}, el guión
\sfilename{hgwebdir.cgi} le permite publicar una lista específica de
repositorios.  Para hacerlo, cree una sección \texttt{paths}, con los
contenidos de la siguiente forma:
\begin{codesample2}
  [paths]
  repo1 = /mi/ruta/a/un/repo
  repo2 = /ruta/a/otro/repo
\end{codesample2}
En este caso, la ruta virtual (el componente que aparecerá en el URL)
está en el lado derecho de cada definición, mientras que la ruta al
repositorio está a la derecha.  Note que no tiene que haber relación
alguna entre la ruta virtual que elija y el lugar del repositorio en
su sistema de ficheros.

Si lo desea, puede usar los dos mecanismos \texttt{collections} y
\texttt{paths} simultáneamente en un sólo fichero de configuración.

\begin{note}
  Si varios repositorios tienen la misma ruta virtual,
  \sfilename{hgwebdir.cgi} no reportará error.  Pero se comportará
  impredeciblemente.
\end{note}

\subsection{Descarga de ficheros fuente}

La interfaz web de Mercurial permite a los ususarios descargar
un conjunto de cualquier revisión.  Este fichero contendrá una réplica
del directorio de trabajo en la revisión en cuestión, pero no
contendrá una copia de los datos del repositorio.

De forma predeterminada esta característica no está habilitada.  Para
lograrlo adicione un \rcitem{web}{allow\_archive} a la sección \rcsection{web}
de su fichero \hgrc.

\subsection{Opciones de configuración en Web}

Las interfaces web de Mercurial(la orden \hgcmd{serve}, y los guiones
\sfilename{hgweb.cgi} y \sfilename{hgwebdir.cgi}) tienen varias
opciones de configuración para establecer. Todas ellas en la sección
\rcsection{web}.
\begin{itemize}
\item[\rcitem{web}{allow\_archive}] Determina cuáles tipos de ficheros
  de descarga soportará Mercurial.  Si habilita esta característica,
  los usuarios de la interfaz web podrán descargar una copia de la
  revisión del repositorio que estén viendo. Para activar la
  característica de descarga de fichero, el valor tendrá una secuencia
  de palabras extraídas de la lista de abajo.
  \begin{itemize}
  \item[\texttt{bz2}] Un fichero \command{tar} con el método de
    compresión \texttt{bzip2}.  Tiene la mejor taza de compresión,
    pero usa más tiempo de procesamiento en el servidor.
  \item[\texttt{gz}] Un fichero \command{tar}, comprimido con
    \texttt{gzip}.
  \item[\texttt{zip}] Un fichero \command{zip}, comprimido con LZW.
    Este formato posee la peor tasa de compresió, pero es muy usado en
    el mundo Windows.
  \end{itemize}
  Si da una lista vacía o no tiene la entrada
  \rcitem{web}{allow\_archive}, esta característica se deshabilitará.
  A continuación un ejemplo de cómo habilitar los tres formatos soportados.
  \begin{codesample4}
    [web]
    allow_archive = bz2 gz zip
  \end{codesample4}
\item[\rcitem{web}{allowpull}] Booleano.  Determina si la interfaz web
  permite a los usuarios remotos emplear \hgcmd{pull} y \hgcmd{clone}
  sobre el repositorio~HTTP.  Si se coloca \texttt{no} o
  \texttt{false}, solamente la porción de los procesos
  ``orientados-a-humanos'' se habilita de la interfaz web.
\item[\rcitem{web}{contact}] Cadena.  Una cadena en forma libre(pero
  preferiblemente corta) que identifica a la persona o grupo a cargo
  del repositorio.  Usualmente contiene el nombre y la dirección de
  correo electrónico de una persona o de una lista de correo.  Aveces
  tiene sentido colocar esta opción en el fichero \sfilename{.hg/hgrc}
  del repositorio, pero en otras oportunidades en el \hgrc\ global si
  todos los repositorios tienen un único mantenedor.
\item[\rcitem{web}{maxchanges}] Entero.  La cantidad máxima de
  conjuntos de cambios a mostrar de forma predeterminada en cada página.
\item[\rcitem{web}{maxfiles}] Entero.  La cantidad máxima
  predeterminada de ficheros modificados a desplegar en una página.
\item[\rcitem{web}{stripes}] Entero.  Si la interfaz web despliega
  ``franjas'' para facilitar la visualización alineada de filas cuando
  se ve una tabla, este valor controla la cantidad de filas en cada
  franja.
\item[\rcitem{web}{style}] Controla la plantilla que Mercurial usa para
  desplegar la interfaz web.  Mercurial viene con dos plantillas web,
  llamadas \texttt{default} y \texttt{gitweb} (La primera es
  visualmente más atractiva).  Puede especificar una plantilla propia;
  consulte el capítulo~\ref{chap:template}.  A continuación mostramos
  cómo habilitar el estilo \texttt{gitweb}.
  \begin{codesample4}
    [web]
    style = gitweb
  \end{codesample4}
\item[\rcitem{web}{templates}] Ruta.  Directorio en el que se buscarán
  los ficheros plantilla.  De forma predeterminada, busca en el
  directorio en el cual fue instalado.
\end{itemize}
Si usa \sfilename{hgwebdir.cgi}, puede añadir otras opciones de
configuración en una sección \section{web} del fichero
\sfilename{hgweb.config} en lugar del fichero \hgrc\, si lo considera
más conveniente.  Estas opciones son \rcitem{web}{motd} y
\rcitem{web}{style}.

\subsubsection{Opciones específicas para repositorios individuales}

Ciertas opciones de configuración de \rcsection{web} deben estar
ubicadas en el \sfilename{.hg/hgrc} de un repositorio en lugar del
fichero del usuario o el \hgrc global.
\begin{itemize}
\item[\rcitem{web}{description}] Cadena.  Una cadena de forma
  libre(preferiblemente corta) que describa los contenidos o el
  propósito del repositorio.
\item[\rcitem{web}{name}] Cadena.  El nombre para visualizar en la
  interfaz web del repositorio. Sustituye el nombre predeterminado, el
  cual es el último componente de la ruta del repositorio.
\end{itemize}

\subsubsection{Opciones específicas a la orden \hgcmd{serve}}

Algunas opciones en la sección \rcsection{web} de un fichero \hgrc\
son de uso exclusivo para la orden \hgcmd{serve}.
\begin{itemize}
\item[\rcitem{web}{accesslog}] Ruta.  El nombre del fichero en el cual
  se escribe la bitácora de acceso.  En principio, la orden
  \hgcmd{serve} escribe esta información a la salida estándar, no a un
  fichero. Las líneas de la bitácora se escriben en un formato de
  fichero ``combinado'' estándar, usado por casi todos los servidores
  web.
\item[\rcitem{web}{address}] Cadena.  La dirección local en la cual el
  servidor debe escuchar peticiones entrantes.  En principio, el
  servidor escucha en todas las direcciones.
\item[\rcitem{web}{errorlog}] Ruta.  El nombre de un fichero en el
  cual escribir la bitácora de error.  En principio, la orden
  \hgcmd{serve} escribe esta información en la salida de error
  estándar, no a un fichero.
\item[\rcitem{web}{ipv6}] Booleano.  Si se usa o no el protocolo
  IPv6. En principio, IPv6 no se usa.
\item[\rcitem{web}{port}] Entero.  El número del puerto~TCP en el cuál
  el servidor escuchará.  El puerto predeterminado es el~8000.
\end{itemize}

\subsubsection{Elegir el fichero \hgrc\ correcto para las
  configuraciones de \rcsection{web}}

Es importante recordar que un servidor web como Apache o
\texttt{lighttpd} se ejecutarán bajo el usuario~ID que generalmente no
es el suyo  Los guiones CGI ejecutados por su servidor, tales como
\sfilename{hgweb.cgi}, se ejecutarán también con el usuario~ID.

Si añade opciones \rcsection{web} a su fichero personal \hgrc\, Los
guiones CGI no leerán tal fichero \hgrc\.  Tales configuraciones
solamente afectarán el comportamiento de la orden \hgcmd{serve} cuando
usted lo ejecuta.  Para logar que los guiones CGI vean sus
configuraciones, o bien cree un fichero \hgrc\ en el directorio hogar
del usuario ID que ejecuta su servidor web, o añada tales
configuraciones al fichero global \hgrc.


%%% Local Variables: 
%%% mode: latex
%%% TeX-master: "00book"
%%% End: 
