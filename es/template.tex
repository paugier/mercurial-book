\chapter{Personalizar los mensajes de Mercurial}
\label{chap:template}

Mercurial provee un poderoso mecanismo que permite controlar como
despliega la información.  El mecanismo se basa en plantillas.  Puede
usar plantillas para generar salida específica para una orden
particular o para especificar la visualización completa de la interfaz
web embebida.

\section{Usar estilos que vienen con Mercurial}
\label{sec:style}

Hay ciertos estilos listos que vienen con Mercurial.  Un estilo es
simplemente una plantilla predeterminada que alguien escribió e
instaló en un sitio en el cual Mercurial puede encontrarla.

Antes de dar un vistazo a los estilos que trae Mercurial, revisemos su
salida usual.

\interaction{template.simple.normal}

Es en cierta medida informativa, pero ocupa mucho espacio---cinco
líneas de salida por cada conjunto de cambios.  El estilo
\texttt{compact} lo reduce a tres líneas, presentadas de forma
suscinta.

\interaction{template.simple.compact}

El estilo de la \texttt{bitácora de cambios} vislumbra el poder
expresivo del sistema de plantillas de Mercurial.  Este estilo busca
seguir los estándares de bitácora de cambios del proyecto
GNU\cite{web:changelog}.

\interaction{template.simple.changelog}

No es una sorpresa que el estilo predeterminado de Mercurial se llame
\texttt{default}\ndt{predeterminado}.

\subsection{Especificar un estilo predeterminado}

Puede modificar el estilo de presentación que Mercurial usará para
toda orden vía el fichero \hgrc\, indicando el estilo que prefiere
usar.

\begin{codesample2}
  [ui]
  style = compact
\end{codesample2}

Si escribe un estilo, puede usarlo bien sea proveyendo la ruta a su
fichero de estilo o copiando su fichero de estilo a un lugar en el
cual Mercurial pueda encontrarlo(típicamente el subdirectorio
\texttt{templates} de su directorio de instalación de Mercurial).

\section{Órdenes que soportan estilos y plantillas}

Todas las órdenes de Mercurial``relacionadas con \texttt{log}'' le
permiten usar estilos y plantillas: \hgcmd{incoming}, \hgcmd{log},
\hgcmd{outgoing} y \hgcmd{tip}.

Al momento de la escritura del manual estas son las únicas órdenes que
soportan estilos y plantillas.  Dado que son las órdenes más
importantes que necesitan personalización, no ha habido muchas
solicitudes desde la comunidad de usuarios de Mercurial para añadir
soporte de plantillas y estilos a otras órdenes.

\section{Cuestiones básicas de plantillas}

Una plantilla de Mercurial es sencillamente una pieza de texto.
Cierta porción nunca cambia, otras partes se \emph{expanden}, o
reemplazan con texto nuevo cuando es necesario.

Antes de continuar, veamos de nuevo un ejemplo sencillo de la salida
usual de Mercurial:

\interaction{template.simple.normal}

Ahora, ejecutemos la misma orden, pero usemos una plantilla para
modificar su salida:

\interaction{template.simple.simplest}

El ejemplo anterior ilustra la plantilla más sencilla posible;  es
solamente una porción estática de código que se imprime una vez por
cada conjunto de cambios.  La opción \hgopt{log}{--template} de la
orden \hgcmd{log} indica a Mercurial usar el texto dado como la
plantilla cuando se imprime cada conjunto de cambios.

Observe que la cadena de plantilla anterior termina con el texto
``\Verb+\n+''.  Es una \emph{secuencia de control}, que le indica a
Mercurial imprimira una nueva línea al final de cada objeto de la
plantilla.  Si omite esta nueva línea, Mercurial colocará cada pieza
de salida seguida.  Si desea más detalles acerca de secuencias de
control, vea la sección~\ref{sec:template:escape}.

Una plantilla que imprime una cadena fija de texto siempre no es muy
útil; intentemos algo un poco más complejo.

\interaction{template.simple.simplesub}

Como puede ver, la cadena ``\Verb+{desc}+'' en la plantilla ha sido
reemplazada en la salida con la descricipción de cada conjunto de
cambios.  Cada vez que Mercurial encuentra texto encerrado entre
corchetes(``\texttt{\{}'' y ``\texttt{\}}''), intentará reemplazar los
corchetes y el texto con la expansión de lo que sea está adentro.
Para imprimir un corchete de forma literal, debe escaparlo, como se
describe en la sección~\ref{sec:template:escape}.

\section{Palabras claves más comunes en las plantillas}
\label{sec:template:keyword}

Puede empezar a escribir plantillas sencillas rápidamente con las
palabras claves descritas a continuación:

\begin{itemize}
\item[\tplkword{author}] Cadena.  El autor NO modificado del conjunto
  de cambios.
\item[\tplkword{branches}] Cadena.  El nombre de la rama en la cual se
  consignó el conjunto de cambios.  Será vacía si el nombre de la rama es
  \texttt{default}.
\item[\tplkword{date}] Información de fecha.  La fecha en la cual se
  consignó el conjunto de cambios.  \emph{No} es legible por un
  humano, debe pasarla por un firltro que la desplegará
  apropiadamente.  En la sección~\ref{sec:template:filter} hay más
  detalles acerca de filtros.  La fecha se expresa como un par de
  números.  El primer número corresponde a una marca de tiempo UNIX
  UTC(segundos desde el primero de enero de 1970); la segunda es el
  corrimiento horario de la zona horaria del UTC en la cual se encontraba
  quien hizo la consignación, en segundos.
\item[\tplkword{desc}] Cadena.  La descripción en texto del conjunto
  de cambios.
\item[\tplkword{files}] Lista de cadenas.  Todos los ficheros
  modificados, adicionados o eliminados por este conjunto de cambios.
\item[\tplkword{file\_adds}] Lista de cadenas.  Ficheros adicionados
  por este conjunto de cambios.
\item[\tplkword{file\_dels}] Lista de cadenas.  Ficheros eliminados
  por este conjunto de cambios.
\item[\tplkword{node}] Cadena.  El hash de identificación de este
  conjunto de cambios como una cadena hexadecimal de 40 caracteres.
\item[\tplkword{parents}] Lista de cadenas.  Los ancestros del
  conjunto de cambios.
\item[\tplkword{rev}] Entero.  El número de revisión del repositorio
  local.
\item[\tplkword{tags}] Lista de cadenas.  Todas las etiquetas
  asociadas al conjunto de cambios.
\end{itemize}

Unos experimentos sencillos nos mostrarán qué esperar cuando usamos
estas palabras claves; puede ver los resultados en la
figura~\ref{fig:template:keywords}.

\begin{figure}
  \interaction{template.simple.keywords}
  \caption{Template keywords in use}
  \label{fig:template:keywords}
\end{figure}

Como mencionamos anteriormente, la palabra clave de fecha no produce
salida legible por un humano, debemos tratarla de forma especial.
Esto involucra usar un \emph{filtro}, acerca de lo cual hay más en la
sección~\ref{sec:template:filter}.

\interaction{template.simple.datekeyword}

\section{Secuencias de Control}
\label{sec:template:escape}

El motor de plantillas de Mercurial reconoce las secuencias de control
más comunmente usadas dentro de las cadenas.  Cuando ve un backslash
(``\Verb+\+''), ve el caracter siguiente y sustituye los dos
caracteres con un reemplazo sencillo, como se describe a continuación:

\begin{itemize}
\item[\Verb+\textbackslash\textbackslash+] Backslash, ``\Verb+\+'',
  ASCII~134.
\item[\Verb+\textbackslash n+] Nueva línea, ASCII~12.
\item[\Verb+\textbackslash r+] Cambio de línea, ASCII~15.
\item[\Verb+\textbackslash t+] Tab, ASCII~11.
\item[\Verb+\textbackslash v+] Tab Vertical, ASCII~13.
\item[\Verb+\textbackslash \{+] Corchete abierto, ``\Verb+{+'', ASCII~173.
\item[\Verb+\textbackslash \}+] Corchete cerrado, ``\Verb+}+'', ASCII~175.
\end{itemize}

Como se indicó arriba, si desea que la expansión en una plantilla
contenga un caracter literal ``\Verb+\+'', ``\Verb+{+'', o
  ``\Verb+{+'', debe escaparlo.

\section{Uso de filtros con palabras claves}
\label{sec:template:filter}

Algunos de los resultados de la expansión de la plantilla no son
fáciles de usar de inmediato.  Mercurial le permite especificar una
cadena de \emph{filtros} opcionales para modificar el resultado de
expandir una palabra clave.  Ya ha visto el filtro usual
\tplkwfilt{date}{isodate} en acción con anterioridad para hacer
legible la fecha.

A continuación hay una lista de los filtros de Mercurial más
comunmente usados.  Ciertos filtros pueden aplicarse a cualquier
texto, otros pueden usarse únicamente en circunstancias específicas.
El nombre de cada filtro está seguido de la indicación de dónde puede
ser usado y una descripción de su efecto.

\begin{itemize}
\item[\tplfilter{addbreaks}] Cualquier texto. Añade una etiqueta XHTML
  ``\Verb+<br/>+'' antes del final de cada línea excepto en la final.
  Por ejemplo, ``\Verb+foo\nbar+'' se convierte en ``\Verb+foo<br/>\nbar+''.
\item[\tplkwfilt{date}{age}] palabra clave \tplkword{date}.  Muestra
  la edad de la fecha, relativa al tiempo actual. Ofrece una cadena como
  ``\Verb+10 minutes+''.
\item[\tplfilter{basename}] Cualquier texto, pero de utilidad sobre
  todo en palabras claves relativas a \tplkword{ficheros}.  Trata el
  texto como una ruta, retornando el nombre base.  Por ejemplo,
  ``\Verb+foo/bar/baz+'', se convierte en ``\Verb+baz+''.
\item[\tplkwfilt{date}{date}] \tplkword{date} palabra clave.  Mostrar
  la fecha en un formato similar a la orden \tplkword{date} de 
  in a similar format to the Unix, pero con la zona horaria incluída.
  Una cadena como ``\Verb+Mon Sep 04 15:13:13 2006 -0700+''.
\item[\tplkwfilt{author}{domain}] Cualquier texto, pero de mayor
  utilidad para la palabra clave \tplkword{author}.  Encuentra la
  primera cadena que luce como una dirección de correo electrónico, y
  extrae solamente el componente del dominio.  Por ejemplo, de 
  ``\Verb+Bryan O'Sullivan <bos@serpentine.com>+'' se extrae
  ``\Verb+serpentine.com+''.
\item[\tplkwfilt{author}{email}] Cualquier texto, pero de mayor
  utilidad para la palabra clave \tplkword{author}.  Extrae la primera
  cadena que luce como una dirección de correo. Por ejemplo, de
  ``\Verb+Bryan O'Sullivan <bos@serpentine.com>+'' extrae 
  ``\Verb+bos@serpentine.com+''.
\item[\tplfilter{escape}] Cualquier texto.  Reemplaza los caracteres
  especiales de XML/XHTML: ``\Verb+&+'', ``\Verb+<+'' y ``\Verb+>+''
  con las entidades XML.
\item[\tplfilter{fill68}] Cualquier texto.  lograr que el texto ocupe
  las primeras 68 columnas.  Es útil emplearlo antes de pasar el texto
  por el filtro \tplfilter{tabindent}, y queremos que aún quepa en una
  ventana de fuente fija y 80 columnas.
\item[\tplfilter{fill76}] Cualquier texto.  Lograr que el texto quepa
  en 76 columnas.
\item[\tplfilter{firstline}] Cualquier texto.  Mostrar la primera
  línea de texto sin saltos de línea.
\item[\tplkwfilt{date}{hgdate}] \tplkword{date} palabra clave.
  Mostrar la fecha como un par de números legibles.  Muestra una
  cadena como ``\Verb+1157407993 25200+''.
\item[\tplkwfilt{date}{isodate}] \tplkword{date} palabra clave.
  Mostrar la fecha como una cadena de texto en el formato.  Muestra
  una cadena como ``\Verb+2006-09-04 15:13:13 -0700+''.
\item[\tplfilter{obfuscate}] Cualquier texto, pero de mayor utilidad
  para la palabra clave \tplkword{author}.  Muestra el campo de texto
  como una secuencia de entidades XML.  Esto ayuda a eliminar ciertos
  robots estúpidos de adquisición de correo.
\item[\tplkwfilt{author}{person}] Cualquier texto, útil sobre todo
  para la palabra clave \tplkword{author}.  Muestra el texto que hay
  antes de la dirección de correo electrónico.  Por ejemplo,
  ``\Verb+Bryan O'Sullivan <bos@serpentine.com>+'' mostraría
  ``\Verb+Bryan O'Sullivan+''.
\item[\tplkwfilt{date}{rfc822date}] \tplkword{date} palabra clave.
  Muestra una fecha con el mismo formato que se usa en los encabezados
  de correo.  Mostraría una cadena como
 ``\Verb+Mon, 04 Sep 2006 15:13:13 -0700+''.
\item[\tplkwfilt{node}{short}] Hash del conjunto de cambios.  Muestra
  la forma corta de un hash de conjunto de cambios, 
  of a changeset hash, p.e.~una cadena hexadecimal de 12 bytes.
\item[\tplkwfilt{date}{shortdate}] \tplkword{date} palabra clave.
  Mostrar año, mes y día de una fecha.  Muestrauna cadena como
  ``\Verb+2006-09-04+''.
\item[\tplfilter{strip}] Cualquier texto.  Elimina todos los espacios
  en blanco al principio y al final de la cadena.
\item[\tplfilter{tabindent}] Cualquier texto.  Muestra el texto con
  todas las líneas excepto la primera que comience con el caracter tab.
\item[\tplfilter{urlescape}] Cualquier texto.  Escapa todos los
  caracteres que se consideren como ``especiales'' por los parsers de
  URL.  Por ejemplo, \Verb+foo bar+ se convierte en \Verb+foo%20bar+.
\item[\tplkwfilt{author}{user}] Cualquier texto, útil sobre todo para
  la palabra clave \tplkword{author}.  Retorna el ``usuario'' de una
  dirección de correo.  Por ejemplo, 
  ``\Verb+Bryan O'Sullivan <bos@serpentine.com>+'' se convierte en 
  ``\Verb+bos+''.
\end{itemize}

\begin{figure}
  \interaction{template.simple.manyfilters}
  \caption{Filtros de plantilla en acción}
  \label{fig:template:filters}
\end{figure}

\begin{note}
  Si trata de aplicar un filtro a una porción de datos que no puede
  procesarse, Mercurial fallará e imprimirá una excepción de Python.
  Por ejemplo, el tratar de usar la salida de la palabra clave
  \tplkword{desc} con el filtro \tplkwfilt{date}{isodate} no resultará
  algo útil.
\end{note}

\subsection{Combinar filtros}

Combinar filtros es para generar una salida en la forma como usted lo
desea es muy sencillo.  La cadena de filtros siguientes arman una
descripción, después aseguran que cabe limpiamente en 68 columnas, y
las indenta con 8~caracteres (por lo menos en sistemas tipo Unix, en
los que el tab por convención se extiende en 8~caracteres).

\interaction{template.simple.combine}

Observe el uso de ``\Verb+\t+'' (un caracter tab) en la plantilla para
forzar que la primera línea se indente; esto es necesario para lograr
que la primera línea luzca indentada;  es necesario debido a que
\tplkword{tabindent} indenta todas las líneas \emph{excepto} la primera.

Tenga en cuenta que el orden de los filtros importa.  El primer filtro
se aplica primero al resultado de la palabra clave; el segundo al
resultado de la aplicación del primer filtro y así sucesivamente.  Por
ejemplo, usar \Verb+fill68|tabindent+ es muy distinto al resultado de
usar \Verb+tabindent|fill68+.


\section{De plantillas a estilos}

Una plantilla provee una forma rápida y sencilla para dar formato a
una salida.  Las plantillas pueden volvers verbosas, y es útil poder
darle un nombre a una plantilla.  Un fichero de estilo es una
plantilla con un nombre, almacenado en un fichero.

Más aún, al usar un fichero de estilo se dispara el poder del motor de
plantillas en un nivel imposible de alcanzar usando las opción
\hgopt{log}{--template} desde la línea de órdenes.


\subsection{Los ficheros de estilo más sencillos}

Nuestro fichero sencillo de estilo contiene una sola línea:

\interaction{template.simple.rev}

Se le indica a Mercurial, ``si está imprimiendo un conjunto de
cambios, use el texto de la derecha como la plantilla''.

\subsection{Sintaxis de ficheros de estilo}

Las reglas de sintaxis para un fichero de estilo son sencillas:

\begin{itemize}
\item El fichero se procesa línea por línea.

\item Se ignoran el espacio en blanco circundante.

\item Se omiten las líneas en blanco.

\item Si una línea comienza con los caracteres ``\texttt{\#}'' o
  ``\texttt{;}'', la línea completa se trata como un comentario, y se
  omite como si fuera vacía.

\item Una línea comienza con una palabra clave.  Esta debe comenzar
  con una caracter alfabético o una raya al piso, y puede contener
  subsecuentemente cualquier caracter alfanumérico o una raya al
  piso.  (En notación de expresiones regulares debe coincidir con
  \Verb+[A-Za-z_][A-Za-z0-9_]*+.)

\item El próximo elemento debe ser un caracter ``\texttt{=}'', que
  puede estar precedido o seguido por una cantidad arbitraria de
  espacio.

\item Si el resto de la línea comienza y termina con caracteres
  encerrados entre caracteres de comillas(bien sea sencillas o
  dobles),  se trata como cuerpo de la plantilla.

\item Si el resto de la línea \emph{no} comienza con una comilla, se
  trata como el nombre de un fichero; los contenidos de este fichero
  se leerán y se usarán como cuerpo de la plantilla.
\end{itemize}

\section{Ejemplos de ficheros de estilos}

Para ilustrar la creación de un fichero de estilo, construiremos
algunos ejemplos.  En lugar de ofrecer un fichero completo de estilo y
analizarlo, replicaremos el proceso usual de desarrollo de un fichero
de estilo comenzando con algo muy sencillo, y avanzando por una serie
de ejemplos sucesivos más completos.

\subsection{Identificar equivocaciones en ficheros de estilo}

Si Mercurial encuentra un problema en un fichero de estilo en el cual
usted está trabajando, imprime un mensaje de error suscinto, cuando
usted identifique lo que significa, resulta muy útil.

\interaction{template.svnstyle.syntax.input}

Tenga en cuenta que \filename{broken.style} trata de definir la
palabra clave \texttt{changeset}, pero omite dar un contenido para esta.
Cuando se le indica a Mercurial que use este fichero de estilo, se
queja inmediatamente.

\interaction{template.svnstyle.syntax.error}

Este mensaje de error luce intimidante, pero no es muy difícil de
seguir:

\begin{itemize}
\item El primer componente es la forma como Mercurial dice ``me rindo''.
  \begin{codesample4}
    \textbf{abort:} broken.style:1: parse error
  \end{codesample4}

\item A continuación viene el nombre del fichero que contiene el error.
  \begin{codesample4}
    abort: \textbf{broken.style}:1: parse error
  \end{codesample4}

\item Siguendo el nombre del fichero viene el número de línea en la
  que se encontró el error.
  \begin{codesample4}
    abort: broken.style:\textbf{1}: parse error
  \end{codesample4}

\item Finalmente, la descripción de lo que falló.
  \begin{codesample4}
    abort: broken.style:1: \textbf{parse error}
  \end{codesample4}
  La descripción del problema no siempre es clara(como en este caso),
  pero aunque sea críptica, casi siempre es trivial la inspección
  visual de la línea en el fichero de estilo y encontrar lo que está
  mal.
\end{itemize}

\subsection{Identificar de forma única un repositorio}

Si desea identificar un repositorio de Mercurial ``de forma única''
con una cadena corta como identificador, puede usar la primera
revisión en el repositorio.
\interaction{template.svnstyle.id} 
No es garantía de unicidad, pero no es útill en ciertos casos:
many cases.
\begin{itemize}
\item No funcionará en un repositorio completamente vacío, porque un
  repositorio así no tiene una revisión~zero.
\item Tampoco funcionará en caso(muy raro) cuando el repositorio sea
  una fusión de dos repositorios independientes y tiene los dos
  directorios por ahí.
\end{itemize}
Hay ciertos casos en los cuales podría colocar el identificador:
\begin{itemize}
\item Como una llave en la tabla de una base de datos que administra
  repositorios en un servidor.
\item Como una parte del par \{\emph{ID~repositorio}, \emph{ID~revisión}\}.
  Almacene esta información de forma independiente cuando ejecute
  construcciones automatizadas u otras actividades, de forma que pueda
  ``reconstruir'' posteriormente en caso de ser necesario.
\end{itemize}

\subsection{Mostrando salida parecida a Subversion}

Intentemos emular la salida usual que usa otro sistema de control de
revisiones,  Subversion.
\interaction{template.svnstyle.short}

Dado que la salida de Subversion es sencilla, es fácil copiar y pegar
una porción de su salida en un fichero, y reemplazar el texto
producido previamente por Subversion con valores base que quisiéramos
ver expandidos.
\interaction{template.svnstyle.template}

Esta plantilla difiere en algunos detalles de la salida producida por
Subversion:
\begin{itemize}
\item Subversion imprime una fecha ``legible'' (el ``\texttt{Wed, 27 Sep
    2006}'' en el ejemplo de salida anterior) en paréntesis.  El motor
  de plantillas de Mercurial no ofrece una forma sencilla de desplegar
  una fecha en este formato sin imprimir también la hora y la zona horaria.
\item Emulamos las líneas de ``separación'' de subversion con caracteres
  ``\texttt{-}'' en una línea.  Usamos la palabra clave
  \tplkword{header} del motor de plantillas para imprimir una línea de
  separación como la primera línea de salida(ver más abajo), para
  lograr una salida similara a la de Subversion.
\item La salida de subversion incluye un conteo en el encabezado del
  número de líneas en el mensaje de consinación.  No podemos
  replicarlo en Mercurial; el motor de plantilla no ofrece en la
  actualidad un filtro que cuente la cantidad de objetos que se le
  pasen.
\end{itemize}
No me tomó más de un minuto o dos de trabajo para reemplazar texto
literal de un ejemplo de salida de la salida de Subversion con ciertas
palabras claves y filtros para ofrecer la plantilla anterior.  El
fichero de estilo se refiere sencillamente a la plantilla.
\interaction{template.svnstyle.style}

Podríamos haber incluído el texto del fichero plantilla directamente
en el fichero de estilo encerrando entre comillas y reemplazando las
nuevas líneas con secuencias ``\verb!\n!'', pero haría muy difícil de
leer el archivo de estilos.  La facilidad para leer es importante
cuando está decidiendo si un texto pertenece a un fichero de estilo o
a un fichero de plantilla incluído en el estilo.  Si el fichero de
estilo luce muy grande o complicado, si inserta una pieza de texto
literal, mejor colóquelo en una plantilla.

%%% Local Variables: 
%%% mode: latex
%%% TeX-master: "00book"
%%% End: 
