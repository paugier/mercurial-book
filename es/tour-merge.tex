\chapter{Una gira de Mercurial: fusionar trabajo}
\label{chap:tour-merge}

Hasta ahora hemos cubierto cómo clonar un repositorio, hacer cambios,
y jalar o empujar dichos cambios de un repositorio a otro. Nuestro
siguiente paso es \emph{fusionar} cambios de repositorios separados.

% TODO cambié streams por líneas. check please
\section{Fusionar líneas de trabajo}

Fusionar es una parte fundamental de trabajar con una herramienta 
de control distribuido de versiones.
\begin{itemize}
\item Alicia y Roberto tienen cada uno una copia personal del
    repositorio de un proyecto en el que están trabajando. Alicia
    arregla un fallo en su repositorio; Roberto añade una nueva
    característica en el suyo. Ambos desean que el repositorio
    compartido contenga el arreglo del fallo y la nueva
    característica.
\item Frecuentemente trabajo en varias tareas diferentes en un mismo
    proyecto al mismo tiempo, cada una aislada convenientemente de las
    otras en su propio repositorio. Trabajar de esta manera significa
    que a menudo debo fusionar una parte de mi propio trabajo con
    otra.
\end{itemize}

Como fusionar es una operación tan necesaria y común, Mercurial la
facilita. Revisemos el proceso. Empezaremos clonando (otro)
% TODO poner interrogante de apertura
repositorio (ve lo seguido que aparecen?) y haciendo un cambio en él.
\interaction{tour.merge.clone}
Ahora deberíamos tener dos copias de \filename{hello.c} con contenidos
diferentes.  El historial de los dos repositorios diverge ahora, como
se ilustra en la figura~\ref{fig:tour-merge:sep-repos}.
\interaction{tour.merge.cat}

\begin{figure}[ht]
  \centering
  \grafix{tour-merge-sep-repos}
  \caption{Historial reciente divergente de los repositorios
      \dirname{my-hello} y \dirname{my-new-hello}}
  \label{fig:tour-merge:sep-repos}
\end{figure}

Ya sabemos que jalar los cambios desde nuestro repositorio
\dirname{my-hello} no tendrá efecto en el directorio de trabajo.
\interaction{tour.merge.pull}
Sin embargo, el comando \hgcmd{pull} dice algo acerca de
``frentes''\ndt{El autor se refiere a \emph{heads} aquí.}.  

\subsection{Conjuntos de cambios de frentes}

Un frente es un cambio que no tiene descendientes, o hijos, como
también se les conoce. La revisión de punta es, por tanto, un frente,
porque la revisión más reciente en un repositorio no tiene ningún
% TODO cambio en la redacción de la frase, pero espero que conserve el
% sentido. Querido human@, apruebe o corrija :D
hijo. Sin embargo, un repositorio puede contener más de un frente.

\begin{figure}[ht]
  \centering
  \grafix{tour-merge-pull}
  \caption{Contenidos del repositorio después de jalar
      \dirname{my-hello} a \dirname{my-new-hello}}
  \label{fig:tour-merge:pull}
\end{figure}

En la figura~\ref{fig:tour-merge:pull} usted puede ver el efecto que
tiene jalar los cambios de \dirname{my-hello} a \dirname{my-new-hello}.
El historial que ya existía en \dirname{my-new-hello} se mantiene
intacto, pero fue añadida una nueva revisión. Refiriéndonos a la
figura~\ref{fig:tour-merge:sep-repos}, podemos ver que el \emph{ID del
conjunto de cambios} se mantiene igual en el nuevo repositorio, pero
el \emph{número de revisión} ha cambiado.  (Incidentalmente, éste es un
buen ejemplo de porqué no es seguro usar números de revisión cuando se
habla de conjuntos de cambios).  Podemos ver los frentes en un
repositorio usando el comando \hgcmd{heads}\ndt{Frentes.}.
\interaction{tour.merge.heads}

\subsection{Hacer la fusión}

% TODO poner interrogante de apertura
Qué pasa si tratamos de usar el comando usual, \hgcmd{update}, para
actualizar el nuevo frente?
\interaction{tour.merge.update}
Mercurial nos indica que el comando \hgcmd{update} no hará la fusión;
no actualizará el directorio de trabajo cuando considera que lo que
deseamos hacer es una fusión, a menos que lo obliguemos a hacerlo.
En vez de \hgcmd{update}, usamos el comando \hgcmd{merge} para hacer
la fusión entre los dos frentes.
\interaction{tour.merge.merge}

\begin{figure}[ht]
  \centering
  \grafix{tour-merge-merge}
  \caption{Directorio de trabajo y repositorio durante la fusión, y
  consignación consecuente}
  \label{fig:tour-merge:merge}
\end{figure}

Esto actualiza el directorio de trabajo, de tal forma que contenga los
cambios de \emph{ambos} frentes, lo que se ve reflejado tanto en la
salida de \hgcmd{parents} como en los contenidos de \filename{hello.c}.
\interaction{tour.merge.parents}

\subsection{Consignar los resultados de la fusión}

Siempre que hacemos una fusión, \hgcmd{parents} mostrará dos padres
hasta que consignemos (\hgcmd{commit}) los resultados de la fusión.
\interaction{tour.merge.commit}
Ahora tenemos una nueva revisión de punta; note que tiene \emph{los
dos} frentes anteriores como sus padres. Estos son las mismas
revisiones que mostró previamente el comando \hgcmd{parents}.
\interaction{tour.merge.tip}
En la figura~\ref{fig:tour-merge:merge} usted puede apreciar una
representación de lo que pasa en el directorio de trabajo durante la
fusión cuando se hace la consignación. Durante la fusión, el
directorio de trabajo tiene dos conjuntos de cambios como sus padres,
y éstos se vuelven los padres del nuevo conjunto de cambios.

\section{Fusionar cambios con conflictos}

La mayoría de las fusiones son algo simple, pero a veces usted se
encontrará fusionando cambios donde más de uno de ellos afecta las
mismas secciones de los mismos ficheros. A menos que ambas
modificaciones sean idénticas, el resultado es un \emph{conflicto}, en
donde usted debe decidir cómo reconciliar ambos cambios y producir un
resultado coherente.

\begin{figure}[ht]
  \centering
  \grafix{tour-merge-conflict}
  \caption{Cambios con conflictos a un documento}
  \label{fig:tour-merge:conflict}
\end{figure}

La figura~\ref{fig:tour-merge:conflict} ilustra un ejemplo con dos
cambios generando conflictos en un documento. Empezamos con una sola
versión de el fichero; luego hicimos algunos cambios; mientras tanto,
alguien más  hizo cambios diferentes en el mismo texto. Lo que debemos
hacer para resolver el conflicto causado por ambos cambios es decidir
cómo debe quedar finalmente el fichero.

Mercurial no tiene ninguna utilidad integrada para manejar conflictos.
En vez de eso, ejecuta un programa externo llamado \command{hgmerge}.
Es un guión de línea de comandos que es instalado junto con Mercurial;
usted puede modificarlo para que se comporte como usted lo desee. Por
defecto, lo que hace es tratar de encontrar una de varias herramientas
para fusionar que es probable que estén instaladas en su sistema.
Primero se intenta con unas herramientas para fusionar cambios
automáticamente; si esto no tiene éxito (porque la fusión demanda
una guía humana) o dichas herramientas no están presentes, el guión
intenta con herramientas gráficas para fusionar.

También es posible hacer que Mercurial ejecute otro programa o guión
en vez de \command{hgmerge}, definiendo la variable de entorno
\envar{HGMERGE} con el nombre del programa de su preferencia.

\subsection{Usar una herramienta gráfica para fusión}

Mi herramienta favorita para hacer fusiones es \command{kdiff3}, y la
usaré para describir las características comunes de las herramientas
gráficas para hacer fusiones. Puede ver una captura de pantalla de
\command{kdiff3} ejecutándose, en la
figura~\ref{fig:tour-merge:kdiff3}.  El tipo de fusión que la
herramienta hace se conoce como \emph{fusión de tres vías}, porque hay
tres versiones diferentes del fichero en que estamos interesados.
Debido a esto la herramienta divide la parte superior de la ventana en
tres paneles.
\begin{itemize}
\item A la izquierda está la revisión \emph{base} del fichero, p.ej.~la
    versión más reciente de la que descienden las dos versiones que
    estamos tratando de fusionar.
\item En la mitad está ``nuestra'' versión del fichero, con las
    modificaciones que hemos hecho.
\item A la derecha está la versión del fichero de ``ellos'', la que
    forma parte del conjunto de cambios que estamos tratando de
    fusionar.
\end{itemize}
En el panel inferior se encuentra el \emph{resultado} actual de la
fusión. Nuestra tarea es reemplazar todo el texto rojo, que muestra
los conflictos sin resolver, con una fusión adecuada de ``nuestra''
versión del fichero y la de ``ellos''.

Los cuatro paneles están \emph{enlazados}; si avanzamos vertical o
horizontalmente en cualquiera de ellos, los otros son actualizados
para mostrar las secciones correspondientes del fichero que tengan
asociado.

\begin{figure}[ht]
  \centering
  \grafix[width=\textwidth]{kdiff3}
  \caption{Usando \command{kdiff3} para fusionar versiones de un
  fichero}
  \label{fig:tour-merge:kdiff3}
\end{figure}

En cada conflicto del fichero podemos escoger resolverlo usando
cualquier combinación del texto de la revisión base, la nuestra, o la
de ellos. También podemos editar manualmente el fichero en que queda
la fusión, si es necesario hacer cambios adicionales.

Hay \emph{muchas} herramientas para fusionar ficheros disponibles. Se
diferencian en las plataformas para las que están disponibles, y en
sus fortalezas y debilidades particulares. La mayoría están afinadas
para fusionar texto plano, mientras que otras están pensadas para
formatos de ficheros especializados (generalmente XML).

% TODO traduje "worked" como "real"
\subsection{Un ejemplo real}

En este ejemplo, reproduciremos el historial de modificaciones al
fichero de la figura~\ref{fig:tour-merge:conflict} mostrada
anteriormente.  Empecemos creando un repositorio con la versión base
de nuestro documento.
\interaction{tour-merge-conflict.wife}
Clonaremos el repositorio y haremos un cambio al fichero.
\interaction{tour-merge-conflict.cousin}
Y haremos otro clon, para simular a alguien más haciendo un cambio al
mismo fichero. (Esto introduce la idea de que no es tan inusual hacer
fusiones consigo mismo, cuando usted aísla tareas en repositorios
separados, y de hecho encuentra conflictos al hacerlo.)
\interaction{tour-merge-conflict.son}
Ahora que tenemos dos versiones diferentes de nuestro fichero,
crearemos un entorno adecuado para hacer la fusión.
\interaction{tour-merge-conflict.pull}

En este ejemplo, no usaré el comando normal de Mercurial para hacer la
fusión (\command{hgmerge}), porque lanzaría mi linda herramienta
automatizada para correr ejemplos dentro de una interfaz gráfica de
usuario. En vez de eso, definiré la variable de entorno
\envar{HGMERGE} para indicarle a Mercurial que use el comando
\command{merge}. Este comando forma parte de la instalación base de
muchos sistemas Unix y similares. Si usted está ejecutando este
ejemplo en su computador, no se moleste en definir \envar{HGMERGE}.
\interaction{tour-merge-conflict.merge}
Debido a que \command{merge} no puede resolver los conflictos que
aparecen, él deja \emph{marcadores de fusión} en el fichero con
conflictos, indicando si provienen de nuestra versión o de la de
ellos.

Mercurial puede saber ---por el código de salida del comando
\command{merge}--- que no fue posible hacer la fusión exitosamente,
así que nos indica qué comandos debemos ejecutar si queremos rehacer
la fusión. Esto puede ser útil si, por ejemplo, estamos ejecutando una
herramienta gráfica de fusión y salimos de ella porque nos confundimos
o cometimos un error.

Si la fusión ---automática o manual--- falla, no hay nada que nos
impida ``arreglar'' los ficheros afectados por nosotros mismos, y
consignar los resultados de nuestra fusión:
% TODO este mercurial no tiene el comando resolve. Revisar si sigue
% siendo necesario
\interaction{tour-merge-conflict.commit}

\section{Simplificar el ciclo jalar-fusionar-consignar}
\label{sec:tour-merge:fetch}

El proceso de fusionar cambios delineado anteriomente es directo, pero
requiere la ejecución de tres comandos en sucesión.
\begin{codesample2}
  hg pull
  hg merge
  hg commit -m 'Fusionados cambios remotos'
\end{codesample2}
En la consignación final usted debe proveer un mensaje adecuado, que
casi siempre es un fragmento de texto ``de relleno'' carente de valor
particular.

Sería agradable reducir la cantidad de pasos necesarios, si fuera
posible. De hecho, Mercurial es distribuido junto con una extensión
llamada \hgext{fetch}\ndt{Descargar, traer.} que hace precisamente
esto.

Mercurial cuenta con un mecanismo de extensión flexible que le permite
% TODO lets people => permite a usuarios
a sus usuarios extender su funcionalidad, manteniendo el núcleo de
Mercurial pequeño y fácil de manejar. Algunas extensiones añaden
nuevos comandos que usted puede usar desde la línea de comandos,
mientras que otros funcionan ``tras bambalinas'', por ejemplo,
añadiendo funcionalidad al servidor.

La extensión \hgext{fetch} añade un comando llamado, no
sorpresivamente, \hgcmd{fetch}.  Esta extensión actúa como una
combinación de \hgcmd{pull}, \hgcmd{update} y \hgcmd{merge}.  Empieza
jalando cambios de otro repositorio al repositorio actual. Si
encuentra que los cambios añaden un nuevo frente en el repositorio
actual, inicia una fusión, y luego consigna el resultado de la misma
con un mensaje generado automáticamente. Si no se añadieron nuevos
frentes, actualiza el directorio de trabajo con el nuevo conjunto de
cambios de punta.

Activar la extensión \hgext{fetch} es fácil. Edite su
\sfilename{.hgrc}, y vaya a (o cree) la sección
\rcsection{extensions}.  Luego añada una línea que diga simplemente
``\Verb+fetch +''.
\begin{codesample2}
  [extensions]
  fetch =
\end{codesample2}
(Normalmente, a la derecha del ``\texttt{=}'' debería aparecer la
ubicación de la extensión, pero como el comando \hgext{fetch} es parte
de la distribución estándar, Mercurial sabe dónde buscarla.)

%%% Local Variables: 
%%% mode: latex
%%% TeX-master: "00book"
%%% End: 
