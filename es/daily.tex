\chapter{Mercurial día a día}
\label{chap:daily}

\section{Cómo indicarle a Mercurial qué ficheros seguir}

Mercurial no trabaja con ficheros en su repositorio a menos que usted
se lo indique explícitamente.  La orden \hgcmd{status} le mostrará
cuáles ficheros son desconocidos para Mercurial; se emplea un
``\texttt{?}'' para mostrar tales ficheros.

Para indicarle a Mercurial que tenga en cuenta un fichero, emplee la
orden \hgcmd{add}. Una vez que haya adicionado el fichero, la línea
referente al fichero al aplicar la orden \hgcmd{status} para tal
fichero cambia de ``\texttt{?}'' a ``\texttt{A}''.
\interaction{daily.files.add}

Después de invocar \hgcmd{commit}, los ficheros que haya adicionado
antes de consignar no se listarán en la salida de \hgcmd{status}.  La
razón para esto es que \hgcmd{status} solamente le muestra aquellos
ficheros ``interesantes''---los que usted haya modificado o a aquellos
sobre los que usted haya indicado a Mercurial hacerles algo---de forma
predeterminada. Si tiene un repositorio que contiene miles de
ficheros, inusualmente deseará saber cuáles de ellos están siendo
seguidos por Mercurial, pero que no han cambiado.  (De todas maneras,
puede obtener tal información; más adelante hablaremos de ello.)


Cuando usted añade un fichero, Mercurial no hace nada con él inmediatamente.
A cambio, tomará una instantánea del estado del fichero la próxima vez
que usted consigne. Continuará haciendo seguimiento a los cambios que
haga sobre el fichero cada vez que consigne, hasta que usted lo elimine.

\subsection{Nombramiento explícito e implícito de ficheros}

Mercurial tiene un comportamiento útil en el cual si a una orden,
le pasa el nombre de un directorio, todas las órdenes lo tratarán como
``Deseo operar en cada fichero de este directorio y sus 
subdirectorios''.
\interaction{daily.files.add-dir}
Tenga en cuenta que en este ejemplo Mercurial imprimió los nombres de
los ficheros que se adicionaron, mientras que no lo hizo en el ejemplo
anterior cuando adicionamos el fichero con nombre \filename{a}.

En el último caso hicimos explícito el nombre del fichero que
deseábamos adicionar en la línea de órdenes, y Mercurial asume en
tales casos que usted sabe lo que está haciendo y no imprime
información alguna.

Cuando hacemos \emph{implícitos} los nombres de los ficheros dando el
nombre de un directorio, Mercurial efectúa un paso extra al imprimir
el nombre de cada fichero con el que va a hacer algo.  Esto para
aclarar lo que está sucediendo, y reducir en lo posible una sorpresa
silenciosa pero fatal.  Este comportamiento es común a la mayoría de
órdenes en Mercurial.

\subsection{Nota al margen:Mercurial trata ficheros, no directorios}

Mercurial no da seguimiento a la información de los directorios.  En
lugar de eso tiene en cuenta las rutas de los ficheros.  Antes  de
crear un fichero, primero crea todos los directorios que hagan falta
para completar la ruta del fichero. Después de borrar un fichero,
borra todos los directorios vacíos que estuvieran en la ruta del
fichero borrado. Suena como una diferencia trivial, pero tiene una
consecuencia práctica menor: no es posible representar un directorio
completamente vacío en Mercurial.

Los directorios vacíos son inusualmente útiles, hay soluciones
alternativas no intrusivas que puede emplear para obtener el efecto
apropiado. Los desarrolladores de Mercurial pensaron que la
complejidad necesaria para administrar directorios vacíos no valía la
pena frente al beneficio limitado que esta característica podría traer.

Si necesita un directorio vacío en su repositorio, hay algunas formas
de lograrlo. Una es crear un directorio, después hacer \hgcmd{add} a
un fichero ``escondido'' dentro de ese directorio. En sistemas tipo
Unix, cualquier fichero cuyo nombre comience con un punto
(``\texttt{.}'') se trata como escondido por la mayoría de
herramientas GUI. Esta aproximación se ilustra en la figura~\ref{ex:daily:hidden}.

\begin{figure}[ht]
  \interaction{daily.files.hidden}
  \caption{Simular un directorio vacío con un fichero escondido}
  \label{ex:daily:hidden}
\end{figure}

Otra forma de abordar la necesidad de un fichero vacío es crear
simplemente uno en sus guiones de construcción antes de ser necesarios.

\section{Cómo dejar de hacer seguimiento a un fichero}

Si decide que un fichero no pertenece a su repositorio, use la orden
\hgcmd{remove}; se borrará el fichero y le indicará a Mercurial que
deje de hacerle seguimiento.  Los ficheros eliminados se representan
con ``\texttt{R}'' al usar \hgcmd{status}.
\interaction{daily.files.remove}

Después de hacer \hgcmd{remove} a un fichero, Mercurial dejará de
hacer seguimiento al mismo, incluso si recrea el fichero con el mismo
nombre en su directorio de trabajo. Si decide recrear un fichero con
el mismo nombre y desea que Mercurial le haga seguimiento, basta con
hacerle \hgcmd{add}. Mercurial sabrá que el fichero recientemente
adicionado no está relacionado con el fichero anterior que tenía el
mismo nombre.

\subsection{Al eliminar un fichero no se afecta su historial}

Es preciso tener en cuenta que al eliminar un fichero se tiene
dos efectos solamente.
\begin{itemize}
\item Se elimina la versión actual del fichero del directorio de 
trabajo.
\item Mercurial deja de hacer seguimiento a los cambios del fichero
  desde la próxima consignación.
\end{itemize}
Al eliminar un fichero \emph{no} se altera de ninguna manera el
\emph{historial} del mismo.

Si actualiza su directorio de trabajo a un conjunto de cambios en el
cual el fichero que eliminó aún era tenido en cuenta, reaparecerá en
el directorio de trabajo, con los contenidos que este tenía cuando se
consignó tal conjunto de cambios.  Si usted actualiza el directorio de
trabajo a un conjunto de cambios posterior en el cual el fichero había
sido eliminado, Mercurial lo eliminará de nuevo del directorio de
trabajo.

\subsection{Ficheros perdidos}

Mercurial considera como \emph{perdido} un fichero que usted borró,
pero para el que no se usó \hgcmd{remove}.  Los ficheros perdidos se
representan con ``\texttt{!}'' al visualizar \hgcmd{status}.
Las órdenes de Mercurial generalmente no harán nada con los ficheros
perdidos.
\interaction{daily.files.missing}

Si su repositorio contiene un fichero que \hgcmd{status} reporta como
perdido, y desea que el mismo se vaya, se puede usar 
\hgcmdargs{remove}{\hgopt{remove}{--after}} posteriormente para
indicarle a Mercurial que usted deseaba borrar tal fichero.
\interaction{daily.files.remove-after}

Por otro lado, si borró un fichero perdido por accidente, puede usar
\hgcmdargs{revert}{\emph{nombre de fichero}} para recuperar el
fichero. Reaparecerá sin modificaciones.
\interaction{daily.files.recover-missing}

\subsection{Nota al margen: ¿Por qué decirle explícitamente a Mercurial
  que elimine un fichero?}

Es posible que se haya preguntado por qué Mercurial exige que usted le
indique explícitamente que está borrando un fichero. Al principio del
desarrollo de Mercurial, este permitía que usted borrara el fichero
sin más; Mercurial se daría cuanta de la ausencia del fichero
automáticamente después de la ejecución de \hgcmd{commit}, y dejaba de
hacer seguimiento al fichero.  En la práctica, resultaba muy sencillo
borrar un fichero accidentalmente sin darse cuenta.

\subsection{Atajo útil---agregar y eliminar ficheros en un solo paso}

Mercurial ofrece una orden combinada, \hgcmd{addremove}, que agrega
los ficheros que no tienen seguimiento y marca los ficheros faltantes
como eliminados.
\interaction{daily.files.addremove}
La orden \hgcmd{commit} su puede usar con la opción \hgopt{commit}{-A} 
que aplica el agregar-eliminar, seguido inmediatamente de una
consignación.
\interaction{daily.files.commit-addremove}

\section{Copiar ficheros}

Mercurial ofrece la orden \hgcmd{copy} para hacer una nueva copia de
un fichero.  Cuando se copia un fichero con esta orden, Mercurial
lleva un registro indicando que el nuevo fichero es una copia del
fichero original. Trata de forma especial los ficheros copiados cuando
usted hace una fusión con el trabajo de alguien más.

\subsection{Resultados de copiar un fichero durante una fusión}

Durante una fusión los cambios ``siguen'' una copia.  Para ilustrar
lo que esto significa, haremos un ejemplo.  Comenzaremos con el mini
repositorio usual que contiene un solo fichero
\interaction{daily.copy.init}
Debemos trabajar algo en paralelo, de forma que tengamos algo para
fusionar. Aquí clonamos el repositorio.
\interaction{daily.copy.clone}
De vuelta en el repositorio, usemos la orden \hgcmd{copy} para hacer
una copia del primer fichero que creamos.
\interaction{daily.copy.copy}

Si vemos la salida de la orden \hgcmd{status}, el fichero copiado luce
como un fichero que se ha añadido normalmente.
\interaction{daily.copy.status}
Pero si usamos la opción \hgopt{status}{-C} de la orden \hgcmd{status},
imprimirá otra línea: el fichero \emph{desde} el cual fue copiado nuestro
fichero recién añadido.
\interaction{daily.copy.status-copy}

Ahora, en el repositorio que clonamos, hagamos un cambio en
paralelo. Adicionaremos una línea de contenido al fichero original que
creamos.
\interaction{daily.copy.other}
Hemos modificado el fichero \filename{file} en este
repositorio. Cuando jalemos los cambios del primer repositorio y
fusionemos las dos cabezas, Mercurial propagará los cambios que hemos
hecho localmente en \filename{file} a su copia, \filename{new-file}.
\interaction{daily.copy.merge}

\subsection{¿Por qué los cambios se reflejan en las copias?}
\label{sec:daily:why-copy}

Este comportamiento de cambios en ficheros que se propagan a las
copias de los ficheros parecería esotérico, pero en la mayoría de
casos es absolutamente deseable.

Es indispensable recordar que esta propagación \emph{solamente} sucede
cuando fusionamos.  Por lo tanto si sobre un fichero se usa
\hgcmd{copy}, y se modifica el fichero original durante el curso
normal de su trabajo, nada pasará.

Lo segundo a tener en cuenta es que las modificaciones solamente se
propagarán en las copias únicamente si los repositorios de los cuales
está jalando los cambios \emph{no saben} de la copia.

Explicaremos a continuación la razón de este comportamiento de
Mercurial. Digamos que yo he aplicado un arreglo de un fallo importante a un
fichero fuente y consigné los cambios.  Por otro lado, usted decidió hacer
\hgcmd{copy} sobre el fichero en su repositorio, sin saber acerca del
fallo o sin ver el arreglo, y ha comenzado a trabajar sobre su copia
del fichero.

Si jala y fusiona mis cambios y Mercurial \emph{no hubiera} propagado
los cambios en las copias, su fichero fuente tendría el fallo, a menos
que usted haya recordado propagar el arreglo del fallo a mano, el
mismo \emph{permanecería} en su copia del fichero.

Mercurial previene esta clase de problemas, gracias a la propagación
automática del cambio que arregló el fallo del fichero original. Hasta
donde sé, Mercurial es el \emph{único} sistema de control de
revisiones que propaga los cambios en las copias de esta forma.

Cuando su historial de cambios tiene un registro de la copia y la
subsecuente fusión, usualmente no es necesario propagar los cambios el
fichero original a las copias del mismo, y por esta razón Mercurial
propaga únicamente los cambios en las copias hasta este punto y no más
allá.


\subsection{Cómo hacer que los cambios \emph{no} sigan a la copia?}

Si por algún motivo usted decide que esta característica de
propagación automática de cambios en las copias no es para usted, use
la orden usual de sus sistema para copiar ficheros (En sistemas tipo
Unix, es \command{cp}), posteriormente use \hgcmd{add} sobre la nueva
copia hecha a mano.  Antes de hacerlo, de todas maneras, relea la
sección~\ref{sec:daily:why-copy}, y tome una decisión asegurándose que
este comportamiento no es el apropiado para su caso específico.

\subsection{Comportamiento de la orden \hgcmd{copy}}

Cuando usa la orden \hgcmd{copy}, Mercurial hace una copia de cada
fichero fuente del directorio actual. Esto significa que si usted hace
modificaciones a un fichero, y le aplica \hgcmd{copy} sin haber
consignado primero los cambios, la nueva copia contendrá también las
modificaciones que haya hecho hasta ese punto. (Este comportamiento me
parece poco intuitivo, y por tal motivo lo menciono.)

La orden \hgcmd{copy} actúa de forma parecida a la orden \command{cp}
de Unix (puede usar el alias \hgcmd{cp} si le es más cómodo).  El
último argumento es el \emph{destino}, y todos los argumentos previos
son las \emph{fuentes}.  Si solamente indica un fichero como la
fuente, y el destino no existe, se crea un fichero nuevo con ese nombre.
\interaction{daily.copy.simple}
Si el destino es un directorio, Mercurial copia las fuentes en este.
\interaction{daily.copy.dir-dest}
La copia de un directorio es recursiva, y preserva la estructura del
directorio fuente.
\interaction{daily.copy.dir-src}
Si tanto la fuente como el destino son directorios, la estructura de
la fuente se recrea en el directorio destino.
\interaction{daily.copy.dir-src-dest}

De la misma forma como la orden \hgcmd{rename}, si copia un fichero
manualmente y desea que Mercurial sepa que ha copiado un fichero,
basta con aplicar la opción \hgopt{copy}{--after} a la orden 
\hgcmd{copy}.
\interaction{daily.copy.after}

\section{Renombrar ficheros}

La necesidad de renombrar un fichero es más común que hacer una copia
del fichero.  La razón por la cual discutí la orden \hgcmd{copy} antes
de hablar acerca de cambiar el nombre de los ficheros, es que
Mercurial trata el renombrar un fichero de la misma forma que una
copia.  Por lo tanto, saber lo que hace Mercurial cuando usted copia
un fichero, le indica qué esperar cuando renombra un fichero.

Cuando usa la orden \hgcmd{rename}, Mercurial hace una copia de cada
fichero fuente, lo borra y lo marca como fichero eliminado.
\interaction{daily.rename.rename}
La orden \hgcmd{status} muestra la nueva copia del fichero como
añadido y el fichero inicial de la copia, como eliminado.
\interaction{daily.rename.status}
De la misma forma como se usa la orden \hgcmd{copy}, debemos usar la
opción \hgopt{status}{-C} de la orden \hgcmd{status} para verificar
que el fichero añadido realmente comienza a  ser seguido por Mercurial
como una copia del fichero original, ahora eliminado.
\interaction{daily.rename.status-copy}

Igual que con \hgcmd{remove} y \hgcmd{copy}, puede indicársele a
Mercurial acerca de un renombramiento inmediato con la opción
\hgopt{rename}{--after}.   El comportamiento de la orden
\hgcmd{rename} y las opciones que acepta, son similares a la orden
\hgcmd{copy} en casi todo.

\subsection{Renombrar ficheros y fusionar cambios}

Dado que el renombrar de Mercurial se implementa como un
copiar-y-eliminar, la misma propagación de cambios ocurre cuando usted
fusiona después de renombrar como después de hacer una copia.

Si Yo modifico un fichero y usted lo renombra a un nuevo fichero, y
posteriormente fusionamos nuestros cambios respectivos, mi
modificación al fichero bajo su nombre original se propagará en el
fichero con el nuevo nombre. (Es lo que se esperaría como ``lo hace,''
pero, no todos los sistemas de control de revisiones lo logran.)

El hecho de que los cambios sigan la copia es una característica que
puede subestimar diciendo ``si, puede ser útil,'' debería ser claro
que el seguimiento de cambios de un renombramiento es importante
definitivamente.  Sin esto, sería muy sencillo que los cambios se
volvieran huérfanos cuando los ficheros se renombran.

\subsection{Cambios de nombre divergentes y fusión}

El caso de renombramiento con nombres divergentes ocurre cuando dos
desarrolladores comienzan  con un fichero---llamémoslo
\filename{foo}---en sus repositorios respectivos.

\interaction{rename.divergent.clone}
Ana renombra el fichero a \filename{bar}.
\interaction{rename.divergent.rename.anne}
Mientras que Roberto lo renombra como \filename{quux}.
\interaction{rename.divergent.rename.bob}

Veo esto como un conflicto porque cada desarrollador ha expresado
intenciones diferentes acerca de cómo considera debería haberse
renombrado el fichero.

¿Qué cree que debería pasar cuando fusionen su trabajo?
El comportamiento de Mercurial es que siempre preserva \emph{ambos}
nombres cuando fusiona  los conjuntos de cambios que contienen nombres
divergentes.
\interaction{rename.divergent.merge}

Tenga en cuenta que Mercurial le advierte acerca de nombres
divergentes, pero deja que usted decida qué hacer con la divergencia
después de la fusión.

\subsection{Cambios de nombre convergentes y fusión}

Otra clase de conflicto al cambiar el nombre ocurre cuando dos
personas eligen renombrar diferentes ficheros \emph{fuente} al mismo
\emph{destino}. En este caso Mercurial aplica su maquinaria de fusión
usual, y le permite a usted guiarlo a una solución adecuada.

\subsection{Otros casos límites relacionados con renombramientos}

Mercurial tiene un fallo de mucho tiempo en el cual no es capaz de
fusionar cuando por un lado hay un fichero con un nombre dado,
mientras que en otro hay un directorio con el mismo nombre. Esto está
documentado como~\bug{29}.
\interaction{issue29.go}

\section{Recuperarse de equivocaciones}

Mercurial tiene unas órdenes poderosas que le ayudarán a recuperarse
de equivocaciones comunes.

La orden \hgcmd{revert} le permite deshacer cambios que haya hecho a
su directorio de trabajo. Por ejemplo, Si aplicó \hgcmd{add} a un
fichero por accidente, ejecute \hgcmd{revert} con el nombre del
fichero que añadió, y en tanto que el fichero no haya sido tocado de
forma alguna, no será adicionado, ni seguido por Mercurial.  También
puede usar \hgcmd{revert} para deshacerse de cambios erróneos a un
fichero.

Tenga en cuenta que la orden \hgcmd{revert} se usa para cambios que no
han sido consignados. Cuando haya consignado un cambio, si decide que
era un error, puede hacer algo todavía, pero sus opciones pueden ser
más limitadas.

Para obtener información acerca de la orden \hgcmd{revert} y detalles
de cómo tratar con cambios consignados, vea el capítulo~\ref{chap:undo}.

%%% Local Variables: 
%%% mode: latex
%%% TeX-master: "00book"
%%% End: 
