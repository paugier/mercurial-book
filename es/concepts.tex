\chapter{Tras bambalinas}
\label{chap:concepts}

A diferencia de varios sistemas de control de revisiones, los
conceptos en los que se fundamenta Mercurial son lo suficientemente
simples como para entender fácilmente cómo funciona el software. 
Saber esto no es necesario, pero considero útil tener un ``modelo
mental'' de qué es lo que sucede.

Comprender esto me da la confianza de que Mercurial ha sido
cuidadosamente diseñado para ser tanto \emph{seguro} como
\emph{eficiente}.  Y tal vez con la misma importancia, si es fácil
para mí hacerme a una idea adecuada de qué está haciendo el software
cuando llevo a cabo una tarea relacionada con control de revisiones,
es menos probable que me sosprenda su comportamiento.

En este capítulo, cubriremos inicialmente los conceptos centrales
del diseño de Mercurial, y luego discutiremos algunos detalles
interesantes de su implementación.

\section{Registro del historial de Mercurial}

\subsection{Seguir el historial de un único fichero}

Cuando Mercurial sigue las modificaciones a un fichero, guarda el
historial de dicho fichero en un objeto de metadatos llamado
\emph{filelog}\ndt{Fichero de registro}.  Cada entrada en el fichero
de registro contiene suficiente información para reconstruir una
revisión del fichero que se está siguiendo. Los ficheros de registro
son almacenados como ficheros el el directorio
\sdirname{.hg/store/data}. Un fichero de registro contiene dos tipos
de información: datos de revisiones, y un índice para ayudar a
Mercurial a buscar revisiones eficientemente.

El fichero de registro de un fichero grande, o con un historial muy
largo, es guardado como ficheros separados para datos (sufijo
``\texttt{.d}'') y para el índice (sufijo ``\texttt{.i}''). Para
ficheros pequeños con un historial pequeño, los datos de revisiones y
el índice son combinados en un único fichero ``\texttt{.i}''. La
correspondencia entre un fichero en el directorio de trabajo y el
fichero de registro que hace seguimiento a su historial en el
repositorio se ilustra en la figura~\ref{fig:concepts:filelog}.

\begin{figure}[ht]
  \centering
  \grafix{filelog}
  \caption{Relación entre ficheros en el directorio de trabajo y
  ficheros de registro en el repositorio}
  \label{fig:concepts:filelog}
\end{figure}

\subsection{Administración de ficheros monitoreados}

Mercurial usa una estructura llamada \emph{manifiesto} para
% TODO collect together => centralizar
centralizar la información que maneja acerca de los ficheros que
monitorea. Cada entrada en el manifiesto contiene información acerca
de los ficheros involucrados en un único conjunto de cambios. Una
entrada registra qué ficheros están presentes en el conjunto de
cambios, la revisión de cada fichero, y otros cuantos metadatos del
mismo.

\subsection{Registro de información del conjunto de cambios}

La \emph{bitácora de cambios} contiene información acerca de cada
conjunto de cambios. Cada revisión indica quién consignó un cambio, el
comentario para el conjunto de cambios, otros datos relacionados con
el conjunto de cambios, y la revisión del manifiesto a usar.

\subsection{Relaciones entre revisiones}

Dentro de una bitácora de cambios, un manifiesto, o un fichero de
registro, cada revisión conserva un apuntador a su padre inmediato
(o sus dos padres, si es la revisión de una fusión). Como menciońe
anteriormente, también hay relaciones entre revisiones \emph{a través}
de estas estructuras, y tienen naturaleza jerárquica.

Por cada conjunto de cambios en un repositorio, hay exactamente una
revisión almacenada en la bitácora de cambios. Cada revisión de la
bitácora de cambios contiene un apuntador a una única revisión del
manifiesto. Una revisión del manifiesto almacena un apuntador a una
única revisión de cada fichero de registro al que se le hacía
seguimiento cuando fue creado el conjunto de cambios. Estas relaciones
se ilustran en la figura~\ref{fig:concepts:metadata}.

\begin{figure}[ht]
  \centering
  \grafix{metadata}
  \caption{Relaciones entre metadatos}
  \label{fig:concepts:metadata}
\end{figure}

Como lo muestra la figura, \emph{no} hay una relación ``uno a uno''
entre las revisiones en el conjunto de cambios, el manifiesto, o el
fichero de registro. Si el manifiesto no ha sido modificado de un
conjunto de cambios a otro, las entradas en la bitácora de cambios
para esos conjuntos de cambios apuntarán a la misma revisión del
manifiesto. Si un fichero monitoreado por Mercurial no sufre ningún
cambio de un conjunto de cambios a otro, la entrada para dicho fichero
en las dos revisiones del manifiesto apuntará a la misma revisión de
su fichero de registro.

\section{Almacenamiento seguro y eficiente}

La base común de las bitácoras de cambios, los manifiestos, y los
ficheros de registros es provista por una única estructura llamada el
\emph{revlog}\ndt{Contracción de \emph{revision log}, registro de
revisión.}.

\subsection{Almacenamiento eficiente}

El revlog provee almacenamiento eficiente de revisiones por medio del
mecanismo de \emph{deltas}\ndt{Diferencias.}.  En vez de almacenar una
copia completa del fichero por cada revisión, almacena los cambios
necesarios para transformar una revisión anterior en la nueva
revisión. Para muchos tipos de fichero, estos deltas son típicamente
de una fracción porcentual del tamaño de una copia completa del
fichero.

Algunos sistemas de control de revisiones obsoletos sólo pueden
manipular deltas de ficheros de texto plano. Ellos o bien almacenan
los ficheros binarios como instantáneas completas, o codificados en
alguna representación de texto plano adecuada, y ambas alternativas
son enfoques que desperdician bastantes recursos. Mercurial puede
manejar deltas de ficheros con contenido binario arbitrario; no
necesita tratar el texto plano como un caso especial.

\subsection{Operación segura}
\label{sec:concepts:txn}

Mercurial sólo \emph{añade} datos al final de los ficheros de revlog. Nunca
modifica ninguna sección de un fichero una vez ha sido escrita. Esto es más
robusto y eficiente que otros esquemas que requieren modificar o reescribir
datos.

Adicionalmente, Mercurial trata cada escritura como parte de una
\emph{transacción}, que puede cubrir varios ficheros. Una transacción es
\emph{atómica}: o bien la transacción tiene éxito y entonces todos sus efectos
son visibles para todos los lectores, o la operación completa es cancelada.
% TODO atomicidad no existe de acuerdo a DRAE, reemplazar
Esta garantía de atomicidad implica que, si usted está ejecutando dos copias de
Mercurial, donde una de ellas está leyendo datos y la otra los está escribiendo,
el lector nunca verá un resultado escrito parcialmente que podría confundirlo.

El hecho de que Mercurial sólo hace adiciones a los ficheros hace más fácil
proveer esta garantía transaccional. A medida que sea más fácil hacer
operaciones como ésta, más confianza tendrá usted en que sean hechas
correctamente.

\subsection{Recuperación rápida de datos}

Mercurial evita ingeniosamente un problema común a todos los sistemas de control
de revisiones anteriores> el problema de la
\emph{recuperación\ndt{\emph{Retrieval}. Recuperación en el sentido de traer los
datos, o reconstruirlos a partir de otros datos, pero no debido a una falla o
calamidad, sino a la operación normal del sistema.} ineficiente de datos}.
Muchos sistemas de control de revisiones almacenan los contenidos de una
revisión como una serie incremental de modificaciones a una ``instantánea''.
Para reconstruir una versión cualquiera, primero usted debe leer la instantánea,
y luego cada una de las revisiones entre la instantánea y su versión objetivo.
Entre más largo sea el historial de un fichero, más revisiones deben ser leídas,
y por tanto toma más tiempo reconstruir una versión particular.

\begin{figure}[ht]
  \centering
  \grafix{snapshot}
  \caption{Instantánea de un revlog, con deltas incrementales}
  \label{fig:concepts:snapshot}
\end{figure}

La innovación que aplica Mercurial a este problema es simple pero efectiva.
Una vez la cantidad de información de deltas acumulada desde la última
instantánea excede un umbral fijado de antemano, se almacena una nueva
instantánea (comprimida, por supuesto), en lugar de otro delta. Esto hace
posible reconstruir \emph{cualquier} versión de un fichero rápidamente. Este
enfoque funciona tan bien que desde entonces ha sido copiado por otros sistemas
de control de revisiones.

La figura~\ref{fig:concepts:snapshot} ilustra la idea. En una entrada en el
fichero índice de un revlog, Mercurial almacena el rango de entradas (deltas)
del fichero de datos que se deben leer para reconstruir una revisión en
particular.

\subsubsection{Nota al margen: la influencia de la compresión de vídeo}

Si le es familiar la compresión de vídeo, o ha mirado alguna vez una emisión de
TV a través de cable digital o un servicio de satélite, puede que sepa que la 
mayor parte de los esquemas de compresión de vídeo almacenan cada cuadro del
mismo como un delta contra el cuadro predecesor. Adicionalmente, estos esquemas
usan técnicas de compresión ``con pérdida'' para aumentar la tasa de
compresión, por lo que los errores visuales se acumulan a lo largo de una
cantidad de deltas inter-cuadros.

Ya que existe la posibilidad de que un flujo de vídeo se ``pierda''
ocasionalmente debido a fallas en la señal, y para limitar la acumulación de
errores introducida por la compresión con pérdidas, los codificadores de vídeo
insertan periódicamente un cuadro completo (también llamado ``cuadro clave'') en
el flujo de vídeo; el siguiente delta es generado con respecto a dicho cuadro.
Esto quiere decir que si la señal de vídeo se interrumpe, se reanudará una vez
se reciba el siguiente cuadro clave. Además, la acumulación de errores de
codificación se reinicia con cada cuadro clave.

\subsection{Identificación e integridad fuerte}

Además de la información de deltas e instantáneas, una entrada en un
% TODO de pronto aclarar qué diablos es un hash?
revlog contiene un hash criptográfico de los datos que representa.
Esto hace difícil falsificar el contenido de una revisión, y hace
fácil detectar una corrupción accidental.

Los hashes proveen más que una simple revisión de corrupción: son
usados como los identificadores para las revisiones. 
% TODO no entendí completamente la frase a continuación
Los hashes de
identificación de conjuntos de cambios que usted ve como usuario final
son de las revisiones de la bitácora de cambios. Aunque los ficheros
de registro y el manifiesto también usan hashes, Mercurial sólo los
usa tras bambalinas.

Mercurial verifica que los hashes sean correctos cuando recupera
revisiones de ficheros y cuando jala cambios desde otro repositorio.
Si se encuentra un problema de integridad, Mercurial se quejará y
detendrá cualquier operación que esté haciendo.

Además del efecto que tiene en la eficiencia en la recuperación, el
uso periódico de instantáneas de Mercurial lo hace más robusto frente
a la corrupción parcial de datos. Si un fichero de registro se
corrompe parcialmente debido a un error de hardware o del sistema, a
menudo es posible reconstruir algunas o la mayoría de las revisiones a
partir de las secciones no corrompidas del fichero de registro, tanto
antes como después de la sección corrompida. Esto no sería posible con
un sistema de almacenamiento basado únicamente en deltas.

\section{Historial de revisiones, ramas y fusiones}

Cada entrada en el revlog de Mercurial conoce la identidad de la
revisión de su ancestro inmediato, al que se conoce usualmente como su
\emph{padre}. De hecho, una revisión contiene sitio no sólo para un
padre, sino para dos. Mercurial usa un hash especial, llamado el
``ID nulo'', para representar la idea de ``no hay padre aquí''. Este
hash es simplemente una cadena de ceros.

En la figura~\ref{fig:concepts:revlog} usted puede ver un ejemplo de
la estructura conceptual de un revlog. Los ficheros de registro,
manifiestos, y bitácoras de cambios comparten la misma estructura;
sólo difieren en el tipo de datos almacenados en cada delta o
instantánea.

La primera revisión en un revlog (al final de la imagen) tiene como
padre al ID nulo, en las dos ranuras disponibles para padres. En una
revisión normal, la primera ranura para padres contiene el ID de la
revisión padre, y la segunda contiene el ID nulo, señalando así que la
revisión sólo tiene un padre real. Un par de revisiones que tenga el
mismo ID padre son ramas. Una revisión que representa una fusión entre
ramas tiene dos IDs de revisión normales en sus ranuras para padres.

\begin{figure}[ht]
  \centering
  \grafix{revlog}
  \caption{}
  \label{fig:concepts:revlog}
\end{figure}

\section{El directorio de trabajo}

% TODO revisar párrafo, no me convence la traducción
En el directorio de trabajo, Mercurial almacena una instantánea de los
ficheros  del repositorio como si fueran los de un conjunto de cambios
particular.

El directorio de trabajo ``sabe'' qué conjunto de cambios contiene.
Cuando usted actualiza el directorio de trabajo para que contenga un
conjunto de cambios particular, Mercurial busca la revisión adecuada
del manifiesto para averiguar qué ficheros estaba monitoreando cuando
se hizo la consignación del conjunto de cambios, y qué revisión de
cada fichero era la actual en ese momento. Luego de eso, recrea una
copia de cada uno de esos ficheros, con los mismos contenidos que
tenían cuando fue consignado el conjunto de cambios.

El \emph{estado de directorio}\ndt{dirstate, en inglés en el
original.} contiene el conocimiento de Mercurial acerca del directorio
de trabajo. Allí se detalla a qué conjunto de cambios es actualizado
el directorio de trabajo, y todos los ficheros que Mercurial está
monitoreando en este directorio.

Tal como la revisión de un revlog tiene espacio para dos padres, para
que pueda representar tanto una revisión normal (con un solo padre) o
una fusión de dos revisiones anteriores, el estado de directorio tiene
espacio para dos padres. Cuando usted usa el comando \hgcmd{update},
el conjunto de cambios al que usted se actualiza es almacenado en la
casilla destinada al ``primer padre'', y un ID nulo es almacenado en
la segunda. Cuando usted hace una fusión (\hgcmd{merge}) con otro
conjunto de cambios, la casilla para el primer padre permanece sin
cambios, y la casilla para el segundo es actualizada con el conjunto
de cambios con el que usted acaba de hacer la fusión. El comando
\hgcmd{parents} le indica cuáles son los padres del estado de
directorio.

\subsection{Qué pasa en una consignación}

El estado de directorio almacena información sobre los padres para
algo más que mero registro. Mercurial usa los padres del estado de
directorio como \emph{los padres de un nuevo conjunto de cambios}
cuando usted hace una consignación.

\begin{figure}[ht]
  \centering
  \grafix{wdir}
  \caption{El directorio de trabajo puede tener dos padres}
  \label{fig:concepts:wdir}
\end{figure}

La figura~\ref{fig:concepts:wdir} muestra el estado normal del
directorio de trabajo, que tiene un único conjunto de cambios como
padre. Dicho conjunto de cambios es la \emph{punta}, el conjunto de
cambios más reciente en el repositorio que no tiene hijos.

\begin{figure}[ht]
  \centering
  \grafix{wdir-after-commit}
  \caption{El directorio de trabajo obtiene nuevos padres luego de una
  consignación}
  \label{fig:concepts:wdir-after-commit}
\end{figure}

Es útil pensar en el directorio de trabajo como en ``el conjunto de
cambios que estoy a punto de enviar''. Cualquier fichero que usted le
diga a Mercurial que fue añadido, borrado, renombrado o copiado, se
verá reflejado en ese conjunto de cambios, como también se verán las
modificaciones a cualquiera de los ficheros que Mercurial ya esté
monitoreando; el nuevo conjunto de cambios dentrá los padres del
directorio de trabajo como propios.

Luego de una consignación, Mercurial actualizará los padres del
directorio de trabajo, de tal manera que el primer padre sea el ID del
nuevo conjunto de cambios, y el segundo sea el ID nulo. Esto puede
verse en la figura~\ref{fig:concepts:wdir-after-commit}.  Mercurial no
toca ninguno de los ficheros del directorio de trabajo cuando usted
hace la consignación; sólo modifica el estado de directorio para
anotar sus nuevos padres.

\subsection{Creación de un nuevo frente}

Es perfectamente normal actualizar el directorio de trabajo a un
conjunto de cambios diferente a la punta actual. Por ejemplo, usted
podría desear saber en qué estado se encontraba su proyecto el martes
pasado, o podría estar buscando en todos los conjuntos de cambios para
saber cuándo se introdujo un fallo. En casos como éstos, la acción
natural es actualizar el directorio de trabajo al conjunto de cambios
de su interés, y examinar directamente los ficheros en el directorio
de trabajo para ver sus contenidos tal como estaban en el momento de
hacer la consignación. El efecto que tiene esto se muestra en la
figura~\ref{fig:concepts:wdir-pre-branch}.

\begin{figure}[ht]
  \centering
  \grafix{wdir-pre-branch}
  \caption{El directorio de trabajo, actualizado a un conjunto de
  cambios anterior}
  \label{fig:concepts:wdir-pre-branch}
\end{figure}

Una vez se ha actualizado el directorio de trabajo a un conjunto de
cambios anterior, qué pasa si se hacen cambios, y luego se hace una
consignación? Mercurial se comporta en la misma forma que describí
anteriormente. Los padres del directorio de trabajo se convierten en
los padres del nuevo conjunto de cambios. Este nuevo conjunto de
cambios no tiene hijos, así que se convierte en la nueva punta. Y el
repositorio tiene ahora dos conjuntos de cambios que no tienen hijos;
a éstos los llamamos \emph{frentes}. Usted puede apreciar la
estructura que esto crea en la figura~\ref{fig:concepts:wdir-branch}.

\begin{figure}[ht]
  \centering
  \grafix{wdir-branch}
  \caption{Después de una consignación hecha mientras se usaba un
  conjunto de cambios anterior}
  \label{fig:concepts:wdir-branch}
\end{figure}

\begin{note}
    Si usted es nuevo en Mercurial, debería tener en mente un
    ``error'' común, que es usar el comando \hgcmd{pull} sin ninguna
    opción. Por defecto, el comando \hgcmd{pull} \emph{no} actualiza
    el directorio de trabajo, así que usted termina trayendo nuevos
    conjuntos de cambios a su repositorio, pero el directorio de
    trabajo sigue usando el mismo conjunto de cambios que tenía antes
    de jalar. Si usted hace algunos cambios, y luego hace una
    consignación, estará creando un nuevo frente, porque su directorio
    de trabajo no es sincronizado a cualquiera que sea la nueva punta.

    Pongo la palabra ``error'' en comillas porque todo lo que usted
    debe hacer para rectificar la situación es hacer una fusión
    (\hgcmd{merge}), y luego una consignación (\hgcmd{commit}). En
    otras palabras, esto casi nunca tiene consecuencias negativas;
    sólo sorprende a la gente. Discutiré otras formas de evitar este
    comportamiento, y porqué Mercurial se comporta de esta forma,
    inicialmente sorprendente, más adelante.
\end{note}

\subsection{Fusión de frentes}

Cuando usted ejecuta el comando \hgcmd{merge}, Mercurial deja el
primer padre del directorio de trabajo intacto, y escribe como segundo
padre el conjunto de cambios contra el cual usted está haciendo la
fusión, como se muestra en la figura~\ref{fig:concepts:wdir-merge}.

\begin{figure}[ht]
  \centering
  \grafix{wdir-merge}
  \caption{Fusión de dos frentes}
  \label{fig:concepts:wdir-merge}
\end{figure}

Mercurial también debe modificar el directorio de trabajo, para
fusionar los ficheros que él monitorea en los dos conjuntos de
cambios. Con algunas simplificaciones, el proceso es el siguiente, por
cada fichero en los manifiestos de ambos conjuntos de cambios.
\begin{itemize}
\item Si ningún conjunto de cambios ha modificado un fichero, no se
    hace nada con el mismo.
\item Si un conjunto de cambios ha modificado un fichero, y el otro no
    lo ha hecho, se crea una copia del fichero con las modificaciones
    pertinentes en el directorio de trabajo.
\item Si un conjunto de cambios borra un fichero, y el otro no lo ha
    hecho (o también lo borró), se borra dicho fichero del directorio
    de trabajo.
\item Si un conjunto de cambios ha borrado un fichero, pero el otro lo ha
    modificado, se le pregunta al usuario qué hacer: conservar el
    fichero modificado, o borrarlo?
\item Si ambos conjuntos de cambios han modificado un fichero, se
    invoca el programa externo de fusión para definir el nuevo
    contenido del fichero fusionado. Esto puede requerir interacción
    directa de parte del usuario.
\item Si un conjunto de cambios ha modificado un fichero, y el otro ha
    renombrado o copiado el mismo, asegurarse de que los cambios sigan
    al nuevo nombre de fichero.
\end{itemize}
Hay más detalles---hacer una fusión tiene una gran cantidad de casos
especiales---pero éstas son las elecciones más comunes que se ven
involucradas en una fusión. Como usted puede ver, muchos de los casos
son completamente automáticos, y de hecho la mayoría de las fusiones
terminan automáticamente, sin requerir la interacción del usuario para
resolver ningún conflicto.

Cuando considere qué pasa cuando usted hace una consignación después
de una fusión, de nuevo el directorio de trabajo es ``el conjunto de
cambios que estoy a punto de consignar''. Una vez termina su trabajo
el comando \hgcmd{merge}, el directorio de trabajo tiene dos padre;
éstos se convertirán en los padres del nuevo conjunto de cambios.

Mercurial le permite hacer múltiples fusiones, pero usted debe
consignar los resultados de cada fusión sucesivamente. Esto es
necesario porque Mercurial sólo monitorea dos padres, tanto para las
revisiones como para los directorios de trabajo. Aunque técnicamente
es posible fusionar varios conjuntos de trabajo en una sola operación,
la posibilidad de confundir al usuario y crear un desorden terrible en
la fusión se hace incontenible de inmediato.

\section{Otras características de diseño interesantes}

En las secciones anteriores, he tratado de resaltar algunos de los
aspectos más importantes del diseño de Mercurial, para mostrar que se
presta gran cuidado y atención a la confiabilidad y el desempeño. Sin
embargo, la atención a los detalles no para ahí. Hay una cantidad de
aspectos de la construcción de Mercurial que encuentro interesantes
personalmente. Detallaré unos cuantos de ellos aquí, aparte de los
elementos ``importantes'' de arriba, para que, si usted está
% TODO the amount of thinking => (la cantidad de) esfuerzo mental
interesado, pueda obetener una idea mejor de la cantidad de esfuerzo
mental invertido en el diseño de un sistema bien diseñado.


\subsection{Compresión ingeniosa}

Cuando es adecuado, Mercurial almacenará tanto las instantáneas como
los deltas en formato comprimido. Lo hace \emph{tratando} siempre de
comprimir una instantánea o delta, y conservando la versión comprimida
sólo si es más pequeña que la versión sin compresión.

Esto implica que Mercurial hace ``lo correcto'' cuando almacena un
fichero cuyo formato original está comprimido, como un fichero
\texttt{zip} o una imagen JPEG.  Cuando estos tipos de ficheros son
comprimidos por segunda vez, el fichero resultante usualmente es más
grande que la versión comprimida una sola vez, por lo que Mercurial
almacenará el fichero \texttt{zip} o JPEG original.

Los deltas entre revisiones de un fichero comprimido usualmente son
más grandes que las instantáneas del mismo fichero, y Mercurial de
nuevo hace ``lo correcto'' en estos casos. Él encuentra que dicho
delta excede el umbral respecto al cual se debería almacenar una
instantánea completa del fichero, así que almacena la instantánea,
ahorrando espacio de nuevo respecto al enfoque simplista de usar
únicamente deltas.

\subsubsection{Recompresión de red}

Cuando almacena las revisiones en disco, Mercurial usa el algoritmo de
compresión ``deflación'' (el mismo usado en el popular formato de
fichero \texttt{zip}), que provee una buena velocidad con una tasa de
compresión respetable. Sin embargo, cuando se transmiten datos de
revisiones a través de una conexión de red, Mercurial descomprime los
datos comprimidos de las revisiones.

Si la conexión es hecha a través de HTTP, Mercurial recomprime el
flujo completo de datos usando un algoritmo de compresión que brinda
una mejor tasa de compresión (el algoritmo Burrows-Wheeler del
ampliamente usado paquete de compresión \texttt{bzip2}). Esta
combinación de algoritmo y compresión del flujo completo de datos
(en vez de una revisión a la vez) reduce sustancialmente la cantidad
de bytes a transferir, brindando así un mejor desempeño de red sobre
casi todo tipo de redes.

(Si la conexión se hace sobre \command{ssh}, Mercurial \emph{no}
recomprmime el flujo, porque \command{ssh} puede hacer esto por sí
mismo.)

\subsection{Reordenado de lectura/escritura y atomicidad}

Añadir datos al final de un fichero no es todo lo que hace falta para
garantizar que un lector no verá una escritura parcial. Si recuerda la
figura~\ref{fig:concepts:metadata}, las revisiones en la bitácora de
cambios apuntan a revisiones en el manifiesto, y las revisiones en el
manifiesto apuntan a revisiones en ficheros de registro. Esta
jerarquía es deliberada.

Un escritor inicia una transacción al escribir los datos del ficheros
del fichero de registro y el manifiesto, y no escribe nada en la
bitácora de cambios hasta que dichas escrituras hayan terminado. Un
lector empieza leyendo datos de la bitácora de cambios, luego del
manifiesto, y finalmente del fichero de registro.

%TODO revisar párrafo completo, no me gusta el resultado
Como el escritor siempre termina de escribir los datos en el fichero
de registro y en el manifiesto antes de escribir a la bitácora de
cambios, un lector nunca verá un apuntador a una versión parcialmente
escrita de revisiones del manifiesto desde la bitácora de cambios, y
nunca leerá un apuntador a una revisión parcialmente escrita del
fichero de registro desde el manifiesto.

\subsection{Acceso concurrente}

El reordenado de lectura/escritura y la atomicidad garantizan que
Mercurial nunca necesita \emph{bloquear} un repositorio cuando está
leyendo datos, aún si se está escribiendo al repositorio mientras se
hace la lectura. Esto tiene un gran efecto en la escalabilidad; usted
puede tener cualquier cantidad de procesos Mercurial leyendo datos de
un repositorio de manera segura al mismo tiempo, sin importar si se
está escribiendo al mismo o no.

La naturaleza carente de bloqueos de la lectura significa que si usted
está compartiendo un repositorio en un sistema multiusuario, no
necesita dar a los usuarios locales permisos de \emph{escritura} a su
repositorio para que ellos puedan clonarlo o jalar cambios; sólo
necesitan permisos de \emph{lectura}. (Esta \emph{no} es una
%TODO signo de admiración de apertura
característica común entre los sistemas de control de revisiones, así
que no la dé por hecha! Muchos de ellos requieren que los lectores
sean capaces de bloquear el repositorio antes de poder leerlo, y esto
requiere acceso de escritura en al menos un directorio, lo que por
supuesto se convierte en una fuente de todo tipo de problemas
administrativos y de seguridad bastante molestos.)

Mercurial usar bloqueos para asegurarse de que sólo un proceso pueda
escribir a un repositorio al mismo tiempo (el mecanismo de bloqueo es
seguro incluso sobre sistemas de ficheros notoriamente hostiles al
bloqueo, como NFS). Si un repositorio está bloqueado, los escritores
esperarán un buen rato para revisar si el repositorio ya ha sido
desbloqueado, pero si el repositorio sique bloqueado por mucho tiempo,
el proceso que intenta escribir fallará por tiempo de espera máximo.
Esto significa que sus guiones automáticos diarios no se quedarán
esperando para siempre, apilándose si el sistema se cayó sin que nadie
se diera cuenta, por ejemplo. (Sí, el tiempo de espera máximo es
configurable, de cero a infinito).


\subsubsection{Acceso seguro al estado de directorio}

Al igual que con los datos de revisión, Mercurial no requiere un
bloqueo para leer el fichero de estado de directorio; sí se usa un
bloqueo para escribir a él. Para evitar la posibilidad de leer una
copia parcialmente escrita del fichero de estado de directorio,
Mercurial escribe a un fichero con un nombre único en el mismo
directorio del fichero de estado de directorio, y luego renombra
atómicamente este fichero temporal a \filename{dirstate}\ndt{Estado de
directorio.}. Así se garantiza que el fichero llamado
\filename{dirstate} esté completo, y no parcialmente escrito.

\subsection{Evitar movimientos de brazo}

Un aspecto crítico para el desempeño de Mercurial es evitar los
movimientos del brazo de lectura del disco duro, ya que cualquier
movimiento de brazo es mucho más costoso que incluso una operación de
lectura relativamente grande.

Es por esto que, por ejemplo, el estado de directorio es almacenado
como un solo fichero. Si hubiera un estado de directorio por cada
directorio que Mercurial monitorea, el disco haría un movimiento de
brazo por cada directorio. En cambio, Mercurial lee el estado de
directorio completo en un solo paso.

Mercurial también usa un esquema de ``copiar al escribir'' cuando
clona un repositorio en un mismo medio de almacenamiento local. En vez
de copiar cada fichero de revlog del repositorio viejo al nuevo, se
crea un ``enlace duro'', que es una manera sucinta de decir
``estos dos nombres apuntan al mismo fichero''. Cuando Mercurial está
a punto de escribir a uno de los ficheros de revlog, revisa si la
cantidad de nombres apuntando al fichero es de más de uno. Si lo es,
más de un repositorio está usando el fichero, así que Mercurial hace
una nueva copia del fichero, privada para este repositorio.

Algunos desarrolladores de control de revisiones han indicado que la
idea de hacer una copia privada completa de un fichero no es eficiente
desde el punto de vista de almacenamiento. Aunque esto es cierto, el
almacenamiento es barato, y este método brinda el máximo rendimiento
al tiempo que delega la mayor parte del trabajo de manejo de ficheros
al sistema operativo. Un esquema alternativo seguramente reduciría el
desempeño y aumentaría la complejidad del software, cada uno de los
cuales es mucho más importante para la ``sensación'' que se tiene del
software en el trabajo día a día.

\subsection{Otros contenidos del estado de directorio}

Debido a que Mercurial no lo fuerza a indicar si usted está
modificando un fichero, se usa el estado de directorio para almacenar
información extra para poder determinar efecientemente si usted ha
modificado un fichero. Por cada fichero en el directorio de trabajo,
se almacena el momento en que Mercurial modificó por última vez el
fichero, y el tamaño del fichero en ese momento.

cuando usted añade (\hgcmd{add}), remueve (\hgcmd{remove}), renombra
(\hgcmd{rename}) o copia (\hgcmd{copy}) ficheros, Mercurial actualiza
el estado de directorio para saber qué hacer con dichos ficheros
cuando usted haga la consignación.

Cuando Mercurial está revisando el estado de los ficheros en el
directorio de trabajo, revisa primero la fecha de modificación del
fichero. Si no ha cambiado, el fichero no ha sido modificado. Si el
tamaño del fichero ha cambiado, el fichero ha sido modificado. Sólo en
el caso en que el tiempo de modificación ha cambiado, pero el tamaño
no, es necesario leer el contenido del fichero para revisar si ha
cambiado. Almacenar estos pocos datos reduce dramáticamente la
cantidad de datos que Mercurial debe leer, lo que brinda una mejora en
el rendimiento grande, comparado con otros sistemas de control de
revisiones.

%%% Local Variables: 
%%% mode: latex
%%% TeX-master: "00book"
%%% End:
