\chapter{Referencia de Órdenes}
\label{cmdref}

\cmdref{add}{Añade ficheros en la próxima consignación}
\optref{add}{I}{include}
\optref{add}{X}{exclude}
\optref{add}{n}{dry-run}

\cmdref{diff}{imprime los cambios en la historia o el directorio actual}

Mostrar las diferencias entre revisiones para ficheros especificados o
directorios, con el formato unificado diff.  Si desea ver una
descripción del formato unificado diff, ver la sección~\ref{sec:mq:patch}.

De forma predeterminada, esta orden no imprime las diferencias para
los archivos binarios que Mercurial esté siguiendo.  Para controlar
este comportamiento, vea las opciones \hgopt{diff}{-a} y
\hgopt{diff}{--git}.

\subsection{Options}

\loptref{diff}{nodates}

Omite la fecha y hora cuando se muestran los encabezados de las
diferencias.

\optref{diff}{B}{ignore-blank-lines}

No imprime los cambios que solamente insertan o eliminan líneas en
blanco.  Una línea que contiene espacios en blanco no se considera
como una línea en blanco.

\optref{diff}{I}{include}

Incluye ficheros y directorios cuyos nombres coinciden con los
patrones elegidos.

\optref{diff}{X}{exclude}

Excluye los ficheros y directorios cuyos nombres coinciden con los
patrones elegidos.

\optref{diff}{a}{text}

Si no especifica esta opción, \hgcmd{diff} no mostrará las diferencias
de los ficheros que detecte como binarios.  Al especificar \hgopt{diff}{-a}
se forza a \hgcmd{diff} a tratar los ficheros como texto, y generar
diferencias para todos.

Esta opción es útil para los ficherso que son ``texto en mayor
medida'' pero que tienen caracteres NUL.  Si lo usa en ficheros que
contienen muchos datos binarios, la salida será incomprensible.

\optref{diff}{b}{ignore-space-change}

No imprime si el único cambio que en la línea es la cantidad de
espacio en blanco.

\optref{diff}{g}{git}

Mostrar diferencias compatibles con \command{git}.  XXX reference a format
description.

\optref{diff}{p}{show-function}

Mostrar el nombre de la función que contiene el código en una porción
del encabzado usando una heurística simple.  Esta funcionalidad se
habilita de forma predeterminada, así que la opción \hgopt{diff}{-p}
no tiene efectos a menos que cambie el valor de
\rcitem{diff}{showfunc} en la configuración, como en el ejemplo
siguiente.
\interaction{cmdref.diff-p}

\optref{diff}{r}{rev}

Especifique una o más revisiones para comparar.  La orden \hgcmd{diff}
acepta hasta dos opciones \hgopt{diff}{-r} para especificar las
revisiones a comparar.

\begin{enumerate}
\setcounter{enumi}{0}
\item Despliega las diferencias entre la revisión padre y del directorio
  de trabajo.
\item Despliega las diferencias entre el conjunto de cambios
  especificados y el directorio de trabajo.
\item Despliega las diferencias entre dos conjuntos de cambios
  especificados.
\end{enumerate}

Puede especificar dos revisiones usando o bien sea las opciones
\hgopt{diff}{-r} o la notación de rango.  Por ejemplo, las dos
especificaciones de revisiones a continuación son equivalentes:
\begin{codesample2}
  hg diff -r 10 -r 20
  hg diff -r10:20
\end{codesample2}

Cuando especifica dos revisiones, esto tiene significado para
Mercurial.  Esto significa que \hgcmdargs{diff}{-r10:20} producirá un
diff que transformará los ficheros desde los contenidos en la revisión
10 a los contenidos de la revisión 20, mientras que
\hgcmdargs{diff}{-r20:10} significa lo opuesto:  el diff que
transformaría los contenidos de los ficheros de la revisión 20 a los
contenidos de la revisión 10.  No puede invertir el orden de esta
forma si está haciendo un diff frente al directorio de trabajo.

\optref{diff}{w}{ignore-all-space}

\cmdref{version}{imprime la información de versión y derechos de reproducción}

Esta orden despliega la versión de Mercurial que está usando, y su
nota de derechos de reproducción.  Hay cuatro clases de cadenas de
versión posibles:
\begin{itemize}
\item La cadena ``\texttt{unknown}''. Esta versión de Mercurial no fue
  construida en un repositorio de Mercurial, y no puede determinar su
  propia versión.
\item Una cadena numérica corta, tal como ``\texttt{1.1}''. Esta es
  una construcción de una versión de Mercurial que se identifica con
  una etiqueta específica en el repositorio en el cual fue
  armada(Esto no significa necesariamente que está ejecutando una
  versión oficial; alguien pudo haber añadido tal etiqueta a cualquier
  versión del repositorio en el cual armaron Mercurial).
\item Una cadena hexadecimal, tal como ``\texttt{875489e31abe}''.
  Esta es una construcción de una revisión dada de Mercurial.
\item Una cadena hexadecimal seguida por una fecha, tal como
  ``\texttt{875489e31abe+20070205}''.  Esta construcción de la
  revisión de Mercurial fue la construcción de un repositorio que tuvo
  cambios locales que no han sido consignados.
\end{itemize}

\subsection{Consejos y trucos}

\subsubsection{¿Por qué difieren los resultados de \hgcmd{diff} y
  \hgcmd{status}?}
\label{cmdref:diff-vs-status}

Cuando ejecuta la orden \hgcmd{status}, verá una lista de ficheros
para los cuales Mercurial almacenará cambios la próxima vez que
consigne.  Si ejecuta la orden \hgcmd{diff}, verá que imprime
diferencias solamente para un \emph{subconjunto} de los ficheros que
\hgcmd{status} liste.  Hay dos posibles razones para este comportamiento:

La primera es que \hgcmd{status} imprime cierta clase de
modificaciones que \hgcmd{diff} no despliega normalmente.  La orden
\hgcmd{diff} usualmente despliega diferencias unificadas, las cuales
no tienen la habilidad de representar algunos cambios que Mercurial
puede seguir.  Lo más notable es que las diferencias tradicionales no
pueden representar un cambio acerca de la ejecutabilidad de un
fichero, pero Mercurial sí almacena esta información.

Si usa la opción \hgopt{diff}{--git} de \hgcmd{diff}, mostrará
diferencias compatibles con \command{git} que \emph{pueden} desplegar
esta información adicional.

La segunda razón posible para que \hgcmd{diff} esté imprimiendo
diferencias para un subconjunto de ficheros de lo que muestra
\hgcmd{status} es que si usted le invoca sin argumento alguno,
\hgcmd{diff} imprime diferencias frente al primer padre del directorio
de trabajo.  Si ha ejecutado \hgcmd{merge} para fusionar dos conjuntos
de cambios, pero no ha consignado aún los resultados de la fusión,  su
directorio de trabajo tiene dos padres(use \hgcmd{parents} para
verlos).  Mientras que \hgcmd{status} imprime modificaciones relativas
a \emph{ambos} padres después de una fusión que no se ha consignado,
\hgcmd{diff} opera aún relativo solamente al primer padre.  Puede
lograr que imprima las diferencias relativas al segundo padre
especificando tal padre con la opción \hgopt{diff}{-r}.  No hay forma
de hacer que imprima las diferencias relativas a los dos padres.

\subsubsection{Generar diferencias seguras en binarios}

Si usa la opción \hgopt{diff}{-a} para forzar que Mercurial imprima
las diferencias de los ficheros que so o bien ``casi completamente
texto'' o contienen muchos datos binarios, tales diferencias no pueden
aplicarse subsecuentemente a la orden \hgcmd{import} de Mercurial o a
la orden \command{patch} del sistema.

Si desea generar una diferencia de un fichero binario que es seguro
para usarlo como entrada a la orden \hgcmd{import}, use la opción
\hgcmd{diff}{--git} cuando genere el parche.  La orden \command{patch}
del sistema no puede tratar con parches binarios.

%%% Local Variables: 
%%% mode: latex
%%% TeX-master: "00book"
%%% End: 
