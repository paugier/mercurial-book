\chapter{Licencia de Publicación Abierta}
\label{cha:opl}

Versión 1.0, 8 Junio de 1999

\section{Requerimientos en versiones modificadas y no modificadas}

Los trabajos bajo Publicación Abierta pueden reproducirse y
distribuirse enteros o en porciones, en cualquier medio físico o
electrónico, siempre y cuando se respeten los términos de esta 
licencia, y se incorpore esta licencia o su referencia (con cualquiera
de las opciones elegidas por el autor y el editor) en la reproducción.

A continuación mostramos la forma correcta de incorporar por referencia:

\begin{quote}
  Copyright (c) \emph{año} por \emph{nombre del autor o designado}.
  Este material puede distribuirse solamente bajo los términos y
  condiciones especificados por la Licencia de Publicación Abierta,
  v\emph{x.y} o  posterior (la última versión disponible está en
  \url{http://www.opencontent.org/openpub/}).
\end{quote}

La referencia debe estar seguida inmediatamente por cualquier opción
elegida por el(os) autor(es) y/o editor(es) del documento (consulte la
sección~\ref{sec:opl:options}).

Se permite la redistribución comercial de los materiales sujetos a la
Publicación Abierta.

Cualquier publicación en forma estándar de libro (papel) requerirá
citar al editor y autor original.  Los nombres del editor y el autor
aparecerán en todas las superficies externas del libro.  En todas las
superficies externas el nombre del editor deberá aparecer en tamaño de
la misma medida que el título del trabajo y será citado como poseedor
con respecto al título.

\section{Derechos de reproducción}

El derecho de reproducción de cada Publicación Abierta pertenece
al(os) autor(es) o designados.

\section{Alcance de la licencia}

Los términos de licencia dsecritos aplican a todos los trabajos bajo
licencia de publicación abierta a menos que se indique de otra forma
en este documento.

La simple agregación de trabajos de Publicación Abierta o una porción
de trabajos de Publicación Abierta con otros trabajos o programas en
el mismo medio no causarán que esta licencia se aplique a los otros
trabajos.  Los agregados deberán contener una nota que especifique la
inclusión de matrial de Publicación Abierta y una nota de derechos de
reproducción acorde.

\textbf{Separabilidad}. Si cualquier porción de esta licencia no es
aplicable en alguna jurisdicción, las porciones restantes se
mantienen.

\textbf{Sin garantía}.  Los trabajos de Publicación Abierta se
licencian y ofrecen ``como están'' sin garantía de ninguna clase,
expresa o implícita, incluyendo, pero no limitados a las garantías de
mercabilidad y adaptabilidad para un propósito particular o garantía
de no infracción.

\section{Requerimientos sobre trabajos modificados}

Todas las versiones modificadas de documentos cubiertos por esta
licencia, incluyendo traducciones, antologías, compilaciones y
documentos parciales, deben seguir estos requerimientos:

\begin{enumerate}
\item La versión modificada debe estar etiquetada como tal.
\item La persona que hace la modificación debe estar identificada y la
  modificación con fecha.
\item El dar crédito al autor original y al editor si se requiere de
  acuerdo a las prácticas académicas de citas.
\item Debe identificarse el lugar del documento original sin
  modificación.
\item No puede usarse el(os) nombre(s) del autor (de los autores) para
  implicar relación alguna con el documento resultante sin el permiso
  explícito del autor (o de los autores).
\end{enumerate}

\section{Recomendaciones de buenas prácticas}

Adicional a los requerimientos de esta licencia, se solicita a los
redistribuidores y se recomienda en gran medida que:

\begin{enumerate}
\item Si está distribuyendo trabajaos de Publicación Abierta en copia
  dura o CD-ROM, envíe una notificación por correo a los autores
  acerca de su intención de redistribuir por lo menos con treinta días
  antes de que su manuscrito o el medio se congelen, para permitir a
  los autores tiempo para proveer documentos actualizados.  Esta
  notificación debería describir las modificaciones, en caso de que
  haya, al documento.
\item Todas las modificaciones sustanciales (incluyendo eliminaciones)
  deben estar marcadas claramente en el documento o si no descritas en
  un adjunto del documento.
\item Finalmente, aunque no es obligatorio bajo esta licencia, se
  considera de buenos modales enviar una copia gratis de cualquier
  expresión en copia dura o CD-ROM de un trabajo licenciado con
  Publicación Abierta a el(os) autor(es).
\end{enumerate}

\section{Opciones de licencia}
\label{sec:opl:options}

El(os) autor(es) y/o editor de un documento licenciado con Publicación
Abierta pueden elegir ciertas opciones añadiendo información a la
referencia o a la copia de la licencia.  Estas opciones se consideran
parte de la instancia de la licencia y deben incluirse con la
licencia (o su incorporación con referencia) en trabajos derivados.

\begin{enumerate}[A]
\item Prohibir la distribución de versiones substancialmente
  modificadas sin el permiso explícito del(os) autor(es).  Se definen
  ``modificaciones substanciales'' como cambios en el contenido
  semántico del documento, y se excluyen simples cambios en el formato
  o correcciones tipográficas.

  Para lograr esto, añada la frase ``Se prohibe la distribución de
  versiones substancialmente modificadas de este documento sin el
  permiso explícito del dueño de los derechos de reproducción.'' a la
  referencia de la licencia o a la copia.

\item Está prohibido prohibir cualquier publicación de este trabajo o
  derivados como un todo o una parte en libros estándar (de papel) con
  propósitos comerciales a menos que se obtenga un permiso previo del
  dueño de los derechos de reproducción.

  Para lograrlo, añada la frase ``La distribución del trabajo o
  derivados en cualquier libro estándar (papel) se prohibe a menos que
  se obtenga un permiso previo del dueño de los derechos de
  reproducción.'' a la referencia de la licencia o la copia.
\end{enumerate}

%%% Local Variables: 
%%% mode: latex
%%% TeX-master: "00book"
%%% End: 
