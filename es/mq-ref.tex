\chapter{Referencia de las Colas de Mercurial}
\label{chap:mqref}

\section{Referencia de órdenes MQ}
\label{sec:mqref:cmdref}

Si desea dar un vistazo a las órdenes que ofrece MQ, use la orden
\hgcmdargs{help}{mq}.

\subsection{\hgxcmd{mq}{qapplied}---imprimir los parches aplicados}

La orden \hgxcmd{mq}{qapplied} imprime la pila actual de parches
aplicados.  Los parches se imprimen en orden de antigüedad, primero
los más antiguos y después los más recientes, por lo tanto el último
parche de la lista es el que está en el ``tope''.

\subsection{\hgxcmd{mq}{qcommit}---consignar cambios en la cola del repositorio}

La orden \hgxcmd{mq}{qcommit} consigna cualquier cambio sobresaliente
en el repositorio \sdirname{.hg/patches}.  Esta orden solamente
funciona si el directorio \sdirname{.hg/patches} es un repositorio,
p.e.~usted creó el directorio con
\hgcmdargs{qinit}{\hgxopt{mq}{qinit}{-c}} o ejecutó
\hgcmd{init} en el directorio después de correr \hgxcmd{mq}{qinit}.

Esta orden es un atajo para \hgcmdargs{commit}{--cwd .hg/patches}.

\subsection{\hgxcmd{mq}{qdelete}---eliminar un parche del fichero
  \sfilename{series}}

La orden \hgxcmd{mq}{qdelete} elimina la entrada del fichero
\sfilename{series} para el parche en el directorio
\sdirname{.hg/patches}.  No sca el parche si ha sido aplicado.  De
forma predeterminada no borra el fichero del parche; use la opción
\hgxopt{mq}{qdel}{-f} para hacerlo.

Opciones:
\begin{itemize}
\item[\hgxopt{mq}{qdel}{-f}] Elimina el fichero del parche.
\end{itemize}

\subsection{\hgxcmd{mq}{qdiff}---imprimir la diferencia del último
  parche aplicado}

La orden \hgxcmd{mq}{qdiff} imprime un diff del parche más
recientemente aplicado.  Es equivalente a \hgcmdargs{diff}{-r-2:-1}.

\subsection{\hgxcmd{mq}{qfold}---fusionar (``integrar'') varios parches en
  uno solo}

La orden \hgxcmd{mq}{qfold} fusiona muchos parches en el último parche
aplicado, de tal forma que el último parche aplicado es la unión de
todos los cambios de los parches en cuestión.

Los parches a fusionar no deben haber sido aplicados;
\hgxcmd{mq}{qfold} saldrá indicando un error si alguno ha sido
aplicado.  El orden en el cual los parches se pliegan es
significativo; \hgcmdargs{qfold}{a b} significa ``aplique el parche
más reciente, seguido de \texttt{a}, y seguido de \texttt{b}''.

Los comentarios de los parches integrados se colocan al final de los
comentarios del parche destino, con cada bloque de comentarios
separado con tres asteriscos(``\texttt{*}'').  Se usa la opción
\hgxopt{mq}{qfold}{-e} para editar el mensaje de consignación para el
conjunto de cambios/parches después de completarse el pliegue.

Opciones:
\begin{itemize}
\item[\hgxopt{mq}{qfold}{-e}] Edita el mensaje de consignación y la
  descripción del parche del parche que se ha integrado.
\item[\hgxopt{mq}{qfold}{-l}] Usa los contenidos del fichero dado como
  el nuevo mensaje de consignación y descripción del parche para el
  parche a integrar.
\item[\hgxopt{mq}{qfold}{-m}] Usa el texto dado como el mensaje de
  consignación y descripción del parche para el parche integrado.
\end{itemize}

\subsection{\hgxcmd{mq}{qheader}---desplegar el encabezado/descripción
  de un parche}

La orden \hgxcmd{mq}{qheader} imprime el encabezado o descripción de
un parche.  De forma predeterminada, imprime el encabezado del último
parche aplicado. Si se da un argumento, imprime el encabezado del
parche referenciado.

\subsection{\hgxcmd{mq}{qimport}---importar el parche de un tercero en
  la cola}

La orden \hgxcmd{mq}{qimport} añade una entrada de un parche externo
al fichero \sfilename{series} y copia el parche en el directorio
\sdirname{.hg/patches}.  Añade la entrada inmediatamente después del
último parche aplicado, pero no introduce el parche.

Si el directorio \sdirname{.hg/patches} es un repositorio, 
\hgxcmd{mq}{qimport} automáticamente hace un \hgcmd{add} del parche
importado.

\subsection{\hgxcmd{mq}{qinit}---preparar un repositorio para trabajar
  con MQ}

La orden \hgxcmd{mq}{qinit} prepara un repositorio para trabajar con
MQ.  Crea un directorio llamado \sdirname{.hg/patches}.

Opciones:
\begin{itemize}
\item[\hgxopt{mq}{qinit}{-c}] Crea \sdirname{.hg/patches} como un
  repositorio por sí mismo.  También crea un fichero
  \sfilename{.hgignore} que ignorará el fichero \sfilename{status}.
\end{itemize}

Cuando el directorio \sdirname{.hg/patches} es un repositorio, las órdenes
\hgxcmd{mq}{qimport} y \hgxcmd{mq}{qnew} hacen \hgcmd{add}
automáticamente a los parches nuevos.

\subsection{\hgxcmd{mq}{qnew}---crear un parche nuevo}

La orden \hgxcmd{mq}{qnew} crea un parche nuevo.  Exige un argumento,
el nombre que se usará para tal parche.  El parche recién creado está
vacío inicialmente.   Se añade al fichero \sfilename{series} después
del último parche aplicado, y se introduce en el tope de ese parche.

Si \hgxcmd{mq}{qnew} encuentra ficheros modificados en el directorio
de trabajo, rehusará crear un parche nuevo a meos que se emplee
\hgxopt{mq}{qnew}{-f} la opción(ver más adelante).  Este
comportamiento le permite hacer \hgxcmd{mq}{qrefresh} al último parche
aplicado antes de aplicar un parche nuevo encima de este.

Opciones:
\begin{itemize}
\item[\hgxopt{mq}{qnew}{-f}] Crea un parche nuevo si los contenidos
  del directorio actual han sido modificados.  Cualquier modificación
  significativa se añade al parche recientemente creado,  de tal forma
  que al finalizar la orden, el directorio de trabajo no lucirá
  modificado.
\item[\hgxopt{mq}{qnew}{-m}] Usa el texto dado como el mensaje de
  consignación.  Este texto se almacenará al principio del fichero del
  parche, antes de los datos del parche.
\end{itemize}

\subsection{\hgxcmd{mq}{qnext}---imprimir el nombre del próximo parche}

La orden \hgxcmd{mq}{qnext} imprime el nombre del siguiente parche en
el fichero \sfilename{series} a continuación del último parche
aplicado.  Este parche sería el próximo parche a aplicar si se
ejecutara la orden \hgxcmd{mq}{qpush}.

\subsection{\hgxcmd{mq}{qpop}---sustraer parches de la pila}

La orden \hgxcmd{mq}{qpop} elimina los parches aplicados del tope de
la pila de parches aplicados.  De forma predeterminada solamente
remueve un parche.

Esta orden elimina los conjuntos de cambios que representan los
parches sustraídos del repositorio, y actualiza el directorio de
trabajo para deshacer los efectos de los parches.

Esta orden toma un argumento opcional, que usa como el nombre o el
índice del parche que desea sustraer.  Si se da el nombre, sustraerá
los parches hasta que el parche nombrado sea el último parche
aplicado.  Si se da un número, \hgxcmd{mq}{qpop} lo trata como un
índice dentro del fichero \sfilename{series},  contando desde
cero(no cuenta las líneas vacías o aquellas que sean únicamente
comentarios).  Sustrae los parches hasta que el parche identificado
por el índice sea el último parche aplicado.

La orden \hgxcmd{mq}{qpop} no lee o escribe parches en el fichero
\sfilename{series}.  \hgxcmd{mq}{qpop} se constituye por tanto en una
forma segura de sustraer un parche del fichero \sfilename{series} o un
parche que ha eliminado o renombrado completamente.  En los dos
últimos casos, use el nombre del parche tal como lo hizo cuando lo
aplicó.

De forma predeterminada, la orden \hgxcmd{mq}{qpop} no sustraerá
parche alguno si el directorio de trabajo ha sido modificado.  Puede
modificar este comportamiento con la opción \hgxopt{mq}{qpop}{-f}, que
revierte todas las modificaciones del directorio de trabajo.

Opciones:
\begin{itemize}
\item[\hgxopt{mq}{qpop}{-a}] Sustrae todos los parches aplicados.
  Restaura el repositorio al estado antes de haber aplicado parche alguno.
\item[\hgxopt{mq}{qpop}{-f}] Revertir forzadamente cualquier
  modificación del directorio de trabajo cuando se hace sustracciones.
\item[\hgxopt{mq}{qpop}{-n}] Sustraer un parche de la cola dado un nombre.
\end{itemize}

La orden \hgxcmd{mq}{qpop} elimina una línea del final del fichero
\sfilename{status} por cada parche que se sustrae.

\subsection{\hgxcmd{mq}{qprev}---imprimir el nombre del parche anterior}

La orden \hgxcmd{mq}{qprev} imprime el nombre del parche en el fichero
\sfilename{series} que está antes del último parche aplicado.   Este
se volverá el último parche aplicado si ejecuta \hgxcmd{mq}{qpop}.

\subsection{\hgxcmd{mq}{qpush}---introducir parches a la pila}
\label{sec:mqref:cmd:qpush}

La orden \hgxcmd{mq}{qpush} añade parches a la pila.  De forma
predeterminada añade solamente un parche.

Esta orden crea un conjunto de cambios que representa cada parche
aplicado y actualiza el directorio de trabajo aplicando los efectos de
los parches.

Los datos predeterminados cuando se crea un conjunto de cambios
corresponde a:
\begin{itemize}
\item La fecha de consignación y zona horaria corresponden a la hora
  actual de la zona.  Dado que tales datos se usan para computar la
  identidad de un conjunto de cambios, significa que si hace
  \hgxcmd{mq}{qpop} a un parche y \hgxcmd{mq}{qpush} de nuevo, el
  conjunto de cambios que introduzca tendrá una identidad distinta a
  la del conjunto de cambios que sustrajo.
\item El autor es el mismo que el predeterminado usado por la orden
  \hgcmd{commit}.
\item El mensaje de consignación es cualquier texto del fichero del
  parche que viene antes del primer encabezado del diff.  Si no hay
  tal texto, un mensaje predeterminado se sua para identificar el
  nombre del parche.
\end{itemize}
Su un parche contiene un encabezado de parche de Mercurial (XXX add
link), la información en el encabezado del parche tiene precedencia
sobre el predeterminado.

Opciones:
\begin{itemize}
\item[\hgxopt{mq}{qpush}{-a}] Introduce todos los parches que no han
  sido aplicados del fichero \sfilename{series} hasta que no haya nada
  más para introducir.
\item[\hgxopt{mq}{qpush}{-l}] Añade el nombre del parche al final del
  mensaje de consignación
\item[\hgxopt{mq}{qpush}{-m}] Si un parche no se aplica limpiamente,
  usa la entrada para un parche en otra cola almacenada para computar
  los parámetros en una fusión de tres, y aplica una fusión de tres
  fuentes usando la maquinaria usual de Mercurial.  Usa la resolución
  de la fusión como el contenido del parche nuevo.
\item[\hgxopt{mq}{qpush}{-n}] Usa la cola mencionada si se está
  fusionando en la introducción.
\end{itemize}

La orden \hgxcmd{mq}{qpush} lee, pero no modifica el fichero
\sfilename{series}.  Añade al final del fichero \hgcmd{status} una
línea por cada parche que se introduce.

\subsection{\hgxcmd{mq}{qrefresh}---actualiza el último parche aplicado}

La orden \hgxcmd{mq}{qrefresh} actualiza el último parche aplicado.
Modifica el parche, elimina el último conjunto de cambios que
representó el parche, y crea un nuevo conjunto de cambios para
representar el parche modificado.

La orden \hgxcmd{mq}{qrefresh} busca las siguientes modificaciones:
\begin{itemize}
\item Los cambios al mensaje de consignación, p.e.~el texto antes del
  primer encabezado de diff en el fichero del parche, se replejan en
  el nuevo conjunto de cambios que representa el parche.
\item Las modificaciones a los ficheros a los que se les da
  seguimiento en el directorio de trabajo se añade al parche.
\item Los cambios a los archivos a los que se les da seguimiento con
  \hgcmd{add}, \hgcmd{copy}, \hgcmd{remove}, o \hgcmd{rename}.  Se
  añaden al parche los ficheros añadidos, copiados y renombrados,
  mientras que los ficheros eliminados y las fuentes renombradas se
  eliminan.
\end{itemize}

Incluso si \hgxcmd{mq}{qrefresh} no detecta cambios, de todas maneras
recrea el conjunto de cambios que representa el cambio.  Esto causa
que la identidad del conjunto de cambios difiera del conjunto de
cambios previo que identificó al parche.

Opciones:
\begin{itemize}
\item[\hgxopt{mq}{qrefresh}{-e}] Modificar la descripción de la
  consignación y el parche con el editor de texto preferido.
\item[\hgxopt{mq}{qrefresh}{-m}] Modificar el mensaje de consignación
  y la descripción del parche con el texto dado.
\item[\hgxopt{mq}{qrefresh}{-l}] Modificar el mensaje de consignación
  y la descripción del parche con el texto del fichero dado.
\end{itemize}

\subsection{\hgxcmd{mq}{qrename}---renombrar un parche}

La orden \hgxcmd{mq}{qrename} renombra un parche y cambia la entrada
del parche en el fichero \sfilename{series}.

Con un argumento sencillo, \hgxcmd{mq}{qrename} renombra el último
parche aplicado.  Con dos argumentos, renombra el primer argumento con
el segundo.

\subsection{\hgxcmd{mq}{qrestore}---restaurar el estado almacenado de
  la cola}

XXX No idea what this does.

\subsection{\hgxcmd{mq}{qsave}---almacena el estado actual de la cola}

XXX Likewise.

\subsection{\hgxcmd{mq}{qseries}---imprime la serie completa de parches}

La orden \hgxcmd{mq}{qseries} imprime la serie completa de parches del
fichero \sfilename{series}.  Imprime solamente los nombres de los
parches sin las líneas en blanco o comentarios.  Imprime primero el
primero y de último, el último aplicado.

\subsection{\hgxcmd{mq}{qtop}---imprime el nombre del parche actual}

\hgxcmd{mq}{qtop} imprime el nombre del último parche aplicado.

\subsection{\hgxcmd{mq}{qunapplied}---imprimir los parches que aún no
  se han aplicado}

La orden \hgxcmd{mq}{qunapplied} imprime los nombres de los parches
del fichero \sfilename{series} que todavía no han sido aplicados.  Los
imprime de acuerdo al orden en el cual serían introducidos.

\subsection{\hgcmd{strip}---remover una revisión y sus descendientes}

La orden \hgcmd{strip} remueve una revisión, y todos sus descendientes
del repositorio.  Deshace los efectos de las revisiones removidas del
repositorio, y actualiza el directorio de trabajo hasta el primer
padre de la revisión removida.

La orden \hgcmd{strip} almacena una copia de segurida de los conjuntos
de cambios en un agrupamiento, de forma tal que puedan ser reaplicados
en caso de que se hayan removido por equivocación.

Opciones:
\begin{itemize}
\item[\hgopt{strip}{-b}] Almacenar conjuntos de cambios no
  relacionados que se han mezclado con los conjuntos de cambios que
  están en franjas con el agrupamiento de copia de seguridad.
\item[\hgopt{strip}{-f}] Si una rama tiene varias ramas principales
  remueve todos los frentes. XXX This should be renamed, y usa
  \texttt{-f} para desagrupar revisiones cuando hay cambios pendientes.
\item[\hgopt{strip}{-n}] No almacene la copia de seguridad agrupada.
\end{itemize}

\section{Referencia de ficheros de MQ}

\subsection{El fichero \sfilename{series}}

El fichero \sfilename{series} contiene una lista de los nombres de
todos los parches que MQ puede aplicar.  Se representa como una lista
de nombres, uno por línea.  Se ignora el espacio en blanco al
principio y al final.

Las líneas pueden contener comentario.  Un comentario comienza con el
caracter ``\texttt{\#}'', y va hasta el final de la línea.  Se ignoran
las líneas vacías y las que solamente contengan comentarios.

En algún momento podría editar el fichero \sfilename{series} a mano,
por tal motivo se admiten comentarios y líneas en blanco como se
menciono anteriormente.  Por ejemplo, puede poner en comentario un
parche temporalmente y \hgxcmd{mq}{qpush} omitirá tal parche cuando
los aplique.  También puede cambiar el orden en el cual se aplican los
parches, reordenando las entradas en el fichero \sfilename{series}.

También es posible colocar el fichero \sfilename{series} bajo control
de revisiones;  también es favorable colocar todos los parches que refiera
bajo control de revisiones.  Si crea un directorio de parches con la
opción \hgxopt{mq}{qinit}{-c} de \hgxcmd{mq}{qinit}, esto se hará
automáticamente.

\subsection{El fichero \sfilename{status}}

El fichero \sfilename{status} contiene los nombres y los hashes de los
conjuntos de cambios de todos los parches que MQ ha aplicado.  A
diferencia del fichero \sfilename{series}, este NO ha sido diseñado
para ser editado.  No debería colocar este fichero bajo el control de
revisiones o modificarlo de forma alguna.  MQ lo usa estrictamente
para administración interna.

%%% Local Variables: 
%%% mode: latex
%%% TeX-master: "00book"
%%% End: 
