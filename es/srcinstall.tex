\chapter{Instalar Mercurial desde las fuentes}
\label{chap:srcinstall}
cosa.lala dfsdf

\section{En un sistema tipo Unix}
\label{sec:srcinstall:unixlike}

Si usa un sistema tipo Unix que tiene una versión suficientemente
reciente de Python (2.3~o superior) disponible, es fácil instalar
Mercurial desde las fuentes.
\begin{enumerate}
\item Descargue un paquete fuente reciente de
  \url{http://www.selenic.com/mercurial/download}.
\item Descomprímalo:
  \begin{codesample4}
    gzip -dc mercurial-\emph{version}.tar.gz | tar xf -
  \end{codesample4}
\item Vaya al directorio fuente y ejecute el guión de instalación.
  Esto armará Mercurial y lo instalará en su directorio casa:
  \begin{codesample4}
    cd mercurial-\emph{version}
    python setup.py install --force --home=\$HOME
  \end{codesample4}
\end{enumerate}
Cuando termine la instalación, Mercurial estará en el subdirectorio
\texttt{bin} de su directorio casa.  No olvide asegurarse de que este
directorio esté presente en el camino de búsqueda de su intérprete de
órdenes.

Probablemente necesitará establecer la variable de ambiente
\envar{PYTHONPATH} de tal forma que los ejecutables de Mercurial
puedan encontrar el resto de los paquetes de Mercurial.  Por ejemplo,
en mi portátil, la establecía a \texttt{/home/bos/lib/python}.  La
ruta exacta que usted use dependerá de como ha sido construído Python
en su sistema, pero debería ser fácil deducirla.  Si no está seguro,
mire lo que haya mostrado el script en el paso anterior, y vea dónde
se instalaron los contenidos del directorio \texttt{mercurial} se
instalaron.

\section{En Windows}

Armar e instalar Mercurial en Windows requiere una variedad de
herramientas, cierta suficiencia técnica y paciencia considerable.
Personalmente, \emph{no le recomiendo} hacerlo si es un ``usuario
casual''.  A menos que intente hacer hacks a Mercurial, le recomiendo
que mejor use un paquete binario.

Si está decidido a construir Mercurial desde las fuentes en Windows,
siga el ``camino difícil'' indicado en el wiki de Mercurial en
\url{http://www.selenic.com/mercurial/wiki/index.cgi/WindowsInstall},
y espere que el proceso sea realmente un trabajo duro.

%%% Local Variables: 
%%% mode: latex
%%% TeX-master: "00book"
%%% End: 
