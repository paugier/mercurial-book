\chapter{Añadir funcionalidad con extensiones}
\label{chap:hgext}

A pesar de que el corazón de Mercurial es muy completo desde el punto
de vista de funcionalidad, carece de características rimbombantes
deliberadamente.  Esta aproximación de preservar la simplicidad
mantiene el programa sencillo tanto para mantenedores como para
usuarios.

Si embargo Mercurial no le cierra las posibilidades a un conjunto
inflexible de órdenes: usted puede añadir características como
\emph{extensiones} (aveces llamadas \emph{añadidos}\ndt{plugins}).  Ya
hemos discutido algunas de estas extensiones en capítulos anteriores:
\begin{itemize}
\item La sección~\ref{sec:tour-merge:fetch} cubre la extensión
  \hgext{fetch}; que combina jalar cambios y fusionarlos con los
  cambios locales en una sola orden: \hgxcmd{fetch}{fetch}.
\item En el capítulo~\ref{chap:hook}, cubrimos muchas extensiones que
  son útiles en funcionalidades relacionadas con ganchos: Los
  \hgext{acl} añaden listas de control de acceso; \hgext{bugzilla}
  añade integración con el sistema de seguimiento de fallos Bugzilla; y
  \hgext{notify} envía notificaciones por correo de nuevos cambios.
\item La extensión de administración de parches MQ es tan invaluable
  que amerita dos capítulos y un apéndice por sí misma.
  El capítulo~\ref{chap:mq} cubre lo básico; el
  capítulo~\ref{chap:mq-collab} discute temas avanzados; y el
  apéndice~\ref{chap:mqref} muestra en detalle cada orden.
\end{itemize}

En este capítulo cubriremos algunas extensiones adicionales
disponibles para Mercurial, y daremos un vistazo a la maquinaria que
necesita conocer en caso de que desee escribir una extensión.
\begin{itemize}
\item En la sección~\ref{sec:hgext:inotify}, discutiremos la
  posibilidad de mejorar el desempeño \emph{en gran medida} con la extensión
  \hgext{inotify}.
\end{itemize}

\section{Mejorar el desempeño con la extensión \hgext{inotify}}
\label{sec:hgext:inotify}

¿Desea lograr que las operaciones más comunmente usadas de Mercurial se
ejecuten centenas de veces más rápido? ¡A leer!

Mercurial tiene gran desempeño bajo circunstancias normales.  Por
ejemplo, cuando ejecuta la orden \hgcmd{status}, Mercurial tiene que
revisar casi todos los ficheros y directorios en su repositorio de
forma que pueda desplegar el estado de los ficheros.  Muchas otras
órdenes tienen que hacer tal trabajo tras bambalinas;  por ejemplo la
orden \hgcmd{diff} usa la maquinaria de estado para evitar hacer
operaciones de comparación costosas en ficheros que obviamente no han
cambiado.

Dado que obtener el estado de los ficheros es crucial para obtener
buen desempeño, los autores de Mercurial han optimizado este código en
la medida de lo posible.  Sin embargo, no puede obviarse el hecho de
que cuando ejecuta \hgcmd{status}, Mercurial tendrá que hacer por lo
menos una costosa llamada al sistema por cada archivo administrado
para determinar si ha cambiado desde la última vez que se consignó.
Para un repositorio suficientemente grande, puede tardar bastante
tiempo.

Para mostrar en números la magnitud de este efect, creé un repositorio
que contenía 150.000 archivos administrador.  Tardó diez segundos para
ejecutar \hgcmd{status}, a pesar de que \emph{ninguno} de los ficheros
había sido modificado.

Muchos sistemas operativos modernos contienen una facilidad de
notificación de archivos.  Si un programa se registra con un servicio
apropiado, el sistema operativo le notificará siempre que un fichero
de interés haya sido creado, modificado o borrado.  En sistemas Linux,
el componente del núcleo que lo hace se llama \texttt{inotify}.

La extensión \hgext{inotify} habla con el componente \texttt{inotify}
del núcleo para optimizar las órdenes de \hgcmd{status}.  La extensión
tiene dos componentes.  Un daemonio está en el fondo recibiendo
notificaciones del subsistema \texttt{inotify}.  También escucha
conexiones de una orden regular de Mercurial.  La extensión modifica
el comportamiento de Mercurial de tal forma que, en lugar de revisar
el sistema de ficheros, le pregunta al daemonio.  Dado que el daemonio
tiene información perfecta acerca del estado del repositorio, puede
responder instantáneamente con el resultado, evitando la necesidad de
revisar cada directorio y fichero del repositorio.

Retomando los diez segundos que medí al ejecutar la orden
\hgcmd{status} de Mercurial sobre un repositorio de 150.000
ficheros. Con la extensión \hgext{inotify} habilitada, el tiempo se
disipó a 0.1~seconds, un factor \emph{cien veces} más rápido.

Antes de continuar, tenga en cuenta algunos detalles:
\begin{itemize}
\item La extensión \hgext{inotify} es específica de Linux.  Porque se
  enlaza directamente con el subsistema \texttt{inotify} del núcleo
  Linux, no funciona en otros sistemas operativos.
\item Debería funcionar en cualquier distribución Linux a partir de
  comienzos del 2005.  Las distribuciones más antiguas deben tener un
  kernel sin \texttt{inotify}, o una versión de \texttt{glibc} que no
  tiene necesariamente el soporte para la interfaz.
\item No todos los sistemas de ficheros pueden usarse con la extensión
  \hgext{inotify}.  Los sistemas de ficheros tales como NFS no lo
  soportan, por ejemplo, si está corriendo Mercurial en vaios
  sistemas, montados todos sobre el mismo sistema de ficheros en red.
  El sistema \texttt{inotify} del kernel no tiene forma de saber
  acerca de los cambios hechos en otro sistema.  La mayoría de
  sistemas de ficheros locales (p.e.~ext3, XFS, ReiserFS) deberían
  funcionar bien.
\end{itemize}

Hacia mayo de 2007 la extensión \hgext{inotify} no venía de forma
predeterminada en Mercurial\ndt{Desde el 2008 para kernels 2.6 viene
  en Mercurial, pero no está activada de forma predeterminada}, y es
un poco más compleja de activar que otras extensiones.  Pero la mejora
en el desempeño bien vale la pena!

La extensión venía en dos partes: un conjunto de parches al código
fuente de Mercurial, y una librería de interfaces de Python hacia el
subsistema \texttt{inotify}.
\begin{note}
  Hay \emph{dos} librerías de enlace de Python hacia \texttt{inotify}.
  Una de ellas se llama \texttt{pyinotify}, y en algunas
  distribuciones de Linux se encuentra como \texttt{python-inotify}.
  Esta es la que \emph{no} necesita, puesto que tiene muchos fallos,
  y es ineficiente para ser práctica.
\end{note}
Para comenzar, es mejor tener una copia de Mercurial funcional
instalada:
\begin{note}
  Si sigue las instrucciones a continuación, estará
  \emph{reemplazando} y sobreescribiendo cualquier instalación previa
  de Mercurial que pudiera tener, con el código de Mercurial ``más
  reciente y peligrosa''.  No diga que no se le advirtio!
\end{note}
\begin{enumerate}
\item Clone el repositorio de interfaz entre Python e
  \texttt{inotify}. Ármelo e instálelo:
  \begin{codesample4}
    hg clone http://hg.kublai.com/python/inotify
    cd inotify
    python setup.py build --force
    sudo python setup.py install --skip-build
  \end{codesample4}
\item Clone el repositorio \dirname{crew} de Mercurial.  Clone el
  repositorio de parches de \hgext{inotify} de forma tal que las colas
  de Mercurial puedan aplicar los parches sobre el repositorio \dirname{crew}.
  \begin{codesample4}
    hg clone http://hg.intevation.org/mercurial/crew
    hg clone crew inotify
    hg clone http://hg.kublai.com/mercurial/patches/inotify inotify/.hg/patches
  \end{codesample4}
\item Asegúrese de instalar la extensión Colas de Mercurial
  \hgext{mq} y que estén habilitadas.  Si nunca ha usado MQ, lea la
  sección~\ref{sec:mq:start} para poder comenzar rápidamente.
\item Vaya al repositorio de \dirname{inotify} y aplique todos los
  parches de \hgext{inotify} con la opción \hgxopt{mq}{qpush}{-a} de
  la orden \hgxcmd{mq}{qpush}.
  \begin{codesample4}
    cd inotify
    hg qpush -a
  \end{codesample4}
  Si obtiene un mensaje de error de \hgxcmd{mq}{qpush}, no debería
  continuar.  Mejor pida ayuda.
\item Arme e instale la versión parchada de Mercurial.
  \begin{codesample4}
    python setup.py build --force
    sudo python setup.py install --skip-build
  \end{codesample4}
\end{enumerate}
Una vez que haya armado una versión funcional parchada de Mercurial,
todo lo que necesita es habilitar la extensión \hgext{inotify}
colocando una entrada en su \hgrc.
\begin{codesample2}
  [extensions]
  inotify =
\end{codesample2}
Cuando la extensión \hgext{inotify} esté habilitada, Mercurial
iniciará transparente y automáticamente el daemonio de estado la
primera vez que ejecute un comando que requiera estado del
repositorio.  Ejecuta un daemoniot de estado por repositorio.

El daemonio de estado se inicia silenciosamente y se ejecuta en el
fondo. Si mira a la lista de procesos en ejecución después de
habilitar la extensión \hgext{inotify} y ejecuta unos pocos comandos
en diferentes repositorios, verá que hay algunos procesos de
\texttt{hg} por ahí, esperando actualizaciones del kernel y
solicitudes de Mercurial.

La primera vez que ejecuta un comando de Mercurial en un repositorio
cuando tiene la extensión \hgext{inotify} habilitada, correrá casi con
el mismo desempeño que una orden usual de Mercurial.  Esto es debido a
que el estado del daemonio necesita aplicar una búsqueda normal sobre
el estado para poder tener una línea de partida frente a la cual
aplicar posteriormente actualizaciones del núcleo.  De todas formas,
\emph{todo} comando posterior que haga cualquier clase de revisión del
estado debería ser notablemente más rápido en repositorios con incluso
un tamaño modesto.  Aún mejor, a medida que su repositorio sea más
grande, mejor desempeño verá.  El daemonio \hgext{inotify} hace
operaciones de estado de forma casi instantánea en repositorios de
todos los tamaños!

Si lo desea, puede iniciar manualmente un daemonio de estado con la orden
\hgxcmd{inotify}{inserve}.  Esto le da un control un poco más fino
acerca de cómo debería ejecutarse el daemonio.  Esta orden solamente
estará disponible cuando haya habilitado la extensión \hgext{inotify}.

Cuando esté usando la extensión \hgext{inotify},
\emph{no debería ver diferencia} en el comportamiento de Mercurial,
con la única excepción de que los comandos relacionados con el estado
deberían ejectuarse mucho más rápido que como solían hacerlo.  Debería
esperar específicamente que las órdenes no deberían ofrecer salidas
distintas; ni ofrecer resultados diferentes.  Si alguna de estas
situaciones ocurre, por favor reporte el fallo.

\section{Soporte flexible de diff con la extensión \hgext{extdiff}}
\label{sec:hgext:extdiff}

La orden predeterminada \hgcmd{diff} de Mercurial despliega diffs en
texto plano unificadas.
\interaction{extdiff.diff}
Si dese emplear una herramienta externa para desplegar las
modificaciones, querrá usar la extensión \hgext{extdiff}.  Esta le
permitirá usar por ejemplo una herramienta gráfica de diff.

La extensión \hgext{extdiff} viene con Mercurial, y es fácil
configurarl.  En la sección \rcsection{extensions} de su \hgrc,
basta con añadir una entrada de una línea para habilitar la extensión.
\begin{codesample2}
  [extensions]
  extdiff =
\end{codesample2}
Esto introduce una orden llamada \hgxcmd{extdiff}{extdiff}, que de
forma predeterminada usa su orden del sistema \command{diff} para
generar un diff unificado de la misma forma que lo hace el comando
predeterminado \hgcmd{diff}.
\interaction{extdiff.extdiff}
El resultado no será exactamente el mismo que con la orden interna
\hgcmd{diff}, puesto que la salida de \command{diff} varía de un
sistema a otro, incluso pasando las mismas opciones.

Como lo indican las líneas``\texttt{making snapshot}'', la orden
\hgxcmd{extdiff}{extdiff} funciona creando dos instantáneas de su
árbol de fuentes.  La primera instantánea es la revisión fuente; la
segunda es la revisión objetivo del directorio de trabajo.  La orden
\hgxcmd{extdiff}{extdiff} genera estas instantáneas en un directorio
temporal, pasa el nombre de cada directorio a un visor de diffs
temporal y borra los directorios temporales.  Por cuestiones de
eficiencia solamente genera instantáneas de los directorios y ficheros
que han cambiado entre dos revisiones.

Los nombres de los directorios de instantáneas tienen los mismos
nombres base de su repositorio.  Si su repositorio tiene por ruta
\dirname{/quux/bar/foo}, \dirname{foo} será el nombre de cada
instantánea de directorio.  Cada instantánea de directorio tiene sus
identificadores de conjuntos de cambios al final del nombre en caso de
que sea apropiado.  Si una instantánea viene de la revisión
\texttt{a631aca1083f}, el directorio se llamará
\dirname{foo.a631aca1083f}.  Una instantánea del directorio de trabajo
no tendrá el identificador del conjunto de cambios, y por lo tanto
será solamente \dirname{foo} en este ejemplo.  Para ver cómo luce en
la práctica, veamos de nuevo el ejemplo \hgxcmd{extdiff}{extdiff}
antes mencionado.  Tenga en cuenta que los diffs tienen los nombres de
las instantáneas de directorio dentro de su encabezado.

La orden \hgxcmd{extdiff}{extdiff} acepta dos opciones importantes.
La opción \hgxopt{extdiff}{extdiff}{-p} le permite elegir un programa
para ver las diferencias, en lugar de \command{diff}.  Con la opción
\hgxopt{extdiff}{extdiff}{-o} puede cambiar las opciones que
\hgxcmd{extdiff}{extdiff} pasa a tal programa(de forma predeterminada
las opciones son``\texttt{-Npru}'', que tienen sentido únicamente si
está usando \command{diff}).  En otros aspectos, la orden
\hgxcmd{extdiff}{extdiff} actúa de forma similar a como lo hace la
orden \hgcmd{diff} de Mercurial: usted usa los mismos nombres de
opciones, sintaxis y argumentos para especificar las revisiones y los
ficheros que quiere, y así sucesivamente.

Por ejemplo, para ejecutar la orden usual del sistema \command{diff},
para lograr que se generen diferencias de contexto (con la opción
\cmdopt{diff}{-c}) en lugar de diferencias unificadas, y cinco líneas
de contexto en lugar de las tres predeterminadas(pasando \texttt{5}
como argumento a la opción \cmdopt{diff}{-C}).
\interaction{extdiff.extdiff-ctx}

Es sencillo lanzar unas herramienta usual de diferencias.  Para lanzar
el visor \command{kdiff3}:
\begin{codesample2}
  hg extdiff -p kdiff3 -o ''
\end{codesample2}

Si su orden para visualizar diferencias no puede tratar con
directorios, puede usar un poco de scripting para lograrlo.  Un
ejemplo de un script con la extensión \hgext{mq} junto con la orden
\command{interdiff} está en la sección~\ref{mq-collab:tips:interdiff}.

\subsection{Definición de alias de comandos}

Acordarse de todas las opciones de las órdenes
\hgxcmd{extdiff}{extdiff} y el visor de diferencias de su preferencia
puede ser dispendioso, y por lo tanto la extensión \hgext{extdiff} le
permite definir \emph{nuevas} órdenes que invocarán su visor de
diferencias con las opciones exactas.

Basta con editar su fichero \hgrc, y añadir una sección llamada
\rcsection{extdiff}.  Dentro de esta sección puede definir varias
órdenes. Mostraremos como añadir la orden \texttt{kdiff3}.  Después de
definido, puede teclear ``\texttt{hg kdiff3}'' y la extensión a
\hgext{extdiff} ejecutará la orden \command{kdiff3}.
\begin{codesample2}
  [extdiff]
  cmd.kdiff3 =
\end{codesample2}
Si deja vacía la porción derecha de la definición, como en el ejemplo,
la extensión \hgext{extdiff} usa el nombre de la orden se definirá
como el nombre del programa externo a ejecutar.  Pero tales nombres no
tienen por qué ser iguales.  Definimos ahora la orden llamada
 ``\texttt{hg wibble}'', que ejecuta \command{kdiff3}.
\begin{codesample2}
  [extdiff]
  cmd.wibble = kdiff3
\end{codesample2}

También puede especificar las opciones predeterminadas con las cuales
desea invocar el visor de diferencias.  Se usa el prefijo ``\texttt{opts.}'',
seguido por el nombre de la orden a la cual se aplican las opciones.
En este ejemplos se define la orden ``\texttt{hg vimdiff}'' que
ejecuta la extensión \texttt{DirDiff} del editor \command{vim}.
\begin{codesample2}
  [extdiff]  
  cmd.vimdiff = vim
  opts.vimdiff = -f '+next' '+execute "DirDiff" argv(0) argv(1)'
\end{codesample2}

\section{Uso de la extensión \hgext{transplant} para seleccionar}
\label{sec:hgext:transplant}

Need to have a long chat with Brendan about this.

\section{Enviar cambios vía correo electrónico con la extensión \hgext{patchbomb}}
\label{sec:hgext:patchbomb}

Varios proyectos tienen la cultura de ``revisión de cambios'', en la
cual la gente envía sus modificaciones a una lista de correo para que
otros las lean y comenten antes de consignar la versión final a un
repositorio compartido.  Algunos proyectos tienen personas que actúan
como cancerberos; ellos aplican los cambios de otras personas a un
repositorio para aquellos que no tienen acceso.

Mercurial facilita enviar cambios por correo para revisión o
aplicación gracias a su extensión \hgext{patchbomb}.  La extensión es
tan popular porque los cambios se formatean como parches y es usual
que se envía un conjunto de cambios por cada correo.  Enviar una gran
cantidad de cambios por correos se llama por tanto ``bombardear'' el
buzón de entrada del destinatario, de ahí su nombre ``bombardeo de
parches''.

Como es usual, la configuración básica de la extensión
\hgext{patchbomb} consta de una o dos líneas en su \hgrc.
\begin{codesample2}
  [extensions]
  patchbomb =
\end{codesample2}
Cuando haya habilitado la extensión, dispondrá de una nueva orden,
llamada \hgxcmd{patchbomb}{email}.

La forma mejor y más segura para invocar la orden
\hgxcmd{patchbomb}{email} es ejecutarla \emph{siempre} con la opción
\hgxopt{patchbomb}{email}{-n}; que le mostrará lo que la orden
\emph{enviaría}, sin enviar nada.  Una vez que haya dado un vistazo a
los cambios y verificado que está enviando los correctos, puede volver
a ejecutar la misma orden, sin la opción \hgxopt{patchbomb}{email}{-n}.

La orden \hgxcmd{patchbomb}{email} acepta la misma clase de sintaxis
de revisiones como cualquier otra orden de Mercurial.  Por ejemplo,
enviará todas las revisiones entre la 7 y la \texttt{punta}, inclusive.
\begin{codesample2}
  hg email -n 7:tip
\end{codesample2}
También puede especificar un \emph{repositorio} para comparar.  Si
indica un repositoro sin revisiones, la orden \hgxcmd{patchbomb}{email}
enviará todas las revisiones en el repositorio local que no están
presentes en el repositorio remoto.  Si especifica revisiones
adicionalmente o el nombre de una rama(la última con la opción
\hgxopt{patchbomb}{email}{-b}), respetará las revisiones enviadas.

Ejecutar la orden \hgxcmd{patchbomb}{email} sin los nombres de
aquellas personas a las cuales desea enviar el correo es completamente
seguro: si lo hace, solicitará tales valores de forma interactiva.
(Si está usando Linux o un sistema tipo Unix, tendrá capacidades
estilo--\texttt{readline} aumentadas cuando ingrese tales encabezados,
lo cual es sumamente útil.)

Cuando envíe una sola revisión, la orden \hgxcmd{patchbomb}{email}
de forma predeterminada usará la primera línea de descripción del
conjunto de cambios como el tema del único mensaje que se enviará.

Si envía varias revisiones, la orden \hgxcmd{patchbomb}{email} enviará
normalmente un mensaje por conjunto de cambios.  colocará como
prefacio un mensaje introductorio en el cual usted debería describir
el propósito de la serie de cambios que está enviando.

\subsection{Cambiar el comportamiento de las bombas de parches}

Cada proyecto tiene sus propias convenciones para enviar cambios en un
correo electrónico; la extensión \hgext{patchbomb} intenta acomodarse
a diferentes variaciones gracias a las opciones de la línea de órdenes:
\begin{itemize}
\item Puede escribir un tema para el mensaje introductorio en la línea
  de órdenes con la opciń \hgxopt{patchbomb}{email}{-s}.  Toma un
  argumento: el tema del mensaje a usar.
\item Para cambiar el correo electrónico del campo del cual se
  origina, use la opción \hgxopt{patchbomb}{email}{-f}.  Toma un
  argumento, el correo electrónico a usar.
\item El comportamiento predeterminado es enviar diferencias
  unificadas (consulte la sección~\ref{sec:mq:patch} si desea una
  descripción del formato), una por mensaje.  Puede enviar un conjunto
  binario\ndt{binary bundle} con la opción \hgxopt{patchbomb}{email}{-b}.
\item Las diferencias unificadas están precedidas por un encabezado de
  metadatos.  Puede omitirlo, y enviar diferencias sin adornos con la
  opción \hgxopt{patchbomb}{email}{--plain}.
\item Las diferencias usualmente se envían ``en línea'', como parte
  del cuerpo del mensaje con la descripción del parche.  Que facilita a
  a la mayor cantidad de lectores citar y responder partes de un diff,
  dado que algunos clientes de correo solamente citarán la primera
  parte MIME del cuerpo de un mensaje.  Si prefiere enviar la
  descripción y el diff en partes separadas del cuerpo, use la opción
  \hgxopt{patchbomb}{email}{-a}.
\item En lugar de enviar mensajes de correo puede escribirlos a un
  fichero con formato-\texttt{mbox}- con la opción
  \hgxopt{patchbomb}{email}{-m}.  La opción recibe un argumento, el
  nombre del fichero en el cual escribir.
\item Si desea añadir un resumen con formato-\command{diffstat} en
  cada parche, y uno como mensaje introductorio, use la opción
  \hgxopt{patchbomb}{email}{-d}.  La orden \command{diffstat}
  despliega una tabla que contiene el nombre de cada fichero parchado,
  el número de líneas afectadas, y un historgrama mostrando cuánto ha
  sido modificado cada fichero.  Lo cual ofrece a los lectores una
  mirada cuantitativa de cuan complejo es el parche.
\end{itemize}

%%% Local Variables: 
%%% mode: latex
%%% TeX-master: "00book"
%%% End: 
