\chapter{Una gira de Mercurial: lo básico}
\label{chap:tour-basic}

\section{Instalar Mercurial en su sistema}
\label{sec:tour:install}
Hay paquetes binarios precompilados de Mercurial disponibles para cada
sistema operativo popular. Esto hace fácil empezar a usar Mercurial
en su computador inmediatamente.

\subsection{Linux}

Dado que cada distribución de Linux tiene sus propias herramientas de
manejo de paquetes, políticas, y ritmos de desarrollo, es difícil dar
un conjunto exhaustivo de instrucciones sobre cómo instalar el paquete
de Mercurial. La versión de Mercurial que usted tenga a disposición
puede variar dependiendo de qué tan activa sea la persona que mantiene
el paquete para su distribución.

Para mantener las cosas simples, me enfocaré en instalar Mercurial
desde la línea de comandos en las distribuciones de Linux más
populares. La mayoría de estas distribuciones proveen administradores
de paquetes gráficos que le permitirán instalar Mercurial con un solo
clic; el nombre de paquete a buscar es \texttt{mercurial}.

\begin{itemize}
\item[Debian]
  \begin{codesample4}
    apt-get install mercurial
  \end{codesample4}

\item[Fedora Core]
  \begin{codesample4}
    yum install mercurial
  \end{codesample4}

\item[Gentoo]
  \begin{codesample4}
    emerge mercurial
  \end{codesample4}

\item[OpenSUSE]
  \begin{codesample4}
    yum install mercurial
  \end{codesample4}

\item[Ubuntu] El paquete de Mercurial de Ubuntu está basado en el de
    Debian. Para instalarlo, ejecute el siguiente comando.
  \begin{codesample4}
    apt-get install mercurial
  \end{codesample4}
  El paquete de Mercurial para Ubuntu tiende a atrasarse con respecto
  a la versión de Debian por un margen de tiempo considerable
  (al momento de escribir esto, 7 meses), lo que en algunos casos
  significará que usted puede encontrarse con problemas que ya habrán
  sido resueltos en el paquete de Debian.
\end{itemize}

\subsection{Solaris}

SunFreeWare, en \url{http://www.sunfreeware.com}, es una buena fuente
para un gran número de paquetes compilados para Solaris para las
arquitecturas Intel y Sparc de 32 y 64 bits, incluyendo versiones
actuales de Mercurial.

\subsection{Mac OS X}

Lee Cantey publica un instalador de Mercurial para Mac OS~X en 
\url{http://mercurial.berkwood.com}.  Este paquete funciona en tanto
en Macs basados en Intel como basados en PowerPC. Antes de que pueda
usarlo, usted debe instalar una versión compatible de Universal
MacPython~\cite{web:macpython}. Esto es fácil de hacer; simplemente
siga las instrucciones de el sitio de Lee.

También es posible instalar Mercurial usando Fink o MacPorts, dos
administradores de paquetes gratuitos y populares para Mac OS X. Si
usted tiene Fink, use \command{sudo apt-get install mercurial-py25}.
Si usa MacPorts, \command{sudo port install mercurial}.

\subsection{Windows}

Lee Cantey publica un instalador de Mercurial para Windows en
\url{http://mercurial.berkwood.com}. Este paquete no tiene
% TODO traducción de it just works. Agreed?
dependencias externas; ``simplemente funciona''.

\begin{note}
    La versión de Windows de Mercurial no convierte automáticamente
    los fines de línea entre estilos Windows y Unix. Si usted desea
    compartir trabajo con usuarios de Unix, deberá hacer un trabajo
    adicional de configuración. XXX Terminar esto.
\end{note}

\section{Arrancando}

Para empezar, usaremos el comando \hgcmd{version} para revisar si
Mercurial está instalado adecuadamente. La información de la versión
que es impresa no es tan importante; lo que nos importa es si imprime
algo en absoluto.

\interaction{tour.version}

% TODO builtin-> integrado?
\subsection{Ayuda integrada}

Mercurial provee un sistema de ayuda integrada. Esto es invaluable
para ésas ocasiones en la que usted está atorado tratando de recordar
cómo ejecutar un comando. Si está completamente atorado, simplemente
ejecute \hgcmd{help}; esto imprimirá una breve lista de comandos,
junto con una descripción de qué hace cada uno. Si usted solicita
ayuda sobre un comando específico (como abajo), se imprime información
más detallada.
\interaction{tour.help}
Para un nivel más impresionante de detalle (que usted no va a
necesitar usualmente) ejecute \hgcmdargs{help}{\hggopt{-v}}. La opción
\hggopt{-v} es la abreviación para \hggopt{--verbose}, y le indica a
Mercurial que imprima más información de lo que haría usualmente.

\section{Trabajar con un repositorio}

En Mercurial, todo sucede dentro de un \emph{repositorio}. El
repositorio para un proyecto contiene todos los archivos que
``pertenecen a'' ése proyecto, junto con un registro histórico de los
archivos de ese proyecto.

No hay nada particularmente mágico acerca de un repositorio; es
simplemente un árbol de directorios en su sistema de archivos que
Mercurial trata como especial. Usted puede renombrar o borrar un
repositorio en el momento que lo desee, usando bien sea la línea de
comandos o su explorador de ficheros.

\subsection{Hacer una copia local de un repositorio}

\emph{Copiar} un repositorio es sólo ligeramente especial. Aunque
usted podría usar un programa normal de copia de archivos para hacer
una copia del repositorio, es mejor usar el comando integrado que
Mercurial ofrece. Este comando se llama \hgcmd{clone}\ndt{Del término
``clonar'' en inglés.}, porque crea una copia idéntica de un
repositorio existente.
\interaction{tour.clone}
Si nuestro clonado tiene éxito, deberíamos tener un directorio local
llamado \dirname{hello}. Este directorio contendrá algunos archivos.
\interaction{tour.ls}
Estos archivos tienen el mismo contenido e historial en nuestro
repositorio y en el repositorio que clonamos.

Cada repositorio Mercurial está completo, es autocontenido e
independiente. Contiene su propia copia de los archivos y la historia
de un proyecto. Un repositorio clonado recuerda la ubicación de la que
fue clonado, pero no se comunica con ese repositorio, ni con ningún
otro, a menos que usted le indique que lo haga.

Lo que esto significa por ahora es que somos libres de experimentar
con nuestro repositorio, con la tranquilidad de saber que es una
% TODO figure out what to say instead of sandbox
``caja de arena'' privada que no afectará a nadie más.

\subsection{Qué hay en un repositorio?}

Cuando miramos en detalle dentro de un repositorio, podemos ver que
contiene un directorio llamado \dirname{.hg}. Aquí es donde Mercurial
mantiene todos los metadatos del repositorio.
\interaction{tour.ls-a}

Los contenidos del directorio \dirname{.hg} y sus subdirectorios son
exclusivos de Mercurial. Usted es libre de hacer lo que desee con
cualquier otro archivo o directorio en el repositorio.

To introduce a little terminology, the \dirname{.hg} directory is the
``real'' repository, and all of the files and directories that coexist
with it are said to live in the \emph{working directory}.  An easy way
to remember the distinction is that the \emph{repository} contains the
\emph{history} of your project, while the \emph{working directory}
contains a \emph{snapshot} of your project at a particular point in
history.

\section{A tour through history}

One of the first things we might want to do with a new, unfamiliar
repository is understand its history.  The \hgcmd{log} command gives
us a view of history.
\interaction{tour.log}
By default, this command prints a brief paragraph of output for each
change to the project that was recorded.  In Mercurial terminology, we
call each of these recorded events a \emph{changeset}, because it can
contain a record of changes to several files.

The fields in a record of output from \hgcmd{log} are as follows.
\begin{itemize}
\item[\texttt{changeset}] This field has the format of a number,
  followed by a colon, followed by a hexadecimal string.  These are
  \emph{identifiers} for the changeset.  There are two identifiers
  because the number is shorter and easier to type than the hex
  string.
\item[\texttt{user}] The identity of the person who created the
  changeset.  This is a free-form field, but it most often contains a
  person's name and email address.
\item[\texttt{date}] The date and time on which the changeset was
  created, and the timezone in which it was created.  (The date and
  time are local to that timezone; they display what time and date it
  was for the person who created the changeset.)
\item[\texttt{summary}] The first line of the text message that the
  creator of the changeset entered to describe the changeset.
\end{itemize}
The default output printed by \hgcmd{log} is purely a summary; it is
missing a lot of detail.

Figure~\ref{fig:tour-basic:history} provides a graphical representation of
the history of the \dirname{hello} repository, to make it a little
easier to see which direction history is ``flowing'' in.  We'll be
returning to this figure several times in this chapter and the chapter
that follows.

\begin{figure}[ht]
  \centering
  \grafix{tour-history}
  \caption{Graphical history of the \dirname{hello} repository}
  \label{fig:tour-basic:history}
\end{figure}

\subsection{Changesets, revisions, and talking to other 
  people}

As English is a notoriously sloppy language, and computer science has
a hallowed history of terminological confusion (why use one term when
four will do?), revision control has a variety of words and phrases
that mean the same thing.  If you are talking about Mercurial history
with other people, you will find that the word ``changeset'' is often
compressed to ``change'' or (when written) ``cset'', and sometimes a
changeset is referred to as a ``revision'' or a ``rev''.

While it doesn't matter what \emph{word} you use to refer to the
concept of ``a~changeset'', the \emph{identifier} that you use to
refer to ``a~\emph{specific} changeset'' is of great importance.
Recall that the \texttt{changeset} field in the output from
\hgcmd{log} identifies a changeset using both a number and a
hexadecimal string.
\begin{itemize}
\item The revision number is \emph{only valid in that repository},
\item while the hex string is the \emph{permanent, unchanging
    identifier} that will always identify that exact changeset in
  \emph{every} copy of the repository.
\end{itemize}
This distinction is important.  If you send someone an email talking
about ``revision~33'', there's a high likelihood that their
revision~33 will \emph{not be the same} as yours.  The reason for this
is that a revision number depends on the order in which changes
arrived in a repository, and there is no guarantee that the same
changes will happen in the same order in different repositories.
Three changes $a,b,c$ can easily appear in one repository as $0,1,2$,
while in another as $1,0,2$.

Mercurial uses revision numbers purely as a convenient shorthand.  If
you need to discuss a changeset with someone, or make a record of a
changeset for some other reason (for example, in a bug report), use
the hexadecimal identifier.

\subsection{Viewing specific revisions}

To narrow the output of \hgcmd{log} down to a single revision, use the
\hgopt{log}{-r} (or \hgopt{log}{--rev}) option.  You can use either a
revision number or a long-form changeset identifier, and you can
provide as many revisions as you want.  \interaction{tour.log-r}

If you want to see the history of several revisions without having to
list each one, you can use \emph{range notation}; this lets you
express the idea ``I want all revisions between $a$ and $b$,
inclusive''.
\interaction{tour.log.range}
Mercurial also honours the order in which you specify revisions, so
\hgcmdargs{log}{-r 2:4} prints $2,3,4$ while \hgcmdargs{log}{-r 4:2}
prints $4,3,2$.

\subsection{More detailed information}

While the summary information printed by \hgcmd{log} is useful if you
already know what you're looking for, you may need to see a complete
description of the change, or a list of the files changed, if you're
trying to decide whether a changeset is the one you're looking for.
The \hgcmd{log} command's \hggopt{-v} (or \hggopt{--verbose})
option gives you this extra detail.
\interaction{tour.log-v}

If you want to see both the description and content of a change, add
the \hgopt{log}{-p} (or \hgopt{log}{--patch}) option.  This displays
the content of a change as a \emph{unified diff} (if you've never seen
a unified diff before, see section~\ref{sec:mq:patch} for an overview).
\interaction{tour.log-vp}

\section{All about command options}

Let's take a brief break from exploring Mercurial commands to discuss
a pattern in the way that they work; you may find this useful to keep
in mind as we continue our tour.

Mercurial has a consistent and straightforward approach to dealing
with the options that you can pass to commands.  It follows the
conventions for options that are common to modern Linux and Unix
systems.
\begin{itemize}
\item Every option has a long name.  For example, as we've already
  seen, the \hgcmd{log} command accepts a \hgopt{log}{--rev} option.
\item Most options have short names, too.  Instead of
  \hgopt{log}{--rev}, we can use \hgopt{log}{-r}.  (The reason that
  some options don't have short names is that the options in question
  are rarely used.)
\item Long options start with two dashes (e.g.~\hgopt{log}{--rev}),
  while short options start with one (e.g.~\hgopt{log}{-r}).
\item Option naming and usage is consistent across commands.  For
  example, every command that lets you specify a changeset~ID or
  revision number accepts both \hgopt{log}{-r} and \hgopt{log}{--rev}
  arguments.
\end{itemize}
In the examples throughout this book, I use short options instead of
long.  This just reflects my own preference, so don't read anything
significant into it.

Most commands that print output of some kind will print more output
when passed a \hggopt{-v} (or \hggopt{--verbose}) option, and less
when passed \hggopt{-q} (or \hggopt{--quiet}).

\section{Making and reviewing changes}

Now that we have a grasp of viewing history in Mercurial, let's take a
look at making some changes and examining them.

The first thing we'll do is isolate our experiment in a repository of
its own.  We use the \hgcmd{clone} command, but we don't need to
clone a copy of the remote repository.  Since we already have a copy
of it locally, we can just clone that instead.  This is much faster
than cloning over the network, and cloning a local repository uses
less disk space in most cases, too.
\interaction{tour.reclone}
As an aside, it's often good practice to keep a ``pristine'' copy of a
remote repository around, which you can then make temporary clones of
to create sandboxes for each task you want to work on.  This lets you
work on multiple tasks in parallel, each isolated from the others
until it's complete and you're ready to integrate it back.  Because
local clones are so cheap, there's almost no overhead to cloning and
destroying repositories whenever you want.

In our \dirname{my-hello} repository, we have a file
\filename{hello.c} that contains the classic ``hello, world'' program.
Let's use the ancient and venerable \command{sed} command to edit this
file so that it prints a second line of output.  (I'm only using
\command{sed} to do this because it's easy to write a scripted example
this way.  Since you're not under the same constraint, you probably
won't want to use \command{sed}; simply use your preferred text editor to
do the same thing.)
\interaction{tour.sed}

Mercurial's \hgcmd{status} command will tell us what Mercurial knows
about the files in the repository.
\interaction{tour.status}
The \hgcmd{status} command prints no output for some files, but a line
starting with ``\texttt{M}'' for \filename{hello.c}.  Unless you tell
it to, \hgcmd{status} will not print any output for files that have
not been modified.  

The ``\texttt{M}'' indicates that Mercurial has noticed that we
modified \filename{hello.c}.  We didn't need to \emph{inform}
Mercurial that we were going to modify the file before we started, or
that we had modified the file after we were done; it was able to
figure this out itself.

It's a little bit helpful to know that we've modified
\filename{hello.c}, but we might prefer to know exactly \emph{what}
changes we've made to it.  To do this, we use the \hgcmd{diff}
command.
\interaction{tour.diff}

\section{Recording changes in a new changeset}

We can modify files, build and test our changes, and use
\hgcmd{status} and \hgcmd{diff} to review our changes, until we're
satisfied with what we've done and arrive at a natural stopping point
where we want to record our work in a new changeset.

The \hgcmd{commit} command lets us create a new changeset; we'll
usually refer to this as ``making a commit'' or ``committing''.  

\subsection{Setting up a username}

When you try to run \hgcmd{commit} for the first time, it is not
guaranteed to succeed.  Mercurial records your name and address with
each change that you commit, so that you and others will later be able
to tell who made each change.  Mercurial tries to automatically figure
out a sensible username to commit the change with.  It will attempt
each of the following methods, in order:
\begin{enumerate}
\item If you specify a \hgopt{commit}{-u} option to the \hgcmd{commit}
  command on the command line, followed by a username, this is always
  given the highest precedence.
\item If you have set the \envar{HGUSER} environment variable, this is
  checked next.
\item If you create a file in your home directory called
  \sfilename{.hgrc}, with a \rcitem{ui}{username} entry, that will be
  used next.  To see what the contents of this file should look like,
  refer to section~\ref{sec:tour-basic:username} below.
\item If you have set the \envar{EMAIL} environment variable, this
  will be used next.
\item Mercurial will query your system to find out your local user
  name and host name, and construct a username from these components.
  Since this often results in a username that is not very useful, it
  will print a warning if it has to do this.
\end{enumerate}
If all of these mechanisms fail, Mercurial will fail, printing an
error message.  In this case, it will not let you commit until you set
up a username.

You should think of the \envar{HGUSER} environment variable and the
\hgopt{commit}{-u} option to the \hgcmd{commit} command as ways to
\emph{override} Mercurial's default selection of username.  For normal
use, the simplest and most robust way to set a username for yourself
is by creating a \sfilename{.hgrc} file; see below for details.

\subsubsection{Creating a Mercurial configuration file}
\label{sec:tour-basic:username}

To set a user name, use your favourite editor to create a file called
\sfilename{.hgrc} in your home directory.  Mercurial will use this
file to look up your personalised configuration settings.  The initial
contents of your \sfilename{.hgrc} should look like this.
\begin{codesample2}
  # This is a Mercurial configuration file.
  [ui]
  username = Firstname Lastname <email.address@domain.net>
\end{codesample2}
The ``\texttt{[ui]}'' line begins a \emph{section} of the config file,
so you can read the ``\texttt{username = ...}'' line as meaning ``set
the value of the \texttt{username} item in the \texttt{ui} section''.
A section continues until a new section begins, or the end of the
file.  Mercurial ignores empty lines and treats any text from
``\texttt{\#}'' to the end of a line as a comment.

\subsubsection{Choosing a user name}

You can use any text you like as the value of the \texttt{username}
config item, since this information is for reading by other people,
but for interpreting by Mercurial.  The convention that most people
follow is to use their name and email address, as in the example
above.

\begin{note}
  Mercurial's built-in web server obfuscates email addresses, to make
  it more difficult for the email harvesting tools that spammers use.
  This reduces the likelihood that you'll start receiving more junk
  email if you publish a Mercurial repository on the web.
\end{note}

\subsection{Writing a commit message}

When we commit a change, Mercurial drops us into a text editor, to
enter a message that will describe the modifications we've made in
this changeset.  This is called the \emph{commit message}.  It will be
a record for readers of what we did and why, and it will be printed by
\hgcmd{log} after we've finished committing.
\interaction{tour.commit}

The editor that the \hgcmd{commit} command drops us into will contain
an empty line, followed by a number of lines starting with
``\texttt{HG:}''.
\begin{codesample2}
  \emph{empty line}
  HG: changed hello.c
\end{codesample2}
Mercurial ignores the lines that start with ``\texttt{HG:}''; it uses
them only to tell us which files it's recording changes to.  Modifying
or deleting these lines has no effect.

\subsection{Writing a good commit message}

Since \hgcmd{log} only prints the first line of a commit message by
default, it's best to write a commit message whose first line stands
alone.  Here's a real example of a commit message that \emph{doesn't}
follow this guideline, and hence has a summary that is not readable.
\begin{codesample2}
  changeset:   73:584af0e231be
  user:        Censored Person <censored.person@example.org>
  date:        Tue Sep 26 21:37:07 2006 -0700
  summary:     include buildmeister/commondefs.   Add an exports and install
\end{codesample2}

As far as the remainder of the contents of the commit message are
concerned, there are no hard-and-fast rules.  Mercurial itself doesn't
interpret or care about the contents of the commit message, though
your project may have policies that dictate a certain kind of
formatting.

My personal preference is for short, but informative, commit messages
that tell me something that I can't figure out with a quick glance at
the output of \hgcmdargs{log}{--patch}.

\subsection{Aborting a commit}

If you decide that you don't want to commit while in the middle of
editing a commit message, simply exit from your editor without saving
the file that it's editing.  This will cause nothing to happen to
either the repository or the working directory.

If we run the \hgcmd{commit} command without any arguments, it records
all of the changes we've made, as reported by \hgcmd{status} and
\hgcmd{diff}.

\subsection{Admiring our new handiwork}

Once we've finished the commit, we can use the \hgcmd{tip} command to
display the changeset we just created.  This command produces output
that is identical to \hgcmd{log}, but it only displays the newest
revision in the repository.
\interaction{tour.tip}
We refer to the newest revision in the repository as the tip revision,
or simply the tip.

\section{Sharing changes}

We mentioned earlier that repositories in Mercurial are
self-contained.  This means that the changeset we just created exists
only in our \dirname{my-hello} repository.  Let's look at a few ways
that we can propagate this change into other repositories.

\subsection{Pulling changes from another repository}
\label{sec:tour:pull}

To get started, let's clone our original \dirname{hello} repository,
which does not contain the change we just committed.  We'll call our
temporary repository \dirname{hello-pull}.
\interaction{tour.clone-pull}

We'll use the \hgcmd{pull} command to bring changes from
\dirname{my-hello} into \dirname{hello-pull}.  However, blindly
pulling unknown changes into a repository is a somewhat scary
prospect.  Mercurial provides the \hgcmd{incoming} command to tell us
what changes the \hgcmd{pull} command \emph{would} pull into the
repository, without actually pulling the changes in.
\interaction{tour.incoming}
(Of course, someone could cause more changesets to appear in the
repository that we ran \hgcmd{incoming} in, before we get a chance to
\hgcmd{pull} the changes, so that we could end up pulling changes that we
didn't expect.)

Bringing changes into a repository is a simple matter of running the
\hgcmd{pull} command, and telling it which repository to pull from.
\interaction{tour.pull}
As you can see from the before-and-after output of \hgcmd{tip}, we
have successfully pulled changes into our repository.  There remains
one step before we can see these changes in the working directory.

\subsection{Updating the working directory}

We have so far glossed over the relationship between a repository and
its working directory.  The \hgcmd{pull} command that we ran in
section~\ref{sec:tour:pull} brought changes into the repository, but
if we check, there's no sign of those changes in the working
directory.  This is because \hgcmd{pull} does not (by default) touch
the working directory.  Instead, we use the \hgcmd{update} command to
do this.
\interaction{tour.update}

It might seem a bit strange that \hgcmd{pull} doesn't update the
working directory automatically.  There's actually a good reason for
this: you can use \hgcmd{update} to update the working directory to
the state it was in at \emph{any revision} in the history of the
repository.  If you had the working directory updated to an old
revision---to hunt down the origin of a bug, say---and ran a
\hgcmd{pull} which automatically updated the working directory to a
new revision, you might not be terribly happy.

However, since pull-then-update is such a common thing to do,
Mercurial lets you combine the two by passing the \hgopt{pull}{-u}
option to \hgcmd{pull}.
\begin{codesample2}
  hg pull -u
\end{codesample2}
If you look back at the output of \hgcmd{pull} in
section~\ref{sec:tour:pull} when we ran it without \hgopt{pull}{-u},
you can see that it printed a helpful reminder that we'd have to take
an explicit step to update the working directory:
\begin{codesample2}
  (run 'hg update' to get a working copy)
\end{codesample2}

To find out what revision the working directory is at, use the
\hgcmd{parents} command.
\interaction{tour.parents}
If you look back at figure~\ref{fig:tour-basic:history}, you'll see
arrows connecting each changeset.  The node that the arrow leads
\emph{from} in each case is a parent, and the node that the arrow
leads \emph{to} is its child.  The working directory has a parent in
just the same way; this is the changeset that the working directory
currently contains.

To update the working directory to a particular revision, give a
revision number or changeset~ID to the \hgcmd{update} command.
\interaction{tour.older}
If you omit an explicit revision, \hgcmd{update} will update to the
tip revision, as shown by the second call to \hgcmd{update} in the
example above.

\subsection{Pushing changes to another repository}

Mercurial lets us push changes to another repository, from the
repository we're currently visiting.  As with the example of
\hgcmd{pull} above, we'll create a temporary repository to push our
changes into.
\interaction{tour.clone-push}
The \hgcmd{outgoing} command tells us what changes would be pushed
into another repository.
\interaction{tour.outgoing}
And the \hgcmd{push} command does the actual push.
\interaction{tour.push}
As with \hgcmd{pull}, the \hgcmd{push} command does not update the
working directory in the repository that it's pushing changes into.
(Unlike \hgcmd{pull}, \hgcmd{push} does not provide a \texttt{-u}
option that updates the other repository's working directory.)

What happens if we try to pull or push changes and the receiving
repository already has those changes?  Nothing too exciting.
\interaction{tour.push.nothing}

\subsection{Sharing changes over a network}

The commands we have covered in the previous few sections are not
limited to working with local repositories.  Each works in exactly the
same fashion over a network connection; simply pass in a URL instead
of a local path.
\interaction{tour.outgoing.net}
In this example, we can see what changes we could push to the remote
repository, but the repository is understandably not set up to let
anonymous users push to it.
\interaction{tour.push.net}

%%% Local Variables: 
%%% mode: latex
%%% TeX-master: "00book"
%%% End: 
