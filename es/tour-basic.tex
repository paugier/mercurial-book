\chapter{Una gira de Mercurial: lo básico}
\label{chap:tour-basic}

\section{Instalar Mercurial en su sistema}
\label{sec:tour:install}
Hay paquetes binarios precompilados de Mercurial disponibles para cada
sistema operativo popular. Esto hace fácil empezar a usar Mercurial
en su computador inmediatamente.

\subsection{Linux}

Dado que cada distribución de Linux tiene sus propias herramientas de
manejo de paquetes, políticas, y ritmos de desarrollo, es difícil dar
un conjunto exhaustivo de instrucciones sobre cómo instalar el paquete
de Mercurial. La versión de Mercurial que usted tenga a disposición
puede variar dependiendo de qué tan activa sea la persona que mantiene
el paquete para su distribución.

Para mantener las cosas simples, me enfocaré en instalar Mercurial
desde la línea de comandos en las distribuciones de Linux más
populares. La mayoría de estas distribuciones proveen administradores
de paquetes gráficos que le permitirán instalar Mercurial con un solo
clic; el nombre de paquete a buscar es \texttt{mercurial}.

\begin{itemize}
\item[Debian]
  \begin{codesample4}
    apt-get install mercurial
  \end{codesample4}

\item[Fedora Core]
  \begin{codesample4}
    yum install mercurial
  \end{codesample4}

\item[Gentoo]
  \begin{codesample4}
    emerge mercurial
  \end{codesample4}

\item[OpenSUSE]
  \begin{codesample4}
    yum install mercurial
  \end{codesample4}

\item[Ubuntu] El paquete de Mercurial de Ubuntu está basado en el de
    Debian. Para instalarlo, ejecute el siguiente comando.
  \begin{codesample4}
    apt-get install mercurial
  \end{codesample4}
  El paquete de Mercurial para Ubuntu tiende a atrasarse con respecto
  a la versión de Debian por un margen de tiempo considerable
  (al momento de escribir esto, 7 meses), lo que en algunos casos
  significará que usted puede encontrarse con problemas que ya habrán
  sido resueltos en el paquete de Debian.
\end{itemize}

\subsection{Solaris}

SunFreeWare, en \url{http://www.sunfreeware.com}, es una buena fuente
para un gran número de paquetes compilados para Solaris para las
arquitecturas Intel y Sparc de 32 y 64 bits, incluyendo versiones
actuales de Mercurial.

\subsection{Mac OS X}

Lee Cantey publica un instalador de Mercurial para Mac OS~X en 
\url{http://mercurial.berkwood.com}.  Este paquete funciona en tanto
en Macs basados en Intel como basados en PowerPC. Antes de que pueda
usarlo, usted debe instalar una versión compatible de Universal
MacPython~\cite{web:macpython}. Esto es fácil de hacer; simplemente
siga las instrucciones del sitio de Lee.

También es posible instalar Mercurial usando Fink o MacPorts, dos
administradores de paquetes gratuitos y populares para Mac OS X. Si
usted tiene Fink, use \command{sudo apt-get install mercurial-py25}.
Si usa MacPorts, \command{sudo port install mercurial}.

\subsection{Windows}

Lee Cantey publica un instalador de Mercurial para Windows en
\url{http://mercurial.berkwood.com}. Este paquete no tiene
% TODO traducción de it just works. Agreed?
dependencias externas; ``simplemente funciona''.

\begin{note}
    La versión de Windows de Mercurial no convierte automáticamente
    los fines de línea entre estilos Windows y Unix. Si usted desea
    compartir trabajo con usuarios de Unix, deberá hacer un trabajo
    adicional de configuración. XXX Terminar esto.
\end{note}

\section{Arrancando}

Para empezar, usaremos el comando \hgcmd{version} para revisar si
Mercurial está instalado adecuadamente. La información de la versión
que es impresa no es tan importante; lo que nos importa es si imprime
algo en absoluto.

\interaction{tour.version}

% TODO builtin-> integrado?
\subsection{Ayuda integrada}

Mercurial provee un sistema de ayuda integrada. Esto es invaluable
para ésas ocasiones en la que usted está atorado tratando de recordar
cómo ejecutar un comando. Si está completamente atorado, simplemente
ejecute \hgcmd{help}; esto imprimirá una breve lista de comandos,
junto con una descripción de qué hace cada uno. Si usted solicita
ayuda sobre un comando específico (como abajo), se imprime información
más detallada.
\interaction{tour.help}
Para un nivel más impresionante de detalle (que usted no va a
necesitar usualmente) ejecute \hgcmdargs{help}{\hggopt{-v}}. La opción
\hggopt{-v} es la abreviación para \hggopt{--verbose}, y le indica a
Mercurial que imprima más información de lo que haría usualmente.

\section{Trabajar con un repositorio}

En Mercurial, todo sucede dentro de un \emph{repositorio}. El
repositorio para un proyecto contiene todos los ficheros que
``pertenecen a'' ése proyecto, junto con un registro histórico de los
ficheros de ese proyecto.

No hay nada particularmente mágico acerca de un repositorio; es
simplemente un árbol de directorios en su sistema de ficheros que
Mercurial trata como especial. Usted puede renombrar o borrar un
repositorio en el momento que lo desee, usando bien sea la línea de
comandos o su explorador de ficheros.

\subsection{Hacer una copia local de un repositorio}

\emph{Copiar} un repositorio es sólo ligeramente especial. Aunque
usted podría usar un programa normal de copia de ficheros para hacer
una copia del repositorio, es mejor usar el comando integrado que
Mercurial ofrece. Este comando se llama \hgcmd{clone}\ndt{Del término
``clonar'' en inglés.}, porque crea una copia idéntica de un
repositorio existente.
\interaction{tour.clone}
Si nuestro clonado tiene éxito, deberíamos tener un directorio local
llamado \dirname{hello}. Este directorio contendrá algunos ficheros.
\interaction{tour.ls}
Estos ficheros tienen el mismo contenido e historial en nuestro
repositorio y en el repositorio que clonamos.

Cada repositorio Mercurial está completo, es autocontenido e
independiente. Contiene su propia copia de los ficheros y la historia
de un proyecto. Un repositorio clonado recuerda la ubicación de la que
fue clonado, pero no se comunica con ese repositorio, ni con ningún
otro, a menos que usted le indique que lo haga.

Lo que esto significa por ahora es que somos libres de experimentar
con nuestro repositorio, con la tranquilidad de saber que es una
% TODO figure out what to say instead of sandbox
``caja de arena'' privada que no afectará a nadie más.

\subsection{Qué hay en un repositorio?}

Cuando miramos en detalle dentro de un repositorio, podemos ver que
contiene un directorio llamado \dirname{.hg}. Aquí es donde Mercurial
mantiene todos los metadatos del repositorio.
\interaction{tour.ls-a}

Los contenidos del directorio \dirname{.hg} y sus subdirectorios son
exclusivos de Mercurial. Usted es libre de hacer lo que desee con
cualquier otro fichero o directorio en el repositorio.

Para introducir algo de terminología, el directorio \dirname{.hg} es
el repositorio ``real'', y todos los ficheros y directorios que
coexisten con él están en el \emph{directorio de trabajo}. Una forma
sencilla de recordar esta distinción es que el \emph{repositorio}
% TODO unificar con Igor, si historia o historial
contiene el \emph{historial} de su proyecto, mientras que el
\emph{directorio de trabajo} contiene una \emph{instantánea} de su
proyecto en un punto particular del historial.

\section{Vistazo rápido al historial}

Una de las primeras cosas que se desea hacer con un repositorio nuevo,
poco conocido, es conocer su historial. El comando \hgcmd{log} nos
permite ver el mismo.
\interaction{tour.log}
Por defecto este programa imprime un párrafo breve por cada cambio al
proyecto que haya sido grabado. Dentro de la terminología de
Mercurial, cada uno de estos eventos es llamado \emph{conjunto de
cambios}, porque pueden contener un registro de cambios a varios
ficheros.

Los campos de la salida de \hgcmd{log} son los siguientes.
\begin{itemize}
    \item[\texttt{changeset}]\hspace{-0.5em}\ndt{Conjunto de cambios.} Este campo
        tiene un número, seguido por un
        % TODO digo mejor seguido por un dos puntos ? string =>
        % cadena?
        \texttt{:}, seguido por una cadena hexadecimal. Ambos son
        \emph{identificadores} para el conjunto de cambios. Hay dos
        identificadores porque el número es más corto y más fácil de
        recordar que la cadena hexadecimal.
        
\item[\texttt{user}]\hspace{-0.5em}\ndt{Usuario.} La identidad de la
    persona que creó el conjunto de cambios. Este es un campo en el
    que se puede almacenar cualquier valor, pero en la mayoría de los
    casos contiene el nombre de una persona y su dirección de correo
    electrónico.
    
\item[\texttt{date}]\hspace{-0.5em}\ndt{Fecha.} La fecha y hora en la
    que el conjunto de cambios fue creado, y la zona horaria en la que
    fue creado. (La fecha y hora son locales a dicha zona horaria;
    ambos muestran la fecha y hora para la persona que creó el
    conjunto de cambios).
    
\item[\texttt{summary}]\hspace{-0.5em}\ndt{Sumario.} 
    La primera línea del texto que usó la persona que creó el conjunto
    de cambios para describir el mismo.
\end{itemize}
El texto impreso por \hgcmd{log} es sólo un sumario; omite una gran
cantidad de detalles.

La figura~\ref{fig:tour-basic:history} es una representación
gráfica del historial del repositorio \dirname{hello}, para hacer más
fácil ver en qué dirección está ``fluyendo'' el historial. Volveremos
a esto varias veces en este capítulo y en los siguientes.

\begin{figure}[ht]
  \centering
  \grafix{tour-history}
  \caption{Historial gráfico del repositorio \dirname{hello}}
  \label{fig:tour-basic:history}
\end{figure}

\subsection{Conjuntos de cambios, revisiones, y comunicándose con
otras personas}

%TODO sloppy => desordenado ?  TODO hablar del inglés? o de español?
Ya que el inglés es un lenguaje notablemente desordenado, y el área de
ciencias de la computación tiene una notable historia de confusión de
% TODO insertar ? al revés. no sé cómo en un teclado de estos.
términos (porqué usar sólo un término cuando cuatro pueden servir?),
el control de revisiones tiene una variedad de frases y palabras que
tienen el mismo significado. Si usted habla acerca del historial de
Mercurial con alguien, encontrará que la expresión ``conjunto de
cambios'' es abreviada a menudo como ``cambio'' o (por escrito)
``cset''\ndt{Abreviatura para la expresión ``changeset'' en inglés.},
y algunas veces un se hace referencia a un conjunto de cambios como
una ``revisión'' o ``rev''\ndt{De nuevo, como abreviación para el
término en inglés para ``revisión'' (``revision'').}.

Si bien no es relevante qué \emph{palabra} use usted para referirse al
concepto ``conjunto de cambios'', el \emph{identificador} que usted
use para referise a ``un \emph{conjunto de cambios} particular'' es
muy importante. Recuerde que el campo \texttt{changeset} en la salida
de \hgcmd{log} identifica un conjunto de cambios usando tanto un
número como una cadena hexadecimal.

\begin{itemize}
    \item El número de revisión \emph{sólo es válido dentro del
        repositorio}.
    \item Por otro lado, la cadena hexadecimal es el
        \emph{identificador permanente e inmutable} que siempre
        identificará ése conjunto de cambios en \emph{todas} las
        copias del repositorio.
\end{itemize}
La diferencia es importante. Si usted le envía a alguien un correo
electrónico hablando acerca de la ``revisión~33'', hay una
probabilidad alta de que la revisión~33 de esa persona \emph{no sea la
misma suya}. Esto sucede porque el número de revisión depende del
orden en que llegan los cambios al repositorio, y no hay ninguna
garantía de que los mismos cambios llegarán en el mismo orden en
diferentes repositorios. Tres cambios dados $a,b,c$ pueden aparecer en
un repositorio como $0,1,2$, mientras que en otro aparecen como
$1,0,2$.

Mercurial usa los números de revisión simplemente como una abreviación
conveniente. Si usted necesita hablar con alguien acerca de un
conjunto de cambios, o llevar el registro de un conjunto de cambios
por alguna otra razón (por ejemplo, en un reporte de fallo), use el
identificador hexadecimal.

\subsection{Ver revisiones específicas}

Para reducir la salida de \hgcmd{log} a una sola revisión, use la  
opción \hgopt{log}{-r} (o \hgopt{log}{--rev}).  Puede usar un número
de revisión o un identificador hexadecimal de conjunto de cambios, y
puede pasar tantas revisiones como desee.

\interaction{tour.log-r}

Si desea ver el historial de varias revisiones sin tener que mencionar
cada una de ellas, puede usar la \emph{notación de rango}; esto le
permite expresar el concepto ``quiero ver todas las revisiones entre
$a$ y $b$, inclusive''.
\interaction{tour.log.range}
Mercurial también respeta el orden en que usted especifica las
revisiones, así que \hgcmdargs{log}{-r 2:4} muestra $2,3,4$ mientras
que \hgcmdargs{log}{-r 4:2} muestra $4,3,2$.

\subsection{Información más detallada}
Aunque la información presentada por \hgcmd{log} es útil si usted sabe
de antemano qué está buscando, puede que necesite ver una descripción
completa del cambio, o una lista de los ficheros que cambiaron, si
está tratando de averiguar si un conjunto de cambios dado es el que
usted está buscando. La opción \hggopt{-v} (or \hggopt{--verbose}) del
comando \hgcmd{log} le da este nivel extra de detalle.
\interaction{tour.log-v}

Si desea ver tanto la descripción como el contenido de un cambio,
añada la opción \hgopt{log}{-p} (o \hgopt{log}{--patch}). Esto muestra
% TODO qué hacemos con diff unificado? convervarlo, por ser la
% acepción usual?
el contenido de un cambio como un \emph{diff unificado} (si usted
nunca ha visto un diff unificado antes, vea la
sección~\ref{sec:mq:patch} para un vistazo global).
\interaction{tour.log-vp}

\section{Todo acerca de las opciones para comandos}

Tomemos un breve descanso de la tarea de explorar los comandos de
Mercurial para hablar de un patrón en la manera en que trabajan; será
útil tener esto en mente a medida que avanza nuestra gira.

Mercurial tiene un enfoque directo y consistente en el manejo de las
opciones que usted le puede pasar a los comandos. Se siguen las
convenciones para opciones que son comunes en sistemas Linux y Unix
modernos.
\begin{itemize}
\item Cada opción tiene un nombre largo. Por ejemplo, el comando
    \hgcmd{log} acepta la opción \hgopt{log}{--rev}, como ya hemos
    visto.
\item Muchas opciones tienen también un nombre corto. En vez de
    \hgopt{log}{--rev}, podemos usar \hgopt{log}{-r}.  (El motivo para
    que algunas opciones no tengan nombres cortos es que dichas
    opciones se usan rara vez.)
\item Las opciones largas empiezan con dos guiones (p.ej.~\hgopt{log}{--rev}),
    mientras que las opciones cortas empiezan con uno (e.g.~\hgopt{log}{-r}).
\item El nombre  y uso de las opciones es consistente en todos los
    comandos. Por ejemplo, cada comando que le permite pasar un ID de
    conjunto de cambios o un número de revisión acepta tanto la opción
    \hgopt{log}{-r} como la \hgopt{log}{--rev}.
\end{itemize}
En los ejemplos en este libro, uso las opciones cortas en vez de las
largas. Esto sólo muestra mis preferencias, así que no le dé
significado especial a eso.

Muchos de los comandos que generan salida de algún tipo mostrarán más
salida cuando se les pase la opción \hggopt{-v} (o
\hggopt{--verbose}\ndt{Prolijo.}), y menos cuando se les pase la opción \hggopt{-q}
(o \hggopt{--quiet}\ndt{Silencioso.}).

\section{Hacer y repasar cambios}

Ahora que tenemos una comprensión adecuada sobre cómo revisar el
historial en Mercurial, hagamos algunos cambios y veamos cómo
examinarlos.

Lo primero que haremos será aislar nuestro experimento en un
repositorio propio. Usaremos el comando \hgcmd{clone}, pero no hace
falta clonar una copia del repositorio remoto. Como ya contamos con
una copia local del mismo, podemos clonar esa. Esto es mucho más
rápido que clonar a través de la red, y en la mayoría de los casos
clonar un repositorio local usa menos espacio en disco también.
\interaction{tour.reclone}
A manera de recomendación, es considerado buena práctica mantener una
copia ``prístina'' de un repositorio remoto a mano, del cual usted
puede hacer clones temporales para crear cajas de arena para cada
tarea en la que desee trabajar. Esto le permite trabajar en múltiples
tareas en paralelo, teniendo cada una de ellas aislada de las otras
hasta que estén completas y usted esté listo para integrar los cambios
de vuelta. Como los clones locales son tan baratos, clonar y destruir
repositorios no consume demasiados recursos, lo que facilita hacerlo
en cualquier momento.

En nuestro repositorio \dirname{my-hello}, hay un fichero
\filename{hello.c} que contiene el clásico programa ``hello,
world''\ndt{Hola, mundo.}. Usaremos el clásico y venerado comando
\command{sed} para editar este fichero y hacer que imprima una segunda
línea de salida. (Estoy usando el comando \command{sed} para hacer
esto sólo porque es fácil escribir un ejemplo automatizado con él.
Dado que usted no tiene esta restricción, probablemente no querrá usar
\command{sed}; use su editor de texto preferido para hacer lo mismo).
\interaction{tour.sed}

El comando \hgcmd{status} de Mercurial nos dice lo que Mercurial sabe
acerca de los ficheros en el repositorio.
\interaction{tour.status}
El comando \hgcmd{status} no imprime nada para algunos ficheros, sólo
una línea empezando con ``\texttt{M}'' para el fichero
\filename{hello.c}. A menos que usted lo indique explícitamente,
\hgcmd{status} no imprimirá nada respecto a los ficheros que no han
sido modificados.

La ``\texttt{M}'' indica que Mercurial se dio cuenta de que nosotros
modificamos \filename{hello.c}.  No tuvimos que \emph{decirle} a
Mercurial que íbamos a modificar ese fichero antes de hacerlo, o que
lo modificamos una vez terminamos de hacerlo; él fue capaz de darse
cuenta de esto por sí mismo.

Es algo útil saber que hemos modificado el fichero \filename{hello.c},
pero preferiríamos saber exactamente \emph{qué} cambios hicimos.
Para averiguar esto, usamos el comando \hgcmd{diff}.
\interaction{tour.diff}

\section{Grabar cambios en un nuevo conjunto de cambios}

Podemos modificar, compilar y probar nuestros cambios, y usar
\hgcmd{status} y \hgcmd{diff} para revisar los mismos, hasta que
estemos satisfechos con los resultados y lleguemos a un momento en el
que sea natural que querramos guardar nuestro trabajo en un nuevo
conjunto de cambios.

El comando \hgcmd{commit} nos permite crear un nuevo conjunto de
cambios. Nos referiremos usualmente a esto como ``hacer una consigna''
o consignar.

\subsection{Definir un nombre de usuario}

Cuando usted trata de ejecutar \hgcmd{commit}\ndt{Hacer una
consignación} por primera vez, no está garantizado que lo logre.
Mercurial registra su nombre y dirección en cada cambio que usted
consigna, para que más adelante otros puedan saber quién es el
responsable de cada cambio. Mercurial trata de encontrar un nombre de
% TODO consigna o consignación?
usuario adecuado con el cual registrar la consignación. Se intenta con
cada uno de los siguientes métodos, en el orden presentado.
\begin{enumerate}
\item Si usted pasa la opción \hgopt{commit}{-u} al comando \hgcmd{commit}
  en la línea de comandos, seguido de un nombre de usuario, se le da a
  esto la máxima precedencia.
\item A continuación se revisa si usted ha definido la variable de
    entorno \envar{HGUSER}.
\item Si usted crea un fichero en su directorio personal llamado
  \sfilename{.hgrc}, con una entrada \rcitem{ui}{username}, se usa
  luego. Para revisar cómo debe verse este fichero, refiérase a la
  sección~\ref{sec:tour-basic:username} más abajo.
\item Si usted ha definido la variable de entorno \envar{EMAIL}, será
    usada a continuación.
\item Mercurial le pedirá a su sistema buscar su nombre de usuario
    % TODO host => máquina
    local, y el nombre de máquina, y construirá un nombre de usuario a
    partir de estos componentes. Ya que esto generalmente termina
    generando un nombre de usuario no muy útil, se imprimirá una
    advertencia si es necesario hacerlo.
\end{enumerate}
Si todos estos procedimientos fallan, Mercurial fallará, e imprimirá
un mensaje de error. En este caso, no le permitirá hacer la
consignación hasta que usted defina un nombre de usuario.

Trate de ver la variable de entorno \envar{HGUSER} y la opción
\hgopt{commit}{-u} del comando \hgcmd{commit} como formas de
\emph{hacer caso omiso} de la selección de nombre de usuario que
Mercurial hace normalmente.  Para uso normal, la manera más simple y
sencilla de definir un nombre de usuario para usted es crear un
fichero \sfilename{.hgrc}; los detalles se encuentran más adelante.

\subsubsection{Crear el fichero de configuración de Mercurial}
\label{sec:tour-basic:username}

Para definir un nombre de usuario, use su editor de texto favorito
para crear un fichero llamado \sfilename{.hgrc} en su directorio
personal. Mercurial usará este fichero para obtener las
configuraciones personalizadas que usted haya hecho. El contenido
inicial de su fichero \sfilename{.hgrc} debería verse así.
\begin{codesample2}
  # Este es un fichero de configuración de Mercurial.
  [ui]
  username = Primernombre Apellido <correo.electronico@dominio.net>
\end{codesample2}
La línea ``\texttt{[ui]}'' define una \emph{section} del fichero de
configuración, así que usted puede leer la línea ``\texttt{username =
...}'' como ``defina el valor del elemento \texttt{username} en la
sección \texttt{ui}''.
Una sección continua hasta que empieza otra nueva, o se llega al final
del fichero. Mercurial ignora las líneas vacías y considera cualquier
texto desde el caracter ``\texttt{\#}'' hasta el final de la línea
como un comentario.

\subsubsection{Escoger un nombre de usuario}

Usted puede usar el texto que desee como el valor del campo de
configuración \texttt{username}, ya que esta información será leída
por otras personas, e interpretada por Mercurial. La convención que
sigue la mayoría de la gente es usar su nombre y dirección de correo,
como en el ejemplo anterior.

\begin{note}
    % TODO web
    El servidor web integrado de Mercurial ofusca las direcciones de
    correo, para dificultar la tarea de las herramientas de
    recolección de direcciones de correo que usan los
    spammers\ndt{Personas que envían correo no solicitado, también
    conocido como correo basura}. Esto reduce la probabilidad de que
    usted empiece a recibir más correo basura si publica un
    repositorio en la red.
\end{note}

\subsection{Escribir un mensaje de consignación}

Cuando consignamos un cambio, Mercurial nos ubica dentro de un editor
de texto, para ingresar un mensaje que describa las modificaciones que
hemos introducido en este conjunto de cambios. Esto es conocido como
un \emph{mensaje de consignación}. Será un registro de lo que hicimos
y porqué lo hicimos, y será impreso por \hgcmd{log} una vez hayamos
hecho la consignación.
\interaction{tour.commit}

El editor en que \hgcmd{commit} nos ubica contendrá una línea vacía,
seguida de varias líneas que empiezan con la cadena ``\texttt{HG:}''.
\begin{codesample2}
  \emph{línea vacía}
  HG: changed hello.c
\end{codesample2}
Mercurial ignora las líneas que empiezan con ``\texttt{HG:}''; sólo
las usa para indicarnos para cuáles ficheros está registrando los
cambios. Modificar o borrar estas líneas no tiene ningún efecto.

\subsection{Escribir un buen mensaje de consignación}

Ya que por defecto \hgcmd{log} sólo muestra la primera línea de un
mensaje de consignación, lo mejor es escribir un mensaje cuya primera
línea tenga significado por sí misma. A continuación se encuentra un
ejemplo de un mensaje de consignación que \emph{no} sigue esta
pauta, y debido a ello tiene un sumario que no es legible.
\begin{codesample2}
  changeset:   73:584af0e231be
  user:        Persona Censurada <persona.censurada@ejemplo.org>
  date:        Tue Sep 26 21:37:07 2006 -0700
  summary:     se incluye buildmeister/commondefs.   Añade un módulo
\end{codesample2}

Con respecto al resto del contenido del mensaje de consignación, no
hay reglas estrictas-y-rápidas. Mercurial no interpreta ni le da
importancia a los contenidos del mensaje de consignación, aunque es
posible que su proyecto tenga políticas que definan una manera
particular de escribirlo.

Mi preferencia personal es usar mensajes de consignación cortos pero
informativos, que me digan algo que no puedo inferir con una mirada
rápida a la salida de \hgcmdargs{log}{--patch}.

\subsection{Cancelar una consignación}

Si usted decide que no desea hacer la consignación mientras está
editando el mensaje de la misma, simplemente cierre su editor sin
guardar los cambios al fichero que está editando. Esto hará que no
pase nada ni en el repositorio ni en el directorio de trabajo.

Si ejecutamos el comando \hgcmd{commit} sin ningún argumento, se
registran todos los cambios que hemos hecho, como lo indican
\hgcmd{status} y \hgcmd{diff}.

\subsection{Admirar nuestro trabajo}

Una vez hemos terminado la consignación, podemos usar el comando
\hgcmd{tip}\ndt{Punta.} para mostrar el conjunto de cambios que acabamos de crear.
La salida de este comando es idéntica a la de \hgcmd{log}, pero sólo
muestra la revisión más reciente en el repositorio.
\interaction{tour.tip}
Nos referimos a la revisión más reciente en el repositorio como la
revisión de punta, o simplemente la punta.

\section{Compartir cambios}

Anteriormente mencionamos que los repositorios en Mercurial están auto
contenidos. Esto quiere decir que el conjunto de cambios que acabamos
de crear sólo existe en nuestro repositorio \dirname{my-hello}. Veamos
unas cuantas formas de propagar este cambio a otros repositorios.

\subsection{Jalar cambios desde otro repositorio}
\label{sec:tour:pull}

Para empezar, clonemos nuestro repositorio \dirname{hello} original,
el cual no contiene el cambio que acabamos de consignar. Llamaremos a
este repositorio temporal \dirname{hello-pull}.
\interaction{tour.clone-pull}

Usaremos el comando \hgcmd{pull} para traer los cambios de
\dirname{my-hello} y ponerlos en \dirname{hello-pull}.  Sin embargo,
traer cambios desconocidos y aplicarlos en un repositorio es una
perspectiva que asusta al menos un poco.  Mercurial cuenta con el
comando \hgcmd{incoming}\ndt{Entrante, o cambios entrantes.} para
decirnos qué cambios \emph{jalaría} el comando \hgcmd{pull} al
repositorio, sin jalarlos.
\interaction{tour.incoming}
(Por supuesto, alguien podría enviar más conjuntos de cambios al
repositorio en el tiempo que pasa entre la ejecución de
\hgcmd{incoming} y la ejecución de \hgcmd{pull} para jalar los
cambios, así que es posible que terminemos jalando cambios que no
esperábamos.)

Traer cambios al repositorio simplemente es cuestión de ejecutar el
comando \hgcmd{pull}, indicándole de qué repositorio debe jalarlos.
\interaction{tour.pull}
Como puede verse por las salidas antes-y-después de \hgcmd{tip}, hemos
jalado exitosamente los cambios en nuestro repositorio. Aún falta un
paso para que podamos ver estos cambios en nuestro directorio de
trabajo.

\subsection{Actualizar el directorio de trabajo}

Hasta ahora hemos pasado por alto la relación entre un repositorio y
su directorio de trabajo. El comando \hgcmd{pull} que ejecutamos en la
sección~\ref{sec:tour:pull} trajo los cambios al repositorio, pero si
revisamos, no hay rastro de esos cambios en el directorio de trabajo.
Esto pasa porque \hgcmd{pull} (por defecto) no modifica el directorio de
trabajo. En vez de eso, usamos el comando
\hgcmd{update}\ndt{Actualizar.} para hacerlo.
\interaction{tour.update}

Puede parecer algo raro que \hgcmd{pull} no actualice el directorio de
trabajo automáticamente. De hecho, hay una buena razón para esto:
usted puede usar \hgcmd{update} para actualizar el directorio de
trabajo al estado en que se encontraba en \emph{cualquier revisión}
del historial del repositorio. Si usted hubiera actualizado el
directorio de trabajo a una revisión anterior---digamos, para buscar
el origen de un fallo---y hubiera corrido un \hgcmd{pull} que hubiera
actualizado el directorio de trabajo automáticamente a la nueva
revisión, puede que no estuviera particularmente contento.

Sin embargo, como jalar-y-actualizar es una secuencia de operaciones
muy común, Mercurial le permite combinarlas al pasar la opción
\hgopt{pull}{-u}
a \hgcmd{pull}.
\begin{codesample2}
  hg pull -u
\end{codesample2}
Si mira de vuelta la salida de \hgcmd{pull} en la
sección~\ref{sec:tour:pull} cuando lo ejecutamos sin la opción \hgopt{pull}{-u},
verá que el comando imprimió un amable recordatorio de que tenemos que
encargarnos explícitamente de actualizar el directorio de trabajo:
\begin{codesample2}
  (run 'hg update' to get a working copy)
\end{codesample2}

Para averiguar en qué revisión se encuentra el directorio de trabajo,
use el comando \hgcmd{parents}.
\interaction{tour.parents}
Si mira de nuevo la figura~\ref{fig:tour-basic:history}, verá flechas
conectando cada conjunto de cambios. En cada caso, el nodo del que la flecha
\emph{sale} es un padre, y el nodo al que la flecha \emph{llega} es 
su hijo. El directorio de trabajo tiene un padre exactamente de la
misma manera; ése es el conjunto de cambios que contiene actualmente
el directorio de trabajo.

Para actualizar el conjunto de trabajo a una revisión particular, pase
un número de revisión o un ID de conjunto de cambios al comando
\hgcmd{update}.
\interaction{tour.older}
Si no indica explícitamente una revisión, \hgcmd{update} actualizará
hasta la revisión de punta, como se vio en la segunda llamada a
\hgcmd{update} en el ejemplo anterior.

\subsection{Empujar cambios a otro repositorio}

Mercurial nos permite empujar cambios a otro repositorio, desde el
% TODO cambié "visitando" por "usando"
repositorio que estemos usando actualmente. De la misma forma que en
el ejemplo de \hgcmd{pull} arriba, crearemos un repositorio temporal
para empujar allí nuestros cambios.
\interaction{tour.clone-push}
El comando \hgcmd{outgoing}\ndt{Saliente. Cambios salientes.} nos dice
qué cambios serían empujados en el otro repositorio.
\interaction{tour.outgoing}
Y el comando \hgcmd{push} se encarga de empujar dichos cambios.
\interaction{tour.push}
Al igual que \hgcmd{pull}, el comando \hgcmd{push} no actualiza el
directorio de trabajo del repositorio en el que estamos empujando los
cambios.  (A diferencia de \hgcmd{pull}, \hgcmd{push} no ofrece la
opción \texttt{-u} para actualizar el directorio de trabajo del otro
repositorio.)

% TODO poner interrogante de apertura
Qué pasa si tratamos de jalar o empujar cambios y el repositorio
receptor ya tiene esos cambios? Nada emocionante.
\interaction{tour.push.nothing}

\subsection{Compartir cambios a través de una red}

Los comandos que hemos presentando en las pocas secciones anteriores
no están limitados a trabajar con repositorios locales. Cada uno de
ellos funciona exactamente de la misma manera a través de una conexión
% TODO poner ndt para URL
de red. Simplemente pase una URL en vez de una ruta local.
\interaction{tour.outgoing.net}
En este ejemplo, podemos ver qué cambios empujaríamos al repositorio
remoto, aunque, de manera entendible, el repositorio remoto está
configurado para no permitir a usuarios anónimos empujar cambios a él.
\interaction{tour.push.net}

%%% Local Variables: 
%%% mode: latex
%%% TeX-master: "00book"
%%% End: 
