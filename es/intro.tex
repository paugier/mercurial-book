\chapter{Introducción}
\label{chap:intro}

\section{Acerca del control de revisiones}

El control de revisiones es el proceso de administrar diferentes
versiones de una pieza de información. En su forma más simple es algo
que la mayoría de gente hace a mano: cada vez que usted modifica un
fichero, lo graba con un nuevo nombre que contiene un número, el
siguiente mayor que el anterior.

Administrar manualmente muchas versiones de un fichero es una tarea
propensa a errores, a pesar de que hace bastante tiempo hay
herramientas que ayudan en este proceso.  Las primeras herramientas
para automatizar el control de revisiones fueron pensadas para que un
usuario administrara un solo fichero.  En las décadas pasadas, el
alcance de las herramientas de control de revisiones ha ido aumentando
considerablemente; ahora manejan muchos ficheros y facilitan el
trabajo en conjunto de varias personas. Las mejores herramientas de
control de revisiones de la actualidad no tienen problema con miles de
personas trabajando en proyectos que consisten de decenas de miles de
ficheros.

\subsection{¿Por qué usar control de revisiones?}

Hay muchas razones por las cuales usted o su equipo desearía usar una
herramienta automática de control de revisiones para un proyecto.
\begin{itemize}
\item Contar con la historia y la evolución de su proyecto, para
  evitar hacer la tarea manualmente. Por cada cambio tendrá una
  bitácora de \emph{quién} lo hizo; \emph{por qué} se hizo;
  \emph{cuándo} se hizo; y de \emph{qué} se trataba el cambio.
\item Cuando trabaja con más personas, los programas de control de
  revisiones facilitan la colaboración.  Por ejemplo, cuando varias
  personas de forma casi simultanea pueden hacer cambios
  incompatibles, el programa le ayudará a identificar y resolver tales
  conflictos.
\item Puede ayudarle a recuperarse de equivocaciones. Si aplica un
  cambio que posteriormente se evidencia como un error, puede
  revertirlo a una versión previa a uno o muchos ficheros. De hecho,
  una herramienta \emph{realmente} buena, incluso puede ayudarle
  efectivamente a darse cuenta exactamente cuándo se introdujo el
  error( para más detalles ver la sección~\ref{sec:undo:bisect}).
\item Le permitirá trabajar simultáneamente, y manejar las diferencias
  entre múltiples versiones de su proyecto.
\end{itemize}
La mayoría de estas razones son igualmente validas ---por lo menos en
teoría--- así esté trabajando en un proyecto solo, o con mucha gente.

Algo fundamental acerca de lo práctico de un sistema de control de
revisiones en estas dos escalas (``un hacker solo'' y ``un equipo
gigantesco'') es cómo se comparan los \emph{beneficios} con los
\emph{costos}.  Una herramienta de control de revisiones que sea
difícil de entender o usar impondrá un costo alto.

Un proyecto de quinientas personas es muy propenso a colapsar
solamente con su peso inmediatamente sin una herramienta de control de
versiones y un proceso. En este caso, el costo de usar control de
revisiones ni siquiera se tiene en cueant, puesto que \emph{sin} él,
el fracaso está casi garantizado.

Por otra parte, un ``arreglo rápido'' de una sola persona, excluiría
la necesidad de usar una herramienta de control de revisiones, porque
casi seguramente, el costo de usar una estaría cerca del costo del
proyecto. ¿No es así?

Mercurial solamente soporta \emph{ambas} escalas de de
desarrollo. Puede aprender lo básico en pocos minutos, y dado su bajo
sobrecosto, puede aplicar el control de revisiones al proyecto más
pequeño con facilidad. Su simplicidad significa que no tendrá que
preocuparse por conceptos obtusos o secuencias de órdenes compitiendo
por espacio mental con lo que sea que \emph{realmente} esté tratando
de hacer.  Al mismo tiempo, Mercurial tiene alto desempeño y su
naturaleza peer-to-peer le permite escalar indoloramente para manejar
grandes proyectos.

Ninguna herramienta de control de revisiones puede salvar un
proyecto mal administrado, pero la elección de herramientas puede
hacer una gran diferencia en la fluidez con la cual puede trabajar en
el proyecto.

\subsection{La cantidad de nombres del control de revisiones}

El control de revisiones es un campo amplio, tan ampli que no hay un
acrónimo o nombre único. A continuación presentamos un listado de
nombres comunes y acrónimos que se podrían encontrar:
\begin{itemize}
\item Control de revisiones (RCS)
\item Manejo de Configuraciones de Programas(SCM), o administracón de
  configuraciones
\item Administración de código fuente
\item Control de Código Fuente, o Control de Fuentes
\item Control de Versiones(VCS)
\end{itemize}
Algunas personas aducen que estos términos tienen significados
diversos, pero en la práctica se sobrelapan tanto que no hay un
acuerdo o una forma adecuada de separarlos.

\section{Historia resumida del control de revisiones}

La herramienta de control de revisiones más antigua conocida es SCCS 
(Sistema de Control de Código), escrito por Marc Rochkind en Bell
Labs, a comienzos de los setentas(1970s).  SCCS operaba sobre ficheros
individuales, y requería que cada persona que trabajara en el proyecto
tuviera acceso a un espacio compartido en un solo sistema.  Solamente
una persona podía modificar un fichero en un momento dado; el
arbitramiento del acceso a los ficheros se hacía con candados. Era
común que la gente pusiera los candados a los ficheros, y que
posteriormente olvidara quitarlos, impidiendo que otro pudiera
modificar los ficheros en cuestión sin la intervención del
administrador.

Walter Tichy desarrolló una alternativa gratutita a SCCS a comienzos
de los ochentas(1980s), llamó a su programa RCS(Sistema de Control de
Revisiones).  Al igual que SCCS, RCS requería que los desarrolladores
trabajaran en un único espacio compartido y colocaran candados a los
ficheros para evitar que varias personas los estuvieran modificando
simultáneamente.

Después en los ochenta, Dick Grune usó RCS como un bloque de
construcción para un conjunto de guiones de línea de comando, que
inicialmente llamó cmt, pero que renombró a CVS(Sistema Concurrente de
Versiones).  La gran innovación de CVS era que permitía a los
desarrolladores trabajar simultáneamente de una forma más o menos
independiente en sus propios espacios de trabajo. Los espacios de
trabajo personales impedian que los desarrolladores se pisaran las
mangueras todo el tiempo, situación común con SCCS y RCS.  Cada
desarrollador tenía una copia de todo el fichero del proyecto y podía
modificar su copia independientemente, Tenían que fusionar sus
ediciones antes de consignar los cambios al repositorio central.

Brian Berliner tomó los scripts originales de Grune y los reescribió
en~C, haciéndolos públicos en 1989, código sobre el cual se ha
desarrollado la versión moderna de CVS.  CVS posteriormente adquirió
la habilidad de operar sobre una conexión de red, dotándolo de una
arquitectura, cliente/servidor. La arquitectura de CVS es
centralizada; La historia del proyecto está únicamente en el
repositorio central.  Los espacios de trabajo de los clientes
contienen únicamente copias recientes de las versiones de los
ficheros, y pocos metadatos para indicar dónde está el servidor. CVS
ha tenido un éxito enorme; Es probablemente el sistema de control de
revisiones más extendido del planeta.

A comienzos de los noventa(1990s), Sun MicroSystems desarrollo un
temprano sistema distribuido de revisión de controles llamado TeamWare
Un espacio de trabajo TeamWare contiene una copia completa de la
historia del proyecto. TeamWare no tiene la noción de repositorio
central. (CVS se basaba en RCS para el almacenamiento de su historia;
TeamWare usaba SCCS.)

A medida que avanzaba la decada de los noventa, se empezño a
evidenciar los problemas de CVS.  Alacena cambios simultáneos a muchos
ficheros de forma individual, en lugar de agruparlos como una
operación única y atómica lógicamente.  No maneja bien su jerarquía de
ficheros bien; es fácil desordenar un repositorio renombrando ficheros
y directorios. Peor aún, su código fuente es difícil de leer y
mantener, lo que hace que su ``umbral de dolor'' para arreglar sus
problemas arquitecturales algo prohibitivo.

En 2001, Jim Blandy y Karl Fogel, dos desarrolladores que habían
trabajado en CVS, comenzaron un proyecto para reemplazarlo con una
herramienta con mejor arquitectura y código más limpio.  El resultado,
Subversion, no se separó del modelo centralizado cliente/servidor de
CVS, pero añadió consignaciones atómicas de varios ficheros, mejor
manejo de nombres de espacios, y otras características que lo hacen
mejor que CVS. Desde su versión inicial, ha ido creciendo en
popularidad.

Más o menos en forma simultánea Graydon Hoare comenzó a trabajar en un
sistema distribuido de control de versiones ambicioso que llamó
Monotone. Mientras que Monotone se enfocaba a evitar algunas fallas de
diseño de CVS con una arquitectura peer-to-peer, fue mucho más
allá(junto con otros proyectos subsecuentes) que unas herramientas de
control de revisiones en varios aspectos innovadores. Usa hashes
criptográficos como identificadores, y tiene una noción integral de 
``confianza'' para código de diversas fuentes.

Mercurial nació en el 2005.  Algunos de sus aspectos de de diseño
fueron influenciados por Monotone, pero Mercurial se enfoca en la
facilidad de uso, gran rendimiento y escalabilidad para proyectos muy
grandes.

\section{Tendencias en el control de revisiones}

Ha habido varias tendencias en el desarrollo y uso de las herramientas
de control de revisiones en las pasadas cuatro décadas, mientras la
gente se ha vuelto familiar con las capacidades de sus herramientas
así mismo con sus limitaciones.

La primera generación comenzó administrando ficheros individuales en
computadores por persona. A pesar de que tales herramientas
representaron un avance importante frente al control de revisiones
manual, su modelo de candados y la limitación a un sólo computador,
determinó equipos de trabajo pequeños y acoplados.

La segunda generación dejó atrás esas limitaciones moviéndose a
arquitecturas centradas en  redes, y administrando proyectos completos
uno a la vez. A medida que los proyectos crecían, nacieron nuevos
problemas. Con la necesidad de comunicación frecuente con los
servidores, escalar estas máquinas se convirtió en un problema en
proyectos realmente grandes. Las redes con poca estabilidad impidieron
que usuarios remotos se conectaran al servidor. A medida que los
proyecos de código abierto comenzaron a ofrecer acceso de sólo lectura
de forma anónima a cualquiera, la gente sin permiso para consignar,
vio que no podían usar tales herramientas para interactuar en un
proyecto de forma natural, puesto que no podían guardar sus cambios.

La generación actual de herramientas de control de revisiones es de
forma natural peer-to-peer.  Todos estos sistemas han eliminado la
dependencia de un único servidor central, y han permitido que la
gente distribuya sus datos de control de revisiones donde realmente se
necesita. La colaboración a través de Internet ha cambiado las
limitantes tecnológicas a la cuestión de elección y consenso. Las
herramientas modernas pueden operar sin conexión indefinidamenta y
autónomamente, necesitando una conexión de red solamente para
sincronizar los cambios con otro repositorio.

\section{Algunas ventajas del control distribuido de revisiones}

A pesar de que las herramientas para el control distribuido de
revisiones lleva varios años siendo tan robusto y usable como la
generación previa de su contraparte, personas que usan herramientas
más antiguas no se han percatado de sus ventajas.  Hay gran cantidad
de situaciones en las cuales las herramientas distribuidas brillan
frente a las centralizadas.

Para un desarrollador individual, las herramientas distribuidas casi
siempre son más rápidas que las centralizadas. Por una razón sencilla:
Una herramienta centralizada necesita comunicarse por red para las
operaciones más usuales, debido a que los metadatos se almacenan en
una sola copia en el servidor central. Una herramienta distribuida
almacena todos sus metadatos localmente.  Con todo lo demás de la
misma forma, comunicarse por red tiene un sobrecosto en una
herramienta centralizada. No subestime el valor de una herramienta de
respuesta rápida: Usted empleará mucho tiempo interactuando con su
programa de control de revisiones.

Las herramientas distribuidas son indiferentes a los caprichos de su
infraestructura de servidores, de nuevo, debido a la replicación de
metadatos en tantos lugares. Si usa un sistema centralizado y su
servidor explota, ojalá los medios físicos de su copia de seguridad
sean confiables, y que su última copia sea reciente y además
funcione. Con una herramienta distribuida tiene tantas copias de
seguridad disponibles como computadores de contribuidores.

La confiabilidad de su red afectará las herramientas distribuidas de
una forma mucho menor que las herramientas centralizadas No puede
siquiera usar una herramienta centralizada sin conexión de red,
excepto con algunas órdenes muy limitadas. Con herramientas
distribuidas, si sus conexión cae mientras usted está trabajando,
podría nisiquiera darse cuenta. Lo único que que no podrá hacer es
comunicarse  con repositorios en otros computadores, algo que es
relativamente raro comparado con las operaciones locales. Si tiene
colaboradores remotos en su equipo, puede ser significante.

\subsection{Ventajas para proyectos de código abierto}

Si descubre un proyecto de código abierto y decide que desea comenzar
a trabajar en él, y ese proyecto usa una herramienta de control
distribuido de revisiones, usted es un par con la gente que se
considera el ``alma'' del proyecto.  Si ellos publican los
repositorios, se puede copiar inmediatamente la historia del proyecto,
hacer cambios y guardar su trabajo, usando las mismas herramientas de
la misma forma que ellos. En contraste, con una herramienta
centralizada, debe usar el programa en un modo ``sólo lectura'' a
menos que alguien le otorgue permisos para consignar cambios en el
repositorio central. Hasta entonces, no podrá almacenar sus cambios y
sus modificaciones locales correrán el riesgo de dañarse cuando trate
de actualizar su vista del repositorio.

\subsubsection{Las bifurcaciones(forks) no son un problema}

Se ha mencionado que las herramientas de control distribuido de
versiones albergan un riesgo a los proyectos de código abierto, puesto
que se vuelve muy sencillo hacer una ``bifurcanción''\ndt{fork.} del
desarrollo del proyecto.  Una bifurcación pasa cuando hay diferencias
de opinión o actitud entre grupos de desarrolladores que desenvoca en
la decisión de la imposibilidad de continuar trabajando juntos. Cada
parte toma una copia más o menos completa del código fuente del
proyecto y toma su propio rumbo.

En algunas ocasiones los líderes de las bifurcaciones reconcilian sus
diferencias. Con un sistema centralizado de control de revisiones, el
proceso \emph{técnico} de reconciliarse es doloroso, y se hace de
forma muy manual.  Tiene que decidir qué historia de revisiones va a
``win'', e injertar los cambios del otro equipo en el árbol de alguna
manera. Con esto usualmente se pierde algo o todo del historial de la
revisión de alguna de las partes.

Lo que las herramientas distribuidas hacen con respecto a las
bifurcaciones, es que las bifurcaciones son la \emph{única} forma de
desarrollar un proyecto. Cada cambio que usted hace es potencialmente
un punto de bifurcación. La gran fortaleza de esta aproximación es que
las herramientas distribuidas de control de revisiones tiene que ser
bueno al \emph{fusionar} las bifurcaciones, porque las bifurcaciones
son absolutamente fundamentales: pasan todo el tiempo.

Si todas las porciones de trabajo que todos hacen todo el tiempo, se
enmarca en términos de bifurcaciones y fusiones, entonces a aquello a
lo que se refiere en el mundo del código abierto a una ``bifurcación''
se convierte \emph{puramente} en una cuestión social. Lo que hacen las
herramientas distribuidas es \emph{disminuir} la posibilidad de una
bifurcación porque:
\begin{itemize}
\item Eliminan la distinción social que las herramientas centralizadas
  imponen: esto entre los miembros (personas con permiso de consignar)
  y forasteros(los que no tienen el permiso).
\item Facilitan la reconciliación después de una bifurcación social,
  porque todo lo que concierne al programa de control de revisiones es
  una fusión.
\end{itemize}

Algunas personas se resisten a las herramientas distribuidas porque
desean mantener control completo sobre sus proyectos, y creen que las
herramientas centralizadas les da tal control. En todo caso, si este
es su parecer, y publica sus repositorios de CVS o Subversion, hay
muchas herramientas disponibles que pueden obtener la historia
completa(A pesar de lo lento) y recrearla en otro sitio que usted no
controla. Siendo así un control ilusorio, puesto que está impidiendo
la fluidez de colaboración en lugar de prevenir que alguien se sienta
impulsado a obtener una copia y hacer una bifurcación con su historia.

\subsection{Ventajas para proyectos comerciales}

Muchos proyectos comerciales tienen grupos de trabajo distribuidos
alrededor del globo.  Quienes contribuyen y están lejos de un
repositorio central verán una ejecución más lenta de las órdenes y tal
vez menos confiabilidad. Los sistemas de control de revisión
comerciales intentan amortiguar estos problemas con adiciones de
replicación remota que usualmente son muy costosos y complicados de
administradr. Un sistema distribuido no padece estos problemas. Mejor
aún, puede colocar varios servidores autorizados, por ejemplo, uno por
sitio, de tal forma que no haya comunicación redundante entre
repositorios sobre enlaces de conexión costosos.

Los sistemas de control de revisiones distribuidos tienden a ser poco
escalables. No es inusual quw costosos sistemas centralizados caigan
ante la carga combinada de unas cuantas docenas de usuarios
concurrentes. De nuevo, las respuestas típicas de replibcación tienden
a ser costosas y complejas de instalar y administrar. Dado que la
carga en un servidor central---si es que tiene uno---es muchas veces
menor con una herramienta distribuida(debido a que los datos están
replicados en todas partes), un sólo servidor económico puede tratar
las necesidades de equipos mucho más grandes, y la replicación para
balancear la carga se vuelve cosa de scripts.

Si tiene un empleado en el campo, se beneficiará grandemente de un
sistema distribuido de control de versiones al resolver problemas en
el sitio del cliente. La herramienta le permitirá generar
construcciones a la medida, probar diferentes arreglos de forma
independiente y buscar de forma eficiente las fuentes de fallos en la
historia y regresiones en los ambientes de los clientes, todo, sin
necesidad de conectarse al servidor de su compañía.

\section{¿Por qué elegir Mercurial?}

Mercurial cuenta con un conjunto único de propiedades que lo hacen
particularmente una buena elección como un sistema de control de
revisiones, puesto que:
\begin{itemize}
\item Es fácil de aprender y usar.
\item Es liviano.
\item Escala de forma excelente.
\item Es fácil de acondicionar.
\end{itemize}

Si los sistemas de control de revisiones le son familiares, debería
estar listo para usar Mercurial en menos de cinco minutos. Si no, va a
tomar unos pocos minutos más. Las órdenes de Mercurial y su conjunto
de características son uniformes y consistentes generalmente, y basta
con que siga unas pocas reglas generales en lugar de un montón de
excepciones.

En un proyecto pequeño, puede comenzar a trabajar con Mercurial poco a
poco. Creando nuevos cambios y ramas, transfiriendo cambios(localmente
o por la red); y las operaciones relacionadas con el estado y la
historia son rápidas. Mercurial buscar ser ligero y no incomodar en su
camino combinando poca sobrecarga cognitiva con operaciones
asombrosamente rápidas.

La utilidad de Mercurial no se limita a proyectos pequeños: está
siendo usado por proyectos con centenas de miles de contribuyentes,
cada uno conteniendo decenas de miles de ficheros y centenas de
megabytes de código fuente

Si la funcionalidad básica de Mercurial no es suficiente para usted,
es muy fácil extenderlo. Mercurial se comporta muy bien con tareas de
scripting y su limpieza interna junto con su implementación en Python
permiten añadir características fácilmente en forma de extensiones.
Hay un buen número de extensiones útiles y populares en este momento,
desde ayudar a identificar fallos hasta el mejoramiento de su
desempeño.

\section{Comparación de Mercurial con otras herramientas}

Antes de leer, por favor tenga en cuenta que esta sección
necesariamente refleja mis propias experiencias, intereses y(tengo que
decirlo) mis preferencias. He usado cada una de las herramientas de
control de versiones listadas a continuación, y en muchos casos por
varios años.


\subsection{Subversion}

Subversion es una herramienta de control de revisiones muy popular,
desarrollada para reemplazar a CVS.  Su arquitectura es centralizada
en cliente/servidor.

Subversion y Mercurial tienen órdenes con nombres similares para hacer
las mismas operaciones, por lo que si le son familiares en una, será
sencillo aprender a usar la otra. Ambas herramientas son portables en
todos los sistemas operativos populares.

Antes de la versión 1.5, Subversion no tenía soporte para fusiones. En
el momento de la escritura, sus capcidades para llevar cuenta de las
funciones son nuevas,
\href{http://svnbook.red-bean.com/nightly/en/svn.branchmerge.advanced.html#svn.branchmerge.advanced.finalword}{complicadas
  y poco estables\NdT{buggy}}.

Mercurial tiene una ventaja considerable en el desempeño sobre
Subversion en cualquier operación de control de revisiones que yo haya
medido. He medido sus ventajas con factores desde dos hasta seis veces
comparando con almacenamiento de ficheros \emph{ra\_local}
Subversion~1.4.3, el cual es el método de acceso más rápido.  En los
escenarios más sencillos incluyendo almacenamiento con la red de por
medio, Subversion se encuentra en desventaja aún mayor. Dado que casi
todas las órdenes de Subversion deben tratar con el servidor y
Subversion no tiene utilidades de replicación sencillas, la capacidad
del servidor y el ancho de banda se convierten en cuellos de botella
para proyectos modestamente grandes.

Adicionalmente, Subversion tiene un sobrecosto en almacentamiento
considerable para evitar transacciones por red en algunas operaciones,
tales como encontrar ficheros modificados(\texttt{status}) y desplegar
información frente a la revisión actual(\texttt{diff}).  Como
resultado, la copia de trabajo de Subversion termina siendo del mismo
tamaño o más grande que un repositorio de Mercurial y el directorio de
trabajo, a pesar de que el repositorio de Mercurial contiene la
historia completa  del proyecto.

Subversion tiene soporte amplio de otras herramientas. Mercurial por
ahora está bastante atrás en este aspecto.  Esta diferencia está
disminuyendo, y algunas de las herramientas GUI\NdT{Interfaz de
  Usuario Gráfica}, eclipsan sus equivalentes de Subversion. Al igual
que Mercurial, Subversion tiene un excelente manual de usuario.

Dado que Subversion no almacena la historia de revisiones en el
cliente, es muy bueno para administrar proyectos que tienen muchos
ficheros binarios grandes y opacos. Si consigna cincuenta revisiones
de un fichero de 10MB que no es comprimible, el esapacio en el cliente
de Subversion se mantendrá constante mientras que el espacio usado por
cualquier Sistema Distribuido de Control de Revisiones crecerá
rápidamente en proporción con el número de revisiones, debido a que
las diferencias entre cada revisión es grande.

Adicionalmente, generalmente es difícil o más bien, imposible mezclar
diferentes versiones de un fichero binario. La habilidad de Subversion
para permitirle al usuario poner una cerradura  a un fichero, de modo
que tenga un permiso exclusivo para consignar cambios, puede ser una
ventaja significativa en un proyecto donde los ficheros binarios sean
usados ampliamente.

Mercurial puede importar la historia de revisiones de un repositorio
de Subversion. También puede exportar historia de revisiones a un
repositorio de Subversion.  De esta forma es sencillo ``dar un
vistazo'' y usar Mercurial y Subversion en paralelo antes de decidirse
a dar el paso. La conversión de la historia es incremental, de modo
que puede aplicar una conversión inicial, y después conversiones
pequeñas y adicionales posteriormente para traer nuevos cambios.

\subsection{Git}

Git es una herramienta distribuida de control de revisiones
desarrollada para administrar el arbol del Kernel de Linux.  Al igual
que Mercurial los principios de su diseño fueron influenciados por 
Monotone.

Git tiene un conjunto de órdenes muy grande en la versión~1.5.0
ofrece~139 órdenes individuales.  Tiene cierta reputación de ser
difícil de aprender. Comparado con Git, Mercurial tiene un fuerte
enfoque hacia la facilidad.

En términos de rendimiento, Git es extremadamente rápido. En muchos
casos, es más rápido que Mercurial, por lo menos en Linux, mientras
que Mercurial se comporta mejor en otras operaciones.  De todas
maneras en Windows, el desempeño y el nivel general de soporte que
ofrece Git, al momento de la escritura, está lejos detrás de
Mercurial.

Mientras que el repositorio de Mercurial no requiere mantenimiento, el
repositorio de Git requiere frecuentes ``repacks'' a sus metadatos.
Sin estos, el desempeño se degrada y el espacio crece rápidamente. Un
servidor que contenga repositorios de Git que no sean reempacados
rigurosa y frecuentemente requerirá trabajo-intenso de disco durante
las copias de seguridad, y ha habido situaciones en copias de
seguridad diaria que toman más de 24 horas como resultado. Un
repositorio recién reempacado de Git es un poco más pequeño que un
repositorio de Mercurial, pero un repositorio sin reempacar es varios
órdenes de magnitud más grande.

El corazón de Git está escrito en C.  Muchas órdenes de Git están
implementadas como scripts de shell o Perl y la calidad de esos
scripts varía ampliamente. He encontrado muchas situaciones en las
cuales los scripts no tuvieron en cuenta la presencia de errores que
podrían haber sido fatales.

Mercurial puede importar el historial de revisiones de un repositorio
de Git.

\subsection{CVS}

CVS es probablemente la herramienta de control de revisiones más
ampliamente usada en el planeta.  Debido a su edad y su poca pulcritud
nterna, ha sido ligeramente mantenido en muchos años.

Tiene una arquitectura centralizada cliente/servidor. No agrupa
cambios relacionadso en consignaciones atómicas, pemitiendo que con
facilidad la gente ``rompa la construcción'': una persona puede
consignar exitósamente parte del cambio y estar bloqueada por la
necesidad de una mezcla, forzando a otras personas a ver solamente una
porción del trabajo que estaban buscando hacer.  Esto afecta también
la forma como usted trabaja con la historia del proyecto. Si quiere
ver todas las modificaciones que alguien hizo como parte de una tarea,
necesitará inspeccionar manualmente las descripciones y las marcas de
tiempo de cambio de cada fichero involucrado(esto, si usted saber
cuáles eran tales ficheros).

CVS tiene una noción confusa de etiquetas y ramas que yo no trataría
incluso de describir.  No soporta renombramiento de ficheros o
directorios de buena forma, facilitando el corromper un
repositorio. Casi no tiene chequeo de consistencia interna, por lo
tanto es casi imposible identificar por que o cómo se corrompió un
repositorio. Yo no recomendaría un repositorio de CVS para proyecto
alguno, ni existente ni nuevo.

Mercurial puede importar la historia de revisiones de CVS.  De todas
maneras hay ciertos trucos para aplicar; los cuales también son
necesarios para cualquier herramienta importadora de historial de
CVS. Debido a la falta de atomicidad de cambios y el no versionamiento
de la jerarquía del sistema de archivos, es imposible reconstruir la
completamente la historia de CVS acertadamente; hay cierto trabajo de
conjetura involucrado y los renombramientos tampoco se
mostrarán. Debido a que gran parte de la administración avanzada de
CVS tiene que hacerse manualmente y por lo tanto es proclive al error,
es común que los importadores de CVS encuentren muchos problemas con
repositorios corruptos( marcas de tiempo totalmente desubicadas y
archivos que han permanecido con candados por más de una década son
dos de los problemas más interesantes de los que puedo retomar de mi
experiencia personal).

Mercurial puede importar la historia de revisiones de un repositorio
CVS.

\subsection{Herramientas comerciales}

Perforce tiene una arquitectura centralizada cliente/servidor sin
almacenamiento de dato alguno en el lado del cliente. A diferencia de
las herramientas modernas de control de revisiones, Perforce requiere
que un usuario ejecute una orden para informar al servidor acerca de
todo fichero que se vaya a editar.

El rendimiento de Perforce es muy bueno para equipos pequeños, pero se
degrada rápidamente cuando el número de usuarios va más allá de pocas
docenas. Instalaciones modestamente grandes de Perforce requiere la
organización de proxies para soportar la carga que sus usuarios generan.

\subsection{Elegir una herramienta de control de revisiones}

Con la excepción de CVS, toda las herramientas que se han listado
anteriormente tienen fortalezas que los hacen valiosos de acuerdo al
tipo de trabajo. No hay una única herramienta de control de revisiones
que sea la mejor en todas las situaciones.

Por ejemplo, Subversion es una buena elección para trabajar con
edición frecuente de ficheros binarios, debido a su naturaleza
centralizada y soporte para poner candados a ficheros.

Personalmente encuentro las propiedades de simplicidad, desempeño, y
buen soporte de fusiones de Mercurial una combinación llamativa que ha
dado buenos frutos por varios años.


\section{Migrar de otra herramienta hacia Mercurial}

Mercurial viene con una extensión llamada \hgext{convert}, que puede
importar historiales de revisiones de forma incremental desde varias
herramientas de control de revisiones. Por ``incremental'', quiero
decir que puede migrar toda la historia de un proyecto en una primera
instancia y después volver a ejecutar la migración posteriormente para
obtener los nuevos cambios que han sucedido después de la migración
inicial.

A continuación presentamos las herramientas de revisiones que la
orden\hgext{convert} soporta:
\begin{itemize}
\item Subversion
\item CVS
\item Git
\item Darcs
\end{itemize}

Adicionalmente, \hgext{convert} puede exportar cambios de Mercurial
hacia Subversion.  Lo que hace posible probar Subversion y Mercurial
en paralelo antes de lanzarse a un migración total, sin arriesgarse a
perder trabajo alguno.

La orden \hgxcmd{conver}{convert} es sencilla de usar. Basta con
apuntarla hacia la ruta o el URL del repositorio fuente, opcionalmente
darle el nombre del nombre del repositorio destino y comenzará a hacer
su trabajo. Después de la conversión inicial, basta con invocar de
nuevo la orden para importar nuevos cambios.


%%% Local Variables: 
%%% mode: latex
%%% TeX-master: "00book"
%%% End: 
