\chapter{Usos avanzados de las Colas de Mercurial}
\label{chap:mq-collab}

Auunque es fácil aprender los usos más directos de las Colas de
Mercurial, tener algo de disciplina junto con algunas de las
capacidadees menos usadas de MQ hace posible trabajar en entornos de
desarrollo complejos.

En este capítulo, usaré como ejemplo una técnica que he usado para
administrar el desarrollo de un controlador de dispositivo Infiniband
para el kernel de Linux. El controlador en cuestión es grande
(al menos en lo que se refiere a controladores), con 25,000 líneas de
código esparcidas en 35 ficheros fuente. Es mantenido por un equipo
pequeño de desarrolladores. 

Aunque mucho del material en este capítulo es específico de Linux, los
mismos principios aplican a cualquier base de código de la que usted
no sea el propietario principal, y sobre la que usted necesita hacer
un montón de desarrollo.

\section{El problema de múltiples objetivos}

El kernel de Linux cambia con rapidez, y nunca ha sido estable
internamente; los desarrolladores hacen cambios drásticos entre
%TODO no encontré una traducción adecuada para "release". Por eso el
%cambio
versiones frecuentemente. Esto significa que una versión del
controlador que funciona bien con una versión particular del kernel ni
siquiera \emph{compilará} correctamente contra, típicamente, cualquier
otra versión.

Para mantener un controlador, debemos tener en cuenta una buena
cantidad de versiones de Linux en mente.
\begin{itemize}
\item Un objetivo es el árbol de desarrollo principal del kernel de
  Linux. En este caso el mantenimiento del código es compartido
  parcialmente por otros desarrolladores en la comunidad del kernel, 
  %TODO drive-by. 
  quienes hacen modificaciones ``de-afán'' al controlador a medida que 
  desarrollan y refinan subsistemas en el kernel.
  %TODO backport
\item También mantenemos algunos ``backports'' para versiones antiguas
  del kernel de Linux, para dar soporte a las necesidades de los
  clientes que están corriendo versiones antiguas de Linux que no
  incorporan nuestros controladores. (Hacer el \emph{backport} de un
  pedazo de código es modificarlo para que trabaje en una versión
  de su entorno objetivo anterior a aquella para la cual fue escrito.)
\item Finalmente, nosotros liberamos nuestro software de acuerdo a un
  cronograma que no necesariamente está alineado con el que usan los
  distribuidores de Linux y los desarrolladores del kernel, así que
  podemos entregar nuevas características a los clientes sin forzarlos
  a actualizar kernels completos o distribuciones.
\end{itemize}

\subsection{Aproximaciones tentadoras que no funcionan adecuadamente}

Hay dos maneras estándar de mantener una porción de software que debe
funcionar en muchos entornos diferentes.

La primera es mantener varias ramas, cada una pensada para un único
entorno. El problema de esta aproximación es que usted debe tener una
disciplina férrea con el flujo de cambios entre repositorios. Una
nueva característica o un arreglo de fallo deben empezar su vida en un
repositorio ``prístino'', y luego propagarse a cada repositorio de
backport. Los cambios para backports están más limitados respecto a
las ramas a las que deberían propagarse; un cambio para backport que
es aplicado a una rama en la que no corresponde probablemente hará que
el controlador no compile.

La segunda es mantener un único árbol de código fuente lleno de
declaraciones que activen o desactiven secciones de código dependiendo
del entorno objetivo. Ya que estos ``ifdefs'' no están permitidos en
el árbol del kernel de Linux, debe seguirse algún proceso manual o
automático para eliminarlos y producir un árbol limpio. Una base de
código mantenida de esta manera se convierte rápidamente en un nido de
ratas de bloques condicionales que son difíciles de entender y
mantener.

%TODO canónica?
Ninguno de estos enfoques es adecuado para situaciones en las que
usted no es ``dueño'' de la copia canónica de un árbol de fuentes. En
el caso de un controlador de Linux que es distribuido con el kernel
estándar, el árbol de Linux contiene la copia del código que será
considerada por el mundo como la canónica. La versión oficial de
``mi'' controlador puede ser modificada por gente que no conozco, sin
que yo siquiera me entere de ello hasta después de que los cambios
aparecen en el árbol de Linus.

Estos enfoques tienen la debilidad adicional de dificultar la
%TODO upstream. no no es río arriba
generación de parches bien formados para enviarlos a la versión
oficial.

En principio, las Colas de Mercurial parecen ser un buen candidato
para administrar un escenario de desarrollo como el de arriba. Aunque
este es de hecho el caso, MQ tiene unas cuantas características
adicionales que hacen el trabajo más agradable.

\section{Aplicar parches condicionalmente mediante guardias}

Tal vez la mejor manera de conservar la cordura con tantos entornos
objetivo es poder escoger parches específicos para aplicar para cada
situación. MQ provee una característica llamada ``guardias''
(que se origina del comando \texttt{guards} de Quilt) que hace
precisamente ésto. Para empezar, creemos un repositorio sencillo para
experimentar.
\interaction{mq.guards.init}
Esto nos brinda un pequeño repositorio que contiene dos parches que no
tienen ninguna dependencia respecto al otro, porque tocan ficheros
diferentes.

La idea detrás de la aplicación condicional es que usted puede
``etiquetar'' un parche con un \emph{guardia}, que simplemente es una
cadena de texto de su elección, y luego decirle a MQ que seleccione
guardias específicos para usar cuando aplique parches. MQ entonces
aplicará, u omitirá, un parche vigilado, dependiendo de los guardias
que usted haya seleccionado.

Un parche puede tener una cantidad arbitraria de guardias; cada uno es
\emph{positivo} (``aplique el parche si este guardia es
seleccionado'') o \emph{negativo} (``omita este parche si este guardia
es seleccionado''). Un parche sin guardias siempre es aplicado.

\section{Controlar los guardias de un parche}

%TODO tal vez no decir determinar, sino definir?
El comando \hgxcmd{mq}{qguard} le permite determinar qué guardias
deben aplicarse a un parche, o mostrar los guardias que están en
efecto. Sin ningún argumento, el comando muestra los guardias del
parche actual de la parte más alta de la pila.
\interaction{mq.guards.qguard}
Para poner un guardia positivo en un parche, prefije el nombre del
guardia con un ``\texttt{+}''.
\interaction{mq.guards.qguard.pos}
Para poner un guardia negativo en un parche, prefije el nombre del
guardia con un ``\texttt{-}''.
\interaction{mq.guards.qguard.neg}

\begin{note}
  El comando \hgxcmd{mq}{qguard} \emph{pone} los guardias en un
  parche; no los \emph{modifica}. Esto significa que si usted ejecuta
  \hgcmdargs{qguard}{+a +b} sobre un parche, y luego
  \hgcmdargs{qguard}{+c} en el mismo parche, el único guardia sobre el
  parche después del comando será \texttt{+c}.
\end{note}

Mercurial almacena los guardias en el fichero \sfilename{series}; la
forma en que son almacenados es fácil tanto de entender como de editar
a mano. (En otras palabras, usted no tiene que usar el comando
\hgxcmd{mq}{qguard} si no lo desea; está bien simplemente editar el
fichero \sfilename{series})
\interaction{mq.guards.series}

\section{Selecccionar los guardias a usar}

%TODO tal vez no decir determinar, sino definir?
El comando \hgxcmd{mq}{qselect} determina qué guardias están activos
en cualquier momento. El efecto de esto es determinar qué parches
aplicará MQ la próxima vez que usted ejecute \hgxcmd{mq}{qpush}.  No
tiene ningún otro efecto; en particular, no hace nada a los parches
que ya han sido aplicados.

Sin argumentos, el comando \hgxcmd{mq}{qselect} lista los guardias en
efecto actualmente, uno por cada línea de salida. Cada argumento es
tratado como el nombre de un guardia a aplicar.
\interaction{mq.guards.qselect.foo}
Si está interesado, los guardias seleccionados actualmente están
almacenados en el fichero \sfilename{guards}.
\interaction{mq.guards.qselect.cat}
Podemos ver el efecto que tienen los guardias seleccionados cuando
ejecutamos \hgxcmd{mq}{qpush}.
\interaction{mq.guards.qselect.qpush}

Un guardia no puede empezar con un  caracter ``\texttt{+}'' o
``\texttt{-}''. El nombre del guardia no debe contener espacios en
blanco, pero muchos otros caracteres son aceptables. Si usted trata de
usar un guardia con un nombre inválido, MQ se quejará:
\interaction{mq.guards.qselect.error} 
Cambiar los guardias seleccionados cambia los parches que son
aplicados.
\interaction{mq.guards.qselect.quux} 
Usted puede ver en el ejemplo de abajo que los guardias negativos
tienen precedencia sobre los guardias positivos.
\interaction{mq.guards.qselect.foobar}

\section{Reglas de MQ para aplicar parches}

Las reglas que MQ usa para decidir si debe aplicar un parche son las
siguientes.
\begin{itemize}
\item Un parche sin guardias es aplicado siempre.
\item Si el parche tiene algún guardia negativo que corresponda con
  cualquiera de los guardias seleccionados, se salta el parche.
\item Si el parche tiene algún guardia positivo que corresponda con
  cualquiera de los guardias seleccionados, se aplica el parche.
\item Si el parche tiene guardias positivos o negativos, pero ninguno
  corresponde con cualquiera de los guardias seleccionados, se salta
  el parche.
\end{itemize}

\section{Podar el entorno de trabajo}

En el trabajo del controlador de dispositivo que mencioné
anteriormente, yo no aplico los parches a un árbol normal del kernel
de Linux. En cambio, uso un repositorio que sólo contiene una
instantánea de los ficheros fuente y de cabecera que son relevantes
para el desarrollo de Infiniband. Este repositorio tiene un~1\% del
tamaño del repositorio del kernel, por lo que es más fácil trabajar
con él.

Luego escojo una versión ``base'' sobre la cual son aplicados los
parches. Es una instantánea del árbol del kernel de Linux en una
revisión de mi elección. Cuando tomo la instantánea, almaceno el ID de
conjunto de cambios en el mensaje de consignación. Ya que la
instantánea preserva la ``forma'' y el contenido de las partes
relevantes del árbol del kernel, puedo aplicar mis parches sobre mi
pequeño repositorio o sobre un árbol normal del kernel.

Normalmente, el árbol base sobre el que se aplican los parches debería
ser una instantánea de un árbol de desarrollo muy reciente. Esto
facilita mucho el desarrollo de parches que puedan ser enviados al
árbol oficial con pocas o ninguna modificación.

\section{Dividir el fichero \sfilename{series}}

Yo categorizo los parches en el fichero \sfilename{series} en una
serie de grupos lógicos. Cada sección de parches similares empieza con
un bloque de comentarios que describen el propósito de los parches que
le siguen.

La secuencia de grupos de parches que mantengo se muestra a
continuación. El orden de los grupos es importante; explicaré porqué
luego de que presente los grupos.
\begin{itemize}
\item El grupo ``aceptado''. Son parches que el equipo de desarrollo
  ha enviado al mantenedor del subsistema Infiniband, y que él ha
  aceptado, pero que no están presentes en la instantánea en la cual
  está basada el repositorio pequeño. Estos son parches de
  ``sólo lectura'', presentes únicamente para transformar el árbol en
  un estado similar al del repositorio del mantenedor oficial.
\item El grupo ``revisar''. Parches que yo he enviado, pero sobre los
  que que el mantenedor oficial ha solicitado modificaciones antes de
  aceptarlos.
\item El grupo ``pendiente''. Parches que no he enviado al mantenedor
  oficial, pero que ya están terminados. Estos parches serán de
  ``sólo lectura'' por un buen tiempo. Si el mantenedor oficial los
  acepta cuando los envíe, los moveré al final del grupo ``aceptado''.
  Si él solicita que modificaciones en alguno de ellos, los moveré al
  principio del grupo ``revisar''.
\item El grupo ``en proceso''. Parches que están siendo activamente
  desarrollados, y no deberían ser enviados a ninguna parte aún.
\item El grupo ``backport''. Parches que adaptan el árbol de fuentes a
    versiones antiguas del árbol del kernel.
\item El grupo ``no enviar''. Parches que por alguna razón nunca deben
  ser enviados al mantenedor oficial del kernel. Por ejemplo, alguno
  de esos parches podría cambiar las cadenas de identificación
  embebidas del controlador para hacer más fácil la distinción, en
  pruebas de campo, entre una versión del controlador de
  salida-del-árbol  y una versión entregada por un vendedor de alguna
  distribución.
\end{itemize}

Ahora volvemos a las razones para ordenar los grupos de parches en
esta manera. Quisiéramos que los parches del fondo de la pila sean tan
estables como sea posible, para no tener que revisar parches más
arriba debido a cambios de contexto. Poner los parches que nunca
cambiarán en el primer lugar del fichero \sfilename{series} sirve a
este propósito.

También desearíamos que los parches que sabemos que debemos modificar
sean aplicados sobre un árbol de fuentes que se parezca al oficial
tanto como sea posible. Es por esto que mantenemos los parches
aceptados disponibles por una buena cantidad de tiempo.

Los parches ``backport'' y ``no enviar'' flotan al final del fichero
\sfilename{series}. Los parches de backport deben ser aplicados encima
de todos los otros parches, y los parches ``no enviar'' pueden
perfectamente quedarse fuera del camino.

\section{Mantener la serie de parches}

En mi trabajo, uso varios guardias para controlar qué parches deben
ser aplicados.

\begin{itemize}
\item Los parches ``aceptados'' son vigilados con
  \texttt{accepted}. Yo habilito este guardia la mayoría de las veces.
  Cuando aplico los parches sobre un árbol donde los parches ya están
  %TODO no será ``desactivar este guardia''? si sí, corregir versión
  %en inglés también
  presentes, puedo desactivar este parche, y los parches que lo siguen
  se aplicarán sin problemas.
\item Los parches que están ``terminados'', pero no han sido enviados,
  no tienen guardias. Si estoy aplicando la pila de parches a una
  copia del árbol oficial, no necesito habilitar ningún guardia para
  obtener un árbol de fuentes razonablemente seguro.
\item Los parches que necesitan revisión antes de ser reenviados
  tienen el guardia \texttt{rework}.
\item Para aquellos parches que aún están bajo desarrollo, uso
  \texttt{devel}.
\item Un parche de backport puede tener varios guardias, uno para cada
  versión del kernel a la que aplica. Por ejemplo, un parche que hace
  backport de un segmento de código a~2.6.9 tendrá un guardia~\texttt{2.6.9}.
\end{itemize}
La variedad de guardias me brinda una flexibilidad considerable para
determinar qué tipo de árbol de fuentes acabaré por obtener. En la
mayoría de las situaciones, la selección de guardias apropiados es
automatizada durante el proceso de compilación, pero puedo ajustar
manualmente los guardias a usar para circunstancias poco comunes.

\subsection{The art of writing backport patches}

Using MQ, writing a backport patch is a simple process.  All such a
patch has to do is modify a piece of code that uses a kernel feature
not present in the older version of the kernel, so that the driver
continues to work correctly under that older version.

A useful goal when writing a good backport patch is to make your code
look as if it was written for the older version of the kernel you're
targeting.  The less obtrusive the patch, the easier it will be to
understand and maintain.  If you're writing a collection of backport
patches to avoid the ``rat's nest'' effect of lots of
\texttt{\#ifdef}s (hunks of source code that are only used
conditionally) in your code, don't introduce version-dependent
\texttt{\#ifdef}s into the patches.  Instead, write several patches,
each of which makes unconditional changes, and control their
application using guards.

There are two reasons to divide backport patches into a distinct
group, away from the ``regular'' patches whose effects they modify.
The first is that intermingling the two makes it more difficult to use
a tool like the \hgext{patchbomb} extension to automate the process of
submitting the patches to an upstream maintainer.  The second is that
a backport patch could perturb the context in which a subsequent
regular patch is applied, making it impossible to apply the regular
patch cleanly \emph{without} the earlier backport patch already being
applied.

\section{Useful tips for developing with MQ}

\subsection{Organising patches in directories}

If you're working on a substantial project with MQ, it's not difficult
to accumulate a large number of patches.  For example, I have one
patch repository that contains over 250 patches.

If you can group these patches into separate logical categories, you
can if you like store them in different directories; MQ has no
problems with patch names that contain path separators.

\subsection{Viewing the history of a patch}
\label{mq-collab:tips:interdiff}

If you're developing a set of patches over a long time, it's a good
idea to maintain them in a repository, as discussed in
section~\ref{sec:mq:repo}.  If you do so, you'll quickly discover that
using the \hgcmd{diff} command to look at the history of changes to a
patch is unworkable.  This is in part because you're looking at the
second derivative of the real code (a diff of a diff), but also
because MQ adds noise to the process by modifying time stamps and
directory names when it updates a patch.

However, you can use the \hgext{extdiff} extension, which is bundled
with Mercurial, to turn a diff of two versions of a patch into
something readable.  To do this, you will need a third-party package
called \package{patchutils}~\cite{web:patchutils}.  This provides a
command named \command{interdiff}, which shows the differences between
two diffs as a diff.  Used on two versions of the same diff, it
generates a diff that represents the diff from the first to the second
version.

You can enable the \hgext{extdiff} extension in the usual way, by
adding a line to the \rcsection{extensions} section of your \hgrc.
\begin{codesample2}
  [extensions]
  extdiff =
\end{codesample2}
The \command{interdiff} command expects to be passed the names of two
files, but the \hgext{extdiff} extension passes the program it runs a
pair of directories, each of which can contain an arbitrary number of
files.  We thus need a small program that will run \command{interdiff}
on each pair of files in these two directories.  This program is
available as \sfilename{hg-interdiff} in the \dirname{examples}
directory of the source code repository that accompanies this book.
\excode{hg-interdiff}

With the \sfilename{hg-interdiff} program in your shell's search path,
you can run it as follows, from inside an MQ patch directory:
\begin{codesample2}
  hg extdiff -p hg-interdiff -r A:B my-change.patch
\end{codesample2}
Since you'll probably want to use this long-winded command a lot, you
can get \hgext{hgext} to make it available as a normal Mercurial
command, again by editing your \hgrc.
\begin{codesample2}
  [extdiff]
  cmd.interdiff = hg-interdiff
\end{codesample2}
This directs \hgext{hgext} to make an \texttt{interdiff} command
available, so you can now shorten the previous invocation of
\hgxcmd{extdiff}{extdiff} to something a little more wieldy.
\begin{codesample2}
  hg interdiff -r A:B my-change.patch
\end{codesample2}

\begin{note}
  The \command{interdiff} command works well only if the underlying
  files against which versions of a patch are generated remain the
  same.  If you create a patch, modify the underlying files, and then
  regenerate the patch, \command{interdiff} may not produce useful
  output.
\end{note}

The \hgext{extdiff} extension is useful for more than merely improving
the presentation of MQ~patches.  To read more about it, go to
section~\ref{sec:hgext:extdiff}.

%%% Local Variables: 
%%% mode: latex
%%% TeX-master: "00book"
%%% End: 
