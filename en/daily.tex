\chapter{Mercurial in daily use}
\label{chap:daily}

\section{Telling Mercurial which files to track}

Mercurial does not work with files in your repository unless you tell
it to manage them.  The \hgcmd{status} command will tell you which
files Mercurial doesn't know about; it uses a ``\texttt{?}'' to
display such files.

To tell Mercurial to track a file, use the \hgcmd{add} command.  Once
you have added a file, the entry in the output of \hgcmd{status} for
that file changes from ``\texttt{?}'' to ``\texttt{A}''.
\interaction{daily.files.add}

After you run a \hgcmd{commit}, the files that you added before the
commit will no longer be listed in the output of \hgcmd{status}.  The
reason for this is that \hgcmd{status} only tells you about
``interesting'' files---those that you have modified or told Mercurial
to do something with---by default.  If you have a repository that
contains thousands of files, you will rarely want to know about files
that Mercurial is tracking, but that have not changed.  (You can still
get this information; we'll return to this later.)

Once you add a file, Mercurial doesn't do anything with it
immediately.  Instead, it will take a snapshot of the file's state the
next time you perform a commit.  It will then continue to track the
changes you make to the file every time you commit, until you remove
the file.

\subsection{Explicit versus implicit file naming}

A useful behaviour that Mercurial has is that if you pass the name of
a directory to a command, every Mercurial command will treat this as
``I want to operate on every file in this directory and its
subdirectories''.
\interaction{daily.files.add-dir}
Notice in this example that Mercurial printed the names of the files
it added, whereas it didn't do so when we added the file named
\filename{a} in the earlier example.

What's going on is that in the former case, we explicitly named the
file to add on the command line, so the assumption that Mercurial
makes in such cases is that we know what you were doing, and it
doesn't print any output.

However, when we \emph{imply} the names of files by giving the name of
a directory, Mercurial takes the extra step of printing the name of
each file that it does something with.  This makes it more clear what
is happening, and reduces the likelihood of a silent and nasty
surprise.  This behaviour is common to most Mercurial commands.

\subsection{Aside: Mercurial tracks files, not directories}

Mercurial does not track directory information.  Instead, it tracks
the path to a file.  Before creating a file, it first creates any
missing directory components of the path.  After it deletes a file, it
then deletes any empty directories that were in the deleted file's
path.  This sounds like a trivial distinction, but it has one minor
practical consequence: it is not possible to represent a completely
empty directory in Mercurial.

Empty directories are rarely useful, and there are unintrusive
workarounds that you can use to achieve an appropriate effect.  The
developers of Mercurial thus felt that the complexity that would be
required to manage empty directories was not worth the limited benefit
this feature would bring.

If you need an empty directory in your repository, there are a few
ways to achieve this. One is to create a directory, then \hgcmd{add} a
``hidden'' file to that directory.  On Unix-like systems, any file
name that begins with a period (``\texttt{.}'') is treated as hidden
by most commands and GUI tools.  This approach is illustrated in
figure~\ref{ex:daily:hidden}.

\begin{figure}[ht]
  \interaction{daily.files.hidden}
  \caption{Simulating an empty directory using a hidden file}
  \label{ex:daily:hidden}
\end{figure}

Another way to tackle a need for an empty directory is to simply
create one in your automated build scripts before they will need it.

\section{How to stop tracking a file}

Once you decide that a file no longer belongs in your repository, use
the \hgcmd{remove} command; this deletes the file, and tells Mercurial
to stop tracking it.  A removed file is represented in the output of
\hgcmd{status} with a ``\texttt{R}''.
\interaction{daily.files.remove}

\subsection{Missing files}

Mercurial considers a file that you have deleted, but not used
\hgcmd{remove} to delete, to be \emph{missing}.  A missing file is
represented with ``\texttt{!}'' in the output of \hgcmd{status}.
Mercurial commands will not generally do anything with missing files.
\interaction{daily.files.missing}

If your repository contains a file that \hgcmd{status} reports as
missing, and you want the file to stay gone, you can run
\hgcmdargs{remove}{\hgopt{remove}{--after}} at any time later on, to
tell Mercurial that you really did mean to remove the file.
\interaction{daily.files.remove-after}

On the other hand, if you deleted the missing file by accident, use
\hgcmdargs{revert}{\emph{filename}} to recover the file.  It will
reappear, in unmodified form.
\interaction{daily.files.recover-missing}

\subsection{Aside: why tell Mercurial explicitly to 
  remove a file?}

You might wonder why Mercurial requires you to explicitly tell it that
you are deleting a file.  Early during the development of Mercurial,
it let you delete a file however you pleased; Mercurial would notice
the absence of the file automatically when you next ran a
\hgcmd{commit}, and stop tracking the file.  In practice, this made it
too easy to accidentally remove a file without noticing.

\subsection{Useful shorthand---adding and removing files
  in one step}

Mercurial offers a combination command, \hgcmd{addremove}, that adds
untracked files and marks missing files as removed.  
\interaction{daily.files.addremove}
The \hgcmd{commit} command also provides a \hgopt{commit}{-A} option
that performs this same add-and-remove, immediately followed by a
commit.
\interaction{daily.files.commit-addremove}

\section{Copying files}

Mercurial provides a \hgcmd{copy} command that lets you make a new
copy of a file.  When you copy a file using this command, Mercurial
makes a record of the fact that the new file is a copy of the original
file.  It treats these copied files specially when you merge your work
with someone else's.

What happens during a merge is that changes ``follow'' a copy.  To
best illustrate what this means, let's create an example.  We'll start
with the usual tiny repository that contains a single file.
\interaction{daily.copy.init}
We need to do some work in parallel, so that we'll have something to
merge.  So let's clone our repository.
\interaction{daily.copy.clone}
Back in our initial repository, let's use the \hgcmd{copy} command to
make a copy of the first file we created.
\interaction{daily.copy.copy}

If we look at the output of the \hgcmd{status} command afterwards, the
copied file looks just like a normal added file.
\interaction{daily.copy.status}
But if we pass the \hgopt{status}{-C} option to \hgcmd{status}, it
prints another line of output: this is the file that our newly-added
file was copied \emph{from}.
\interaction{daily.copy.status-copy}

Now, back in the repository we cloned, let's make a change in
parallel.  We'll add a line of content to the original file that we
created.
\interaction{daily.copy.other}
Now we have a modified \filename{file} in this repository.  When we
pull the changes from the first repository, and merge the two heads,
Mercurial will propagate the changes that we made locally to
\filename{file} into its copy, \filename{new-file}.
\interaction{daily.copy.merge}

\subsection{Why should changes follow copies?}
\label{sec:daily:why-copy}

This behaviour, of changes to a file propagating out to copies of the
file, might seem esoteric, but in most cases it's highly desirable.

First of all, remember that this propagation \emph{only} happens when
you merge.  So if you \hgcmd{copy} a file, and subsequently modify the
original file during the normal course of your work, nothing will
happen.

The second thing to know is that modifications will only propagate
across a copy as long as the repository that you're pulling changes
from \emph{doesn't know} about the copy.

The reason that Mercurial does this is as follows.  Let's say I make
an important bug fix in a source file, and commit my changes.
Meanwhile, you've decided to \hgcmd{copy} the file in your repository,
without knowing about the bug or having seen the fix, and you have
started hacking on your copy of the file.

If you pulled and merged my changes, and Mercurial \emph{didn't}
propagate changes across copies, your source file would now contain
the bug, and unless you remembered to propagate the bug fix by hand,
the bug would \emph{remain} in your copy of the file.

By automatically propagating the change that fixed the bug from the
original file to the copy, Mercurial prevents this class of problem.
To my knowledge, Mercurial is the \emph{only} revision control system
that propagates changes across copies like this.

Once your change history has a record that the copy and subsequent
merge occurred, there's usually no further need to propagate changes
from the original file to the copied file, and that's why Mercurial
only propagates changes across copies until this point, and no
further.

\subsection{How to make changes \emph{not} follow a copy}

If, for some reason, you decide that this business of automatically
propagating changes across copies is not for you, simply use your
system's normal file copy command (on Unix-like systems, that's
\command{cp}) to make a copy of a file, then \hgcmd{add} the new copy
by hand.  Before you do so, though, please do reread
section~\ref{sec:daily:why-copy}, and make an informed decision that
this behaviour is not appropriate to your specific case.

\subsection{Behaviour of the \hgcmd{copy} command}

The \hgcmd{copy} command acts similarly to the Unix \command{cp}
command.  The last argument is the \emph{destination}, and all prior
arguments are \emph{sources}.
If you pass it a single file as the source, and the destination
does not exist, it creates a new file with that name.
\interaction{daily.copy.simple}
If the destination is a directory, Mercurial copies its sources into
that directory.
\interaction{daily.copy.dir-dest}
Copying a directory is recursive, and preserves the directory
structure of the source.
\interaction{daily.copy.dir-src}
If the source and destination are both directories, the source tree is
recreated in the destination directory.
\interaction{daily.copy.dir-src-dest}

\section{Renaming files}

To rename a file that is tracked by Mercurial, use the \hgcmd{rename}
command.  This command behaves similarly to the Unix \command{mv}
command (and in fact you can use the alias \hgcmd{mv} if you wish).
If the last argument is a directory, \hgcmd{rename} moves all files
identified by earlier arguments into that directory.  Otherwise, it
renames a single file or directory to the name given in the last
argument.

As with \hgcmd{remove}, you can tell Mercurial about a rename after
the fact using the \hgopt{remove}{--after} option.

%%% Local Variables: 
%%% mode: latex
%%% TeX-master: "00book"
%%% End: 
