\chapter{Basic Concepts}
\label{chap:concepts}

This chapter introduces some of the basic concepts behind distributed
version control systems such as Mercurial.

\section{Repository}
\label{sec:concepts:repo}
The repository is a directory where Mercurial stores the history for the
files under revision control.

\subsection{Where?}
% where is this repository you speak of?
XXX

\subsection{How?}
% How are the changes stored?
XXX

\subsection{Structure}
\label{sec:concepts:structure}
% What's the structure of the repository?
A typical Mercurial repository is a directory which contains a checked out
working copy (see section~\ref{sec:concepts:workingcopy}) as well as
\sdirname{.hg} directory.  Figure~\ref{ex:concepts:dirlist} shows the
contents of a freshly created repository.  This repository does not contain
any revisions. Let's take a look at a repository that has history for
several files.
Figure~\ref{ex:concepts:dirlist2} shows the contents of a repository keeping
history on two files.  We see the checked out copies of the files
\filename{foo} and \filename{bar}, as well as the files containing their
histories \filename{foo.i} and \filename{bar.i}, respectively. Additionally,
we see the \filename{changelog.i} and \filename{00manifest.i} files. These
contain the repository-wide revision data, such as the commit message, and
the list of files in the repository during the commit.

\begin{figure}[ht]
  \interaction{concepts.dirlist}
  \caption{Contents of a freshly created repository}
  \label{ex:concepts:dirlist}
\end{figure}

\begin{figure}[ht]
  \interaction{concepts.dirlist2}
  \caption{Contents of a repository tracking two files}
  \label{ex:concepts:dirlist2}
\end{figure}

\subsection{hgrc}
% .hg/hgrc
XXX

\subsection{Creating a Repository}
% hg init
Creating a repository is quick and painless.  One uses the \hgcmd{init}
command as figure~\ref{ex:concepts:hginit} demonstrates.  The one argument
passed to the \hgcmd{init} command is the name of the repository. The name
can be any string usable as a directory name.

\begin{caution}
If you do not specify a name of the repository, the current working
directory will be used instead.
\end{caution}

\begin{figure}[ht]
  \interaction{concepts.hginit}
  \caption{Creating a new repository}
  \label{ex:concepts:hginit}
\end{figure}

\subsection{Remote Repositories}
\label{sec:concepts:remoterepo}
In addition to repositories stored on the local file system, Mercurial
supports so called \emph{remote repositories}.  These remote repositories
can be accessed via several different methods.  See
section~\ref{sec:XXX:remotesetup} for instructions how to set up remote
repositories.
% XXX: reference the proper section!

\subsubsection{SSH}
\label{sec:concepts:remoterepo:ssh}
Mercurial can use \command{ssh} to send and receive changes. The remote
repository is identified by an URL. The basic format for the URL is:

\begin{verbatim}
ssh://[user@]host/path
\end{verbatim}

Where \cmdargs{user} is optional, and the \cmdargs{path} is path to the
repository --- either an absolute or relative to the user's home directory
--- on the remote host with hostname: \cmdargs{host}.

\begin{note}
If the path for the remote repository is absolute there will be two
consecutive slashes.  E.g., if the remote path is \dirname{/repos/hgbook},
the URL would look something like the following:

\begin{verbatim}
ssh://someuser@remotebox//repos/hgbook
\end{verbatim}

Relative paths have only one slash and are relative to the user's home
directory.
\end{note}

\subsubsection{HTTP \& HTTPS}
\label{sec:concepts:remoterepo:http}
The other protocol supported is HTTP as well as HTTPS.  The repository URL
is very much like that of the \command{ssh}.

\begin{verbatim}
http://[user@]remotebox/path
\end{verbatim}

Just as before, the username is optional.
% XXX: is it optional for both push & pull or just for pull?
This time however, the path is relative to the HTTP server root.  

\section{Working Copy}
\label{sec:concepts:workingcopy}
XXX

\section{Revisions}
\label{sec:concepts:revs}
XXX

%%% Local Variables: 
%%% mode: latex
%%% TeX-master: "00book"
%%% End:

