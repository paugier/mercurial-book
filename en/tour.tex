\chapter{A lightning tour of Mercurial}
\label{chap:tour}

\section{Installing Mercurial on your system}
\label{sec:tour:install}

Prebuilt binary packages of Mercurial are available for every popular
operating system.  These make it easy to start using Mercurial on your
computer immediately.

\subsection{Linux}

Because each Linux distribution has its own packaging tools, policies,
and rate of development, it's difficult to give a comprehensive set of
instructions on how to install Mercurial binaries.  The version of
Mercurial that you will end up with can vary depending on how active
the person is who maintains the package for your distribution.

To keep things simple, I will focus on installing Mercurial from the
command line under the most popular Linux distributions.  Most of
these distributions provide graphical package managers that will let
you install Mercurial with a single click; the package name to look
for is \texttt{mercurial}.

\begin{itemize}
\item[Debian]
  \begin{codesample4}
    apt-get install mercurial
  \end{codesample4}

\item[Fedora Core]
  \begin{codesample4}
    yum install mercurial
  \end{codesample4}

\item[Gentoo]
  \begin{codesample4}
    emerge mercurial
  \end{codesample4}

\item[OpenSUSE]
  \begin{codesample4}
    yum install mercurial
  \end{codesample4}

\item[Ubuntu] Ubuntu's Mercurial package is particularly old, and you
  should not use it.  If you know how, you can rebuild and install the
  Debian package.  It's probably easier to build Mercurial from source
  and simply run that; see section~\ref{sec:srcinstall:unixlike} for
  details.
\end{itemize}

\subsection{Mac OS X}

Lee Cantey publishes an installer of Mercurial for Mac OS~X at
\url{http://mercurial.berkwood.com}.  This package works on both
Intel-~and Power-based Macs.  Before you can use it, you must install
a compatible version of Universal MacPython~\cite{web:macpython}.  This
is easy to do; simply follow the instructions on Lee's site.

\subsection{Solaris}

XXX.

\subsection{Windows}

Lee Cantey publishes an installer of Mercurial for Windows at
\url{http://mercurial.berkwood.com}.  This package has no external
dependencies; it ``just works''.

\begin{note}
  The Windows version of Mercurial does not automatically convert line
  endings between Windows and Unix styles.  If you want to share work
  with Unix users, you must do a little additional configuration
  work. XXX Flesh this out.
\end{note}

\section{Getting started}

To begin, we'll use the \hgcmd{version} command to find out whether
Mercurial is actually installed properly.  The actual version
information that it prints isn't so important; it's whether it prints
anything at all that we care about.
\interaction{tour.version}

\subsection{Built-in help}

Mercurial provides a built-in help system.  This invaluable for those
times when you find yourself stuck trying to remember how to run a
command.  If you are completely stuck, simply run \hgcmd{help}; it
will print a brief list of commands, along with a description of what
each does.  If you ask for help on a specific command (as below), it
prints more detailed information.
\interaction{tour.help}
For a more impressive level of detail (which you won't usually need)
run \hgcmdargs{help}{\hggopt{-v}}.  The \hggopt{-v} option is short
for \hggopt{--verbose}, and tells Mercurial to print more information
than it usually would.

\section{Working with a repository}

In Mercurial, everything happens inside a \emph{repository}.  The
repository for a project contains all of the files that ``belong to''
that project, along with a historical record of the project's files.

There's nothing particularly magical about a repository; it is simply
a directory tree in your filesystem that Mercurial treats as special.
You can rename delete a repository any time you like, using either the
command line or your file browser.

\subsection{Making a copy of a repository}

\emph{Copying} a repository is just a little bit special.  While you
could use a normal file copying command to make a copy of a
repository, it's best to use a built-in command that Mercurial
provides.  This command is called \hgcmd{clone}, because it creates an
identical copy of an existing repository.
\interaction{tour.clone}
If our clone succeeded, we should now have a local directory called
\dirname{hello}.  This directory will contain some files.
\interaction{tour.ls}
These files have the same contents and history in our repository as
they do in the repository we cloned.

Every Mercurial repository is complete, self-contained, and
independent.  It contains its own private copy of a project's files
and history.  A cloned repository remembers the location of the
repository it was cloned from, but it does not communicate with that
repository, or any other, unless you tell it to.

What this means for now is that we're free to experiment with our
repository, safe in the knowledge that it's a private ``sandbox'' that
won't affect anyone else.

%%% Local Variables: 
%%% mode: latex
%%% TeX-master: "00book"
%%% End: 
