\chapter{A lightning tour of Mercurial}
\label{chap:tour}

\section{Installing Mercurial on your system}
\label{sec:tour:install}

Prebuilt binary packages of Mercurial are available for every popular
operating system.  These make it easy to start using Mercurial on your
computer immediately.

\subsection{Linux}

Because each Linux distribution has its own packaging tools, policies,
and rate of development, it's difficult to give a comprehensive set of
instructions on how to install Mercurial binaries.  The version of
Mercurial that you will end up with can vary depending on how active
the person is who maintains the package for your distribution.

To keep things simple, I will focus on installing Mercurial from the
command line under the most popular Linux distributions.  Most of
these distributions provide graphical package managers that will let
you install Mercurial with a single click; the package name to look
for is \texttt{mercurial}.

\begin{itemize}
\item[Debian]
  \begin{codesample4}
    apt-get install mercurial
  \end{codesample4}

\item[Fedora Core]
  \begin{codesample4}
    yum install mercurial
  \end{codesample4}

\item[Gentoo]
  \begin{codesample4}
    emerge mercurial
  \end{codesample4}

\item[OpenSUSE]
  \begin{codesample4}
    yum install mercurial
  \end{codesample4}

\item[Ubuntu] Ubuntu's Mercurial package is particularly old, and you
  should not use it.  If you know how, you can rebuild and install the
  Debian package.  It's probably easier to build Mercurial from source
  and simply run that; see section~\ref{sec:srcinstall:unixlike} for
  details.
\end{itemize}

\subsection{Mac OS X}

Lee Cantey publishes an installer of Mercurial for Mac OS~X at
\url{http://mercurial.berkwood.com}.  This package works on both
Intel-~and Power-based Macs.  Before you can use it, you must install
a compatible version of Universal MacPython~\cite{web:macpython}.  This
is easy to do; simply follow the instructions on Lee's site.

\subsection{Solaris}

XXX.

\subsection{Windows}

Lee Cantey publishes an installer of Mercurial for Windows at
\url{http://mercurial.berkwood.com}.  This package has no external
dependencies; it ``just works''.

\begin{note}
  The Windows version of Mercurial does not automatically convert line
  endings between Windows and Unix styles.  If you want to share work
  with Unix users, you must do a little additional configuration
  work. XXX Flesh this out.
\end{note}


%%% Local Variables: 
%%% mode: latex
%%% TeX-master: "00book"
%%% End: 
