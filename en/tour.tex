\chapter{A lightning tour of Mercurial}
\label{chap:tour}

\section{Installing Mercurial on your system}
\label{sec:tour:install}

\subsection{Linux}

All major Linux distributions provide a prebuilt Mercurial package.
Because each Linux distribution has its own packaging tools, policies,
and rate of development, it's difficult to give a comprehensive set of
instructions on how to install Mercurial binaries, and the version of
Mercurial that you will end up with can vary widely.  

To keep things simple, I will focus on installing Mercurial from the
command line under the most popular Linux distributions.  Most of
these distributions provide graphical package managers that will let
you install Mercurial with a single click; the package name to look
for is \texttt{mercurial}.

\subsubsection{Debian}

\begin{codesample2}
  apt-get install mercurial
\end{codesample2}

\subsubsection{Fedora Core}

\begin{codesample2}
  yum install mercurial
\end{codesample2}

\subsubsection{Gentoo}

\begin{codesample2}
  emerge mercurial
\end{codesample2}

\subsubsection{OpenSUSE}

\begin{codesample2}
  yum install mercurial
\end{codesample2}

\subsubsection{Ubuntu}

\begin{codesample2}
  apt-get install mercurial
\end{codesample2}

\subsection{Mac OS X}

Lee Cantey publishes an installer of Mercurial for Mac OS~X at
\url{http://mercurial.berkwood.com}.  This package works on both
Intel- and Power-based Macs, but requires you to install Universal
Python before you can use it.  This is easy to do; simply follow the
instructions on Lee's site.

\subsection{Solaris}

XXX.

\subsection{Windows}

Lee Cantey publishes an installer of Mercurial for Windows at
\url{http://mercurial.berkwood.com}.  This package has no external
dependencies; it ``just works''.

\begin{note}
  The Windows version of Mercurial does not automatically convert line
  endings between Windows and Unix styles.  If you want to share work
  with Unix users, you must do a little additional configuration
  work. XXX Flesh this out.
\end{note}

%%% Local Variables: 
%%% mode: latex
%%% TeX-master: "00book"
%%% End: 
