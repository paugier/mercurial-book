\chapter{Behind the scenes}
\label{chap:concepts}

Unlike many revision control systems, the concepts upon which
Mercurial is built are simple enough that it's easy to understand how
the software really works.  Knowing this certainly isn't necessary,
but I find it useful to have a ``mental model'' of what's going on.

This understanding gives me confidence that Mercurial has been
carefully designed to be both \emph{safe} and \emph{efficient}.  And
just as importantly, if it's easy for me to retain a good idea of what
the software is doing when I perform a revision control task, I'm less
likely to be surprised by its behaviour.

In this chapter, we'll initially cover the core concepts behind
Mercurial's design, then continue to discuss some of the interesting
details of its implementation.

\section{Mercurial's historical record}

\subsection{Tracking the history of a single file}

When Mercurial tracks modifications to a file, it stores the history
of that file in a metadata object called a \emph{filelog}.  Each entry
in the filelog contains enough information to reconstruct one revision
of the file that is being tracked.  Filelogs are stored as files in
the \sdirname{.hg/store/data} directory.  A filelog contains two kinds
of information: revision data, and an index to help Mercurial to find
a revision efficiently.

A file that is large, or has a lot of history, has its filelog stored
in separate data (``\texttt{.d}'' suffix) and index (``\texttt{.i}''
suffix) files.  For small files without much history, the revision
data and index are combined in a single ``\texttt{.i}'' file.  The
correspondence between a file in the working directory and the filelog
that tracks its history in the repository is illustrated in
figure~\ref{fig:concepts:filelog}.

\begin{figure}[ht]
  \centering
  \grafix{filelog}
  \caption{Relationships between files in working directory and filelogs in repository}
  \label{fig:concepts:filelog}
\end{figure}

\subsection{Managing tracked files}

Mercurial uses a structure called a \emph{manifest} to collect
together information about the files that it tracks.  Each entry in
the manifest contains information about the files present in a single
changeset.  An entry records which files are present in the changeset,
the revision of each file, and a few other pieces of file metadata.

\subsection{Recording changeset information}

The \emph{changelog} contains information about each changeset.  Each
revision records who committed a change, the changeset comment, other
pieces of changeset-related information, and the revision of the
manifest to use.

\subsection{Relationships between revisions}

Within a changelog, a manifest, or a filelog, each revision stores a
pointer to its immediate parent (or to its two parents, if it's a
merge revision).  As I mentioned above, there are also relationships
between revisions \emph{across} these structures, and they are
hierarchical in nature.

For every changeset in a repository, there is exactly one revision
stored in the changelog.  Each revision of the changelog contains a
pointer to a single revision of the manifest.  A revision of the
manifest stores a pointer to a single revision of each filelog tracked
when that changeset was created.  These relationships are illustrated
in figure~\ref{fig:concepts:metadata}.

\begin{figure}[ht]
  \centering
  \grafix{metadata}
  \caption{Metadata relationships}
  \label{fig:concepts:metadata}
\end{figure}

As the illustration shows, there is \emph{not} a ``one to one''
relationship between revisions in the changelog, manifest, or filelog.
If the manifest hasn't changed between two changesets, the changelog
entries for those changesets will point to the same revision of the
manifest.  If a file that Mercurial tracks hasn't changed between two
changesets, the entry for that file in the two revisions of the
manifest will point to the same revision of its filelog.

\section{Safe, efficient storage}

The underpinnings of changelogs, manifests, and filelogs are provided
by a single structure called the \emph{revlog}.

\subsection{Efficient storage}

The revlog provides efficient storage of revisions using a
\emph{delta} mechanism.  Instead of storing a complete copy of a file
for each revision, it stores the changes needed to transform an older
revision into the new revision.  For many kinds of file data, these
deltas are typically a fraction of a percent of the size of a full
copy of a file.

Some obsolete revision control systems can only work with deltas of
text files.  They must either store binary files as complete snapshots
or encoded into a text representation, both of which are wasteful
approaches.  Mercurial can efficiently handle deltas of files with
arbitrary binary contents; it doesn't need to treat text as special.

\subsection{Safe operation}
\label{sec:concepts:txn}

Mercurial only ever \emph{appends} data to the end of a revlog file.
It never modifies a section of a file after it has written it.  This
is both more robust and efficient than schemes that need to modify or
rewrite data.

In addition, Mercurial treats every write as part of a
\emph{transaction} that can span a number of files.  A transaction is
\emph{atomic}: either the entire transaction succeeds and its effects
are all visible to readers in one go, or the whole thing is undone.
This guarantee of atomicity means that if you're running two copies of
Mercurial, where one is reading data and one is writing it, the reader
will never see a partially written result that might confuse it.

The fact that Mercurial only appends to files makes it easier to
provide this transactional guarantee.  The easier it is to do stuff
like this, the more confident you should be that it's done correctly.

\subsection{Fast retrieval}

Mercurial cleverly avoids a pitfall common to all earlier
revision control systems: the problem of \emph{inefficient retrieval}.
Most revision control systems store the contents of a revision as an
incremental series of modifications against a ``snapshot''.  To
reconstruct a specific revision, you must first read the snapshot, and
then every one of the revisions between the snapshot and your target
revision.  The more history that a file accumulates, the more
revisions you must read, hence the longer it takes to reconstruct a
particular revision.

\begin{figure}[ht]
  \centering
  \grafix{snapshot}
  \caption{Snapshot of a revlog, with incremental deltas}
  \label{fig:concepts:snapshot}
\end{figure}

The innovation that Mercurial applies to this problem is simple but
effective.  Once the cumulative amount of delta information stored
since the last snapshot exceeds a fixed threshold, it stores a new
snapshot (compressed, of course), instead of another delta.  This
makes it possible to reconstruct \emph{any} revision of a file
quickly.  This approach works so well that it has since been copied by
several other revision control systems.

Figure~\ref{fig:concepts:snapshot} illustrates the idea.  In an entry
in a revlog's index file, Mercurial stores the range of entries from
the data file that it must read to reconstruct a particular revision.

\subsubsection{Aside: the influence of video compression}

If you're familiar with video compression or have ever watched a TV
feed through a digital cable or satellite service, you may know that
most video compression schemes store each frame of video as a delta
against its predecessor frame.  In addition, these schemes use
``lossy'' compression techniques to increase the compression ratio, so
visual errors accumulate over the course of a number of inter-frame
deltas.

Because it's possible for a video stream to ``drop out'' occasionally
due to signal glitches, and to limit the accumulation of artefacts
introduced by the lossy compression process, video encoders
periodically insert a complete frame (called a ``key frame'') into the
video stream; the next delta is generated against that frame.  This
means that if the video signal gets interrupted, it will resume once
the next key frame is received.  Also, the accumulation of encoding
errors restarts anew with each key frame.

\subsection{Identification and strong integrity}

Along with delta or snapshot information, a revlog entry contains a
cryptographic hash of the data that it represents.  This makes it
difficult to forge the contents of a revision, and easy to detect
accidental corruption.  

Hashes provide more than a mere check against corruption; they are
used as the identifiers for revisions.  The changeset identification
hashes that you see as an end user are from revisions of the
changelog.  Although filelogs and the manifest also use hashes,
Mercurial only uses these behind the scenes.

Mercurial verifies that hashes are correct when it retrieves file
revisions and when it pulls changes from another repository.  If it
encounters an integrity problem, it will complain and stop whatever
it's doing.

In addition to the effect it has on retrieval efficiency, Mercurial's
use of periodic snapshots makes it more robust against partial data
corruption.  If a revlog becomes partly corrupted due to a hardware
error or system bug, it's often possible to reconstruct some or most
revisions from the uncorrupted sections of the revlog, both before and
after the corrupted section.  This would not be possible with a
delta-only storage model.

\section{Revision history, branching, and merging}

Every entry in a Mercurial revlog knows the identity of its immediate
ancestor revision, usually referred to as its \emph{parent}.  In fact,
a revision contains room for not one parent, but two.  Mercurial uses
a special hash, called the ``null ID'', to represent the idea ``there
is no parent here''.  This hash is simply a string of zeroes.

In figure~\ref{fig:concepts:revlog}, you can see an example of the
conceptual structure of a revlog.  Filelogs, manifests, and changelogs
all have this same structure; they differ only in the kind of data
stored in each delta or snapshot.

The first revision in a revlog (at the bottom of the image) has the
null ID in both of its parent slots.  For a ``normal'' revision, its
first parent slot contains the ID of its parent revision, and its
second contains the null ID, indicating that the revision has only one
real parent.  Any two revisions that have the same parent ID are
branches.  A revision that represents a merge between branches has two
normal revision IDs in its parent slots.

\begin{figure}[ht]
  \centering
  \grafix{revlog}
  \caption{}
  \label{fig:concepts:revlog}
\end{figure}

\section{The working directory}

In the working directory, Mercurial stores a snapshot of the files
from the repository as of a particular changeset.

The working directory ``knows'' which changeset it contains.  When you
update the working directory to contain a particular changeset,
Mercurial looks up the appropriate revision of the manifest to find
out which files it was tracking at the time that changeset was
committed, and which revision of each file was then current.  It then
recreates a copy of each of those files, with the same contents it had
when the changeset was committed.

The \emph{dirstate} contains Mercurial's knowledge of the working
directory.  This details which changeset the working directory is
updated to, and all of the files that Mercurial is tracking in the
working directory.

Just as a revision of a revlog has room for two parents, so that it
can represent either a normal revision (with one parent) or a merge of
two earlier revisions, the dirstate has slots for two parents.  When
you use the \hgcmd{update} command, the changeset that you update to
is stored in the ``first parent'' slot, and the null ID in the second.
When you \hgcmd{merge} with another changeset, the first parent
remains unchanged, and the second parent is filled in with the
changeset you're merging with.  The \hgcmd{parents} command tells you
what the parents of the dirstate are.

\subsection{What happens when you commit}

The dirstate stores parent information for more than just book-keeping
purposes.  Mercurial uses the parents of the dirstate as \emph{the parents of a new changeset} when you perform a commit.

\begin{figure}[ht]
  \centering
  \grafix{wdir}
  \caption{The working directory can have two parents}
  \label{fig:concepts:wdir}
\end{figure}

Figure~\ref{fig:concepts:wdir} shows the normal state of the working
directory, where it has a single changeset as parent.  That changeset
is the \emph{tip}, the newest changeset in the repository that has no
children.

\begin{figure}[ht]
  \centering
  \grafix{wdir-after-commit}
  \caption{The working directory gains new parents after a commit}
  \label{fig:concepts:wdir-after-commit}
\end{figure}

It's useful to think of the working directory as ``the changeset I'm
about to commit''.  Any files that you tell Mercurial that you've
added, removed, renamed, or copied will be reflected in that
changeset, as will modifications to any files that Mercurial is
already tracking; the new changeset will have the parents of the
working directory as its parents.

After a commit, Mercurial will update the parents of the working
directory, so that the first parent is the ID of the new changeset,
and the second is the null ID.  This is shown in
figure~\ref{fig:concepts:wdir-after-commit}.  Mercurial doesn't touch
any of the files in the working directory when you commit; it just
modifies the dirstate to note its new parents.

\subsection{Creating a new head}

It's perfectly normal to update the working directory to a changeset
other than the current tip.  For example, you might want to know what
your project looked like last Tuesday, or you could be looking through
changesets to see which one introduced a bug.  In cases like this, the
natural thing to do is update the working directory to the changeset
you're interested in, and then examine the files in the working
directory directly to see their contents as they were when you
committed that changeset.  The effect of this is shown in
figure~\ref{fig:concepts:wdir-pre-branch}.

\begin{figure}[ht]
  \centering
  \grafix{wdir-pre-branch}
  \caption{The working directory, updated to an older changeset}
  \label{fig:concepts:wdir-pre-branch}
\end{figure}

Having updated the working directory to an older changeset, what
happens if you make some changes, and then commit?  Mercurial behaves
in the same way as I outlined above.  The parents of the working
directory become the parents of the new changeset.  This new changeset
has no children, so it becomes the new tip.  And the repository now
contains two changesets that have no children; we call these
\emph{heads}.  You can see the structure that this creates in
figure~\ref{fig:concepts:wdir-branch}.

\begin{figure}[ht]
  \centering
  \grafix{wdir-branch}
  \caption{After a commit made while synced to an older changeset}
  \label{fig:concepts:wdir-branch}
\end{figure}

\begin{note}
  If you're new to Mercurial, you should keep in mind a common
  ``error'', which is to use the \hgcmd{pull} command without any
  options.  By default, the \hgcmd{pull} command \emph{does not}
  update the working directory, so you'll bring new changesets into
  your repository, but the working directory will stay synced at the
  same changeset as before the pull.  If you make some changes and
  commit afterwards, you'll thus create a new head, because your
  working directory isn't synced to whatever the current tip is.

  I put the word ``error'' in quotes because all that you need to do
  to rectify this situation is \hgcmd{merge}, then \hgcmd{commit}.  In
  other words, this almost never has negative consequences; it just
  surprises people.  I'll discuss other ways to avoid this behaviour,
  and why Mercurial behaves in this initially surprising way, later
  on.
\end{note}

\subsection{Merging heads}

When you run the \hgcmd{merge} command, Mercurial leaves the first
parent of the working directory unchanged, and sets the second parent
to the changeset you're merging with, as shown in
figure~\ref{fig:concepts:wdir-merge}.

\begin{figure}[ht]
  \centering
  \grafix{wdir-merge}
  \caption{Merging two heads}
  \label{fig:concepts:wdir-merge}
\end{figure}

Mercurial also has to modify the working directory, to merge the files
managed in the two changesets.  Simplified a little, the merging
process goes like this, for every file in the manifests of both
changesets.
\begin{itemize}
\item If neither changeset has modified a file, do nothing with that
  file.
\item If one changeset has modified a file, and the other hasn't,
  create the modified copy of the file in the working directory.
\item If one changeset has removed a file, and the other hasn't (or
  has also deleted it), delete the file from the working directory.
\item If one changeset has removed a file, but the other has modified
  the file, ask the user what to do: keep the modified file, or remove
  it?
\item If both changesets have modified a file, invoke an external
  merge program to choose the new contents for the merged file.  This
  may require input from the user.
\item If one changeset has modified a file, and the other has renamed
  or copied the file, make sure that the changes follow the new name
  of the file.
\end{itemize}
There are more details---merging has plenty of corner cases---but
these are the most common choices that are involved in a merge.  As
you can see, most cases are completely automatic, and indeed most
merges finish automatically, without requiring your input to resolve
any conflicts.

When you're thinking about what happens when you commit after a merge,
once again the working directory is ``the changeset I'm about to
commit''.  After the \hgcmd{merge} command completes, the working
directory has two parents; these will become the parents of the new
changeset.

Mercurial lets you perform multiple merges, but you must commit the
results of each individual merge as you go.  This is necessary because
Mercurial only tracks two parents for both revisions and the working
directory.  While it would be technically possible to merge multiple
changesets at once, the prospect of user confusion and making a
terrible mess of a merge immediately becomes overwhelming.

\section{Other interesting design features}

In the sections above, I've tried to highlight some of the most
important aspects of Mercurial's design, to illustrate that it pays
careful attention to reliability and performance.  However, the
attention to detail doesn't stop there.  There are a number of other
aspects of Mercurial's construction that I personally find
interesting.  I'll detail a few of them here, separate from the ``big
ticket'' items above, so that if you're interested, you can gain a
better idea of the amount of thinking that goes into a well-designed
system.

\subsection{Clever compression}

When appropriate, Mercurial will store both snapshots and deltas in
compressed form.  It does this by always \emph{trying to} compress a
snapshot or delta, but only storing the compressed version if it's
smaller than the uncompressed version.

This means that Mercurial does ``the right thing'' when storing a file
whose native form is compressed, such as a \texttt{zip} archive or a
JPEG image.  When these types of files are compressed a second time,
the resulting file is usually bigger than the once-compressed form,
and so Mercurial will store the plain \texttt{zip} or JPEG.

Deltas between revisions of a compressed file are usually larger than
snapshots of the file, and Mercurial again does ``the right thing'' in
these cases.  It finds that such a delta exceeds the threshold at
which it should store a complete snapshot of the file, so it stores
the snapshot, again saving space compared to a naive delta-only
approach.

\subsubsection{Network recompression}

When storing revisions on disk, Mercurial uses the ``deflate''
compression algorithm (the same one used by the popular \texttt{zip}
archive format), which balances good speed with a respectable
compression ratio.  However, when transmitting revision data over a
network connection, Mercurial uncompresses the compressed revision
data.

If the connection is over HTTP, Mercurial recompresses the entire
stream of data using a compression algorithm that gives a better
compression ratio (the Burrows-Wheeler algorithm from the widely used
\texttt{bzip2} compression package).  This combination of algorithm
and compression of the entire stream (instead of a revision at a time)
substantially reduces the number of bytes to be transferred, yielding
better network performance over almost all kinds of network.

(If the connection is over \command{ssh}, Mercurial \emph{doesn't}
recompress the stream, because \command{ssh} can already do this
itself.)

\subsection{Read/write ordering and atomicity}

Appending to files isn't the whole story when it comes to guaranteeing
that a reader won't see a partial write.  If you recall
figure~\ref{fig:concepts:metadata}, revisions in the changelog point to
revisions in the manifest, and revisions in the manifest point to
revisions in filelogs.  This hierarchy is deliberate.

A writer starts a transaction by writing filelog and manifest data,
and doesn't write any changelog data until those are finished.  A
reader starts by reading changelog data, then manifest data, followed
by filelog data.

Since the writer has always finished writing filelog and manifest data
before it writes to the changelog, a reader will never read a pointer
to a partially written manifest revision from the changelog, and it will
never read a pointer to a partially written filelog revision from the
manifest.

\subsection{Concurrent access}

The read/write ordering and atomicity guarantees mean that Mercurial
never needs to \emph{lock} a repository when it's reading data, even
if the repository is being written to while the read is occurring.
This has a big effect on scalability; you can have an arbitrary number
of Mercurial processes safely reading data from a repository safely
all at once, no matter whether it's being written to or not.

The lockless nature of reading means that if you're sharing a
repository on a multi-user system, you don't need to grant other local
users permission to \emph{write} to your repository in order for them
to be able to clone it or pull changes from it; they only need
\emph{read} permission.  (This is \emph{not} a common feature among
revision control systems, so don't take it for granted!  Most require
readers to be able to lock a repository to access it safely, and this
requires write permission on at least one directory, which of course
makes for all kinds of nasty and annoying security and administrative
problems.)

Mercurial uses locks to ensure that only one process can write to a
repository at a time (the locking mechanism is safe even over
filesystems that are notoriously hostile to locking, such as NFS).  If
a repository is locked, a writer will wait for a while to retry if the
repository becomes unlocked, but if the repository remains locked for
too long, the process attempting to write will time out after a while.
This means that your daily automated scripts won't get stuck forever
and pile up if a system crashes unnoticed, for example.  (Yes, the
timeout is configurable, from zero to infinity.)

\subsubsection{Safe dirstate access}

As with revision data, Mercurial doesn't take a lock to read the
dirstate file; it does acquire a lock to write it.  To avoid the
possibility of reading a partially written copy of the dirstate file,
Mercurial writes to a file with a unique name in the same directory as
the dirstate file, then renames the temporary file atomically to
\filename{dirstate}.  The file named \filename{dirstate} is thus
guaranteed to be complete, not partially written.

\subsection{Avoiding seeks}

Critical to Mercurial's performance is the avoidance of seeks of the
disk head, since any seek is far more expensive than even a
comparatively large read operation.

This is why, for example, the dirstate is stored in a single file.  If
there were a dirstate file per directory that Mercurial tracked, the
disk would seek once per directory.  Instead, Mercurial reads the
entire single dirstate file in one step.

Mercurial also uses a ``copy on write'' scheme when cloning a
repository on local storage.  Instead of copying every revlog file
from the old repository into the new repository, it makes a ``hard
link'', which is a shorthand way to say ``these two names point to the
same file''.  When Mercurial is about to write to one of a revlog's
files, it checks to see if the number of names pointing at the file is
greater than one.  If it is, more than one repository is using the
file, so Mercurial makes a new copy of the file that is private to
this repository.

A few revision control developers have pointed out that this idea of
making a complete private copy of a file is not very efficient in its
use of storage.  While this is true, storage is cheap, and this method
gives the highest performance while deferring most book-keeping to the
operating system.  An alternative scheme would most likely reduce
performance and increase the complexity of the software, each of which
is much more important to the ``feel'' of day-to-day use.

\subsection{Other contents of the dirstate}

Because Mercurial doesn't force you to tell it when you're modifying a
file, it uses the dirstate to store some extra information so it can
determine efficiently whether you have modified a file.  For each file
in the working directory, it stores the time that it last modified the
file itself, and the size of the file at that time.  

When you explicitly \hgcmd{add}, \hgcmd{remove}, \hgcmd{rename} or
\hgcmd{copy} files, Mercurial updates the dirstate so that it knows
what to do with those files when you commit.

When Mercurial is checking the states of files in the working
directory, it first checks a file's modification time.  If that has
not changed, the file must not have been modified.  If the file's size
has changed, the file must have been modified.  If the modification
time has changed, but the size has not, only then does Mercurial need
to read the actual contents of the file to see if they've changed.
Storing these few extra pieces of information dramatically reduces the
amount of data that Mercurial needs to read, which yields large
performance improvements compared to other revision control systems.

%%% Local Variables: 
%%% mode: latex
%%% TeX-master: "00book"
%%% End:
