% Bug ID.
\newcommand{\bug}[1]{\index{Mercurial bug
    database!\href{http://www.selenic.com/mercurial/bts/issue#1}{bug
      ~#1}}\href{http://www.selenic.com/mercurial/bts/issue#1}{Mercurial
    bug no.~#1}}

% File name in the user's home directory.
\newcommand{\tildefile}[1]{\texttt{\~{}/#1}}

% File name.
\newcommand{\filename}[1]{\texttt{#1}}

% Directory name.
\newcommand{\dirname}[1]{\texttt{#1}}

% File name, with index entry.
% The ``s'' prefix comes from ``special''.
\newcommand{\sfilename}[1]{\index{\texttt{#1} file}\texttt{#1}}

% Directory name, with index entry.
\newcommand{\sdirname}[1]{\index{\texttt{#1} directory}\texttt{#1}}

% Mercurial extension.
\newcommand{\hgext}[1]{\index{\texttt{#1} extension}\texttt{#1}}

% Mercurial command.
\newcommand{\hgcmd}[1]{\index{\texttt{#1} command}``\texttt{hg #1}''}

% Mercurial command, with arguments.
\newcommand{\hgcmdargs}[2]{\index{\texttt{#1} command}``\texttt{hg #1 #2}''}

% Shell/system command.
\newcommand{\command}[1]{\index{\texttt{#1} command}\texttt{#1}}

% Shell/system command, with arguments.
\newcommand{\cmdargs}[2]{\index{\texttt{#1} command}``\texttt{#1 #2}''}

% Mercurial command option.
\newcommand{\hgopt}[2]{\index{\texttt{#1} command!\texttt{#2} option}\texttt{#2}}

% Mercurial global option.
\newcommand{\hggopt}[1]{\index{global options!\texttt{#1} option}\texttt{#1}}

% Shell/system command option.
\newcommand{\cmdopt}[2]{\index{\texttt{#1} command!\texttt{#2} option}\texttt{#2}}

% Command option.
\newcommand{\option}[1]{\texttt{#1}}

% Software package.
\newcommand{\package}[1]{\index{\texttt{#1} package}\texttt{#1}}

% Section name from a hgrc file.
\newcommand{\rcsection}[1]{\index{\texttt{hgrc} file!\texttt{#1} section}\texttt{[#1]}}

% Named item in a hgrc file section.
\newcommand{\rcitem}[2]{\index{\texttt{hgrc} file!\texttt{#1}
    section!\texttt{#2} entry}\texttt{#1.#2}}

% hgrc file.
\newcommand{\hgrc}{\index{\texttt{hgrc} file}\texttt{hgrc}}

% Hook name.
\newcommand{\hook}[1]{\index{\texttt{#1} hook}\index{hooks!\texttt{#1}}\texttt{#1}}

% Environment variable.
\newcommand{\envar}[1]{\index{\texttt{#1} environment
    variable}\index{environment variables!\texttt{#1}}\texttt{#1}}

% Python module.
\newcommand{\pymod}[1]{\index{\texttt{#1} module}\texttt{#1}}

% Python class in a module.
\newcommand{\pymodclass}[2]{\index{\texttt{#1} module!\texttt{#2}
    class}\texttt{#1.#2}}

% Note: blah blah.
\newsavebox{\notebox}
\newenvironment{note}%
  {\begin{lrbox}{\notebox}\begin{minipage}{0.7\textwidth}\textbf{Note:}\space}%
  {\end{minipage}\end{lrbox}\fbox{\usebox{\notebox}}}

% Code sample, eating 4 characters of leading space.
\DefineVerbatimEnvironment{codesample4}{Verbatim}{frame=single,gobble=4,numbers=left,commandchars=\\\{\}}

% Code sample, eating 2 characters of leading space.
\DefineVerbatimEnvironment{codesample2}{Verbatim}{frame=single,gobble=2,numbers=left,commandchars=\\\{\}}

% Interaction from the examples directory.
\newcommand{\interaction}[1]{\VerbatimInput[frame=single,numbers=left,commandchars=\\\{\}]{examples/#1.out}}

% Graphics inclusion.
\ifpdf
  \newcommand{\grafix}[1]{\includegraphics{#1.pdf}}
\else
  \newcommand{\grafix}[1]{\includegraphics{#1.png}}
\fi

%%% Local Variables: 
%%% mode: latex
%%% TeX-master: "00book"
%%% End: 
