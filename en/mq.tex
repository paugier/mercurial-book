\chapter{Managing change with Mercurial Queues}
\label{chap:mq}

\section{The patch management problem}
\label{sec:mq:patch-mgmt}

Here is a common scenario: you need to install a software package from
source, but you find a bug that you must fix in the source before you
can start using the package.  You make your changes, forget about the
package for a while, and a few months later you need to upgrade to a
newer version of the package.  If the newer version of the package
still has the bug, you must extract your fix from the older source
tree and apply it against the newer version.  This is a tedious task,
and it's easy to make mistakes.

This is a simple case of the ``patch management'' problem.  You have
an ``upstream'' source tree that you can't change; you need to make
some local changes on top of the upstream tree; and you'd like to be
able to keep those changes separate, so that you can apply them to
newer versions of the upstream source.

The patch management problem arises in many situations.  Probably the
most visible is that a user of an open source software project will
contribute a bug fix or new feature to the project's maintainers in the
form of a patch.

Distributors of operating systems that include open source software
often need to make changes to the packages they distribute so that
they will build properly in their environments.

When you have few changes to maintain, it is easy to manage a single
patch using the standard \texttt{diff} and \texttt{patch} programs.
Once the number of changes grows, it starts to makes sense to maintain
patches as discrete ``chunks of work,'' so that for example a single
patch will contain only one bug fix (the patch might modify several
files, but it's doing ``only one thing''), and you may have a number
of such patches for different bugs you need fixed and local changes
you require.  In this situation, if you submit a bug fix patch to the
upstream maintainers of a package and they include your fix in a
subsequent release, you can simply drop that single patch when you're
updating to the newer release.

Maintaining a single patch against an upstream tree is a little
tedious and error-prone, but not difficult.  However, the complexity
of the problem grows rapidly as the number of patches you have to
maintain increases.  With more than a tiny number of patches in hand,
understanding which ones you have applied and maintaining them moves
from messy to overwhelming.

Fortunately, Mercurial includes a powerful extension, Mercurial Queues
(or simply ``MQ''), that massively simplifies the patch management
problem.

\section{The prehistory of Mercurial Queues}
\label{sec:mq:history}

During the late 1990s, several Linux kernel developers started to
maintain ``patch series'' that modified the behaviour of the Linux
kernel.  Some of these series were focused on stability, some on
feature coverage, and others were more speculative.

The sizes of these patch series grew rapidly.  In 2002, Andrew Morton
published some shell scripts he had been using to automate the task of
managing his patch queues.  Andrew was successfully using these
scripts to manage hundreds (sometimes thousands) of patches on top of
the Linux kernel.

\subsection{A patchwork quilt}
\label{sec:mq:quilt}


In early 2003, Andreas Gruenbacher and Martin Quinson borrowed the
approach of Andrew's scripts and published a tool called ``patchwork
quilt''~\cite{web:quilt}, or simply ``quilt''
(see~\cite{gruenbacher:2005} for a paper describing it).  Because
quilt substantially automated patch management, it rapidly gained a
large following among open source software developers.

Quilt manages a \emph{stack of patches} on top of a directory tree.
To begin, you tell quilt to manage a directory tree; it stores away
the names and contents of all files in the tree.  To fix a bug, you
create a new patch (using a single command), edit the files you need
to fix, then ``refresh'' the patch.  

The refresh step causes quilt to scan the directory tree; it updates
the patch with all of the changes you have made.  You can create
another patch on top of the first, which will track the changes
required to modify the tree from ``tree with one patch applied'' to
``tree with two patches applied''.

You can \emph{change} which patches are applied to the tree.  If you
``pop'' a patch, the changes made by that patch will vanish from the
directory tree.  Quilt remembers which patches you have popped,
though, so you can ``push'' a popped patch again, and the directory
tree will be restored to contain the modifications in the patch.  Most
importantly, you can run the ``refresh'' command at any time, and the
topmost applied patch will be updated.  This means that you can, at
any time, change both which patches are applied and what
modifications those patches make.

Quilt knows nothing about revision control tools, so it works equally
well on top of an unpacked tarball or a Subversion repository.

\subsection{From patchwork quilt to Mercurial Queues}
\label{sec:mq:quilt-mq}

In mid-2005, Chris Mason took the features of quilt and wrote an
extension that he called Mercurial Queues, which added quilt-like
behaviour to Mercurial.

The key difference between quilt and MQ is that quilt knows nothing
about revision control systems, while MQ is \emph{integrated} into
Mercurial.  Each patch that you push is represented as a Mercurial
changeset.  Pop a patch, and the changeset goes away.

This integration makes understanding patches and debugging their
effects \emph{enormously} easier.  Since every applied patch has an
associated changeset, you can use \hgcmdargs{log}{\emph{filename}} to
see which changesets and patches affected a file.  You can use the
\hgext{bisect} extension to binary-search through all changesets and
applied patches to see where a bug got introduced or fixed.  You can
use the \hgcmd{annotate} command to see which changeset or patch
modified a particular line of a source file.  And so on.

Because quilt does not care about revision control tools, it is still
a tremendously useful piece of software to know about for situations
where you cannot use Mercurial and MQ.
\section{Getting started with Mercurial Queues}
\label{sec:mq:start}

Because MQ is implemented as an extension, you must explicitly enable
before you can use it.  (You don't need to download anything; MQ ships
with the standard Mercurial distribution.)  To enable MQ, edit your
\tildefile{.hgrc} file, and add the lines in figure~\ref{ex:mq:config}.

\begin{figure}[h]
  \begin{codesample4}
    [extensions]
    hgext.mq =
  \end{codesample4}
  \label{ex:mq:config}
  \caption{Contents to add to \tildefile{.hgrc} to enable the MQ extension}
\end{figure}

Once the extension is enabled, it will make a number of new commands
available.  To verify that the extension is working, you can use
\hgcmd{help} to see if the \hgcmd{qinit} command is now available; see
the example in figure~\ref{ex:mq:enabled}.

\begin{figure}[h]
  \interaction{mq.qinit-help.help}
  \caption{How to verify that MQ is enabled}
  \label{ex:mq:enabled}
\end{figure}

You can use MQ with \emph{any} Mercurial repository; to start, simply
prepare the repository using the \hgcmd{qinit} command (see
figure~\ref{ex:mq:qinit}).  This command creates an empty directory
called \filename{.hg/patches}, where MQ will keep its metadata.  As
with many Mercurial commands, the \hgcmd{qinit} command prints nothing
if it succeeds.

\begin{figure}[h]
  \interaction{mq.tutorial.qinit}
  \caption{Preparing a repository for use with MQ}
  \label{ex:mq:qinit}
\end{figure}

\begin{figure}[h]
  \interaction{mq.tutorial.qnew}
  \caption{Creating a new patch}
  \label{ex:mq:qnew}
\end{figure}

To commence work on a new patch, use the \hgcmd{qnew} command.  This
command takes one argument, the name of the patch to create.  MQ will
use this as the name of an actual file in the \filename{.hg/patches}
directory, as you can see in figure~\ref{ex:mq:qnew}.

Now also present in the \filename{.hg/patches} directory are two new
files, \filename{series} and \filename{status}.  The \filename{series}
file lists all of the patches that MQ knows about for this repository,
with one patch per line.  The \filename{status} file lists all of the
patches that MQ has \emph{applied} in this repository.

\begin{note}
  You may sometimes want to edit the \filename{series} file by hand;
  for example, to change the sequence in which some patches are
  applied.  However, manually editing the \filename{status} file is
  almost always a bad idea, as it's easy to corrupt MQ's idea of what
  is happening.
\end{note}

%%% Local Variables: 
%%% mode: latex
%%% TeX-master: "00book"
%%% End: 
