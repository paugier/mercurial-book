\chapter{A tour of Mercurial: merging work}
\label{chap:tour-merge}

We've now covered cloning a repository, making changes in a
repository, and pulling or pushing changes from one repository into
another.  Our next step is \emph{merging} changes from separate
repositories.

\section{Merging streams of work}

Merging is a fundamental part of working with a distributed revision
control tool.
\begin{itemize}
\item Alice and Bob each have a personal copy of a repository for a
  project they're collaborating on.  Alice fixes a bug in her
  repository; Bob adds a new feature in his.  They want the shared
  repository to contain both the bug fix and the new feature.
\item I frequently work on several different tasks for a single
  project at once, each safely isolated in its own repository.
  Working this way means that I often need to merge one piece of my
  own work with another.
\end{itemize}

Because merging is such a common thing to need to do, Mercurial makes
it easy.  Let's walk through the process.  We'll begin by cloning yet
another repository (see how often they spring up?) and making a change
in it.
\interaction{tour.merge.clone}
We should now have two copies of \filename{hello.c} with different
contents.
\interaction{tour.merge.cat}

We already know that pulling changes from our \dirname{my-hello}
repository will have no effect on the working directory.
\interaction{tour.merge.pull}
However, the \hgcmd{pull} command says something about ``heads''.  

A head is a change that has no descendants.  The tip revision is thus
a head, but a repository can contain more than one head.  We can view
them using the \hgcmd{heads} command.
\interaction{tour.merge.heads}
What happens if we try to use the normal \hgcmd{update} command to
update to the new tip?
\interaction{tour.merge.update}
Mercurial is telling us that the \hgcmd{update} command won't do a
merge.  Instead, we use the \hgcmd{merge} command to merge the two
heads.
\interaction{tour.merge.merge}
This updates the working directory so that it contains changes from
both heads, which is reflected in both the output of \hgcmd{parents}
and the contents of \filename{hello.c}.
\interaction{tour.merge.parents}
Whenever we've done a merge, \hgcmd{parents} will display two parents
until we \hgcmd{commit} the results of the merge.
\interaction{tour.merge.commit}
We now have a new tip revision; notice that it has \emph{both} of
our former heads as its parents.  These are the same revisions that
were previously displayed by \hgcmd{parents}.
\interaction{tour.merge.tip}

%%% Local Variables: 
%%% mode: latex
%%% TeX-master: "00book"
%%% End: 
