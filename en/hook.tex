\chapter{Handling repository events with hooks}
\label{chap:hook}

Mercurial offers a powerful mechanism to let you perform automated
actions in response to events that occur in a repository.  In some
cases, you can even control Mercurial's response to those events.

The name Mercurial uses for one of these actions is a \emph{hook}.
Hooks are called ``triggers'' in some revision control systems, but
the two names refer to the same idea.

\section{An overview of hooks in Mercurial}

Here is a brief list of the hooks that Mercurial supports. For each
hook, we indicate when it is run, and a few examples of common tasks
you can use it for.  We will revisit each of these hooks in more
detail later.
\begin{itemize}
\item[\small\hook{changegroup}] This is run after a group of
  changesets has been brought into the repository from elsewhere.  In
  other words, it is run after a \hgcmd{pull} or \hgcmd{push} into a
  repository, but not after a \hgcmd{commit}.  You can use this for
  performing an action once for the entire group of newly arrived
  changesets.  For example, you could use this hook to send out email
  notifications, or kick off an automated build or test.
\item[\small\hook{commit}] This is run after a new changeset has been
  created in the local repository, typically using the \hgcmd{commit}
  command.
\item[\small\hook{incoming}] This is run once for each new changeset
  that is brought into the repository from elsewhere.  Notice the
  difference from \hook{changegroup}, which is run once per
  \emph{group} of changesets brought in.  You can use this for the
  same purposes as the \hook{changegroup} hook; it's simply more
  convenient sometimes to run a hook once per group of changesets,
  while othher times it's handier once per changeset.
\item[\small\hook{outgoing}] This is run after a group of changesets
  has been transmitted from this repository to another.  You can use
  this, for example, to notify subscribers every time changes are
  cloned or pulled from the repository.
\item[\small\hook{prechangegroup}] This is run before starting to
  bring a group of changesets into the repository.  It cannot see the
  actual changesets, because they have not yet been transmitted.  If
  it fails, the changesets will not be transmitted.  You can use this
  hook to ``lock down'' a repository against incoming changes.
\item[\small\hook{precommit}] This is run before starting a commit.
  It cannot tell what files are included in the commit, or any other
  information about the commit.  If it fails, the commit will not be
  allowed to start.  You can use this to perform a build and require
  it to complete successfully before a commit can proceed, or
  automatically enforce a requirement that modified files pass your
  coding style guidelines.
\item[\small\hook{preoutgoing}] This is run before starting to
  transmit a group of changesets from this repository.  You can use
  this to lock a repository against clones or pulls from remote
  clients.
\item[\small\hook{pretag}] This is run before creating a tag.  If it
  fails, the tag will not be created.  You can use this to enforce a
  uniform tag naming convention.
\item[\small\hook{pretxnchangegroup}] This is run after a group of
  changesets has been brought into the local repository from another,
  but before the transaction completes that will make the changes
  permanent in the repository.  If it fails, the transaction will be
  rolled back and the changes will disappear from the local
  repository.  You can use this to automatically check newly arrived
  changes and, for example, roll them back if the group as a whole
  does not build or pass your test suite.
\item[\small\hook{pretxncommit}] This is run after a new changeset has
  been created in the local repository, but before the transaction
  completes that will make it permanent.  Unlike the \hook{precommit}
  hook, this hook can see which changes are present in the changeset,
  and it can also see all other changeset metadata, such as the commit
  message.  You can use this to require that a commit message follows
  your local conventions, or that a changeset builds cleanly.
\item[\small\hook{preupdate}] This is run before starting an update or
  merge of the working directory.
\item[\small\hook{tag}] This is run after a tag is created.
\item[\small\hook{update}] This is run after an update or merge of the
  working directory has finished.
\end{itemize}
Each of the hooks with a ``\texttt{pre}'' prefix has the ability to
\emph{control} an activity.  If the hook succeeds, the activity may
proceed; if it fails, the activity is either not permitted or undone,
depending on the hook.

\section{Hooks and security}

\subsection{Hooks are run with your privileges}

When you run a Mercurial command in a repository, and the command
causes a hook to run, that hook runs on your system, under your user
account, with your privilege level.  Since hooks are arbitrary pieces
of executable code, you should treat them with an appropriate level of
suspicion.  Do not install a hook unless you are confident that you
know who created it and what it does.

In some cases, you may be exposed to hooks that you did not install
yourself.  If you work with Mercurial on an unfamiliar system,
Mercurial will run hooks defined in that system's global \hgrc\ file.

If you are working with a repository owned by another user, Mercurial
will run hooks defined in that repository.  For example, if you
\hgcmd{pull} from that repository, and its \sfilename{.hg/hgrc}
defines a local \hook{outgoing} hook, that hook will run under your
user account, even though you don't own that repository.

\begin{note}
  This only applies if you are pulling from a repository on a local or
  network filesystem.  If you're pulling over http or ssh, any
  \hook{outgoing} hook will run under the account of the server
  process, on the server.
\end{note}

XXX To see what hooks are defined in a repository, use the
\hgcmdargs{config}{hooks} command.  If you are working in one
repository, but talking to another that you do not own (e.g.~using
\hgcmd{pull} or \hgcmd{incoming}), remember that it is the other
repository's hooks you should be checking, not your own.

\subsection{Hooks do not propagate}

In Mercurial, hooks are not revision controlled, and do not propagate
when you clone, or pull from, a repository.  The reason for this is
simple: a hook is a completely arbitrary piece of executable code.  It
runs under your user identity, with your privilege level, on your
machine.

It would be extremely reckless for any distributed revision control
system to implement revision-controlled hooks, as this would offer an
easily exploitable way to subvert the accounts of users of the
revision control system.

Since Mercurial does not propagate hooks, if you are collaborating
with other people on a common project, you should not assume that they
are using the same Mercurial hooks as you are, or that theirs are
correctly configured.  You should document the hooks you expect people
to use.

In a corporate intranet, this is somewhat easier to control, as you
can for example provide a ``standard'' installation of Mercurial on an
NFS filesystem, and use a site-wide \hgrc\ file to define hooks that
all users will see.  However, this too has its limits; see below.

\subsection{Hooks can be overridden}

Mercurial allows you to override a hook definition by redefining the
hook.  You can disable it by setting its value to the empty string, or
change its behaviour as you wish.

If you deploy a system-~or site-wide \hgrc\ file that defines some
hooks, you should thus understand that your users can disable or
override those hooks.

\subsection{Ensuring that critical hooks are run}

Sometimes you may want to enforce a policy that you do not want others
to be able to work around.  For example, you may have a requirement
that every changeset must pass a rigorous set of tests.  Defining this
requirement via a hook in a site-wide \hgrc\ won't work for remote
users on laptops, and of course local users can subvert it at will by
overriding the hook.

Instead, you can set up your policies for use of Mercurial so that
people are expected to propagate changes through a well-known
``canonical'' server that you have locked down and configured
appropriately.

One way to do this is via a combination of social engineering and
technology.  Set up a restricted-access account; users can push
changes over the network to repositories managed by this account, but
they cannot log into the account and run normal shell commands.  In
this scenario, a user can commit a changeset that contains any old
garbage they want.

When someone pushes a changeset to the server that everyone pulls
from, the server will test the changeset before it accepts it as
permanent, and reject it if it fails to pass the test suite.  If
people only pull changes from this filtering server, it will serve to
ensure that all changes that people pull have been automatically
vetted.

\section{A short tutorial on using hooks}
\label{sec:hook:simple}

It is easy to write a Mercurial hook.  Let's start with a hook that
runs when you finish a \hgcmd{commit}, and simply prints the hash of
the changeset you just created.  The hook is called \hook{commit}.

\begin{figure}[ht]
  \interaction{hook.simple.init}
  \caption{A simple hook that runs when a changeset is committed}
  \label{ex:hook:init}
\end{figure}

All hooks follow the pattern in example~\ref{ex:hook:init}.  You add
an entry to the \rcsection{hooks} section of your \hgrc\.  On the left
is the name of the event to trigger on; on the right is the action to
take.  As you can see, you can run an arbitrary shell command in a
hook.  Mercurial passes extra information to the hook using
environment variables (look for \envar{HG\_NODE} in the example).

\subsection{Performing multiple actions per event}

Quite often, you will want to define more than one hook for a
particular kind of event, as shown in example~\ref{ex:hook:ext}.
Mercurial lets you do this by adding an \emph{extension} to the end of
a hook's name.  You extend a hook's name by giving the name of the
hook, followed by a full stop (the ``\texttt{.}'' character), followed
by some more text of your choosing.  For example, Mercurial will run
both \texttt{commit.foo} and \texttt{commit.bar} when the
\texttt{commit} event occurs.

\begin{figure}[ht]
  \interaction{hook.simple.ext}
  \caption{Defining a second \hook{commit} hook}
  \label{ex:hook:ext}
\end{figure}

To give a well-defined order of execution when there are multiple
hooks defined for an event, Mercurial sorts hooks by extension, and
executes the hook commands in this sorted order.  In the above
example, it will execute \texttt{commit.bar} before
\texttt{commit.foo}, and \texttt{commit} before both.

It is a good idea to use a somewhat descriptive extension when you
define a new hook.  This will help you to remember what the hook was
for.  If the hook fails, you'll get an error message that contains the
hook name and extension, so using a descriptive extension could give
you an immediate hint as to why the hook failed (see
section~\ref{sec:hook:perm} for an example).

\subsection{Controlling whether an activity can proceed}
\label{sec:hook:perm}

In our earlier examples, we used the \hook{commit} hook, which is
run after a commit has completed.  This is one of several Mercurial
hooks that run after an activity finishes.  Such hooks have no way of
influencing the activity itself.

Mercurial defines a number of events that occur before an activity
starts; or after it starts, but before it finishes.  Hooks that
trigger on these events have the added ability to choose whether the
activity can continue, or will abort.  

The \hook{pretxncommit} hook runs after a commit has all but
completed.  In other words, the metadata representing the changeset
has been written out to disk, but the transaction has not yet been
allowed to complete.  The \hook{pretxncommit} hook has the ability to
decide whether the transaction can complete, or must be rolled back.

If the \hook{pretxncommit} hook exits with a status code of zero, the
transaction is allowed to complete; the commit finishes; and the
\hook{commit} hook is run.  If the \hook{pretxncommit} hook exits with
a non-zero status code, the transaction is rolled back; the metadata
representing the changeset is erased; and the \hook{commit} hook is
not run.

\begin{figure}[ht]
  \interaction{hook.simple.pretxncommit}
  \caption{Using the \hook{pretxncommit} hook to control commits}
  \label{ex:hook:pretxncommit}
\end{figure}

The hook in example~\ref{ex:hook:pretxncommit} checks that a commit
comment contains a bug ID.  If it does, the commit can complete.  If
not, the commit is rolled back.

\section{Writing your own hooks}

When you are writing a hook, you might find it useful to run Mercurial
either with the \hggopt{-v} option, or the \rcitem{ui}{verbose} config
item set to ``true''.  When you do so, Mercurial will print a message
before it calls each hook.

\subsection{Choosing how your hook should run}
\label{sec:hook:lang}

You can write a hook either as a normal program---typically a shell
script---or as a Python function that is executed within the Mercurial
process.

Writing a hook as an external program has the advantage that it
requires no knowledge of Mercurial's internals.  You can call normal
Mercurial commands to get any added information you need.  The
trade-off is that external hooks are slower than in-process hooks.

An in-process Python hook has complete access to the Mercurial API,
and does not ``shell out'' to another process, so it is inherently
faster than an external hook.  It is also easier to obtain much of the
information that a hook requires by using the Mercurial API than by
running Mercurial commands.

If you are comfortable with Python, or require high performance,
writing your hooks in Python may be a good choice.  However, when you
have a straightforward hook to write and you don't need to care about
performance (probably the majority of hooks), a shell script is
perfectly fine.

\subsection{Hook parameters}
\label{sec:hook:param}

Mercurial calls each hook with a set of well-defined parameters.  In
Python, a parameter is passed as a keyword argument to your hook
function.  For an external program, a parameter is passed as an
environment variable.

Whether your hook is written in Python or as a shell script, the
hook-specific parameter names and values will be the same.  A boolean
parameter will be represented as a boolean value in Python, but as the
number 1 (for ``true'') or 0 (for ``false'') as an environment
variable for an external hook.  If a hook parameter is named
\texttt{foo}, the keyword argument for a Python hook will also be
named \texttt{foo} Python, while the environment variable for an
external hook will be named \texttt{HG\_FOO}.

\subsection{Hook return values and activity control}

A hook that executes successfully must exit with a status of zero if
external, or return boolean ``false'' if in-process.  Failure is
indicated with a non-zero exit status from an external hook, or an
in-process hook returning boolean ``true''.  If an in-process hook
raises an exception, the hook is considered to have failed.

For a hook that controls whether an activity can proceed, zero/false
means ``allow'', while non-zero/true/exception means ``deny''.

\subsection{Writing an external hook}

When you define an external hook in your \hgrc\ and the hook is run,
its value is passed to your shell, which interprets it.  This means
that you can use normal shell constructs in the body of the hook.

An executable hook is always run with its current directory set to a
repository's root directory.

Each hook parameter is passed in as an environment variable; the name
is upper-cased, and prefixed with the string ``\texttt{HG\_}''.

With the exception of hook parameters, Mercurial does not set or
modify any environment variables when running a hook.  This is useful
to remember if you are writing a site-wide hook that may be run by a
number of different users with differing environment variables set.
In multi-user situations, you should not rely on environment variables
being set to the values you have in your environment when testing the
hook.

\subsection{Telling Mercurial to use an in-process hook}

The \hgrc\ syntax for defining an in-process hook is slightly
different than for an executable hook.  The value of the hook must
start with the text ``\texttt{python:}'', and continue with the
fully-qualified name of a callable object to use as the hook's value.

The module in which a hook lives is automatically imported when a hook
is run.  So long as you have the module name and \envar{PYTHONPATH}
right, it should ``just work''.

The following \hgrc\ example snippet illustrates the syntax and
meaning of the notions we just described.
\begin{codesample2}
  [hooks]
  commit.example = python:mymodule.submodule.myhook
\end{codesample2}
When Mercurial runs the \texttt{commit.example} hook, it imports
\texttt{mymodule.submodule}, looks for the callable object named
\texttt{myhook}, and calls it.

\subsection{Writing an in-process hook}

The simplest in-process hook does nothing, but illustrates the basic
shape of the hook API:
\begin{codesample2}
  def myhook(ui, repo, **kwargs):
      pass
\end{codesample2}
The first argument to a Python hook is always a
\pymodclass{mercurial.ui}{ui} object.  The second is a repository object;
at the moment, it is always an instance of
\pymodclass{mercurial.localrepo}{localrepository}.  Following these two
arguments are other keyword arguments.  Which ones are passed in
depends on the hook being called, but a hook can ignore arguments it
doesn't care about by dropping them into a keyword argument dict, as
with \texttt{**kwargs} above.

\section{Hook reference}


\subsection{In-process hook execution}

An in-process hook is called with arguments of the following form:
\begin{codesample2}
  def myhook(ui, repo, **kwargs):
      pass
\end{codesample2}
The \texttt{ui} parameter is a \pymodclass{mercurial.ui}{ui} object.
The \texttt{repo} parameter is a
\pymodclass{mercurial.localrepo}{localrepository} object.  The
names and values of the \texttt{**kwargs} parameters depend on the
hook being invoked, with the following common features:
\begin{itemize}
\item If a parameter is named \texttt{node} or
  \texttt{parent\emph{N}}, it will contain a hexadecimal changeset ID.
  The empty string is used to represent ``null changeset ID'' instead
  of a string of zeroes.
\item Boolean-valued parameters are represented as Python
  \texttt{bool} objects.
\end{itemize}

An in-process hook is called without a change to the process's working
directory (unlike external hooks, which are run in the root of the
repository).  It must not change the process's working directory.  If
it were to do so, it would probably cause calls to the Mercurial API,
or operations after the hook finishes, to fail.

If a hook returns a boolean ``false'' value, it is considered to
have succeeded.  If it returns a boolean ``true'' value or raises an
exception, it is considered to have failed.

\subsection{External hook execution}

An external hook is passed to the user's shell for execution, so
features of that shell, such as variable substitution and command
redirection, are available.  The hook is run in the root directory of
the repository.

Hook parameters are passed to the hook as environment variables.  Each
environment variable's name is converted in upper case and prefixed
with the string ``\texttt{HG\_}''.  For example, if the name of a
parameter is ``\texttt{node}'', the name of the environment variable
representing that parameter will be ``\texttt{HG\_NODE}''.

A boolean parameter is represented as the string ``\texttt{1}'' for
``true'', ``\texttt{0}'' for ``false''.  If an environment variable is
named \envar{HG\_NODE}, \envar{HG\_PARENT1} or \envar{HG\_PARENT2}, it
contains a changeset ID represented as a hexadecimal string.  The
empty string is used to represent ``null changeset ID'' instead of a
string of zeroes.

If a hook exits with a status of zero, it is considered to have
succeeded.  If it exits with a non-zero status, it is considered to
have failed.

\subsection{The \hook{changegroup} hook}
\label{sec:hook:changegroup}

This hook is run after a group of pre-existing changesets has been
added to the repository, for example via a \hgcmd{pull} or
\hgcmd{unbundle}.  This hook is run once per operation that added one
or more changesets.

Parameters to this hook:
\begin{itemize}
\item[\texttt{node}] A changeset ID.  The changeset ID of the first
  changeset in the group that was added.  All changesets between this
  and \index{tags!\texttt{tip}}\texttt{tip}, inclusive, were added by
  a single \hgcmd{pull}, \hgcmd{push} or \hgcmd{unbundle}.
\end{itemize}

See also: \hook{incoming} (section~\ref{sec:hook:incoming}),
\hook{prechangegroup} (section~\ref{sec:hook:prechangegroup}),
\hook{pretxnchangegroup} (section~\ref{sec:hook:pretxnchangegroup})

\subsection{The \hook{commit} hook}
\label{sec:hook:commit}

This hook is run after a new changeset has been created.

Parameters to this hook:
\begin{itemize}
\item[\texttt{node}] A changeset ID.  The changeset ID of the newly
  committed changeset.
\item[\texttt{parent1}] A changeset ID.  The changeset ID of the first
  parent of the newly committed changeset.
\item[\texttt{parent2}] A changeset ID.  The changeset ID of the second
  parent of the newly committed changeset.
\end{itemize}

See also: \hook{precommit} (section~\ref{sec:hook:precommit}),
\hook{pretxncommit} (section~\ref{sec:hook:pretxncommit})

\subsection{The \hook{incoming} hook}
\label{sec:hook:incoming}

This hook is run after a pre-existing changeset has been added to the
repository, for example via a \hgcmd{push}.  If a group of changesets
was added in a single operation, this hook is called once for each
added changeset.

Parameters to this hook:
\begin{itemize}
\item[\texttt{node}] A changeset ID.  The ID of the newly added
  changeset.
\end{itemize}

See also: \hook{changegroup} (section~\ref{sec:hook:changegroup}) \hook{prechangegroup} (section~\ref{sec:hook:prechangegroup}), \hook{pretxnchangegroup} (section~\ref{sec:hook:pretxnchangegroup})

\subsection{The \hook{outgoing} hook}
\label{sec:hook:outgoing}

This hook is run after a group of changesets has been propagated out
of this repository, for example by a \hgcmd{push} or \hgcmd{bundle}
command.

Parameters to this hook:
\begin{itemize}
\item[\texttt{node}] A changeset ID.  The changeset ID of the first
  changeset of the group that was sent.
\item[\texttt{source}] A string.  The source of the of the operation.
  If a remote client pulled changes from this repository,
  \texttt{source} will be \texttt{serve}.  If the client that obtained
  changes from this repository was local, \texttt{source} will be
  \texttt{bundle}, \texttt{pull}, or \texttt{push}, depending on the
  operation the client performed.
\end{itemize}

See also: \hook{preoutgoing} (section~\ref{sec:hook:preoutgoing})

\subsection{The \hook{prechangegroup} hook}
\label{sec:hook:prechangegroup}

This hook is not passed any parameters.

See also: \hook{changegroup} (section~\ref{sec:hook:changegroup}),
\hook{incoming} (section~\ref{sec:hook:incoming}), ,
\hook{pretxnchangegroup} (section~\ref{sec:hook:pretxnchangegroup})

\subsection{The \hook{precommit} hook}
\label{sec:hook:precommit}

This hook is invoked before Mercurial has obtained any of the metadata
for the commit, such as the commit message or date.

Parameters to this hook:
\begin{itemize}
\item[\texttt{parent1}] A changeset ID.  The changeset ID of the first
  parent of the working directory.
\item[\texttt{parent2}] A changeset ID.  The changeset ID of the second
  parent of the working directory.
\end{itemize}
If the commit proceeds, the parents of the working directory will
become the parents of the new changeset.

See also: \hook{commit} (section~\ref{sec:hook:commit}),
\hook{pretxncommit} (section~\ref{sec:hook:pretxncommit})

\subsection{The \hook{preoutgoing} hook}
\label{sec:hook:preoutgoing}

This hook is invoked before Mercurial knows the identities of the
changesets to be transmitted.

Parameters to this hook:
\begin{itemize}
\item[\texttt{source}] A string.  The source of the operation that is
  attempting to obtain changes from this repository.  See the
  documentation for the \texttt{source} parameter to the
  \hook{outgoing} hook, in section~\ref{sec:hook:outgoing}, for
  possible values of this parameter..
\end{itemize}

See also: \hook{outgoing} (section~\ref{sec:hook:outgoing})

\subsection{The \hook{pretag} hook}
\label{sec:hook:pretag}

Parameters to this hook:
\begin{itemize}
\item[\texttt{local}] A boolean.  Whether the tag is local to this
  repository instance (i.e.~stored in \sfilename{.hg/tags}) or managed
  by Mercurial (stored in \sfilename{.hgtags}).
\item[\texttt{node}] A changeset ID.  The ID of the changeset to be tagged.
\item[\texttt{tag}] A string.  The name of the tag to be created.
\end{itemize}

If the tag to be created is revision-controlled, the \hook{precommit}
and \hook{pretxncommit} hooks (sections~\ref{sec:hook:commit}
and~\ref{sec:hook:pretxncommit}) will also be run.

See also: \hook{tag} (section~\ref{sec:hook:tag})

\subsection{The \hook{pretxnchangegroup} hook}
\label{sec:hook:pretxnchangegroup}

Parameters to this hook are the same as for the \hook{changegroup}
hook; see section~\ref{sec:hook:changegroup} for details.

See also: \hook{changegroup} (section~\ref{sec:hook:changegroup}),
\hook{incoming} (section~\ref{sec:hook:incoming}),
\hook{prechangegroup} (section~\ref{sec:hook:prechangegroup})

\subsection{The \hook{pretxncommit} hook}
\label{sec:hook:pretxncommit}

Parameters to this hook are the same as for the \hook{commit} hook;
see section~\ref{sec:hook:commit} for details.

See also: \hook{precommit} (section~\ref{sec:hook:precommit})

\subsection{The \hook{preupdate} hook}
\label{sec:hook:preupdate}

Parameters to this hook:
\begin{itemize}
\item[\texttt{parent1}] A changeset ID.  The ID of the parent that the
  working directory is to be updated to.  If the working directory is
  being merged, it will not change this parent.
\item[\texttt{parent2}] A changeset ID.  Only set if the working
  directory is being merged.  The ID of the revision that the working
  directory is being merged with.
\end{itemize}

See also: \hook{update} (section~\ref{sec:hook:update})

\subsection{The \hook{tag} hook}
\label{sec:hook:tag}

Parameters to this hook are the same as for the \hook{pretag} hook;
see section~\ref{sec:hook:pretag} for details.

If the created tag is revision-controlled, the \hook{commit} hook
(section~\ref{sec:hook:commit}) will also be run.

See also: \hook{pretag} (section~\ref{sec:hook:pretag})

\subsection{The \hook{update} hook}
\label{sec:hook:update}

\begin{itemize}
\item[\texttt{error}] A boolean.  Indicates whether the update or
  merge completed successfully.
\item[\texttt{parent1}] A changeset ID.  The ID of the parent that the
  working directory was updated to.  If the working directory was
  merged, it will not have changed this parent.
\item[\texttt{parent2}] A changeset ID.  Only set if the working
  directory was merged.  The ID of the revision that the working
  directory was merged with.
\end{itemize}

See also: \hook{preupdate} (section~\ref{sec:hook:preupdate})

%%% Local Variables: 
%%% mode: latex
%%% TeX-master: "00book"
%%% End: 
